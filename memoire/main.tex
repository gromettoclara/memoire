\documentclass[a4paper,12pt,twoside]{book}
\usepackage[T1]{fontenc}
\usepackage{inputenc}
\usepackage{fontspec}
\usepackage{lmodern}
\usepackage[english,french]{babel}
\usepackage{xspace} % pour la gestion des espaces après les commandes
\usepackage{minted}
\usepackage{csquotes}
\usepackage{graphicx}
\usepackage{pdfpages}
\usepackage{float}


% Mise en page École des chartes
\usepackage[margin=2.5cm]{geometry} % marges
\usepackage{setspace}
\onehalfspacing % interligne de 1.5
\setlength\parindent{1cm}

% Biblio
\usepackage[backend=biber, sorting=nyt, style=enc, minbibnames=10, maxbibnames=10]{biblatex}
\addbibresource{bibliographie/biblio.bib}
\nocite{*}
\defbibnote{intro}{Cette bibliographie présente toutes les ressources utilisées, de tout type, citées ou non, classées thématiquement.}

% Blocs de code
\usepackage{listings}
\usepackage{color}

% Couleurs personnalisées
\definecolor{gray}{rgb}{0.4,0.4,0.4}
\definecolor{darkblue}{rgb}{0.0,0.0,0.6}
\definecolor{cyan}{rgb}{0.0,0.6,0.6}
\definecolor{lightgray}{rgb}{0.95,0.95,0.95}
\definecolor{keywordcolor}{rgb}{0.5,0.0,0.5}
\definecolor{stringcolor}{rgb}{0.8,0.2,0.2}
\definecolor{functioncolor}{rgb}{0.1,0.5,0.1}
\definecolor{csscolor}{rgb}{0.1,0.4,0.8}
\definecolor{lightgray}{rgb}{0.95, 0.95, 0.95}
\definecolor{darkgray}{rgb}{0.4, 0.4, 0.4}
\definecolor{purple}{rgb}{0.65, 0.12, 0.82}
\definecolor{ocherCode}{rgb}{1, 0.5, 0} % #FF7F00 -> rgb(239, 169, 0)
\definecolor{blueCode}{rgb}{0, 0, 0.93} % #0000EE -> rgb(0, 0, 238)
\definecolor{greenCode}{rgb}{0, 0.6, 0} % #009900 -> rgb(0, 153, 0) 

% Paramètres globaux
\lstset{
	basicstyle=\ttfamily\small,
	columns=fullflexible,
	showstringspaces=false,
	commentstyle=\color{gray}\upshape,
	captionpos=b,
	backgroundcolor=\color{lightgray},
}

% Style pour le langage Python
\lstdefinelanguage{Python}{
	morekeywords={def, return, if, elif, else, for, while, break, continue, and, or, not, in, import, from, as, class, pass, True, False, None, lambda},
	keywordstyle=\color{keywordcolor}\bfseries,
	ndkeywords={self},
	ndkeywordstyle=\color{darkblue}\bfseries,
	commentstyle=\color{gray}\itshape,
	stringstyle=\color{stringcolor},
	identifierstyle=\color{functioncolor},
	morestring=[b]',
	morestring=[b]",
	morecomment=[s]{'''}{'''},
	morecomment=[s]{"""}{"""},
}

% Style pour le langage CSS
\lstdefinelanguage{CSS}{
	morekeywords={color, background, margin, padding, border, width, height, font, display, position, top, bottom, left, right, content},
	keywordstyle=\color{csscolor}\bfseries,
	commentstyle=\color{gray}\itshape,
	stringstyle=\color{stringcolor},
	morestring=[b]",
	morestring=[b]',
	morecomment=[s]{/*}{*/},
}

% Style pour le langage XML
\lstdefinelanguage{XML}
{
	morestring=[b]",
	morestring=[s]{>}{<},
	morecomment=[s]{<?}{?>},
	stringstyle=\color{black},
	identifierstyle=\color{darkblue},
	keywordstyle=\color{cyan},
	morekeywords={xmlns,version,type}% list your attributes here
}

% Style pour HTML avec JavaScript
\lstdefinelanguage{html5}{
	language=HTML,
	morekeywords={html,head,title,meta,link,style,script,body,h1,h2,h3,h4,h5,h6,div,span,p,a,img,ul,ol,li,table,tr,td,th,form,input,button,textarea,select,option,label},
	keywordstyle=\color{cyan}\bfseries,
	identifierstyle=\color{darkblue},
	stringstyle=\color{stringcolor},
	morestring=[b]",
	morestring=[b]',
	commentstyle=\color{gray}\itshape,
	morecomment=[s]{<!--}{-->},
	alsoletter={:},
	alsodigit={-},
	morekeywords=[2]{var,function,let,const,if,else,for,while,do,break,continue,switch,case,default,return,try,catch,finally,new,this,throw,true,false,null,undefined,typeof,instanceof},
	keywordstyle=[2]\color{purple}\bfseries,
	morekeywords=[3]{console,document,window},
	keywordstyle=[3]\color{ocherCode}\bfseries,
	sensitive=true,
	morecomment=[l]{//},
	morecomment=[s]{/*}{*/},
	morestring=[b]',
	morestring=[b]",
}


\usepackage[pdfusetitle, pdfsubject={Mémoire TNAH — Titre}, pdfkeywords={mot1, mot2, mot3}]{hyperref}

\author{Clara Grometto – M2 TNAH — ENC}
\title{Le partage des outils de la recherche}

% Changer le style de description de manière à ce que les acronymes dans la liste des acronymes apparaissent en petites capitales
\usepackage{enumitem}
\setlist[description]{labelwidth=2em, labelsep=.5em, font=\normalfont}

% ACRONYMS
\usepackage[automake, acronym, toc]{glossaries}
\makeglossaries

\setacronymstyle{short-long}

\newacronym{alto}{\textsc{alto}}{\emph{Analysed Layout and Text Object}}
\newacronym{api}{\textsc{api}}{\emph{Application Programming Interface}}
\newacronym{ark}{\textsc{ark}}{\emph{Archival Resource Key}}
\newacronym{bnf}{\textsc{b}n\textsc{f}}{\emph{Bibliothèque Nationale de France}}
\newacronym{cite}{\textsc{cite}}{\emph{Collections, Indices, Texts, and Extensions}}
\newacronym{cnn}{\textsc{cnn}}{\emph{Convolutional Neural Network}}
\newacronym{cpu}{\textsc{cpu}}{\emph{Computing Processing Unit}}
\newacronym{dishas}{\textsc{dishas}}{\emph{Digital Information System for the History of Astral Sciences}}
\newacronym{dom}{\textsc{dom}}{\emph{Document Object Model}}
\newacronym{eida}{\textsc{eida}}{\emph{Editing and analysing hIstorical astronomical Diagrams with Artificial intelligence}}
\newacronym{fair}{\textsc{fair}}{\emph{Findable Accessible Interoperable Reusable}}
\newacronym{gbs}{\textsc{gbs}}{\emph{Google Books Search}}
\newacronym{gpu}{\textsc{gpu}}{\emph{Graphics Processing Unit}}
\newacronym{hn}{\textsc{hn}}{\emph{Humanités Numériques}}
\newacronym{html}{\textsc{html}}{\emph{HyperText Markup Language}}
\newacronym{htr}{\textsc{htr}}{\emph{Handwritten Text Recognition}}
\newacronym{http}{\textsc{http}}{\emph{Hypertext Transfer Protocol}}
\newacronym{ia}{\textsc{ia}}{\emph{Intelligence Artificielle}}
\newacronym{IIIF}{\textsc{iiif}}{\emph{International Image Interoperability Framework}}
\newacronym{imagine}{\textsc{imagine}}{\emph{Laboratoire d’Informatique Gaspard Monge}}
\newacronym{iscd}{\textsc{iscd}}{\emph{Institut des sciences du calcul et des données}}
\newacronym{jpeg}{\textsc{jpeg}}{\emph{Joint Photographic Experts Group}}
\newacronym{json}{\textsc{json}}{\emph{JavaScript Object Notation}}
\newacronym{mpa}{\textsc{mpa}}{\emph{Multiple Page Application}}
\newacronym{ocr}{\textsc{ocr}}{\emph{Optical Character Recognition}}
\newacronym{odd}{\textsc{oca}}{\emph{Open Content Alliance}}
\newacronym{odd}{\textsc{odd}}{\emph{One Document Does it all}}
\newacronym{png}{\textsc{png}}{\emph{Portable Network Graphics}}
\newacronym{ransac}{\textsc{ransac}}{\emph{RANdom SAmple Consensus}}
\newacronym{rdf}{\textsc{rdf}}{\emph{Resource Description Framework}}
\newacronym{sas}{\textsc{sas}}{\emph{Simple Annotation Server}}
\newacronym{shs}{\textsc{shs}}{\emph{Sciences Humaines et Sociales}}
\newacronym{spa}{\textsc{spa}}{\emph{Single Page Application}}
\newacronym{si}{\textsc{si}}{\emph{Systèmes d'Information}}
\newacronym{sru}{\textsc{sru}}{\emph{Search/Retrieve via URL}}
\newacronym{svg}{\textsc{svg}}{\emph{Scalable Vector Graphics}}
\newacronym{syrte}{\textsc{syrte}}{\emph{Systèmes de Référence Temps-Espace}}
\newacronym{tal}{\textsc{tal}}{\emph{Traitement Automatique du Langage}}
\newacronym{tei}{\textsc{tei}}{\emph{Text Encoding Initiative}}
\newacronym{ui}{\textsc{ui}}{\emph{User Interface}}
\newacronym{uri}{\textsc{uri}}{\emph{Uniform Resource Identifier}}
\newacronym{url}{\textsc{url}}{\emph{Uniform Resource Locator}}
\newacronym{urn}{\textsc{urn}}{\emph{Uniform Resource Name}}
\newacronym{ux}{\textsc{ux}}{\emph{User eXperience}}
\newacronym{vhs}{\textsc{vhs}}{\emph{Vision artificielle et analyse Historique de la circulation de l'illustration Scientifique}}
\newacronym{w3c}{\textsc{w3c}}{\emph{World Wide Web Consortium}}
\newacronym{xml}{\textsc{xml}}{\emph{eXtensible Markup Language}}
\newacronym{yolo}{\textsc{yolo}}{\emph{You Only Look Once}}

% COMMANDS
\newcommand{\enc}{École nationale des chartes\xspace}
\newcommand{\fair}{\gls{fair}\xspace}
\newcommand{\api}{\gls{api}\xspace}
\newcommand{\apis}{\gls{api}s\xspace}
\newcommand{\bnf}{\gls{bnf}\xspace}
\newcommand{\eida}{\gls{eida}\xspace}
\newcommand{\aikon}{\textsc{aikon}\xspace}
\newcommand{\syrte}{\gls{syrte}\xspace}
\newcommand{\tei}{\gls{tei}\xspace}
\newcommand{\iiif}{\gls{IIIF}\xspace}
\newcommand{\wit}{\texttt{Witness}\xspace}
\newcommand{\wo}{\texttt{Work}\xspace}
\newcommand{\ser}{\texttt{Serie}\xspace}
\newcommand{\man}{\emph{Manifest}\xspace} % ????? 
\newcommand{\wits}{\texttt{Witnesses}\xspace}
\newcommand{\wos}{\texttt{Works}\xspace}
\newcommand{\sers}{\texttt{Series}\xspace}
\newcommand{\digit}{\texttt{Digitization}\xspace}
\newcommand{\digits}{\texttt{Digitizations}\xspace}
\newcommand{\mans}{\emph{Manifests}\xspace}
\newcommand{\cpu}{\gls{cpu}\xspace}
\newcommand{\cv}{\emph{computer vision}\xspace}
\newcommand{\dishas}{\gls{dishas}\xspace}
\newcommand{\dl}{\emph{deep learning}\xspace}
\newcommand{\enherit}{\gls{enherit}\xspace}
\newcommand{\tal}{\gls{tal}\xspace}
\newcommand{\htr}{\gls{htr}\xspace}
\newcommand{\gpu}{\gls{gpu}\xspace}
\newcommand{\http}{\gls{http}\xspace}
\newcommand{\gbs}{\gls{gbs}\xspace}
\newcommand{\oca}{\gls{oca}\xspace}
\newcommand{\ia}{\gls{ia}\xspace}
\newcommand{\imagine}{\gls{imagine}\xspace}
\newcommand{\iscd}{\gls{iscd}\xspace}
\newcommand{\json}{\gls{json}\xspace}
\newcommand{\html}{\gls{html}\xspace}
\newcommand{\ml}{\emph{machine learning}\xspace}
\newcommand{\svg}{\gls{svg}\xspace}
\newcommand{\svgs}{\gls{svg}s\xspace}
\newcommand{\uri}{\gls{uri}\xspace}
\newcommand{\uris}{\gls{uri}s\xspace}
\newcommand{\URL}{\gls{url}\xspace}
\newcommand{\URLs}{\gls{url}s\xspace}
\newcommand{\urn}{\gls{urn}\xspace}
\newcommand{\urns}{\gls{urn}s\xspace}
\newcommand{\vhs}{\gls{vhs}\xspace}
\newcommand{\yolo}{\gls{yolo}\xspace}
\newcommand{\rdf}{\gls{rdf}\xspace}
\newcommand{\ocr}{\gls{ocr}\xspace}
\newcommand{\ark}{\gls{ark}\xspace}
\newcommand{\odd}{\gls{odd}\xspace}
\newcommand{\xml}{\gls{xml}\xspace}
\newcommand{\alto}{\gls{alto}\xspace}
\newcommand{\cnn}{\gls{cnn}\xspace}
\newcommand{\cnns}{\gls{cnn}s\xspace}
\newcommand{\jpeg}{\gls{jpeg}\xspace}
\newcommand{\png}{\gls{png}\xspace}
\newcommand{\hn}{\gls{hn}\xspace}
\newcommand{\ux}{\gls{ux}\xspace}
\newcommand{\ui}{\gls{ui}\xspace}
\newcommand{\wtc}{\gls{w3c}\xspace}
\newcommand{\shs}{\gls{shs}\xspace}
\newcommand{\dom}{\gls{dom}\xspace}
\newcommand{\si}{\gls{si}\xspace}
\newcommand{\yolov}{YOLOv5\xspace}
\newcommand{\jc}{av. J.-C.\xspace}
\newcommand{\ma}{Moyen Âge\xspace}
\newcommand{\vecto}{vectorisation\xspace}
\newcommand{\gaga}{\emph{Gallic(orpor)a}\xspace}
\newcommand{\spa}{\gls{spa}\xspace}
\newcommand{\sru}{\gls{sru}\xspace}
\newcommand{\mpa}{\gls{mpa}\xspace}
\newcommand{\ransac}{\gls{ransac}\xspace}
\newcommand{\CITE}{\gls{cite}\xspace}
\newcommand{\sas}{\gls{sas}\xspace}
\newcommand{\graphical}{\texttt{Graphical Element}\xspace}
\newcommand{\graphicals}{\texttt{Graphical Elements}\xspace}
\newcommand{\tr}{\texttt{Treatment}\xspace}
\newcommand{\trs}{\texttt{Treatments}\xspace}
\newcommand{\ds}{\texttt{Document Set}\xspace}
\newcommand{\rs}{\texttt{Region Set}\xspace}
\newcommand{\dss}{\texttt{Document Sets}\xspace}
\newcommand{\rss}{\texttt{Region Sets}\xspace}
\def\cdt{\kern-0.5pt\ensuremath\cdot\kern-0.5pt}

% Pour retirer le titre courant d'une page vide avant un chapitre
\newcommand{\clearemptydoublepage}{\newpage{\pagestyle{empty}\cleardoublepage}}

% Pour afficher le titre courant d'un chapitre non numéroté (intro, conclusion, etc.)
\usepackage{fancyhdr}
\pagestyle{fancy}
\fancyfoot{}
\fancyhead[RO,LE]{\thepage}      
\fancyhead[LO]{\small \slshape \leftmark} 
\fancyhead[RE]{\small \slshape \rightmark} 
\renewcommand{\headrulewidth}{0pt}

% Pour des sections non numérotées dans la table des matière
\newcommand\chapterNo[1]{%
 \chapter*{#1}
  \markboth{}{} %vider les en-têtes 
  \markright{\MakeUppercase{#1}}
}

% Création de l'environnement pour citer
\newenvironment{kwote}
  {
    \begin{quote}
    \begin{singlespace}
    \small
  }
  {
    \normalsize
    \end{singlespace}
    \end{quote}
  }


\begin{document}

\onehalfspacing 

\frontmatter

    \begin{titlepage}
    \begin{center}
        
        \bigskip
        
        \begin{large}
            ÉCOLE NATIONALE DES CHARTES
        \end{large}
        \begin{center}\rule{2cm}{0.02cm}\end{center}
        
        \bigskip
        \bigskip
        \bigskip
        \begin{Large}
            \textbf{Clara GROMETTO}\\
        \end{Large}
        \begin{normalsize} \textit{licenciée ès lettres}\\
        \end{normalsize}
        
        \bigskip
        \bigskip
        \bigskip
        
        \begin{Huge}
            \textbf{Le partage des outils de la recherche}\\
        \end{Huge}
        \bigskip
        \bigskip
        \begin{LARGE}
            \textbf{Élaboration d'une plateforme extensive pour le traitement des données visuelles}\\
        \end{LARGE}
        
        \bigskip
        \bigskip
        \bigskip
        \begin{large}
        \end{large}
        \vfill
        
        \begin{large}
            Mémoire 
            pour le diplôme de master \\
            \og Technologies numériques appliquées à l'histoire~\fg\\
            \bigskip
            2024
        \end{large}
        
    \end{center}
\end{titlepage}

    \thispagestyle{empty}	
    \clearemptydoublepage
	
    \chapterNo{Résumé}
\addcontentsline{toc}{chapter}{Résumé}
\medskip	

Ce mémoire porte sur l'intégration des outils numériques dans la recherche en \shs, notamment l'utilisation de la \cv pour l'enrichissement et la sémantification des sources historiques. Le projet central étudié est \eida, porteur du développement de la plateforme \aikon, cette dernière mettant à disposition des outils de \dl pour l'extraction et l'analyse des diagrammes astronomiques de tradition ptoléméenne.

Le mémoire pose la question du niveau de spécificité ou de généralité à prévoir dans le développement des outils numériques~: comment construire une chaîne de traitement à la fois flexible, adaptée à des données variées, et capable de répondre à des besoins spécifiques~? Via cette question, ce travail explore les opportunités liés à l'intégration des technologies numériques dans la recherche en \shs, et relève les défis tenant au partage des outils, impactant le partage des pratiques et des méthodes. 

\textbf{Mots-clés~:} histoire des science~; humanités numérique~; diagrammes astronomiques~; \ia~; \dl~; Python~; modularité~; standardisation technique~; interopérabilité.\\

\textbf{Informations bibliographiques~:} Clara GROMETTO, \textit{La standardisation des outils numériques dans la recherche, Élaboration d'une plateforme extensive pour le traitement des données visuelles basé sur la vision artificielle}, mémoire de master \og Technologies numériques appliquées à l'histoire~\fg, dir. Ségolène Albouy, École nationale des chartes, 2024.
	
\clearemptydoublepage
	
    \chapterNo{Remerciements}
    \addcontentsline{toc}{chapter}{Remerciements}
 
 Avant tout, je tiens à exprimer ma profonde gratitude à toutes les personnes ayant contribué à faire de ce stage une expérience profondément enrichissante sur le plan intellectuel comme personnel. Merci à ma directrice de mémoire Ségolène Albouy et ma tutrice Jade Norindr pour leur soutien, leurs conseils avisés, leur patience, leur disponibilité et leur générosité. Je me sens extrêmement chanceuse d'avoir pu profiter de leur expertise et découvrir leurs qualités humaines, elles forcent l'admiration. Un grand merci à Matthieu Husson ainsi qu'à tous les chercheur.ses d'\eida : Scott, Divna, Samuel, Chen\ldots Merci à Éleonora pour son délicieux cours de philologie. J'ai trouvé à l'Observatoire un environnement valorisant et encourageant, ce qui a été extrêmement précieux.
 
Merci à tous les étudiants de la promotion 2024 du master TNAH pour leur soutien et leur amitié. L'émulation intellectuelle qui règne au sein de notre groupe a été un moteur essentiel et a rendu cette année de master particulièrement stimulante. 
 
 Enfin, merci à mes parents, mes plus fidèles relecteurs, pour leur intérêt curieux et leur soutien indéfectible pendant la douloureuse rédaction de ce mémoire. Et merci à Jingwei pour son enthousiasme à toute épreuve, sa douceur et son oreille attentive. 
    
    \clearemptydoublepage
    
    \part*{Bibliographie}
    \clearemptydoublepage
    \printbibliography[keyword={hist astro},title={Histoire de l'astronomie}]
    \clearemptydoublepage
    \printbibliography[keyword={ia},title={IA~: généralités}]
    \clearemptydoublepage
    \printbibliography[keyword={eda},title={Documentation technique, méthodes, projets annexes}]
    \clearemptydoublepage
    \printbibliography[keyword={edit},title={Problématiques d'édition}]
    \clearemptydoublepage
    
    \phantomsection
    \addcontentsline{toc}{chapter}{Introduction}
    \chapterNo{Introduction}
    \begin{kwote}
``La notion de pensée spirituelle n’a pas de sens et ce que l’on croit relever d’une aptitude intellectuelle extraordinaire consiste, peut-être à 95 \%, en une maîtrise de notre système de signes et de ses combinaisons, qui peut certes confiner à l’art (la possibilité d’enchaîner des centaines, voire des milliers de gestes, d’algorithmes, de recettes ou de formules) mais qui reste majoritairement technique~: une somme d’apprentissages tout à fait accessibles et qui conduisent à des pratiques et des gestes enchaînés de façon de plus en plus rapide avec leur répétition. Nous retrouvons là les propos de Leibniz, de Dagognet et de Granger.''\footcite{guichard_linternet_2014}
\end{kwote}

Guichard souligne ici que la pensée s'ancre profondément dans la technique. À l'heure du numérique, chaque projet de recherche, avec ses sources et ses questions, a des besoins épistémologiques précis qui s’incarnent dans des structures informatiques particulières~: des protocoles, des formats, des dispositifs, des visualisations de données, etc. Au cœur de cette idiosyncrasie, comment créer des outils numériques qui s'inscrivent dans un écosystème plus large, tout en restant attentif aux spécificités de chaque objet et de chaque question de recherche~?

Cette question est centrale dans la construction d'une chaîne de traitement des sources. Le mythe du savant isolé, reclus dans son observatoire, penché sur ses grimoires, fait place à la réalité du collectif. La collaboration est désormais de mise~: pour favoriser le partage et l'accès aux données, pour garantir la reproductibilité des résultats et assurer la cohérence des pratiques. Malgré cela, la prolifération des outils spécialisés, souvent conçus pour un projet particulier, persiste dans le domaine des Humanités Numériques. Non seulement nécessitent-ils la mobilisation de ressources humaines et techniques importantes, mais aussi se révèlent-ils difficilement maintenables sur le long terme. Devant ces constats, la question se pose~: comment mettre en place des formes de mutualisation sur le plan technique~? Cet enjeu se joue à plusieurs niveaux~: ouverture des silos de donnée, scripting, architecture applicative et infrastructure hardware, protocoles pour diffuser et maintenir ces outils, pour collaborer autour de leur développement\ldots

Le principal défi réside dans la conciliation des exigences de généralité et de spécificité, pour créer des outils numériques qui non seulement répondent aux besoins spécifiques du projet qui les porte, mais aussi servent une communauté scientifique plus large et interdisciplinaire. Comment concevoir des systèmes suffisamment flexibles sans sacrifier leur pertinence et leur efficacité~? 

Ce mémoire propose une exploration de ces problématiques en s'appuyant sur un cas d'étude~: la construction de la plateforme \aikon, qui met à disposition des outils basés sur la \cv pour l'enrichissement, la sémantification et l'analyse des données visuelles. 

\section{Mise en contexte}

\subsection{Un changement de paradigme}

La numérisation massive de sources archivistiques et bibliographiques a radicalement transformé le paysage de la recherche en sciences humaines. Les ressources numériques et leurs usages se cessent de se développer et se diversifier. Les ouvrages et les manuscrits scientifiques, notamment, constituent une source essentielle pour la connaissance de l’histoire des sciences.

\begin{kwote}
``The availability of large amounts of digitized historical documents opened the door to the use of computational approaches for their analysis.''\footcite[p.2]{buttner_cordeep_2022}
\end{kwote}

La disponibilité des données numérisées est au fondement de la démarche basée sur des traitements automatiques des sources par des algorithmes de vision artificielle. Ces immenses corpus offrent des opportunités inédites pour comprendre les phénomènes culturels, sociaux et historiques à une échelle sans précédent, soulevant cependant des défis techniques et méthodologiques~: en premier lieu l'accès à la donnée, en second lieu, sa sémantification.  

Le décloisonnement des silos d'information est un enjeu majeur de la gouvernance des données, et le premier pas vers leur exploitation. Le concept d'\textit{open-data} renvoie à une ambition d'ouverture des données pour leur libre circulation et exploitation. Il concerne à la fois la diffusion des sources numériques auprès d'un public très large\footnote{Auprès de la communauté scientifique comme du grand public.} et l'enrichissement des ressources par la communauté scientifique. Le recours à des standards communs et des mécanismes d'échange normalisés est essentiel pour garantir l'interopérabilité des données et faciliter leur intégration dans des diverses infrastructures. En effet, la capacité à intégrer des données provenant de sources multiples et hétérogènes est indispensable pour exploiter pleinement le potentiel des ressources numérisées dans la recherche en \textsc{SHS}. En d'autres termes, il s'agit de briser les silos applicatifs pour concrétiser une \textit{harmonisation virtuelle}, une unification, créant ainsi un écosystème numérique collaboratif, où les données peuvent circuler librement entre les lieux de stockage. 

Au cœur des enjeux de l'open-data, la valorisation de la donnée brute reste en outre un défi à relever. Pour tirer pleinement parti de cette richesse disponible, il est nécessaire de mettre en place des infrastructures adaptées pour les mettre à disposition afin de favoriser la collaboration entre les différents acteurs (producteurs comme réutilisateur.rices). D'où la nécessité d'enrichir la donnée brute, pour la ``faire parler''. 

\begin{kwote}
``Thus, Big Data means to acquire, store, and analyze large amount of data that are generated quickly and are not always structured {[}\ldots{]}.''\footcite[p.26]{klinke_big_2016}
\end{kwote}

Après l'étape de numérisation, le document reste à l'état d'images matricielles directement issues du scan ou de la photographie des pages. Son contenu sémantique, c'est-à-dire le texte et les illustrations, demeure illisible aux machines, et donc caché au lecteur jusqu'à ce qu'il ouvre et lise la copie numérique. L'enjeu est alors de transformer ces images matricielles en données structurées, d'en obtenir de nouvelles représentations sémantiquement riches et manipulables, afin de construire des portails de bases de données permettant une interrogation fine du contenu des documents. 

La numérisation initie donc un processus de transformation de la donnée brute en informations structurées grâce à sa segmentation. Cette étape, en conférant un niveau d'abstraction supérieur aux données, les rend aptes à supporter de nouveaux traitements algorithmiques et des analyses scientifiques. Face à l'explosion des volumes de données, l'automatisation de ces analyses devient indispensable pour appréhender de nouveaux ordres de grandeur et extraire des connaissances.

\begin{kwote}
``[The] interpretation of results is still exclusive to humans, but computers can help us with the steps leading to that destination.''\footcite[p.26]{klinke_big_2016}
\end{kwote}

Des outils numériques vont permettre de décoder et d'interpréter les informations visuelles, faisant ainsi ``parler'' les données brutes~:

\subsection{De nouveaux outils}

Les Humanités Numériques, initialement concentrées sur l'étude du texte, ont bénéficié de l'essor des technologies d'extraction du texte. La Reconnaissance Optique de Caractères (\ocr), puis la Reconnaissance d'Écriture Manuscrite (\htr), ont suscité un intérêt croissant depuis les années 1950, mais c'est à partir des années 1990 que leur développement s'est véritablement accéléré. L'objectif premier est d'automatiser l'extraction de textes à partir d'images numériques de sources historiques. Confrontées à la diversité des polices, des mises en page et des orthographes rencontrées dans les documents, les méthodes se sont récemment tournées vers l'apprentissage profond (\dl). 

Cependant les archives et les sources contiennent aussi un grand nombre d'images. En comparaison de l'intérêt pour l'\htr et l'\ocr, le développement des réseaux de neurones et du \ml pour le traitement automatique du matériau visuel accompagnant le texte est relativement récent. Pourtant, les deux champs partagent une base commune~: le traitement du texte numérisé comme de ses illustrations, en apprentissage machine, revient à manipuler une image matricielle, soit une grille de pixels, et extraire puis encoder des informations sémantiques. 

\subsection{Perspectives ouvertes pour l'image}

Cette convergence technologique offre aujourd'hui la possibilité d'identifier et classifier des motifs récurrents au sein de vastes collections d'images, permettant une exploration plus systématique et à plus grande échelle des éléments visuels présents dans les sources. Les réseaux de neurones profonds ouvrent de nouvelles voies d'interrogation des archives numériques au prisme de leurs illustrations. 

\begin{kwote}
``They [les réseaux de neurone] open up a part of the digital archive for large-scale analysis, which, until now, has been left uncovered~: the millions of images in digitized books, newspapers, periodicals, and historical documents. As a result, they allow us to explore the visual side of the digital turn in historical research. Using these techniques, we can explore visual material in archives using nontextual search methods. Scholars can, for example, find visual material related to a particular topic, or, they can indentify transitions in the use of a particular medium, such as illustrations and photographs.''\footcite[p.2]{wevers_visual_2020}
\end{kwote}

Les représentations sémantiques extraites par les modèles de vision artificielle fournissent un niveau d'abstraction supérieur à l'image, visent à rendre son contenu compréhensible par la machine, offrant ainsi de nouvelles perspectives pour l'analyse et l'exploitation des données de la recherche.

\subsection{Structurer l'information}

L'extraction et l'encodage de texte ouvrent de nouvelles perspectives pour mettre en œuvre des chaînes d'édition partiellement automatisées. Or les formats de l'édition numériques permettent l'introduction d'une approche sémantique de l'exploitation des textes. Au-delà de la simple mise en forme, l'édition numérique implique désormais un balisage sémantique, basé sur des langages comme \xml et \html, et sur des standards comme \tei, rendant les contenus lisibles aussi bien par les humains que par les machines. Ainsi, l'édition ne se limite plus à la création d'un objet matériel, mais vise à structurer les contenus pour favoriser leur exploration et l'ouverture à de nouveaux traitements algorithmiques.

Comment étendre cette structuration aux images, de manière à les rendre exploitables par des traitements algorithmiques~? Les algorithmes de vectorisation automatique permettent de transformer des images géométriques simples, comme les diagrammes astronomiques qui constituent le corpus d'\eida, en données structurées. Cette représentation vectorielle offre un équivalent visuel du balisage sémantique appliqué aux textes, permettant ainsi de capturer le sens mathématique inhérent à ces images.

\subsection{Les données pour des ponts interdisciplinaires}

La combinaison de deux facteurs, l'augmentation exponentielle de la puissance de calcul et de la disponibilité des données visuelles, a permis à la \textit{Computer Vision} d'évoluer rapidement, permettant de créer des réseaux de neurones plus profonds, plus précis, et améliorant la vitesse et l'exactitude de tâches de plus en plus complexes\footcite{klinke_big_2016}. Mais cet avancement des techniques est aussi catalysée par leur application concrète et en contexte, car leur développement repose sur la disponibilité de larges \textit{datasets} annotés pour l'entraînement des modèles et l'évaluation des résultats. Ainsi si l'\ia profite au domaine de la recherche historique sur du matériel visuel, la réciproque est aussi vraie~: 

\begin{kwote}
``humanities could likewise be a boon to the development of more accurate and more sophisticated computer vision techniques. As classification algorithms have been trained on manually tagged sets, structural biases in their classification schemes will be reproduced in the results produced by computer vision techniques. In collaboration with humanities scholars, computer scientists could critically engage with these biases and rethink the way we annotate data sets and measure algorithmic accuracy.''\footcite[p.12]{wevers_visual_2020}
\end{kwote}

La mise à disposition de jeux de données sélectionnés et annotés par des experts améliore les performances et l'exactitude des modèles, faisant rapidement avancer la recherche en \cv. Ainsi le \dl, gourmand en données, profite de l'existence de ces corpus divers et complexes annotés par les chercheurs en \shs pour l'entraînement des modèles, et les historien.nes profitent du changement d'échelle de l'analyse permise par la \cv. La recherche et la technologie rentrent en dialogue, l'une répondant aux besoins de l'autre, créant des collaborations interdisciplinaires bénéfiques. D'où le besoin d'autant plus prégnant dans le domaine du \ml de penser les outils dans le sens de l'ouverture des systèmes afin de garantir l'interopérabilité de ces données annotées.

\subsection{\emph{Pipelines} et \emph{workflows}~: spécificités de l'implémentation de l'IA}

Les problématiques liées à l'intelligence artificielle ne se limitent pas à la construction de modèles performants. Son utilisation soulève également des questions en matière d'architecture matérielle et applicative. Déjà, les modèles d'\ia, particulièrement ceux basés sur l'apprentissage profond, nécessitent la puissance de calcul et les infrastructures \textit{hard-ware} adaptées. L'intégration de ces modèles dans des systèmes applicatifs exige également des algorithmes optimisés pour la parallélisation, ainsi que des solutions pour la gestion des données à grande échelle et pour la réduction de la latence. 

L'intégration de l'\ia dans un processus de traitement de données implique en outre la création de \textit{workflows} itératifs. Ces \textit{workflows} nécessitent une récupération des données en sortie, suivie d'une intervention humaine pour leur correction et leur réintégration dans le processus d'entraînement des modèles. Ces pipelines impliquent une boucle de rétroaction où les données produites par les modèles sont régulièrement évaluées, corrigées par des experts, puis réinjectées dans le processus d'apprentissage.

Alors comment préserver le rôle des chercheur.ses dans ce processus afin de garantir la qualité des résultats~? Comment concevoir des infrastructures applicatives ou logicielles capables de gérer ces flux de données~? Et est-il possible de créer un outil qui permette d'appliquer une méthode unifiée à des ensembles de données hétérogènes, un outil suffisamment flexible et adapté à divers traitements basés sur le \ml~? La réponse à ces questions nécessite une approche transversale, combinant des compétences en informatique, en développement applicatif, et en gestion de la donnée. 

Les enjeux et interrogations qui entourent la mise en place de ce \textit{workflow} m'ont occupé pendant mon stage, et constituent le cœur de ce qui est exposé dans ce mémoire. 

\section{Mission de stage}

La plateforme \aikon -- anciennement \eida -- est déjà dotée d’une chaîne de traitement partiellement automatisée qui intègre des fonctionnalités d’extraction et de recherche de similarités. Ces traitements permettent de repérer et de comparer des éléments visuels, en l'occurrence des diagrammes, au sein d’une base de données constituée d’images numérisées provenant de sources variées~: institutions de conservation ou collections personnelles.

La prochaine étape dans le développement de la plateforme consiste en l'implémentation de la fonctionnalité de vectorisation des diagrammes extraits, visant à affiner encore le niveau de structuration de l’information. Ce processus transforme une image en un ensemble de formes géométriques élémentaires, appelées primitives. Cette représentation, particulièrement adaptée au traitement informatique, permet de traduire les informations visuelles en structures mathématiques, facilitant ainsi leur manipulation, leur analyse et leur exploitation par des méthodes computationnelles.

Mon stage au sein de l'équipe d'histoire des sciences du laboratoire \syrte de l'Observatoire de Paris a consisté à développer un module de vectorisation automatique dans la plateforme \aikon, tout en participant à une réflexion plus large sur l'architecture de cette plateforme. En favorisant une approche modulaire et flexible, son évolution permettra d'intégrer plus facilement de nouvelles fonctionnalités et d'ouvrir la plateforme à d'autres domaines de recherche.

\section{Problématisation}

Comment concilier la nécessité de développer des outils génériques, réutilisables et performants avec la grande variété des données et des problématiques spécifiques à chaque étude~?

La tension s'exprime dans le défi de la personnalisation~: entre généralisation et spécificité. Nous explorerons comment cette dialectique se manifeste à chaque étape d'une chaîne de traitement. Au niveau de la description des données en premier lieu~: comment concilier la richesse des données réelles avec la nécessité de les structurer pour une exploitation efficace~? Par ailleurs, la construction de modèles de \cv s'inscrit au cœur de cette tension~: ils doivent être suffisamment généraux pour pouvoir s'adapter à la complexité du réel, tout en restant assez complexes et spécialisés sur des cas particuliers. Enfin, comment penser l'implémentation de ces modèles dans une plateforme évolutive et modulable~? 

Une question épistémologique sous-tendra tout au long de ce mémoire les question techniques~: comment l'outillage technique collaboratif contribue à façonner les méthodologies de la recherche~? 

\section{Annonce du plan}

Le développement se déclinera en trois partie. Premièrement, nous présenterons le projet \eida et ses objectifs, en le situant dans le paysage de la recherche en étude visuelles, et nous explorerons les enjeux qui découlent de cette inscription dans un contexte plus large. Nous aborderons ensuite les outils de \cv utilisés, les défis liés à leur mise en œuvre, et les perspectives ouvertes en terme d'édition numérique. Enfin, nous présenterons les contours d'une plateforme qui permettrait de rendre à la fois les méthodes et les résultats de l'analyse accessibles à la communauté scientifique, notamment à d'autres projets de recherche en études visuelles~; cette partie sera consacrée à la mise en œuvre technique d'une approche modulaire dans le développement d'une application web. 
    

    \thispagestyle{empty}
    \clearemptydoublepage

\mainmatter

    \part{Chaîne de traitement de la donnée visuelle~: enjeux technologiques et disciplinaires}

\chapter*{Introduction partielle}

\begin{kwote}
``Indeed, not only have the majority of historians of cosmology dismissed
the early Middle Ages due to the fact that there is no ``scientific
progress'' to be observed, but they also tended to disregard visual
representations, limiting their inquiries to the doctrinal aspects of
textual sources.''\footcite[p.16]{obrist_visual_2012}
\end{kwote}

Les illustrations et leur évolution dans les sources d'ordre
scientifique du \ma jusqu'aux cultures occidentales modernes n'ont
été que peu étudiées. De manière plus générale, le rôle de l'image dans
la construction et la diffusion du savoir scientifique soulève des
questions complexes qui restent historiquement délicates à appréhender
et pour lesquelles des outils d'analyse adaptés font défaut. Afin de
combler cette lacune, l'objectif des projets \eida/\vhs est de développer
des méthodes automatisées pour l'extraction et le traitement des
illustrations dans les sources historiques, avec pour but ultime
d'appuyer l'analyse experte des chercheur.ses en histoire des sciences.

L'enrichissement et l'exploration de vastes corpus iconographiques n'est
pas l'apanage de la seule histoire des sciences. Le développement de
projets en Humanités Numériques dans des disciplines annexes constituent
des antécédents ou des points de comparaison riches en enseignements. Si
ce mémoire porte plus spécifiquement sur les développements réalisés
dans le cadre de \eida, les différents projets s'inscrivent dans des
dynamiques de continuité et de partage, créant un écosystème ouvert et
fructueux, permettant d'appréhender au mieux les enjeux et les défis
liés au traitement des résultats des grandes campagnes de numérisation
des institutions et bibliothèques.

Cette première partie explore la complexité des relations qui se forment
à différents niveaux dans le contexte de la fabrication d'outils de
traitement automatique de l'image. Ces outils doivent prendre en compte
les dynamiques spécifiques aux acteurs et au cadre du projet \eida. Par
ailleurs, ils s'insèrent dans un réseau plus vaste incluant des projets
partenaires sur lesquels s'appuyer ou des briques fonctionnelles à
intégrer. La problématique de ce chapitre réside donc dans la
compréhension des liens se tissant à divers niveaux, impactant les
développements, leur ouvrant des pistes comme les contraignant.

Cette première partie présente ce réseau à deux niveaux~: premièrement
les acteurs et le cadre du projet, et deuxièmement un réseau plus vaste,
celui de la recherche en générale, la communauté se regroupant autour de
grands principes d'ouverture garantissant une interopérabilité technique
des données. Enfin, un état de l'art sur l'intelligence artificielle et d'enrichissement de données se concentrera autour du projet \gaga. Les réflexion méthodologiques menées dans le cadre de ce projet soulignent l'importance de partager des pratiques dans le domaine du \ml. 

\begin{kwote}
``Aller à la rencontre d'autres projets, créer des synergies avec d'autres
équipes ou institutions peut être une réponse à cette difficulté de
gestion de la masse. Un effort de standardisation sommé à une réflexion
sur l'interopérabilité pourrait faire dialoguer les corpus et mutualiser
les outils d'analyse, au-delà des clivages entre disciplines et sujets
de recherches."\footcite{jacquot_decrire_2017}
\end{kwote}

\clearemptydoublepage

        \hypertarget{chapitre-1-eida-contexte-institutionnel-et-scientifique}{%
        \chapter{EIDA~: Contexte institutionnel et
scientifique}\label{chapitre-1-eida-contexte-institutionnel-et-scientifique}}

            Le projet \eida (Editing and analysing hIstorical astronomical Diagrams
with Artificial intelligence) a pour ambition de rassembler des sources
provenant de différentes traditions astronomiques et de produire des
outils d'analyse et d'exploitation de la grande diversité de sources
astronomiques issues d'origines géographiques et temporelles variées,
les mettant à disposition de la communauté de la recherche.

Dans une perspective interdisciplinaire, \eida vise un double objectif
scientifique et technique. Exploitant les progrès récents des approches
analytiques basées sur la vision artificielle le projet s'appuie sur le développement d'outils de
\dl capables d'automatiser l'analyse des diagrammes de
l'extraction à la décomposition en composants significatifs, permettant d'envisager leur édition. Ces outils dédiés aux diagrammes
servent leur étude documentaire et épistémique, diachronique ou
synchronique, en s'appuyant sur des corpus à grande échelle.

Les outils développés dans ce cadre s'inscrivent dans un écosystème
collaboratif qui demande un effort de normalisation des outils
développés. On expliquera comment les dispositifs d'automatisation sont
contraints par la pluralité des usages et des acteurs en présence.

            \hypertarget{contexte-disciplinaire}{%
            \section{Contexte disciplinaire}\label{contexte-disciplinaire}}
                Le module de base est un package pour la gestion documentaire, duquel
l'application ne peut se détacher. Celui-ci inclut tout d'abord des
formulaires pour l'intégration des documents dans la base de données. Le
modèle de données permet de décrire différentes entités qui, bien que
liées dans leurs métadonnées, peuvent être intégrées indépendamment. Le
module de base permet également la création de \mans \iiif pour
chaque numérisation, permettant ensuite la visualisation des documents
grâce aux outils open-source dédiés. De ce fait, l'indexation de zones
d'image peut être réalisée manuellement via l'interface Mirador intégrée
à \sas. Ce noyau fonctionnel inclut en outre la sélection de lots de
documents (le ``panier''), sur lesquels pourront être effectués des
traitements groupés paramétrables.

Les briques fondamentales offrent donc les fonctionnalités essentielles
de gestion documentaire (intégration, modèle de données, \iiif). Les
traitements, quant à eux, sont gérés par des modules séparés, et c'est
sur cette structure que repose la modularité et l'évolutivité de
l'application.

Ci-après nous donnons une description détaillée de certaines de ces
fonctionnalités de base.

\hypertarget{description-des-donnees}{%
\subsection{Description des
données}\label{description-des-donnees}}

Le module de base contient un modèle de données suffisamment extensif
pour décrire efficacement une diversité de données, allant de documents
textuels historiques à des tableaux en histoire de l'art. La
tripartition entre témoin (\wit), série (qui contient un ensemble de
témoins), et contenu permet un alignement avec des corpus très
diversifiés et des données potentiellement hétéroclites, telles que des
manuscrits, des documents épistolaires, des inventaires de galeries
d'art, et même pourquoi pas des cartes\ldots{}

Pour ouvrir à cette large diversité de données, la liste des types de
pagination témoin doit être étendue \emph{a minima} d'un nouveau type
``other'', émancipant l'enregistrement des mentions de pagination. Les
développements futurs prévoient aussi la création d'un système pour
ajouter facilement un nouveau type\footnote{Le type de témoin est une
  métadonnée rentrée par l'utilisateur.rice lors de l'enregistrement du
  \wit dans la base de donnée. Il choisit le type dans une liste,
  originellement manuscrit, imprimé ou gravure sur bois.} de \wit
(tel que peinture, catalogue, etc.).

Au fil des développements, des débats ont émergé autour de l'ajout dans
le modèle de données d'un niveau de granularité supplémentaire pour
décrire des images ou zones d'images unitaires
(\graphicals), créant ainsi une entité détachée du fait
qu'elle provienne d'une extraction dans un document. Cette solution
aurait permis une description plus détaillée et plus fine des images,
importante pour des projets axés sur des images uniques, et aurait
favorisé un élargissement du spectre des type de sources pris en charge.
L'utilisateur.rice aurait pu soit importer une image unique (et de manière
optionnelle, la lier à un \wit) via un formulaire, soit sélectionner
une région d'image d'intérêt au sein des extractions (annotations \sas),
laquelle serait enregistrée comme \graphical, puis l'enrichir de
métadonnées. Dans les deux cas l'enregistrement d'un \graphical
aurait donné lieu à la création d'une \digit au format \jpeg.

Sans l'unité de description \graphical, les régions d'images
sont créées uniquement via les annotations \sas.

L'intégration de cette entité au sein du modèle aurait offert plusieurs
avantages en termes de cohérence et de flexibilité. En s'alignant sur
les structures existantes (\wits et \sers), elle aurait permis une
manipulation plus intuitive des images, facilitant ainsi les opérations
de recherche et la création de \emph{Sets} personnalisés. De plus, elle
aurait rationalisé la gestion des annotations \sas, permettant de
sélectionner les plus pertinentes dans la multitude existante.

Cependant, cette approche présente des limites, et on peut trouver des
alternatives. Tout d'abord, la coexistence de \graphicals avec les
annotations \sas, générées par des processus distincts, aurait pu créer
une certaine confusion quant à leur nature et à leur méthode de
création. De plus, la multiplication potentielle de milliers
d'enregistrements aurait pu impacter les performances de la base de
données et complexifier les requêtes. Enfin, le lien sémantique ambigu
et sujet à interprétation subjective entre \graphical et \wit
aurait compliqué les possibilités de corrélation.

Compte tenu de ces limites, il a semblé préférable de maintenir les
annotations \sas pour identifier les instances de base du modèle, sans
créer de nouvelle unité de description. La solution actuelle reste donc
basée sur la création manuelle ou automatique de zones dans les images
via \iiif et \sas, évitant les problèmes de redondance et de confusion.
Bien que l'entité \graphical n'ait pas été implémentée, les
fonctionnalités d'annotation et de sélection d'images sont assurées par
d'autres mécanismes. L'outil Mirador permet d'associer des tags aux
zones d'image, offrant ainsi une première couche d'enrichissement
sémantique. La sélection dans un \emph{set} personnalisé sera possible en
gardant en mémoire une référence contenant des coordonnées du
\emph{crop}. De plus, l'importation d'images individuelles est
réalisable en les considérant comme des \emph{Witness partiels}, ce qui
permet de les intégrer dans le \textit{workflow} existant. Toutefois
l'enrichissement sémantique à un niveau de granularité fin restera
limité~; et la dépendance à l'outil \sas constitue une potentielle dette
technique, susceptible de restreindre les évolutions futures du système.

Afin de mieux répondre aux exigences de modularité, l'évolution du
modèle de données s'oriente non pas vers une description individuelle
des documents, mais vers la gestion des traitements. Cette évolution
implique la création d'une entité \tr
liée à des ensembles de données (\ds et
\rs) potentiellement hétérogènes.

\hypertarget{principe-du-traitement}{%
\subsection{Principe du Traitement}\label{principe-du-traitement}}

Le but fondamental de la plateforme est de pouvoir effectuer plusieurs
actions sur les objets de la base. Afin d'assurer une meilleure
traçabilité et plus de flexibilité, la plateforme abandonne les
lancements automatiques des processus\footnote{C'était initialement le
  cas de l'extraction des entités, dont le lancement était lié à une
  méthode de classe liée à la \digit après soumission d'un
  formulaire d'ajout d'un \wit ou d'une \ser. L'action se
  lançait immédiatement après enregistrement des images d'une
  numérisation dans la plateforme.} au profit d'un système basé sur
l'entité \tr. Chaque traitement est associé à un ensemble
d'objets traités ensemble (\ds ou \rs), à un jeu de
paramètres et à un résultat. Ces informations sont stockées dans une
table dédiée. Cette approche facilite la gestion et le trackage des
processus (notamment, les utilisateur.rices sont notifiés par e-mail à la fin
du \textit{processing}), permet aux utilisateur.rices de consulter un historique de
leurs actions et offre la possibilité de créer des \textit{workflows}
personnalisés en passant par un formulaire de lancement unique mais
extensif.

En permettant de regrouper des documents de types différents (\wos,
\sers, \wits) dans des \dss, on offre à
l'utilisateur.rice la flexibilité de lancer des actions sur des ensembles
d'entités hétérogènes et granulaires. Le traitement est ensuite réparti
sur les entités de niveau inférieur (les témoins). Les \wits ainsi
sélectionnés peuvent être soumis à une large gamme de traitements~: des
fonctions déjà implémentées comme l'exportation (avec choix du
format), l'extraction, la vectorisation, la recherche de similarité~; ou
de nouveaux traitements personnalisés, tels que la visualisation sur une
frise chronologique ou une carte. La modularité de la plateforme est
assurée par un formulaire de lancement configurable, permettant de
l'adapter à différents scénarios d'utilisation, et à l'ajout de modules
personnalisés.

Le \rs fonctionne similairement au \ds, à un niveau
de granularité inférieur (à l'échelle de la zone d'image)\footnote{À
  l'été 2024, l'entité n'existe pas encore dans la base de données, mais
  le processus d'envoi du traitement et les modes de communication entre
  l'application et l'\api prévoient la possibilité de lancer l'inférence
  des modèles sur un ensemble de régions extraites.}.

\hypertarget{extraction-des-zones-dimage-manuelle}{%
\subsection{Extraction manuelle des zones d'image}\label{extraction-des-zones-dimage-manuelle}}

Le choix de la méthode d'extraction des régions d'intérêt dans les
documents constitue un élément clé de la modularité de la plateforme.
Les utilisateur.rices peuvent opter pour une extraction manuelle ou une
extraction automatique basée sur des algorithmes de vision par
ordinateur, adaptée aux traitements à plus grande échelle.

Après importation d'un enregistrement, le flux de travail procède à la
création de \mans \iiif pour chaque numérisation
(\digit) afin de permettre une visualisation grâce à la
plateforme Mirador. Le module de base autorise par la suite
l'extraction manuelle de zones d'intérêt au sein des images. Cette
fonctionnalité est particulièrement utile pour les projets ne souhaitant
pas recourir à des méthodes entièrement automatisées de vision par
ordinateur. L'outil \sas permet de créer des annotations, c'est-à-dire de
définir des régions d'intérêt spécifiques dans les numérisations, et de
les indexer directement dans les \mans \iiif correspondants,
enrichissant ainsi les ressources numériques. De plus, les
développements futurs prévoient la possibilité d'importer des fichiers
d'annotation préexistants en format .\textsc{txt} afin de pouvoir les indexer
manuellement. Par conséquent, le \textit{workflow} de base ne comporte aucun
traitement automatique basé sur la vision (et de fait éventuellement
trop gourmand en puissance de calcul).

L'extraction, qu'elle soit manuelle ou automatique, constitue le
fondement du reste des processus. Une interface est disponible pour
sélectionner un ensemble de documents et effectuer des actions
spécifiques sur les témoins annotés, via le formulaire de traitement qui
s'étend selon un choix de module configuré. Ainsi l'utilisateur.rice n'est
pas limité par un contexte initial, à l'origine deux étapes
indissociables et incontournables (importation et extraction), pour
pouvoir effectuer d'autres actions. Cette modularité permet de
s'affranchir d'un \textit{workflow} linéaire et prédéfini, offrant ainsi une plus
grande adaptabilité aux besoins spécifiques et aux ressources
matérielles des projets.

Pour conclure, l'existence de ce module de base répond à des besoins
élémentaires des projets de recherche en études visuelles. Il fournit un
outil qui permet d'agréger toutes les sources primaires qui concernent
le sujet, de décrire les sources et de les mettre en relation. Il offre
en outre la possibilité d'extraire et visualiser des contenus d'intérêt
(les ``crops'' d'images), ciblant ainsi les instances de base qui
intéressent les chercheur.ses.
            
            \hypertarget{genese-ecosysteme}{%
            \section{Genèse et écosystème du projet}\label{genese-ecosysteme}}
                Spécialiser un modèle d'intelligence artificielle implique de lui
fournir des données pertinentes, diversifiées, et en quantité
suffisante. Cependant, pour certains domaines, dont l'histoire fait
partie, le volume de données disponible est insuffisant. Ce constat est
d'autant plus vrai dans le cas des diagrammes issus de traités
astronomiques~: les corpus de documents scientifiques historiques
contiennent généralement du texte en majeure partie, des tables et des
images, négligeant souvent les diagrammes.\footnote{Exception faite du
  corpus S-VED (\cite{buttner_cordeep_2022}), collection
  d'illustration très diverses contenant entre autre des diagrammes
  historiques. Mais les primitives ne sont pas annotées.}. De plus, ils
sont dénués d'annotations précises sur les éléments constitutifs des
pages~; c'est sans parler de l'inexistence d'un corpus de diagrammes
dont les primitives sont annotées. Or l'annotation est une tâche
chronophage et fastidieuse. Le recours aux données synthétique répond,
mais en partie seulement, à ces problématiques.

\hypertarget{datasets-synthetiques}{%
\subsection{\emph{datasets} synthétiques}\label{datasets-synthetiques}}

Les \textit{datasets} synthétiques sont générés par des algorithmes ou des
méthodes de simulation pour imiter des données réelles, sans être
directement extraites de sources existantes. De tels jeux de données
sont utilisés lorsque les données réelles sont limitées ou difficiles à
obtenir, mais qu'il est cependant nécessaire de contrôler spécifiquement
les caractéristiques des données d'entraînement\footcite{buttner_cordeep_2022}. La génération
d'images a pour but de fabriquer des ensembles de données plus vastes,
plus diversifiés, très variables et assez complexes, répondant aux
caractéristiques des objets d'intérêt du projet, et surtout étiquetés
automatiquement, sans recourir à l'annotation manuelle.

Ces données synthétiques sont assez ressemblantes et complexes pour être
exploitées. Par exemple, docExtractor est un modèle off-the-shell (au
même titre que \yolo) envisagé dans le cadre de la tâche d'extraction des
diagrammes, et qui se veut sépcifique aux données historiques, car il
est entraîné sur des données produites par un générateur de documents
historiques synthétiques~: SynDoc\footcite{monnier_docextractor_2020}. SynDoc
génère des images de manière aléatoire en combinant des éléments
graphiques (fonds, images, texte et bruit) provenant d'un jeu d'image
défini (constitué de 177 images de pages, 15 contextes, plus de 8000
œuvres d'art provenant de WikiArt, des lettrines générées à partir d'une
lettre aléatoire avec 91 fonts possibles, et des dessins, schémas et
textes tirés d'articles aléatoires sur Wikipedia, avec plus de 400
fonts). Les différents éléments s'agencent, intégrant sur le fond
images, texte et bruit, offrant des combinasons et des mises en pages
assez complexes. Chaque élément de contenu est pré-annoté, éliminant
ainsi le besoin d'annotations manuelles pour ces pages.

          \begin{figure}[H]
          \begin{center}
          \includegraphics[height=6.5cm]{figues/syndoc.jpg}
          \end{center}
          \caption{Données synthétiques générées par SynDoc.\footcite[p.46]{norindr_traitement_2023}}
          \label{fig:syndoc} \end{figure}

Pour entraîner le modèle de vectorisation, il a de même été nécessaire
d'utiliser des données synthétiques. Parce qu'annoter les primitives
géométriques dans des images de diagrammes complexes est très
chronophage, le modèle de vectorisation a été pré-formé sur des corpus
artificiels générés dynamiquement. Le script de génération des données
d'entraînement choisit aléatoirement un arrière-plan, y ajoute des mots,
des nombres et des glyphes puis crée artificiellement un diagramme en
insérant des segments, des cercles et des arcs. Le script est conçu pour
que ces diagrammes aient une forte probabilité de présenter des formes
très caractéristiques comme les cercles concentriques et tangents, les
lignes parallèles et les arcs connectés, afin de simuler les structures
typiques. Les primitives sont dessinées avec des
variations aléatoires d'opacité, de largeur et de couleur. Les cercles
peuvent être remplis ou vides. Enfin, du bruit est ajouté en appliquant
un flou gaussien, et en supprimant de petites régions du diagramme pour
imiter la dégradation des documents historiques. Les données
d'entraînement ainsi générées présentent des configurations assez
complexes.

          \begin{figure}[H]
          \begin{center}
          \includegraphics[height=7cm]{figues/vecto_synthetic_data.png}
          \end{center}
          \caption{Données synthétiques générées pour l'entraînement du modèle de vectorisation.\footcite[Figure issue de la présentation de Syrine Kalelli à l'occasion de la conférence \eida 2024~:][]{noauthor_eida_nodate-1}}
          \label{fig:vecto_synthetic} \end{figure}

Enfin, le modèle de similarité présente un troisième exemple, puisque
SegSwap est pré-entraîné sur de la donnée synthétique. Le script de
génération prend des parties aléatoires d'une images et les copie-colle
au-dessus d'une autre image. Les trois images (source, cible et
superposition) sont placées dans le même dataset d'entraînement, ainsi
le modèle apprend à retrouver ce qui, dans la superposition, vient de la
source, et ce qui vient de la cible.

          \begin{figure}[H]
          \begin{center}
          \includegraphics[height=3cm]{figues/segswap_blended_images.png}
          \end{center}
          \caption{Données d'entraînement du modèle Segswap.}
          \label{fig:segswap} \end{figure}

\hypertarget{les-donnees-reelles}{%
\subsection{Les données réelles}\label{les-donnees-reelles}}

S'appuyer sur les modèles \textit{off-the-shelf}, sur de larges \textit{datasets}
généralistes, ou sur des données synthétiques permet une implémentation
facilitée de la vision dans des projets et constitue une base solide.
Toutefois, les sources tenant aux deux projets (\vhs et \eida) sont trop
spécifiques pour se contenter de modèles généralistes ou formés sur des
données artificielles. Même si ces derniers peuvent offrir des performances
de base, ils risquent de manquer de précision et de sensibilité aux
particularités des documents historiques. Les corpus artificiels
présentent des configurations délibérément complexes pour s'approcher le
plus possible des difficultés que le modèle pourrrait rencontrer sur les
données réelle. Elles sont cependant irréalistes et insuffisantes pour
permettre aux modèles de généraliser sur des diagrammes réels.

En atteste la comparaison des performances de docExtractor et \yolov sur
les données d'\eida. docExtractor\footcite{monnier_docextractor_2020}, entraîné
sur des données synthétiques mimant les documents historiques serait en
théorie plus adapté au traitement d'images de pages de manuscrits, avec
du texte et des illustration côté à côte, d'autant qu'il intègre des
outils de traitement du texte (notamment pour la segmentation des
lignes)\footnote{\eida envisage l'implémentation d'un outil d'extraction
  et transcription des labels et des textes qui entourent les diagrammes}.
Pourtant, sans fine-tuning sur des données réelles, il présente des
performances équivalentes à celles de \yolov\footcite[p.45]{norindr_traitement_2023}. Cela
souligne que même les modèles off-the-shelf entraînés sur un corpus
assez spécifique et complexe, mais synthétique, ne dispense pas d'un
entraînement sur des données réelles, au même titre que les modèles très
généralistes comme \yolov.

Alors, le modèle de base \yolov tel que mis à disposition par
Ultralytics est entraîné sur de grands ensembles de données réelles, ce
qui constitue une base solide pour la classification des objets du
monde. L'utilisation de SynDoc permet ensuite de compléter
l'apprentissage initial en exposant le modèle à des exemples variés et
spécifiques aux documents historiques, augmentant ainsi sa capacité de
généralisation. Ces similis de manuscrits anciens offrent l'avantage de
pouvoir être produits en grandes quantités et de couvrir un large
éventail de scénarii et de configurations difficiles à obtenir dans des
ensembles de données réelles. Puis le modèle est entraîné sur les
données de \vhs, qui sont de réelles pages de documents historiques
contenant une large diversité d'illustrations. Ces données apporteront
une dimension supplémentaire de pertinence au modèle, en l'exposant à
des particularités des documents historiques réalistes. Enfin, \yolov
est entraîné sur les données d'\eida, qui sont orientées spécifiquement
vers les diagrammes, afin qu'il détecte uniquement ces derniers.

Quant au modèle de vectorisation développé par Syrine
Kalleli\footcite{kalleli_historical_2024}, il est formé
sur des données synthétiques générées à la volée par un script. Mais le
corpus de diagrammes d'\eida est particulièrement caractéristique et le
modèle n'aurait pu être optimal sans avoir appris sur des images de
diagrammes issus de manuscrits réels. Un corpus d'entraînement de 303
diagrammes extraits de manuscrits et de gravures a donc été constitué et
annoté par les historien.nes. Ces diagrammes sont issus de sources latines, arabes,
grecques, hébreuses ou chinoises, datant du \textsc{xii}\ieme au \textsc{xviii}\ieme siècle, et ils
présentent en guise d'étiquettes plus de 3000 lignes, cercles et arcs. Le
ré-entraînement a permis le transfert des connaissances acquises sur la
tâche de détection des primitives sur les données réalistes.

Il sera également possible d'obtenir des meilleurs résultats sur la
similarité grâce à une évaluation des scores (qui constitue un jeu de
données annotées) et le ré-entraînement du modèle, pour donner des
résultats plus adaptés à la spécificité des données historiques.

D'ailleurs, cette étape d'annotation (le choix des exemples et des
étiquettes) revêt des enjeux importants. L'apprentissage spécifique se
fait à partir de données sélectionnées par les chercheur.ses~: les exemples
sur lequel l'algorithme d'apprentissage va itérer définissent le modèle.
Il est nécessaire de constituer un échantillon de données aléatoire et
représentatif, et de l'annoter en fonction de ce que l'on souhaite
obtenir en prédiction.

L'annotation des jeux de données est non seulement une étape clé, mais
aussi un bel exemple de collaboration chercheur.ses-ingénieur.es. Elle
nécessite la définition de normes pertinentes et rigoureuses. Travail
minutieux et chronophage, l'étiquetage des données peut engendrer des
erreurs et du bruit dans les données, car elle implique la subjectivité
des chercheur.ses et le regard parfois trop précis sur les sources desquels
les annotateurs sont experts.

Voici un exemple rencontré lors de la préparation des données pour
entraîner un modèle de segmentation du contenu textuel. Les sources
arabes et chinoises sont particulièrement verbeuses et les diagrammes
sont très souvent entourés des blocs de commentaires se mélangeant alors
aux légendes et aux labels. Doit-on considérer ces commentaires comme
faisant partie des éléments que l'on souhaite identifier ou bien les ignorer
? Cette décision est importante car si on les ignore, le modèle risque
de passer à côté d'éléments textuels pertinents. En revanche, si on les
inclut, il ramènera des commentaires sans rapport direct avec le
diagramme observé. On voit ici comment la binarité des modèles, qui se
reflète dans les normes d'annotation, est problématique et constitue une
limite au \ml. Un compromis doit être trouvé entre
l'automatisation, qui requiert une normalisation, des définitions
claires et binaires, et la nuance dans l'interprétation des
sources\footnote{Dans le cadre du projet, il a toujors été plus
  intéressant d'opter pour une définition extensive des objets à
  détecter, car prévision d'une correction des traitement. Et il est
  plus facile de supprimer un élémént pas pertinent que d'aller en
  rechercher un, surtout compte tenu de la taille des corpus des
  chercheur.ses. Vaut aussi pour la préparation des données pour
  l'entraînement du modèle d'extraction.}.

La normalisation peut bénéficier à l'écosystème de recherche dans le
domaine de l'\htr et de l'\ocr. À ce titre, il est pertinent d'envisager
l'utilisation du vocabulaire contrôlé SegmOnto pour l'annotation du
contenu textuel entourant les diagrammes. Cela permettrait de créer des
jeux de données réutilisables, à partager avec des projets poursuivant
des objectifs similaires.\footnote{https://segmonto.github.io/}. Encore
une fois, un compromis doit être trouvé entre les besoins de description
des chercheur.ses et les possibilités offertes par les vocabulaires
contrôlés.

Un autre exemple concerne le dernier entraînement du modèle d'extraction
: les résultats montrent que des diagrammes sont encore détectés en
transparence. La question s'est alors posée de chercher à corriger ce
défaut en donnant au modèle, à l'occasion d'un nouvel entraînement,
d'avantage d'exemples négatifs (diagrammes visibles par transparence
mais non annotés). Or il est préférable de se contenter de la correction
ou suppression manuelle de ces prévisions erronées, garantissant que le
modèle parvienne à détecter les diagrammes presque effacés.

Pour assurer la rigueur et la cohérence des annotations, les décisions
prises entre les chercheur.ses et les ingénieur.es peuvent être l'objet d'une
documentation ou d'ateliers d'annotation.

\hypertarget{loeil-de-la-machine-avantages-et-limites}{%
\subsection{L'oeil de la machine~: avantages et
limites}\label{loeil-de-la-machine-avantages-et-limites}}

Bien qu'il soit possible d'optimiser les performances d'un modèle
d'apprentissage automatique en l'entraînant sur un ensemble de données
spécifique, son interprétation des données reste limitée car
fondamentalement binaire, ce qui le rend parfois déficient pour la
recherche en histoire. Ainsi, il gèrera difficilement les cas limites et
ambigüs. La décision d'inclure ou d'exclure ces cas particuliers de
l'ensemble d'entraînement implique un arbitrage délicat. D'un côté, une
inclusion trop restrictive peut compromettre les capacités de
généralisation du modèle, c'est-à-dire sa capacité à s'adapter à de
nouvelles données. À l'inverse, une inclusion trop permissive risque de
dégrader la précision du modèle sur les cas plus typiques. Les
chercheur.ses espérant obtenir un modèle maximaliste, quitte à accepter un
certain degré d'erreur et de devoir supprimer les faux
positifs, de nombreux cas limites ont été inclus. Le cas des diagrammes
visibles en transparence (expliqué précédemment) en est un exemple
éloquent.

Une autre difficulté réside dans la définition même du ``diagramme
astronomique''. Les limites de ce concept ne sont pas si claires et
définitives pour les chercheur.ses, et pourtant le modèle a besoin d'une
définition rigoureuse et cohérente. Il paraît en effet difficile de
considérer les diagrammes astronomiques en dehors du contexte des
pratiques d'autres sciences et disciplines connexes. Par exemple, Le
\emph{Flores Almagesti} -- réécriture de l'Almageste datant du \textsc{xv}\ieme par
l'astronome Giovanni Bianchini -- présente une partie algébrique à
l'ouverture mathématique, induisant la présence de nouveaux types de
diagrammes d'inspiration euclidienne. Pour retracer la source de ces
derniers, il est nécessaire de considérer les traités d'Euclide ou
autres travaux d'algèbre. Ceux-ci ne sont pas des traités
\emph{astronomiques}, bien qu'il ne soit pas certain que ces disinctions
contemporaines aient été aussi rigide à l'époque et aient eu un
quelconque sens pour les acteurs historiques. Les sources byzantines
confirment cette complexité~: les diagrammes y sont nommés
\emph{katagraphai}, indépendamment du domaine scientifique auquel ils
appartiennent. Également, de nombreux travaux astronomiques sont groupés
dans des témoins qui contiennent des œuvres issus de domaines divers.
C'est le cas avec les sources chinoises, comme le \emph{Chongzhen
lishu}, qui se présente généralement annexé d'une série de traités
mathématiques. Par conséquent, les diagrammes euclidiens ont été gardés
lors de la préparation des données, et l'algorithme de détection les
classe comme ``diagramme'', même s'ils ne constituent pas l'objet
principal des chercheur.ses.

En ce qui concerne les autres types de diagrammes non strictement
astronomiques (géométriques, harmoniques, logiques, illustrations de
constellations), une approche plus sélective a été adopté afin d'éviter
un modèle trop maximalistes. Ces éléments, bien que potentiellement
intéressants, n'ont pas été inclus dans la phase de détection
automatique.

Ainsi l'œil de la \cv contraint à des choix méthodologique
potentiellement inconfortables, mais en même temps il peut aider à
mesurer les impulsions des chercheur.ses, à mieux définir les objectifs de
recherche et à prioriser les éléments les plus pertinents. Ainsi, la
vision par ordinateur oblige les chercheur.ses à s'adapter à une logique
algorithmique qui, tout en limitant certaines interprétations
subjectives, offre l'opportunité de développer des modèles conceptuels et des méthodologies très rigoureuses.
            
            \hypertarget{objectifs}{%
            \section{Objectifs}\label{objectifs}}
                La construction d'une plateforme extensive et modulaire pour
démocratiser l'accès à un outil de gestion et de traitement de la donnée
visuelle implique une réflexion approfondie sur les architectures
matérielles et logicielles. Pour toucher des publics diversifiés de la communauté de la recherche, il est
essentiel de penser des infrastructures matérielles diverses, plus ou
moins puissantes et abordables, intégrant ou non des composants comme
les \gpu (qui permettent d'accélérer les calculs intensifs nécessaires à
l'\ia). En effet, la gestion efficace des ressources, la scalabilité,
et la performance sont des aspects à prendre en compte pour
que ces outils puissent être utilisés de manière fiable. Parallèlement, l'architecture logicielle doit être flexible et
évolutive. 

\hypertarget{hardware-une-api-sur-le-gpu}{%
\subsection{Hardware~: une API sur le GPU}\label{hardware-une-api-sur-le-gpu}}

La séparation physique de l'inférence des modèles tient un rôle
important dans l'ouverture de la plateforme.

Le type d'infrastructure de calcul, notamment le \cpu ou le \gpu,
implique des différences dans le traitement et l'analyse des données,
chacun offrant des capacités distinctes adaptées à des besoins
spécifiques. Un \cpu (Central Processing Unit) est le processeur
principal d'un ordinateur, conçu pour gérer une large gamme de tâches
générales et basiques, et utilisé pour les besoins quotidiens. Un \gpu
(Graphics Processing Unit) est spécialisé dans le
traitement des éléments graphiques. Il est conçu pour effectuer un grand
nombre de calculs simples en parallèle grâce à ses nombreux cœurs, ce
qui le rend extrêmement efficace pour des tâches nécessitant un
traitement massif et simultané de données. L'utilisation d'un \gpu est
souvent nécessaire pour les tâches d'\ia, notamment en vision
artificielle, car ces tâches impliquent souvent des opérations de calcul
intensives et parallélisables. Un \gpu, avec sa capacité à gérer des
milliers de \textit{threads} en parallèle, permet d'accélérer l'entraînement
et l'inférence des modèles de vision artificielle, rendant le traitement
plus rapide et plus efficace que sur \cpu.

Discover-Demo est une \api développée comme un module de l'application, répondant au besoin de séparer les algorithmes de vision du reste
de l'application. Cette séparation permet une plus grande flexibilité
dans l'utilisation des ressources de calcul. Elle tourne sur le \gpu
Dishas-ia, dédié quasi exclusivement aux besoins de l'équipe d'histoire
des sciences du \syrte.

          \begin{figure}[H]
	\begin{center}
		\includegraphics[height=7cm]{figues/com_hard_ware.png}
	\end{center}
	\caption{Organisation et communication des infrastructures.}
	\label{fig:com} \end{figure}

Malgré l'importance accordée à l'\ia,
l'interface web et l'\api associée sont conçues pour une analyse complète
des documents historiques, allant de leur importation et stockage à
leurs traitements (divers) et visualisations. Dans une optique
d'extensivité, elles ne doivent être rattachées à aucun processus d'analyse
prédéterminée. Ainsi, toutes les étapes peuvent être
effectuées manuellement ou à l'aide d'algorithmes automatisés.
L'application de base n'intègre pas de traitement de \cv, mais permet de gérer une base de données et des sources avec
leurs numérisations, utilisant le standard \iiif. Elle permet
l'indexation manuelle de zones d'images dans \sas via l'interface Mirador,
permettant la sélection de zones d'images d'intérêt. Pour cela,
l'extraction automatique de zones d'images est séparée en un nouveau
module, mais les fonctionnalités de base de l'application incluent
toujours les outils nécessaires pour effectuer des annotations manuelles
de régions. Cela comprend toutes les fonctions pour indexer un fichier
texte dans \sas, visualiser les régions annotées, et exporter les
résultats. Il devient alors envisageable d'importer des résultats de traitement
(fichiers d'annotation de régions ou de paires de régions similaires) et
de les indexer manuellement pour permettre leur visualisation et analyse
ultérieure.

Chaque traitement peut donc être
réalisé via l'inférence des modèles de vision sur \gpu (comme c'est le
cas pour \eida grâce à l'\api), par l'import d'un fichier de résultats,
manuellement, ou potentiellement par des méthodes locales sur \cpu
(\yolov, par exemple, est assez léger pour tourner en local). La
plateforme permet ainsi une adaptation à des environnements matériels
divers, laissant la possibilité de réaliser les traitements soit
automatiquement via l'\ia, soit manuellement.

Séparer les modèles de vision c'est aussi permettre une bascule vers des
modèles spécialisés. Les modèle développés dans le cadre du projet sont
disponibles mais peuvent facilement être réentraînés pour correspondre
spécifiquement aux données de l'utilisateur.rice, prenant en compte les
besoins de sa recherche.

Pour conclure, grâce à cette séparation des composants \textit{hard-ware}, la
plateforme répond efficacement à une diversité de besoins et permet son
intégration dans des projets aux ressources matérielles variées. Même
sans ressource matérielle capable de faire tourner les modèles de
vision, les utilisateur.rices peuvent toujours exploiter la plateforme web.
Cette conception offre un accès aux outils et méthode à des utilisateur.rices
divers, allant des projets sans ingénieur dédié pour le
développement, aux équipes de recherche disposant de leurs propres
ingénieurs, en passant par des doctorants indépendants ayant des
compétences en programmation mais sans accès à un serveur. L'outil est
pensé pour s'adapter à des environnements variés, des configurations
légères fonctionnant en local, jusqu'à des projets disposant de
ressources matérielles importantes comme un \gpu.

\hypertarget{software-des-modules-separes}{%
\subsection{Software~: des modules
séparés}\label{software-des-modules-separes}}

Le modèle MVC (Model-View-Controler) est une architecture logicielle qui
segmente une application en trois composantes interconnectées. Le Modèle
est chargé de la gestion des données et de la logique métier de
l'application, assurant la manipulation et l'administration des
informations. La Vue est responsable de la présentation visuelle des
données, les mettant en forme visuellement dans un \textit{template}. Le
'Contrôleur', sert d'intermédiaire entre le 'Modèle' et la 'Vue'~: il reçoit
les entrées de l'utilisateur.rice via la 'Vue', traite ces entrées, puis
interagit avec le 'Modèle' pour actualiser les données et, enfin, met à
jour ces modifications dans la Vue. Cette séparation des préoccupations
permet une organisation plus rigoureuse du code, facilitant ainsi la
maintenance, la réutilisabilité et le développement parallèle de chaque
composante.

Le cycle action → mise à jour → affichage induit par ce patron est bien
adapté aux applications web, il est à ce titre utilisé par nombre
d'entre elles, dont \eida fait partie, et par de nombreux \textit{frameworks},
Django y compris.

Bien que le modèle MVC offre déjà une structure prenant en compte la
séparation des préoccupation, \eida cherche à aller au-delà, proposant
une architecture encore plus flexible. La plateforme est conçue pour
permettre aux développeur.ses d'ajouter ou de supprimer des fonctionnalités
de manière indépendante. Cette approche permet de personnaliser
l'application en fonction des besoins spécifiques de chaque projet, sans
avoir à modifier le cœur du système. Les utilisateur.rices peuvent ainsi
partir de la base de la plateforme et la compléter avec des modules sur
mesure.

Voici une transcription de l'arborescence des fichiers de l'application
:

\begin{verbatim}
app/
├── config/
├── logs/
├── mediafiles/
├── regions/
├── similarity/
├── vectorization/
│   ├── templates/
│   ├── __init__.py
│   ├── const.py
│   ├── tasks.py
│   ├── urls.py
│   ├── utils.py
│   ├── views.py
├── webapp/
├── webpack/
├── __init__.py
├── manage.py
├── requirements-base.txt
├── requirements-dev.txt
├── requirements-prod.txt
cantaloupe/
celery/
docs/
gunicorn/
sas/
scripts/
├── .gitignore
├── .pre-commit-config.yaml
├── README.md
├── run.sh
\end{verbatim}

Il a été créées plusieurs unités fonctionnelles pouvant inclure leurs vues, \textit{templates}, utilitaires, etc. Cette approche
permet aux développeur.ses de découper l'application tout en factorisant le code dédié à
plusieurs tâches, ainsi les modules partagent des \textit{statics}, un fichier
de configuration global et des fonctions utilitaires.

Chaque sous-dossier dans \texttt{app/} représente un module fonctionnel. Le
répertoire \texttt{webapp/} contient le module de base, tandis que \texttt{webpack/} est
dédié aux interfaces\footnote{Voir le \hyperlink{chapitre-8-interfaces}{chapitre suivant}}. Les modules
additionnels, autonomes, peuvent s'interfacer les uns avec les autres,
et être développés puis testés de indépendamment. Pour un exemple
détaillé du contenu des fichiers, une description du module dédié à
la vectorisation développée pendant se trouve en annexe \ref{module_vecto}. 

La variable \texttt{INSTALLED\_APPS} du fichier de configuration global permet de
personnaliser l'application en activant ou désactivant les modules
souhaités.

\emph{Modularité}

Cette architecture s'inscrit dans une stratégie de développement
applicatif ouverte et évolutive. La division en unités fonctionnelles et
indépendantes permet d'ajouter ou de supprimer les fonctionnalité
complémentaires au module de base. Ce cadre de développement modulaire
garantit que l'application reste adaptable aux exigences évolutives des
chercheur.ses et des institutions partenaires, la rendant plus robuste et
tolérante aux usages extérieurs et autorisant alors le réemploi du code
par des projets ayant besoin d'effectuer des traitements divers sur
du matériel documentaire pictural numérisé. \aikon est utilisable par des
projets divers, ce qui réduit le temps de développement et ouvre la voie
à des partenariats, aidant alors à pérenniser les outils.

\emph{Maintenance}

L'organisation modulaire du code facilite également la maintenance et
les mises à jour. Avec une telle structure, il devient plus facile
d'isoler les composants pour le développement et le débogage. Les
développeur.ses peuvent travailler sur un module spécifique sans interférer
avec les autres parties du projet. De plus,
l'implémentation d'un module indépendant provoquera moins de conflit
lors du déploiement.

\vspace{2cm}

En résumé, \eida repose sur la constitution de jeux de données de
diagrammes astronomiques extraits de sources relevant de multiples
domaines linguistiques et culturels, sources connectées à l'échelle
afro-eurasienne. Le projet développe des outils numériques permettant
une approche critique des diagrammes astronomiques à large échelle,
fournissant à ce titre une base solide pour une étude approfondie des
schémas de circulation des connaissances astronomiques entre le
\ma et l'époque moderne.

En \cv, \eida s'appuie sur les récents progrès en
\dl, et teste l'application d'une nouvelle génération de
méthodes vectorisation d'images sur des sources
historiques~: manuscrits, \emph{early prints} et xylogravures.

Au centre de ce projet réside une approche intrinsèquement transversale
et interdisciplinaire, qui s'appuie sur des projets antérieurs et
bénéficie de la collaboration avec des équipes dotées de compétences en
ingénierie, ainsi qu'avec le projet \textsc{anr} \vhs~; bien que les objectifs de
ce dernier divergent de ceux d'\eida. Ainsi, au sein même de l'écosystème
d'un même projet de recherche, les livrables doivent s'adapter aux
usages et besoins des chercheur.ses, à la diversité de leurs questions de
recherche et enfin à deux corpus différents dans toute leur
hétérogénéité. De fait, la problématique de modularisation est au cœur
du projet \vhs/\eida. Cette modularisation impose de créer des composants
logiciels indépendants et réutilisables, facilitant ainsi l'adaptation
du projet aux deux contextes, comme nous le développeront plus avant. En
outre, pour garantir une approche conforme aux principes de l'open-source et de la science ouverte, une couche de standardisation doit être
ajoutée. Cette standardisation vise à rendre les composants du projet
interopérables, permettant ainsi une collaboration transparente et la
réutilisation des résultats de recherche au sein de la communauté
scientifique. En intégrant ces principes dans la conception et le
développement du projet, il devient possible de promouvoir la
transparence, la reproductibilité et l'accessibilité des outils
techniques (modèles de vision) et des corpus enrichis.

            
        \clearemptydoublepage
        
\hypertarget{chapitre-2-open-data-et-enjeux-interop}{%
\chapter{Open-data et enjeux d'interopérabilité~: la standardisation technique à grande échelle}\label{chapitre-2-open-data-et-enjeux-interop}}

                Les acteurs impliqués dans la réalisation d'un outil tel que la
plateforme commune \aikon ne se limitent pas aux participants du projets évoqués
dans la partie précédente. Les potentiels utilisateur.rices futurs et les
institutions détentrices des données sont aussi à prendre en compte,
ils orientent les développements et les fonctionnalités des
outils conçus.

Le traitement des sources iconographiques revêt des enjeux spécifiques
du point de vue technique, juridique et de la disponibilité. La remise
en circulation des résultats des traitements, à l'autre bout de la
chaîne, présente d'autres défis et soulève des questions de visibilité,
d'accès, et de médiation qui sont autant d'exigence à prendre en
considération pour garantir l'utilité des résultats pour l'ensemble des
utilisateur.rices et partenaires impliqués.

Enfin, la mutualisation du code induit des enjeux d'organisation devant aussi être pris en considération.

Ces enjeux contraignent le projet à rentrer dans un cadre technique
rigoureux où plusieurs aspects clés doivent être respectés~:
l'utilisation de standards ouverts pour garantir la compatibilité et
l'interopérabilité entre différents systèmes et outils, l'adoption de
formats de données standardisés pour faciliter le partage et l'échange
d'informations, la conformité avec les réglementations juridiques en
matière de droits d'auteur et de protection des données, et la prise en
compte de l'interaction avec des utilisateur.rices divers.

\begin{kwote}
``une compatibilité technique et sémantique, l'interopérabilité des
données aide au décloisonnement entre domaines d'expertise et permet à
des projets de recherche aux objectifs divergents de s'appuyer sur un
socle commun.''\footcite[p. xvii]{albouy_mediation_2019}
\end{kwote}

                \hypertarget{ouverture-donnees}{%
                \section{Ouverture des données~: accès aux sources et partage des résultats de la recherche}\label{ouverture-donnees}}
                        Le module de base est un package pour la gestion documentaire, duquel
l'application ne peut se détacher. Celui-ci inclut tout d'abord des
formulaires pour l'intégration des documents dans la base de données. Le
modèle de données permet de décrire différentes entités qui, bien que
liées dans leurs métadonnées, peuvent être intégrées indépendamment. Le
module de base permet également la création de \mans \iiif pour
chaque numérisation, permettant ensuite la visualisation des documents
grâce aux outils open-source dédiés. De ce fait, l'indexation de zones
d'image peut être réalisée manuellement via l'interface Mirador intégrée
à \sas. Ce noyau fonctionnel inclut en outre la sélection de lots de
documents (le ``panier''), sur lesquels pourront être effectués des
traitements groupés paramétrables.

Les briques fondamentales offrent donc les fonctionnalités essentielles
de gestion documentaire (intégration, modèle de données, \iiif). Les
traitements, quant à eux, sont gérés par des modules séparés, et c'est
sur cette structure que repose la modularité et l'évolutivité de
l'application.

Ci-après nous donnons une description détaillée de certaines de ces
fonctionnalités de base.

\hypertarget{description-des-donnees}{%
\subsection{Description des
données}\label{description-des-donnees}}

Le module de base contient un modèle de données suffisamment extensif
pour décrire efficacement une diversité de données, allant de documents
textuels historiques à des tableaux en histoire de l'art. La
tripartition entre témoin (\wit), série (qui contient un ensemble de
témoins), et contenu permet un alignement avec des corpus très
diversifiés et des données potentiellement hétéroclites, telles que des
manuscrits, des documents épistolaires, des inventaires de galeries
d'art, et même pourquoi pas des cartes\ldots{}

Pour ouvrir à cette large diversité de données, la liste des types de
pagination témoin doit être étendue \emph{a minima} d'un nouveau type
``other'', émancipant l'enregistrement des mentions de pagination. Les
développements futurs prévoient aussi la création d'un système pour
ajouter facilement un nouveau type\footnote{Le type de témoin est une
  métadonnée rentrée par l'utilisateur.rice lors de l'enregistrement du
  \wit dans la base de donnée. Il choisit le type dans une liste,
  originellement manuscrit, imprimé ou gravure sur bois.} de \wit
(tel que peinture, catalogue, etc.).

Au fil des développements, des débats ont émergé autour de l'ajout dans
le modèle de données d'un niveau de granularité supplémentaire pour
décrire des images ou zones d'images unitaires
(\graphicals), créant ainsi une entité détachée du fait
qu'elle provienne d'une extraction dans un document. Cette solution
aurait permis une description plus détaillée et plus fine des images,
importante pour des projets axés sur des images uniques, et aurait
favorisé un élargissement du spectre des type de sources pris en charge.
L'utilisateur.rice aurait pu soit importer une image unique (et de manière
optionnelle, la lier à un \wit) via un formulaire, soit sélectionner
une région d'image d'intérêt au sein des extractions (annotations \sas),
laquelle serait enregistrée comme \graphical, puis l'enrichir de
métadonnées. Dans les deux cas l'enregistrement d'un \graphical
aurait donné lieu à la création d'une \digit au format \jpeg.

Sans l'unité de description \graphical, les régions d'images
sont créées uniquement via les annotations \sas.

L'intégration de cette entité au sein du modèle aurait offert plusieurs
avantages en termes de cohérence et de flexibilité. En s'alignant sur
les structures existantes (\wits et \sers), elle aurait permis une
manipulation plus intuitive des images, facilitant ainsi les opérations
de recherche et la création de \emph{Sets} personnalisés. De plus, elle
aurait rationalisé la gestion des annotations \sas, permettant de
sélectionner les plus pertinentes dans la multitude existante.

Cependant, cette approche présente des limites, et on peut trouver des
alternatives. Tout d'abord, la coexistence de \graphicals avec les
annotations \sas, générées par des processus distincts, aurait pu créer
une certaine confusion quant à leur nature et à leur méthode de
création. De plus, la multiplication potentielle de milliers
d'enregistrements aurait pu impacter les performances de la base de
données et complexifier les requêtes. Enfin, le lien sémantique ambigu
et sujet à interprétation subjective entre \graphical et \wit
aurait compliqué les possibilités de corrélation.

Compte tenu de ces limites, il a semblé préférable de maintenir les
annotations \sas pour identifier les instances de base du modèle, sans
créer de nouvelle unité de description. La solution actuelle reste donc
basée sur la création manuelle ou automatique de zones dans les images
via \iiif et \sas, évitant les problèmes de redondance et de confusion.
Bien que l'entité \graphical n'ait pas été implémentée, les
fonctionnalités d'annotation et de sélection d'images sont assurées par
d'autres mécanismes. L'outil Mirador permet d'associer des tags aux
zones d'image, offrant ainsi une première couche d'enrichissement
sémantique. La sélection dans un \emph{set} personnalisé sera possible en
gardant en mémoire une référence contenant des coordonnées du
\emph{crop}. De plus, l'importation d'images individuelles est
réalisable en les considérant comme des \emph{Witness partiels}, ce qui
permet de les intégrer dans le \textit{workflow} existant. Toutefois
l'enrichissement sémantique à un niveau de granularité fin restera
limité~; et la dépendance à l'outil \sas constitue une potentielle dette
technique, susceptible de restreindre les évolutions futures du système.

Afin de mieux répondre aux exigences de modularité, l'évolution du
modèle de données s'oriente non pas vers une description individuelle
des documents, mais vers la gestion des traitements. Cette évolution
implique la création d'une entité \tr
liée à des ensembles de données (\ds et
\rs) potentiellement hétérogènes.

\hypertarget{principe-du-traitement}{%
\subsection{Principe du Traitement}\label{principe-du-traitement}}

Le but fondamental de la plateforme est de pouvoir effectuer plusieurs
actions sur les objets de la base. Afin d'assurer une meilleure
traçabilité et plus de flexibilité, la plateforme abandonne les
lancements automatiques des processus\footnote{C'était initialement le
  cas de l'extraction des entités, dont le lancement était lié à une
  méthode de classe liée à la \digit après soumission d'un
  formulaire d'ajout d'un \wit ou d'une \ser. L'action se
  lançait immédiatement après enregistrement des images d'une
  numérisation dans la plateforme.} au profit d'un système basé sur
l'entité \tr. Chaque traitement est associé à un ensemble
d'objets traités ensemble (\ds ou \rs), à un jeu de
paramètres et à un résultat. Ces informations sont stockées dans une
table dédiée. Cette approche facilite la gestion et le trackage des
processus (notamment, les utilisateur.rices sont notifiés par e-mail à la fin
du \textit{processing}), permet aux utilisateur.rices de consulter un historique de
leurs actions et offre la possibilité de créer des \textit{workflows}
personnalisés en passant par un formulaire de lancement unique mais
extensif.

En permettant de regrouper des documents de types différents (\wos,
\sers, \wits) dans des \dss, on offre à
l'utilisateur.rice la flexibilité de lancer des actions sur des ensembles
d'entités hétérogènes et granulaires. Le traitement est ensuite réparti
sur les entités de niveau inférieur (les témoins). Les \wits ainsi
sélectionnés peuvent être soumis à une large gamme de traitements~: des
fonctions déjà implémentées comme l'exportation (avec choix du
format), l'extraction, la vectorisation, la recherche de similarité~; ou
de nouveaux traitements personnalisés, tels que la visualisation sur une
frise chronologique ou une carte. La modularité de la plateforme est
assurée par un formulaire de lancement configurable, permettant de
l'adapter à différents scénarios d'utilisation, et à l'ajout de modules
personnalisés.

Le \rs fonctionne similairement au \ds, à un niveau
de granularité inférieur (à l'échelle de la zone d'image)\footnote{À
  l'été 2024, l'entité n'existe pas encore dans la base de données, mais
  le processus d'envoi du traitement et les modes de communication entre
  l'application et l'\api prévoient la possibilité de lancer l'inférence
  des modèles sur un ensemble de régions extraites.}.

\hypertarget{extraction-des-zones-dimage-manuelle}{%
\subsection{Extraction manuelle des zones d'image}\label{extraction-des-zones-dimage-manuelle}}

Le choix de la méthode d'extraction des régions d'intérêt dans les
documents constitue un élément clé de la modularité de la plateforme.
Les utilisateur.rices peuvent opter pour une extraction manuelle ou une
extraction automatique basée sur des algorithmes de vision par
ordinateur, adaptée aux traitements à plus grande échelle.

Après importation d'un enregistrement, le flux de travail procède à la
création de \mans \iiif pour chaque numérisation
(\digit) afin de permettre une visualisation grâce à la
plateforme Mirador. Le module de base autorise par la suite
l'extraction manuelle de zones d'intérêt au sein des images. Cette
fonctionnalité est particulièrement utile pour les projets ne souhaitant
pas recourir à des méthodes entièrement automatisées de vision par
ordinateur. L'outil \sas permet de créer des annotations, c'est-à-dire de
définir des régions d'intérêt spécifiques dans les numérisations, et de
les indexer directement dans les \mans \iiif correspondants,
enrichissant ainsi les ressources numériques. De plus, les
développements futurs prévoient la possibilité d'importer des fichiers
d'annotation préexistants en format .\textsc{txt} afin de pouvoir les indexer
manuellement. Par conséquent, le \textit{workflow} de base ne comporte aucun
traitement automatique basé sur la vision (et de fait éventuellement
trop gourmand en puissance de calcul).

L'extraction, qu'elle soit manuelle ou automatique, constitue le
fondement du reste des processus. Une interface est disponible pour
sélectionner un ensemble de documents et effectuer des actions
spécifiques sur les témoins annotés, via le formulaire de traitement qui
s'étend selon un choix de module configuré. Ainsi l'utilisateur.rice n'est
pas limité par un contexte initial, à l'origine deux étapes
indissociables et incontournables (importation et extraction), pour
pouvoir effectuer d'autres actions. Cette modularité permet de
s'affranchir d'un \textit{workflow} linéaire et prédéfini, offrant ainsi une plus
grande adaptabilité aux besoins spécifiques et aux ressources
matérielles des projets.

Pour conclure, l'existence de ce module de base répond à des besoins
élémentaires des projets de recherche en études visuelles. Il fournit un
outil qui permet d'agréger toutes les sources primaires qui concernent
le sujet, de décrire les sources et de les mettre en relation. Il offre
en outre la possibilité d'extraire et visualiser des contenus d'intérêt
(les ``crops'' d'images), ciblant ainsi les instances de base qui
intéressent les chercheur.ses.
            
                \hypertarget{iiif}{%
                \section{IIIF}\label{iiif}}
                        Spécialiser un modèle d'intelligence artificielle implique de lui
fournir des données pertinentes, diversifiées, et en quantité
suffisante. Cependant, pour certains domaines, dont l'histoire fait
partie, le volume de données disponible est insuffisant. Ce constat est
d'autant plus vrai dans le cas des diagrammes issus de traités
astronomiques~: les corpus de documents scientifiques historiques
contiennent généralement du texte en majeure partie, des tables et des
images, négligeant souvent les diagrammes.\footnote{Exception faite du
  corpus S-VED (\cite{buttner_cordeep_2022}), collection
  d'illustration très diverses contenant entre autre des diagrammes
  historiques. Mais les primitives ne sont pas annotées.}. De plus, ils
sont dénués d'annotations précises sur les éléments constitutifs des
pages~; c'est sans parler de l'inexistence d'un corpus de diagrammes
dont les primitives sont annotées. Or l'annotation est une tâche
chronophage et fastidieuse. Le recours aux données synthétique répond,
mais en partie seulement, à ces problématiques.

\hypertarget{datasets-synthetiques}{%
\subsection{\emph{datasets} synthétiques}\label{datasets-synthetiques}}

Les \textit{datasets} synthétiques sont générés par des algorithmes ou des
méthodes de simulation pour imiter des données réelles, sans être
directement extraites de sources existantes. De tels jeux de données
sont utilisés lorsque les données réelles sont limitées ou difficiles à
obtenir, mais qu'il est cependant nécessaire de contrôler spécifiquement
les caractéristiques des données d'entraînement\footcite{buttner_cordeep_2022}. La génération
d'images a pour but de fabriquer des ensembles de données plus vastes,
plus diversifiés, très variables et assez complexes, répondant aux
caractéristiques des objets d'intérêt du projet, et surtout étiquetés
automatiquement, sans recourir à l'annotation manuelle.

Ces données synthétiques sont assez ressemblantes et complexes pour être
exploitées. Par exemple, docExtractor est un modèle off-the-shell (au
même titre que \yolo) envisagé dans le cadre de la tâche d'extraction des
diagrammes, et qui se veut sépcifique aux données historiques, car il
est entraîné sur des données produites par un générateur de documents
historiques synthétiques~: SynDoc\footcite{monnier_docextractor_2020}. SynDoc
génère des images de manière aléatoire en combinant des éléments
graphiques (fonds, images, texte et bruit) provenant d'un jeu d'image
défini (constitué de 177 images de pages, 15 contextes, plus de 8000
œuvres d'art provenant de WikiArt, des lettrines générées à partir d'une
lettre aléatoire avec 91 fonts possibles, et des dessins, schémas et
textes tirés d'articles aléatoires sur Wikipedia, avec plus de 400
fonts). Les différents éléments s'agencent, intégrant sur le fond
images, texte et bruit, offrant des combinasons et des mises en pages
assez complexes. Chaque élément de contenu est pré-annoté, éliminant
ainsi le besoin d'annotations manuelles pour ces pages.

          \begin{figure}[H]
          \begin{center}
          \includegraphics[height=6.5cm]{figues/syndoc.jpg}
          \end{center}
          \caption{Données synthétiques générées par SynDoc.\footcite[p.46]{norindr_traitement_2023}}
          \label{fig:syndoc} \end{figure}

Pour entraîner le modèle de vectorisation, il a de même été nécessaire
d'utiliser des données synthétiques. Parce qu'annoter les primitives
géométriques dans des images de diagrammes complexes est très
chronophage, le modèle de vectorisation a été pré-formé sur des corpus
artificiels générés dynamiquement. Le script de génération des données
d'entraînement choisit aléatoirement un arrière-plan, y ajoute des mots,
des nombres et des glyphes puis crée artificiellement un diagramme en
insérant des segments, des cercles et des arcs. Le script est conçu pour
que ces diagrammes aient une forte probabilité de présenter des formes
très caractéristiques comme les cercles concentriques et tangents, les
lignes parallèles et les arcs connectés, afin de simuler les structures
typiques. Les primitives sont dessinées avec des
variations aléatoires d'opacité, de largeur et de couleur. Les cercles
peuvent être remplis ou vides. Enfin, du bruit est ajouté en appliquant
un flou gaussien, et en supprimant de petites régions du diagramme pour
imiter la dégradation des documents historiques. Les données
d'entraînement ainsi générées présentent des configurations assez
complexes.

          \begin{figure}[H]
          \begin{center}
          \includegraphics[height=7cm]{figues/vecto_synthetic_data.png}
          \end{center}
          \caption{Données synthétiques générées pour l'entraînement du modèle de vectorisation.\footcite[Figure issue de la présentation de Syrine Kalelli à l'occasion de la conférence \eida 2024~:][]{noauthor_eida_nodate-1}}
          \label{fig:vecto_synthetic} \end{figure}

Enfin, le modèle de similarité présente un troisième exemple, puisque
SegSwap est pré-entraîné sur de la donnée synthétique. Le script de
génération prend des parties aléatoires d'une images et les copie-colle
au-dessus d'une autre image. Les trois images (source, cible et
superposition) sont placées dans le même dataset d'entraînement, ainsi
le modèle apprend à retrouver ce qui, dans la superposition, vient de la
source, et ce qui vient de la cible.

          \begin{figure}[H]
          \begin{center}
          \includegraphics[height=3cm]{figues/segswap_blended_images.png}
          \end{center}
          \caption{Données d'entraînement du modèle Segswap.}
          \label{fig:segswap} \end{figure}

\hypertarget{les-donnees-reelles}{%
\subsection{Les données réelles}\label{les-donnees-reelles}}

S'appuyer sur les modèles \textit{off-the-shelf}, sur de larges \textit{datasets}
généralistes, ou sur des données synthétiques permet une implémentation
facilitée de la vision dans des projets et constitue une base solide.
Toutefois, les sources tenant aux deux projets (\vhs et \eida) sont trop
spécifiques pour se contenter de modèles généralistes ou formés sur des
données artificielles. Même si ces derniers peuvent offrir des performances
de base, ils risquent de manquer de précision et de sensibilité aux
particularités des documents historiques. Les corpus artificiels
présentent des configurations délibérément complexes pour s'approcher le
plus possible des difficultés que le modèle pourrrait rencontrer sur les
données réelle. Elles sont cependant irréalistes et insuffisantes pour
permettre aux modèles de généraliser sur des diagrammes réels.

En atteste la comparaison des performances de docExtractor et \yolov sur
les données d'\eida. docExtractor\footcite{monnier_docextractor_2020}, entraîné
sur des données synthétiques mimant les documents historiques serait en
théorie plus adapté au traitement d'images de pages de manuscrits, avec
du texte et des illustration côté à côte, d'autant qu'il intègre des
outils de traitement du texte (notamment pour la segmentation des
lignes)\footnote{\eida envisage l'implémentation d'un outil d'extraction
  et transcription des labels et des textes qui entourent les diagrammes}.
Pourtant, sans fine-tuning sur des données réelles, il présente des
performances équivalentes à celles de \yolov\footcite[p.45]{norindr_traitement_2023}. Cela
souligne que même les modèles off-the-shelf entraînés sur un corpus
assez spécifique et complexe, mais synthétique, ne dispense pas d'un
entraînement sur des données réelles, au même titre que les modèles très
généralistes comme \yolov.

Alors, le modèle de base \yolov tel que mis à disposition par
Ultralytics est entraîné sur de grands ensembles de données réelles, ce
qui constitue une base solide pour la classification des objets du
monde. L'utilisation de SynDoc permet ensuite de compléter
l'apprentissage initial en exposant le modèle à des exemples variés et
spécifiques aux documents historiques, augmentant ainsi sa capacité de
généralisation. Ces similis de manuscrits anciens offrent l'avantage de
pouvoir être produits en grandes quantités et de couvrir un large
éventail de scénarii et de configurations difficiles à obtenir dans des
ensembles de données réelles. Puis le modèle est entraîné sur les
données de \vhs, qui sont de réelles pages de documents historiques
contenant une large diversité d'illustrations. Ces données apporteront
une dimension supplémentaire de pertinence au modèle, en l'exposant à
des particularités des documents historiques réalistes. Enfin, \yolov
est entraîné sur les données d'\eida, qui sont orientées spécifiquement
vers les diagrammes, afin qu'il détecte uniquement ces derniers.

Quant au modèle de vectorisation développé par Syrine
Kalleli\footcite{kalleli_historical_2024}, il est formé
sur des données synthétiques générées à la volée par un script. Mais le
corpus de diagrammes d'\eida est particulièrement caractéristique et le
modèle n'aurait pu être optimal sans avoir appris sur des images de
diagrammes issus de manuscrits réels. Un corpus d'entraînement de 303
diagrammes extraits de manuscrits et de gravures a donc été constitué et
annoté par les historien.nes. Ces diagrammes sont issus de sources latines, arabes,
grecques, hébreuses ou chinoises, datant du \textsc{xii}\ieme au \textsc{xviii}\ieme siècle, et ils
présentent en guise d'étiquettes plus de 3000 lignes, cercles et arcs. Le
ré-entraînement a permis le transfert des connaissances acquises sur la
tâche de détection des primitives sur les données réalistes.

Il sera également possible d'obtenir des meilleurs résultats sur la
similarité grâce à une évaluation des scores (qui constitue un jeu de
données annotées) et le ré-entraînement du modèle, pour donner des
résultats plus adaptés à la spécificité des données historiques.

D'ailleurs, cette étape d'annotation (le choix des exemples et des
étiquettes) revêt des enjeux importants. L'apprentissage spécifique se
fait à partir de données sélectionnées par les chercheur.ses~: les exemples
sur lequel l'algorithme d'apprentissage va itérer définissent le modèle.
Il est nécessaire de constituer un échantillon de données aléatoire et
représentatif, et de l'annoter en fonction de ce que l'on souhaite
obtenir en prédiction.

L'annotation des jeux de données est non seulement une étape clé, mais
aussi un bel exemple de collaboration chercheur.ses-ingénieur.es. Elle
nécessite la définition de normes pertinentes et rigoureuses. Travail
minutieux et chronophage, l'étiquetage des données peut engendrer des
erreurs et du bruit dans les données, car elle implique la subjectivité
des chercheur.ses et le regard parfois trop précis sur les sources desquels
les annotateurs sont experts.

Voici un exemple rencontré lors de la préparation des données pour
entraîner un modèle de segmentation du contenu textuel. Les sources
arabes et chinoises sont particulièrement verbeuses et les diagrammes
sont très souvent entourés des blocs de commentaires se mélangeant alors
aux légendes et aux labels. Doit-on considérer ces commentaires comme
faisant partie des éléments que l'on souhaite identifier ou bien les ignorer
? Cette décision est importante car si on les ignore, le modèle risque
de passer à côté d'éléments textuels pertinents. En revanche, si on les
inclut, il ramènera des commentaires sans rapport direct avec le
diagramme observé. On voit ici comment la binarité des modèles, qui se
reflète dans les normes d'annotation, est problématique et constitue une
limite au \ml. Un compromis doit être trouvé entre
l'automatisation, qui requiert une normalisation, des définitions
claires et binaires, et la nuance dans l'interprétation des
sources\footnote{Dans le cadre du projet, il a toujors été plus
  intéressant d'opter pour une définition extensive des objets à
  détecter, car prévision d'une correction des traitement. Et il est
  plus facile de supprimer un élémént pas pertinent que d'aller en
  rechercher un, surtout compte tenu de la taille des corpus des
  chercheur.ses. Vaut aussi pour la préparation des données pour
  l'entraînement du modèle d'extraction.}.

La normalisation peut bénéficier à l'écosystème de recherche dans le
domaine de l'\htr et de l'\ocr. À ce titre, il est pertinent d'envisager
l'utilisation du vocabulaire contrôlé SegmOnto pour l'annotation du
contenu textuel entourant les diagrammes. Cela permettrait de créer des
jeux de données réutilisables, à partager avec des projets poursuivant
des objectifs similaires.\footnote{https://segmonto.github.io/}. Encore
une fois, un compromis doit être trouvé entre les besoins de description
des chercheur.ses et les possibilités offertes par les vocabulaires
contrôlés.

Un autre exemple concerne le dernier entraînement du modèle d'extraction
: les résultats montrent que des diagrammes sont encore détectés en
transparence. La question s'est alors posée de chercher à corriger ce
défaut en donnant au modèle, à l'occasion d'un nouvel entraînement,
d'avantage d'exemples négatifs (diagrammes visibles par transparence
mais non annotés). Or il est préférable de se contenter de la correction
ou suppression manuelle de ces prévisions erronées, garantissant que le
modèle parvienne à détecter les diagrammes presque effacés.

Pour assurer la rigueur et la cohérence des annotations, les décisions
prises entre les chercheur.ses et les ingénieur.es peuvent être l'objet d'une
documentation ou d'ateliers d'annotation.

\hypertarget{loeil-de-la-machine-avantages-et-limites}{%
\subsection{L'oeil de la machine~: avantages et
limites}\label{loeil-de-la-machine-avantages-et-limites}}

Bien qu'il soit possible d'optimiser les performances d'un modèle
d'apprentissage automatique en l'entraînant sur un ensemble de données
spécifique, son interprétation des données reste limitée car
fondamentalement binaire, ce qui le rend parfois déficient pour la
recherche en histoire. Ainsi, il gèrera difficilement les cas limites et
ambigüs. La décision d'inclure ou d'exclure ces cas particuliers de
l'ensemble d'entraînement implique un arbitrage délicat. D'un côté, une
inclusion trop restrictive peut compromettre les capacités de
généralisation du modèle, c'est-à-dire sa capacité à s'adapter à de
nouvelles données. À l'inverse, une inclusion trop permissive risque de
dégrader la précision du modèle sur les cas plus typiques. Les
chercheur.ses espérant obtenir un modèle maximaliste, quitte à accepter un
certain degré d'erreur et de devoir supprimer les faux
positifs, de nombreux cas limites ont été inclus. Le cas des diagrammes
visibles en transparence (expliqué précédemment) en est un exemple
éloquent.

Une autre difficulté réside dans la définition même du ``diagramme
astronomique''. Les limites de ce concept ne sont pas si claires et
définitives pour les chercheur.ses, et pourtant le modèle a besoin d'une
définition rigoureuse et cohérente. Il paraît en effet difficile de
considérer les diagrammes astronomiques en dehors du contexte des
pratiques d'autres sciences et disciplines connexes. Par exemple, Le
\emph{Flores Almagesti} -- réécriture de l'Almageste datant du \textsc{xv}\ieme par
l'astronome Giovanni Bianchini -- présente une partie algébrique à
l'ouverture mathématique, induisant la présence de nouveaux types de
diagrammes d'inspiration euclidienne. Pour retracer la source de ces
derniers, il est nécessaire de considérer les traités d'Euclide ou
autres travaux d'algèbre. Ceux-ci ne sont pas des traités
\emph{astronomiques}, bien qu'il ne soit pas certain que ces disinctions
contemporaines aient été aussi rigide à l'époque et aient eu un
quelconque sens pour les acteurs historiques. Les sources byzantines
confirment cette complexité~: les diagrammes y sont nommés
\emph{katagraphai}, indépendamment du domaine scientifique auquel ils
appartiennent. Également, de nombreux travaux astronomiques sont groupés
dans des témoins qui contiennent des œuvres issus de domaines divers.
C'est le cas avec les sources chinoises, comme le \emph{Chongzhen
lishu}, qui se présente généralement annexé d'une série de traités
mathématiques. Par conséquent, les diagrammes euclidiens ont été gardés
lors de la préparation des données, et l'algorithme de détection les
classe comme ``diagramme'', même s'ils ne constituent pas l'objet
principal des chercheur.ses.

En ce qui concerne les autres types de diagrammes non strictement
astronomiques (géométriques, harmoniques, logiques, illustrations de
constellations), une approche plus sélective a été adopté afin d'éviter
un modèle trop maximalistes. Ces éléments, bien que potentiellement
intéressants, n'ont pas été inclus dans la phase de détection
automatique.

Ainsi l'œil de la \cv contraint à des choix méthodologique
potentiellement inconfortables, mais en même temps il peut aider à
mesurer les impulsions des chercheur.ses, à mieux définir les objectifs de
recherche et à prioriser les éléments les plus pertinents. Ainsi, la
vision par ordinateur oblige les chercheur.ses à s'adapter à une logique
algorithmique qui, tout en limitant certaines interprétations
subjectives, offre l'opportunité de développer des modèles conceptuels et des méthodologies très rigoureuses.

\vspace{2cm}

Donc si les stratégies d'accès à la donnée se voudraient universelles,
il existe des limitations techniques ou juridiques demandant une
flexibilité dans les modes d'accès aux sources. Il faut donc laisser
ouvert des modes d'accès plus ``artisanaux'' (fichiers stockés en local,
métadonnées rentrées manuellement) que le standard \iiif, mais aussi
garantir une gestion des droits d'accès aux sources et leur
exploitation, afin de respecter les exigences des
institutions détentrices. 


            
        \clearemptydoublepage
        
\hypertarget{chapitre-3-EDA-image-ia}{%
\chapter{État de l'art~: IA et traitement du volume}\label{chapitre-3-EDA-image-ia}}

            La \bnf constate le besoin d'automatisation pour traiter les volumétries
croissantes des collections numérisées, en outre caractérisées par une
variété considérable, et elle relève les tensions se jouant dans le traitement en
masse de données hétérogènes.

\begin{kwote}

"De plus en plus confrontées à des niveaux de volumétrie et de vélocité
typiques des mégadonnées (big data), les collections numériques de la
\bnf, qui occupent aujourd'hui environ six pétaoctets, sont caractérisées
par une variété considérable. Documents numérisés, tels que par exemple
les livres et manuscrits consultables dans Gallica --- la bibliothèque
numérique de la \bnf ---~; documents nativement numériques comme les
œuvres d'art vidéo, les logiciels, les bases de données, les archives de
l'Internet~; métadonnées bibliographiques et données d'autorité
décrivant les personnes, lieux, organisations, concepts\ldots{} autant
d'ensembles de données diverses en termes de structures, formats,
qualité, contextes de production, fonctions et contenus. Ces ensembles
ont des histoires différentes, issues des changements des supports et
des multiples strates de pratiques documentaires accumulées au fil du
temps. Leur hétérogénéité exige des traitements spécifiques et par
conséquent des compétences et des méthodes particulières, aussi bien
pour les conserver ou les communiquer que pour les analyser (cf
Moiraghi, 2017). Cette hétérogénéité des données, qui découle de
l'amplitude chronologique et de la vocation à l'encyclopédisme
caractéristiques des bibliothèques nationales, s'ajoute à
l'accroissement de la quantité des données en entrée et à l'accélération
conséquente des temps de traitement. La tendance traditionnelle des
bibliothèques à la systématisation des procédures doit dès lors trouver
son équilibre face à la spécificité des données mais aussi des questions
scientifiques propres aux projets de recherche qui les
exploitent".\footcite[p.6]{bermes_patrimoine_2020}  
\end{kwote}

L'état de l'art montre une convergence vers l'utilisation de
l'intelligence artificielle pour traiter efficacement des corpus de
données de plus en plus vastes et complexes. Cette systématisation doit
cependant être équilibrée par une compréhension fine des contextes
locaux et spécifiques des données, afin de garantir la pertinence et
l'efficacité des outils développés.

Les explorations autour du traitement du patrimoine numérique menées à
la \bnf\footnote{\cite{bermes_patrimoine_2020}, \cite{beaudouin_cartographie_2017}, \cite{michez_gallicapix_2021}, \cite{bouchard_presentation_2017}},
en rapport étroit avec des projets de recherche en \hn, illustrent bien
cette tension. Ces projets ont porté sur des ensembles de données
balisés et spécifiques (tel que les sources documentaires numérisées
autour de la guerre 14-18, les publicités de 1910 à 1920, illustrations
du magazine de mode Vogue de 1920 à 1940 ou encore illustrations de
papier peint\footnote{\cite{beaudouin_cartographie_2017}, \cite{michez_gallicapix_2021}}). L'absence de
passage à l'échelle est significatif des difficultés d'un traitement et
d'un enrichissement systématique et standardisé de très grands volumes de données très
hétérogènes. Ainsi, chacun des projets reposait sur des méthodes et des
techniques différentes en raison de la nature des données explorées et
des finalités scientifiques propres à chaque projet.

Cependant des leçons ont été tirées~: les résultats découlent d'un
travail collectif et interdisciplinaire, et se sont appuyés sur des
méthodes standardisées, par exemple pour l'extraction des données et
métadonnées via des \apis\footcite[p.7]{bermes_patrimoine_2020}. Ces
premières applications de l'\ia ont démontré la nécessite de garantir la
reproductibilité des méthodes et d'adopter des normes pouvant mettre en
œuvre un cadre technique interopérable afin de faciliter la
collaboration à plusieurs échelles~: entre les projets de recherche d'un
part, et d'autre part entre les disciplines et les corps de métier
(notamment entre les chercheur.ses et les professionnels des
bibliothèques).

Le projet \gaga\footnote{https://gallicorpora.github.io/} --
bénéficiant de l'appui du DataLab de la \bnf -- se détache néanmoins dans
le paysage des projets estampillés \bnf, car il aspire à s'éloigner le
plus possible de son corpus, voulant concilier reproductibilité des
résultats sur des données diverses et applicabilité à un large corpus
issu des collections numérisées de la Bibliothèque Nationale. Il
illustre en outre les problématiques susmentionnées~: l'inscription dans
un dialogue interdisciplinaire et dans des pratiques normalisées.
L'ambition du projet porte sur le développement d'une chaîne de
traitement automatisée pour la transcription et l'annotation des documents textuels historiques, en diachronie longue, en
partant de leurs numérisations disponibles sur le portail Gallica,
créant ainsi des corpus enrichis, et facilitant leur exploitation et
leur valorisation. En effet les besoins des institution se déplacent de la transcription des textes à leur encodage sémantique automatique en \xml-\tei, afin d'offrir
utilisateur.rices de bibliothèques numériques de nouvelles options pour la
fouille de données. Le projet \gaga s'inscrit dans cette
dynamique, en exploitant le riche corpus de la \bnf, qui met à disposition 193 265
manuscrits, dont 52 188 précédent 1800, et 1 182 471 livres imprimés,
dont 160 335 précédent 1800\footcite{sagot_gallicorpor_2022}.

\begin{kwote}  
``Le nouveau défi à relever aujourd'hui est de transformer ces
numérisations en des ressources enrichies, qui augmentent le texte
extrait et repérable avec de la métadonnée et de l'analyse. Le texte
brut et non annoté ne suffit plus pour la recherche en informatique
appliquée aux documents historiques. De là vient l'impulsion pour le
projet \gaga. Le projet envisage la mise en place d'un pipeline
qui saisit un document numérisé depuis le portail Gallica et renvoie une
ressource numérique très enrichie. En plus d'une description du texte
repérable, la ressource présentera les données structurelles portant sur
la mise en page, ainsi qu'une analyse linguistique du texte extrait et
des métadonnées portant sur le document physique et le fac-similé
numérique''.\footcite{christensen_gallicorpor_2022}.
\end{kwote}

En outre, le but sous-jacent du projet est de produire un prototype qui
pourrait servir d'exemple de chaîne d'acquisition
numérique pour les institutions
patrimoniales. Par conséquent, ce projet se propose aussi d'être une
preuve de concept d'un \emph{modus operandi} pour l'extraction et
l'annotation de textes très divers, créant une sorte de pipeline ultime. Mais avant tout, la chaîne de traitement est destinée à être applicable
à un large corpus de documents mis à disposition par la \bnf. Les
documents du corpus visé proviennent de différentes époques (du \textsc{xv}\ieme au
\textsc{xviii}\ieme siècle) et présentent une grande diversité de mises en page, de
langues (ancien français, moyen français, français classique), et de
supports (manuscrits, imprimés). L'hétérogénéité des sources est censée
faire preuve de faisabilité, et vérifier le potentiel de la méthode
élaborée. Elle vise à prouver que le concept d'une chaîne de traitement
généraliste peut être concrètement appliquée.

\gaga expose ainsi la plupart des questionnements et défis
tenant au montage d'une chaîne de traitement unifiée applicable à une
grande diversité de données. Les choix technologiques, tels que
l'adoption de formats ouverts et interopérables, la prise en compte de
la diversité des modes d'acquisition des données et la spécialisation
des modèles d'\ia, sont au cœur de ces enjeux. Par ailleurs, la question
de l'ouverture des corpus annotés, essentielle pour l'apprentissage
machine, est également un axe d'analyse à considérer.

                \hypertarget{formats-standards}{\section{%
                Formats standards}\label{formats-standards}}
                    Le module de base est un package pour la gestion documentaire, duquel
l'application ne peut se détacher. Celui-ci inclut tout d'abord des
formulaires pour l'intégration des documents dans la base de données. Le
modèle de données permet de décrire différentes entités qui, bien que
liées dans leurs métadonnées, peuvent être intégrées indépendamment. Le
module de base permet également la création de \mans \iiif pour
chaque numérisation, permettant ensuite la visualisation des documents
grâce aux outils open-source dédiés. De ce fait, l'indexation de zones
d'image peut être réalisée manuellement via l'interface Mirador intégrée
à \sas. Ce noyau fonctionnel inclut en outre la sélection de lots de
documents (le ``panier''), sur lesquels pourront être effectués des
traitements groupés paramétrables.

Les briques fondamentales offrent donc les fonctionnalités essentielles
de gestion documentaire (intégration, modèle de données, \iiif). Les
traitements, quant à eux, sont gérés par des modules séparés, et c'est
sur cette structure que repose la modularité et l'évolutivité de
l'application.

Ci-après nous donnons une description détaillée de certaines de ces
fonctionnalités de base.

\hypertarget{description-des-donnees}{%
\subsection{Description des
données}\label{description-des-donnees}}

Le module de base contient un modèle de données suffisamment extensif
pour décrire efficacement une diversité de données, allant de documents
textuels historiques à des tableaux en histoire de l'art. La
tripartition entre témoin (\wit), série (qui contient un ensemble de
témoins), et contenu permet un alignement avec des corpus très
diversifiés et des données potentiellement hétéroclites, telles que des
manuscrits, des documents épistolaires, des inventaires de galeries
d'art, et même pourquoi pas des cartes\ldots{}

Pour ouvrir à cette large diversité de données, la liste des types de
pagination témoin doit être étendue \emph{a minima} d'un nouveau type
``other'', émancipant l'enregistrement des mentions de pagination. Les
développements futurs prévoient aussi la création d'un système pour
ajouter facilement un nouveau type\footnote{Le type de témoin est une
  métadonnée rentrée par l'utilisateur.rice lors de l'enregistrement du
  \wit dans la base de donnée. Il choisit le type dans une liste,
  originellement manuscrit, imprimé ou gravure sur bois.} de \wit
(tel que peinture, catalogue, etc.).

Au fil des développements, des débats ont émergé autour de l'ajout dans
le modèle de données d'un niveau de granularité supplémentaire pour
décrire des images ou zones d'images unitaires
(\graphicals), créant ainsi une entité détachée du fait
qu'elle provienne d'une extraction dans un document. Cette solution
aurait permis une description plus détaillée et plus fine des images,
importante pour des projets axés sur des images uniques, et aurait
favorisé un élargissement du spectre des type de sources pris en charge.
L'utilisateur.rice aurait pu soit importer une image unique (et de manière
optionnelle, la lier à un \wit) via un formulaire, soit sélectionner
une région d'image d'intérêt au sein des extractions (annotations \sas),
laquelle serait enregistrée comme \graphical, puis l'enrichir de
métadonnées. Dans les deux cas l'enregistrement d'un \graphical
aurait donné lieu à la création d'une \digit au format \jpeg.

Sans l'unité de description \graphical, les régions d'images
sont créées uniquement via les annotations \sas.

L'intégration de cette entité au sein du modèle aurait offert plusieurs
avantages en termes de cohérence et de flexibilité. En s'alignant sur
les structures existantes (\wits et \sers), elle aurait permis une
manipulation plus intuitive des images, facilitant ainsi les opérations
de recherche et la création de \emph{Sets} personnalisés. De plus, elle
aurait rationalisé la gestion des annotations \sas, permettant de
sélectionner les plus pertinentes dans la multitude existante.

Cependant, cette approche présente des limites, et on peut trouver des
alternatives. Tout d'abord, la coexistence de \graphicals avec les
annotations \sas, générées par des processus distincts, aurait pu créer
une certaine confusion quant à leur nature et à leur méthode de
création. De plus, la multiplication potentielle de milliers
d'enregistrements aurait pu impacter les performances de la base de
données et complexifier les requêtes. Enfin, le lien sémantique ambigu
et sujet à interprétation subjective entre \graphical et \wit
aurait compliqué les possibilités de corrélation.

Compte tenu de ces limites, il a semblé préférable de maintenir les
annotations \sas pour identifier les instances de base du modèle, sans
créer de nouvelle unité de description. La solution actuelle reste donc
basée sur la création manuelle ou automatique de zones dans les images
via \iiif et \sas, évitant les problèmes de redondance et de confusion.
Bien que l'entité \graphical n'ait pas été implémentée, les
fonctionnalités d'annotation et de sélection d'images sont assurées par
d'autres mécanismes. L'outil Mirador permet d'associer des tags aux
zones d'image, offrant ainsi une première couche d'enrichissement
sémantique. La sélection dans un \emph{set} personnalisé sera possible en
gardant en mémoire une référence contenant des coordonnées du
\emph{crop}. De plus, l'importation d'images individuelles est
réalisable en les considérant comme des \emph{Witness partiels}, ce qui
permet de les intégrer dans le \textit{workflow} existant. Toutefois
l'enrichissement sémantique à un niveau de granularité fin restera
limité~; et la dépendance à l'outil \sas constitue une potentielle dette
technique, susceptible de restreindre les évolutions futures du système.

Afin de mieux répondre aux exigences de modularité, l'évolution du
modèle de données s'oriente non pas vers une description individuelle
des documents, mais vers la gestion des traitements. Cette évolution
implique la création d'une entité \tr
liée à des ensembles de données (\ds et
\rs) potentiellement hétérogènes.

\hypertarget{principe-du-traitement}{%
\subsection{Principe du Traitement}\label{principe-du-traitement}}

Le but fondamental de la plateforme est de pouvoir effectuer plusieurs
actions sur les objets de la base. Afin d'assurer une meilleure
traçabilité et plus de flexibilité, la plateforme abandonne les
lancements automatiques des processus\footnote{C'était initialement le
  cas de l'extraction des entités, dont le lancement était lié à une
  méthode de classe liée à la \digit après soumission d'un
  formulaire d'ajout d'un \wit ou d'une \ser. L'action se
  lançait immédiatement après enregistrement des images d'une
  numérisation dans la plateforme.} au profit d'un système basé sur
l'entité \tr. Chaque traitement est associé à un ensemble
d'objets traités ensemble (\ds ou \rs), à un jeu de
paramètres et à un résultat. Ces informations sont stockées dans une
table dédiée. Cette approche facilite la gestion et le trackage des
processus (notamment, les utilisateur.rices sont notifiés par e-mail à la fin
du \textit{processing}), permet aux utilisateur.rices de consulter un historique de
leurs actions et offre la possibilité de créer des \textit{workflows}
personnalisés en passant par un formulaire de lancement unique mais
extensif.

En permettant de regrouper des documents de types différents (\wos,
\sers, \wits) dans des \dss, on offre à
l'utilisateur.rice la flexibilité de lancer des actions sur des ensembles
d'entités hétérogènes et granulaires. Le traitement est ensuite réparti
sur les entités de niveau inférieur (les témoins). Les \wits ainsi
sélectionnés peuvent être soumis à une large gamme de traitements~: des
fonctions déjà implémentées comme l'exportation (avec choix du
format), l'extraction, la vectorisation, la recherche de similarité~; ou
de nouveaux traitements personnalisés, tels que la visualisation sur une
frise chronologique ou une carte. La modularité de la plateforme est
assurée par un formulaire de lancement configurable, permettant de
l'adapter à différents scénarios d'utilisation, et à l'ajout de modules
personnalisés.

Le \rs fonctionne similairement au \ds, à un niveau
de granularité inférieur (à l'échelle de la zone d'image)\footnote{À
  l'été 2024, l'entité n'existe pas encore dans la base de données, mais
  le processus d'envoi du traitement et les modes de communication entre
  l'application et l'\api prévoient la possibilité de lancer l'inférence
  des modèles sur un ensemble de régions extraites.}.

\hypertarget{extraction-des-zones-dimage-manuelle}{%
\subsection{Extraction manuelle des zones d'image}\label{extraction-des-zones-dimage-manuelle}}

Le choix de la méthode d'extraction des régions d'intérêt dans les
documents constitue un élément clé de la modularité de la plateforme.
Les utilisateur.rices peuvent opter pour une extraction manuelle ou une
extraction automatique basée sur des algorithmes de vision par
ordinateur, adaptée aux traitements à plus grande échelle.

Après importation d'un enregistrement, le flux de travail procède à la
création de \mans \iiif pour chaque numérisation
(\digit) afin de permettre une visualisation grâce à la
plateforme Mirador. Le module de base autorise par la suite
l'extraction manuelle de zones d'intérêt au sein des images. Cette
fonctionnalité est particulièrement utile pour les projets ne souhaitant
pas recourir à des méthodes entièrement automatisées de vision par
ordinateur. L'outil \sas permet de créer des annotations, c'est-à-dire de
définir des régions d'intérêt spécifiques dans les numérisations, et de
les indexer directement dans les \mans \iiif correspondants,
enrichissant ainsi les ressources numériques. De plus, les
développements futurs prévoient la possibilité d'importer des fichiers
d'annotation préexistants en format .\textsc{txt} afin de pouvoir les indexer
manuellement. Par conséquent, le \textit{workflow} de base ne comporte aucun
traitement automatique basé sur la vision (et de fait éventuellement
trop gourmand en puissance de calcul).

L'extraction, qu'elle soit manuelle ou automatique, constitue le
fondement du reste des processus. Une interface est disponible pour
sélectionner un ensemble de documents et effectuer des actions
spécifiques sur les témoins annotés, via le formulaire de traitement qui
s'étend selon un choix de module configuré. Ainsi l'utilisateur.rice n'est
pas limité par un contexte initial, à l'origine deux étapes
indissociables et incontournables (importation et extraction), pour
pouvoir effectuer d'autres actions. Cette modularité permet de
s'affranchir d'un \textit{workflow} linéaire et prédéfini, offrant ainsi une plus
grande adaptabilité aux besoins spécifiques et aux ressources
matérielles des projets.

Pour conclure, l'existence de ce module de base répond à des besoins
élémentaires des projets de recherche en études visuelles. Il fournit un
outil qui permet d'agréger toutes les sources primaires qui concernent
le sujet, de décrire les sources et de les mettre en relation. Il offre
en outre la possibilité d'extraire et visualiser des contenus d'intérêt
(les ``crops'' d'images), ciblant ainsi les instances de base qui
intéressent les chercheur.ses.
                    
                 \hypertarget{specialisation-modeles}{%
                \section{La spécialisation des modèles}\label{specialisation-modeles}}
                La construction d'une plateforme extensive et modulaire pour
démocratiser l'accès à un outil de gestion et de traitement de la donnée
visuelle implique une réflexion approfondie sur les architectures
matérielles et logicielles. Pour toucher des publics diversifiés de la communauté de la recherche, il est
essentiel de penser des infrastructures matérielles diverses, plus ou
moins puissantes et abordables, intégrant ou non des composants comme
les \gpu (qui permettent d'accélérer les calculs intensifs nécessaires à
l'\ia). En effet, la gestion efficace des ressources, la scalabilité,
et la performance sont des aspects à prendre en compte pour
que ces outils puissent être utilisés de manière fiable. Parallèlement, l'architecture logicielle doit être flexible et
évolutive. 

\hypertarget{hardware-une-api-sur-le-gpu}{%
\subsection{Hardware~: une API sur le GPU}\label{hardware-une-api-sur-le-gpu}}

La séparation physique de l'inférence des modèles tient un rôle
important dans l'ouverture de la plateforme.

Le type d'infrastructure de calcul, notamment le \cpu ou le \gpu,
implique des différences dans le traitement et l'analyse des données,
chacun offrant des capacités distinctes adaptées à des besoins
spécifiques. Un \cpu (Central Processing Unit) est le processeur
principal d'un ordinateur, conçu pour gérer une large gamme de tâches
générales et basiques, et utilisé pour les besoins quotidiens. Un \gpu
(Graphics Processing Unit) est spécialisé dans le
traitement des éléments graphiques. Il est conçu pour effectuer un grand
nombre de calculs simples en parallèle grâce à ses nombreux cœurs, ce
qui le rend extrêmement efficace pour des tâches nécessitant un
traitement massif et simultané de données. L'utilisation d'un \gpu est
souvent nécessaire pour les tâches d'\ia, notamment en vision
artificielle, car ces tâches impliquent souvent des opérations de calcul
intensives et parallélisables. Un \gpu, avec sa capacité à gérer des
milliers de \textit{threads} en parallèle, permet d'accélérer l'entraînement
et l'inférence des modèles de vision artificielle, rendant le traitement
plus rapide et plus efficace que sur \cpu.

Discover-Demo est une \api développée comme un module de l'application, répondant au besoin de séparer les algorithmes de vision du reste
de l'application. Cette séparation permet une plus grande flexibilité
dans l'utilisation des ressources de calcul. Elle tourne sur le \gpu
Dishas-ia, dédié quasi exclusivement aux besoins de l'équipe d'histoire
des sciences du \syrte.

          \begin{figure}[H]
	\begin{center}
		\includegraphics[height=7cm]{figues/com_hard_ware.png}
	\end{center}
	\caption{Organisation et communication des infrastructures.}
	\label{fig:com} \end{figure}

Malgré l'importance accordée à l'\ia,
l'interface web et l'\api associée sont conçues pour une analyse complète
des documents historiques, allant de leur importation et stockage à
leurs traitements (divers) et visualisations. Dans une optique
d'extensivité, elles ne doivent être rattachées à aucun processus d'analyse
prédéterminée. Ainsi, toutes les étapes peuvent être
effectuées manuellement ou à l'aide d'algorithmes automatisés.
L'application de base n'intègre pas de traitement de \cv, mais permet de gérer une base de données et des sources avec
leurs numérisations, utilisant le standard \iiif. Elle permet
l'indexation manuelle de zones d'images dans \sas via l'interface Mirador,
permettant la sélection de zones d'images d'intérêt. Pour cela,
l'extraction automatique de zones d'images est séparée en un nouveau
module, mais les fonctionnalités de base de l'application incluent
toujours les outils nécessaires pour effectuer des annotations manuelles
de régions. Cela comprend toutes les fonctions pour indexer un fichier
texte dans \sas, visualiser les régions annotées, et exporter les
résultats. Il devient alors envisageable d'importer des résultats de traitement
(fichiers d'annotation de régions ou de paires de régions similaires) et
de les indexer manuellement pour permettre leur visualisation et analyse
ultérieure.

Chaque traitement peut donc être
réalisé via l'inférence des modèles de vision sur \gpu (comme c'est le
cas pour \eida grâce à l'\api), par l'import d'un fichier de résultats,
manuellement, ou potentiellement par des méthodes locales sur \cpu
(\yolov, par exemple, est assez léger pour tourner en local). La
plateforme permet ainsi une adaptation à des environnements matériels
divers, laissant la possibilité de réaliser les traitements soit
automatiquement via l'\ia, soit manuellement.

Séparer les modèles de vision c'est aussi permettre une bascule vers des
modèles spécialisés. Les modèle développés dans le cadre du projet sont
disponibles mais peuvent facilement être réentraînés pour correspondre
spécifiquement aux données de l'utilisateur.rice, prenant en compte les
besoins de sa recherche.

Pour conclure, grâce à cette séparation des composants \textit{hard-ware}, la
plateforme répond efficacement à une diversité de besoins et permet son
intégration dans des projets aux ressources matérielles variées. Même
sans ressource matérielle capable de faire tourner les modèles de
vision, les utilisateur.rices peuvent toujours exploiter la plateforme web.
Cette conception offre un accès aux outils et méthode à des utilisateur.rices
divers, allant des projets sans ingénieur dédié pour le
développement, aux équipes de recherche disposant de leurs propres
ingénieurs, en passant par des doctorants indépendants ayant des
compétences en programmation mais sans accès à un serveur. L'outil est
pensé pour s'adapter à des environnements variés, des configurations
légères fonctionnant en local, jusqu'à des projets disposant de
ressources matérielles importantes comme un \gpu.

\hypertarget{software-des-modules-separes}{%
\subsection{Software~: des modules
séparés}\label{software-des-modules-separes}}

Le modèle MVC (Model-View-Controler) est une architecture logicielle qui
segmente une application en trois composantes interconnectées. Le Modèle
est chargé de la gestion des données et de la logique métier de
l'application, assurant la manipulation et l'administration des
informations. La Vue est responsable de la présentation visuelle des
données, les mettant en forme visuellement dans un \textit{template}. Le
'Contrôleur', sert d'intermédiaire entre le 'Modèle' et la 'Vue'~: il reçoit
les entrées de l'utilisateur.rice via la 'Vue', traite ces entrées, puis
interagit avec le 'Modèle' pour actualiser les données et, enfin, met à
jour ces modifications dans la Vue. Cette séparation des préoccupations
permet une organisation plus rigoureuse du code, facilitant ainsi la
maintenance, la réutilisabilité et le développement parallèle de chaque
composante.

Le cycle action → mise à jour → affichage induit par ce patron est bien
adapté aux applications web, il est à ce titre utilisé par nombre
d'entre elles, dont \eida fait partie, et par de nombreux \textit{frameworks},
Django y compris.

Bien que le modèle MVC offre déjà une structure prenant en compte la
séparation des préoccupation, \eida cherche à aller au-delà, proposant
une architecture encore plus flexible. La plateforme est conçue pour
permettre aux développeur.ses d'ajouter ou de supprimer des fonctionnalités
de manière indépendante. Cette approche permet de personnaliser
l'application en fonction des besoins spécifiques de chaque projet, sans
avoir à modifier le cœur du système. Les utilisateur.rices peuvent ainsi
partir de la base de la plateforme et la compléter avec des modules sur
mesure.

Voici une transcription de l'arborescence des fichiers de l'application
:

\begin{verbatim}
app/
├── config/
├── logs/
├── mediafiles/
├── regions/
├── similarity/
├── vectorization/
│   ├── templates/
│   ├── __init__.py
│   ├── const.py
│   ├── tasks.py
│   ├── urls.py
│   ├── utils.py
│   ├── views.py
├── webapp/
├── webpack/
├── __init__.py
├── manage.py
├── requirements-base.txt
├── requirements-dev.txt
├── requirements-prod.txt
cantaloupe/
celery/
docs/
gunicorn/
sas/
scripts/
├── .gitignore
├── .pre-commit-config.yaml
├── README.md
├── run.sh
\end{verbatim}

Il a été créées plusieurs unités fonctionnelles pouvant inclure leurs vues, \textit{templates}, utilitaires, etc. Cette approche
permet aux développeur.ses de découper l'application tout en factorisant le code dédié à
plusieurs tâches, ainsi les modules partagent des \textit{statics}, un fichier
de configuration global et des fonctions utilitaires.

Chaque sous-dossier dans \texttt{app/} représente un module fonctionnel. Le
répertoire \texttt{webapp/} contient le module de base, tandis que \texttt{webpack/} est
dédié aux interfaces\footnote{Voir le \hyperlink{chapitre-8-interfaces}{chapitre suivant}}. Les modules
additionnels, autonomes, peuvent s'interfacer les uns avec les autres,
et être développés puis testés de indépendamment. Pour un exemple
détaillé du contenu des fichiers, une description du module dédié à
la vectorisation développée pendant se trouve en annexe \ref{module_vecto}. 

La variable \texttt{INSTALLED\_APPS} du fichier de configuration global permet de
personnaliser l'application en activant ou désactivant les modules
souhaités.

\emph{Modularité}

Cette architecture s'inscrit dans une stratégie de développement
applicatif ouverte et évolutive. La division en unités fonctionnelles et
indépendantes permet d'ajouter ou de supprimer les fonctionnalité
complémentaires au module de base. Ce cadre de développement modulaire
garantit que l'application reste adaptable aux exigences évolutives des
chercheur.ses et des institutions partenaires, la rendant plus robuste et
tolérante aux usages extérieurs et autorisant alors le réemploi du code
par des projets ayant besoin d'effectuer des traitements divers sur
du matériel documentaire pictural numérisé. \aikon est utilisable par des
projets divers, ce qui réduit le temps de développement et ouvre la voie
à des partenariats, aidant alors à pérenniser les outils.

\emph{Maintenance}

L'organisation modulaire du code facilite également la maintenance et
les mises à jour. Avec une telle structure, il devient plus facile
d'isoler les composants pour le développement et le débogage. Les
développeur.ses peuvent travailler sur un module spécifique sans interférer
avec les autres parties du projet. De plus,
l'implémentation d'un module indépendant provoquera moins de conflit
lors du déploiement.
                    
                \hypertarget{normalisation-donnees}{\section{%
                La normalisation des données d'annotation}\label{normalisation-donnees}}
                    Spécialiser un modèle d'intelligence artificielle implique de lui
fournir des données pertinentes, diversifiées, et en quantité
suffisante. Cependant, pour certains domaines, dont l'histoire fait
partie, le volume de données disponible est insuffisant. Ce constat est
d'autant plus vrai dans le cas des diagrammes issus de traités
astronomiques~: les corpus de documents scientifiques historiques
contiennent généralement du texte en majeure partie, des tables et des
images, négligeant souvent les diagrammes.\footnote{Exception faite du
  corpus S-VED (\cite{buttner_cordeep_2022}), collection
  d'illustration très diverses contenant entre autre des diagrammes
  historiques. Mais les primitives ne sont pas annotées.}. De plus, ils
sont dénués d'annotations précises sur les éléments constitutifs des
pages~; c'est sans parler de l'inexistence d'un corpus de diagrammes
dont les primitives sont annotées. Or l'annotation est une tâche
chronophage et fastidieuse. Le recours aux données synthétique répond,
mais en partie seulement, à ces problématiques.

\hypertarget{datasets-synthetiques}{%
\subsection{\emph{datasets} synthétiques}\label{datasets-synthetiques}}

Les \textit{datasets} synthétiques sont générés par des algorithmes ou des
méthodes de simulation pour imiter des données réelles, sans être
directement extraites de sources existantes. De tels jeux de données
sont utilisés lorsque les données réelles sont limitées ou difficiles à
obtenir, mais qu'il est cependant nécessaire de contrôler spécifiquement
les caractéristiques des données d'entraînement\footcite{buttner_cordeep_2022}. La génération
d'images a pour but de fabriquer des ensembles de données plus vastes,
plus diversifiés, très variables et assez complexes, répondant aux
caractéristiques des objets d'intérêt du projet, et surtout étiquetés
automatiquement, sans recourir à l'annotation manuelle.

Ces données synthétiques sont assez ressemblantes et complexes pour être
exploitées. Par exemple, docExtractor est un modèle off-the-shell (au
même titre que \yolo) envisagé dans le cadre de la tâche d'extraction des
diagrammes, et qui se veut sépcifique aux données historiques, car il
est entraîné sur des données produites par un générateur de documents
historiques synthétiques~: SynDoc\footcite{monnier_docextractor_2020}. SynDoc
génère des images de manière aléatoire en combinant des éléments
graphiques (fonds, images, texte et bruit) provenant d'un jeu d'image
défini (constitué de 177 images de pages, 15 contextes, plus de 8000
œuvres d'art provenant de WikiArt, des lettrines générées à partir d'une
lettre aléatoire avec 91 fonts possibles, et des dessins, schémas et
textes tirés d'articles aléatoires sur Wikipedia, avec plus de 400
fonts). Les différents éléments s'agencent, intégrant sur le fond
images, texte et bruit, offrant des combinasons et des mises en pages
assez complexes. Chaque élément de contenu est pré-annoté, éliminant
ainsi le besoin d'annotations manuelles pour ces pages.

          \begin{figure}[H]
          \begin{center}
          \includegraphics[height=6.5cm]{figues/syndoc.jpg}
          \end{center}
          \caption{Données synthétiques générées par SynDoc.\footcite[p.46]{norindr_traitement_2023}}
          \label{fig:syndoc} \end{figure}

Pour entraîner le modèle de vectorisation, il a de même été nécessaire
d'utiliser des données synthétiques. Parce qu'annoter les primitives
géométriques dans des images de diagrammes complexes est très
chronophage, le modèle de vectorisation a été pré-formé sur des corpus
artificiels générés dynamiquement. Le script de génération des données
d'entraînement choisit aléatoirement un arrière-plan, y ajoute des mots,
des nombres et des glyphes puis crée artificiellement un diagramme en
insérant des segments, des cercles et des arcs. Le script est conçu pour
que ces diagrammes aient une forte probabilité de présenter des formes
très caractéristiques comme les cercles concentriques et tangents, les
lignes parallèles et les arcs connectés, afin de simuler les structures
typiques. Les primitives sont dessinées avec des
variations aléatoires d'opacité, de largeur et de couleur. Les cercles
peuvent être remplis ou vides. Enfin, du bruit est ajouté en appliquant
un flou gaussien, et en supprimant de petites régions du diagramme pour
imiter la dégradation des documents historiques. Les données
d'entraînement ainsi générées présentent des configurations assez
complexes.

          \begin{figure}[H]
          \begin{center}
          \includegraphics[height=7cm]{figues/vecto_synthetic_data.png}
          \end{center}
          \caption{Données synthétiques générées pour l'entraînement du modèle de vectorisation.\footcite[Figure issue de la présentation de Syrine Kalelli à l'occasion de la conférence \eida 2024~:][]{noauthor_eida_nodate-1}}
          \label{fig:vecto_synthetic} \end{figure}

Enfin, le modèle de similarité présente un troisième exemple, puisque
SegSwap est pré-entraîné sur de la donnée synthétique. Le script de
génération prend des parties aléatoires d'une images et les copie-colle
au-dessus d'une autre image. Les trois images (source, cible et
superposition) sont placées dans le même dataset d'entraînement, ainsi
le modèle apprend à retrouver ce qui, dans la superposition, vient de la
source, et ce qui vient de la cible.

          \begin{figure}[H]
          \begin{center}
          \includegraphics[height=3cm]{figues/segswap_blended_images.png}
          \end{center}
          \caption{Données d'entraînement du modèle Segswap.}
          \label{fig:segswap} \end{figure}

\hypertarget{les-donnees-reelles}{%
\subsection{Les données réelles}\label{les-donnees-reelles}}

S'appuyer sur les modèles \textit{off-the-shelf}, sur de larges \textit{datasets}
généralistes, ou sur des données synthétiques permet une implémentation
facilitée de la vision dans des projets et constitue une base solide.
Toutefois, les sources tenant aux deux projets (\vhs et \eida) sont trop
spécifiques pour se contenter de modèles généralistes ou formés sur des
données artificielles. Même si ces derniers peuvent offrir des performances
de base, ils risquent de manquer de précision et de sensibilité aux
particularités des documents historiques. Les corpus artificiels
présentent des configurations délibérément complexes pour s'approcher le
plus possible des difficultés que le modèle pourrrait rencontrer sur les
données réelle. Elles sont cependant irréalistes et insuffisantes pour
permettre aux modèles de généraliser sur des diagrammes réels.

En atteste la comparaison des performances de docExtractor et \yolov sur
les données d'\eida. docExtractor\footcite{monnier_docextractor_2020}, entraîné
sur des données synthétiques mimant les documents historiques serait en
théorie plus adapté au traitement d'images de pages de manuscrits, avec
du texte et des illustration côté à côte, d'autant qu'il intègre des
outils de traitement du texte (notamment pour la segmentation des
lignes)\footnote{\eida envisage l'implémentation d'un outil d'extraction
  et transcription des labels et des textes qui entourent les diagrammes}.
Pourtant, sans fine-tuning sur des données réelles, il présente des
performances équivalentes à celles de \yolov\footcite[p.45]{norindr_traitement_2023}. Cela
souligne que même les modèles off-the-shelf entraînés sur un corpus
assez spécifique et complexe, mais synthétique, ne dispense pas d'un
entraînement sur des données réelles, au même titre que les modèles très
généralistes comme \yolov.

Alors, le modèle de base \yolov tel que mis à disposition par
Ultralytics est entraîné sur de grands ensembles de données réelles, ce
qui constitue une base solide pour la classification des objets du
monde. L'utilisation de SynDoc permet ensuite de compléter
l'apprentissage initial en exposant le modèle à des exemples variés et
spécifiques aux documents historiques, augmentant ainsi sa capacité de
généralisation. Ces similis de manuscrits anciens offrent l'avantage de
pouvoir être produits en grandes quantités et de couvrir un large
éventail de scénarii et de configurations difficiles à obtenir dans des
ensembles de données réelles. Puis le modèle est entraîné sur les
données de \vhs, qui sont de réelles pages de documents historiques
contenant une large diversité d'illustrations. Ces données apporteront
une dimension supplémentaire de pertinence au modèle, en l'exposant à
des particularités des documents historiques réalistes. Enfin, \yolov
est entraîné sur les données d'\eida, qui sont orientées spécifiquement
vers les diagrammes, afin qu'il détecte uniquement ces derniers.

Quant au modèle de vectorisation développé par Syrine
Kalleli\footcite{kalleli_historical_2024}, il est formé
sur des données synthétiques générées à la volée par un script. Mais le
corpus de diagrammes d'\eida est particulièrement caractéristique et le
modèle n'aurait pu être optimal sans avoir appris sur des images de
diagrammes issus de manuscrits réels. Un corpus d'entraînement de 303
diagrammes extraits de manuscrits et de gravures a donc été constitué et
annoté par les historien.nes. Ces diagrammes sont issus de sources latines, arabes,
grecques, hébreuses ou chinoises, datant du \textsc{xii}\ieme au \textsc{xviii}\ieme siècle, et ils
présentent en guise d'étiquettes plus de 3000 lignes, cercles et arcs. Le
ré-entraînement a permis le transfert des connaissances acquises sur la
tâche de détection des primitives sur les données réalistes.

Il sera également possible d'obtenir des meilleurs résultats sur la
similarité grâce à une évaluation des scores (qui constitue un jeu de
données annotées) et le ré-entraînement du modèle, pour donner des
résultats plus adaptés à la spécificité des données historiques.

D'ailleurs, cette étape d'annotation (le choix des exemples et des
étiquettes) revêt des enjeux importants. L'apprentissage spécifique se
fait à partir de données sélectionnées par les chercheur.ses~: les exemples
sur lequel l'algorithme d'apprentissage va itérer définissent le modèle.
Il est nécessaire de constituer un échantillon de données aléatoire et
représentatif, et de l'annoter en fonction de ce que l'on souhaite
obtenir en prédiction.

L'annotation des jeux de données est non seulement une étape clé, mais
aussi un bel exemple de collaboration chercheur.ses-ingénieur.es. Elle
nécessite la définition de normes pertinentes et rigoureuses. Travail
minutieux et chronophage, l'étiquetage des données peut engendrer des
erreurs et du bruit dans les données, car elle implique la subjectivité
des chercheur.ses et le regard parfois trop précis sur les sources desquels
les annotateurs sont experts.

Voici un exemple rencontré lors de la préparation des données pour
entraîner un modèle de segmentation du contenu textuel. Les sources
arabes et chinoises sont particulièrement verbeuses et les diagrammes
sont très souvent entourés des blocs de commentaires se mélangeant alors
aux légendes et aux labels. Doit-on considérer ces commentaires comme
faisant partie des éléments que l'on souhaite identifier ou bien les ignorer
? Cette décision est importante car si on les ignore, le modèle risque
de passer à côté d'éléments textuels pertinents. En revanche, si on les
inclut, il ramènera des commentaires sans rapport direct avec le
diagramme observé. On voit ici comment la binarité des modèles, qui se
reflète dans les normes d'annotation, est problématique et constitue une
limite au \ml. Un compromis doit être trouvé entre
l'automatisation, qui requiert une normalisation, des définitions
claires et binaires, et la nuance dans l'interprétation des
sources\footnote{Dans le cadre du projet, il a toujors été plus
  intéressant d'opter pour une définition extensive des objets à
  détecter, car prévision d'une correction des traitement. Et il est
  plus facile de supprimer un élémént pas pertinent que d'aller en
  rechercher un, surtout compte tenu de la taille des corpus des
  chercheur.ses. Vaut aussi pour la préparation des données pour
  l'entraînement du modèle d'extraction.}.

La normalisation peut bénéficier à l'écosystème de recherche dans le
domaine de l'\htr et de l'\ocr. À ce titre, il est pertinent d'envisager
l'utilisation du vocabulaire contrôlé SegmOnto pour l'annotation du
contenu textuel entourant les diagrammes. Cela permettrait de créer des
jeux de données réutilisables, à partager avec des projets poursuivant
des objectifs similaires.\footnote{https://segmonto.github.io/}. Encore
une fois, un compromis doit être trouvé entre les besoins de description
des chercheur.ses et les possibilités offertes par les vocabulaires
contrôlés.

Un autre exemple concerne le dernier entraînement du modèle d'extraction
: les résultats montrent que des diagrammes sont encore détectés en
transparence. La question s'est alors posée de chercher à corriger ce
défaut en donnant au modèle, à l'occasion d'un nouvel entraînement,
d'avantage d'exemples négatifs (diagrammes visibles par transparence
mais non annotés). Or il est préférable de se contenter de la correction
ou suppression manuelle de ces prévisions erronées, garantissant que le
modèle parvienne à détecter les diagrammes presque effacés.

Pour assurer la rigueur et la cohérence des annotations, les décisions
prises entre les chercheur.ses et les ingénieur.es peuvent être l'objet d'une
documentation ou d'ateliers d'annotation.

\hypertarget{loeil-de-la-machine-avantages-et-limites}{%
\subsection{L'oeil de la machine~: avantages et
limites}\label{loeil-de-la-machine-avantages-et-limites}}

Bien qu'il soit possible d'optimiser les performances d'un modèle
d'apprentissage automatique en l'entraînant sur un ensemble de données
spécifique, son interprétation des données reste limitée car
fondamentalement binaire, ce qui le rend parfois déficient pour la
recherche en histoire. Ainsi, il gèrera difficilement les cas limites et
ambigüs. La décision d'inclure ou d'exclure ces cas particuliers de
l'ensemble d'entraînement implique un arbitrage délicat. D'un côté, une
inclusion trop restrictive peut compromettre les capacités de
généralisation du modèle, c'est-à-dire sa capacité à s'adapter à de
nouvelles données. À l'inverse, une inclusion trop permissive risque de
dégrader la précision du modèle sur les cas plus typiques. Les
chercheur.ses espérant obtenir un modèle maximaliste, quitte à accepter un
certain degré d'erreur et de devoir supprimer les faux
positifs, de nombreux cas limites ont été inclus. Le cas des diagrammes
visibles en transparence (expliqué précédemment) en est un exemple
éloquent.

Une autre difficulté réside dans la définition même du ``diagramme
astronomique''. Les limites de ce concept ne sont pas si claires et
définitives pour les chercheur.ses, et pourtant le modèle a besoin d'une
définition rigoureuse et cohérente. Il paraît en effet difficile de
considérer les diagrammes astronomiques en dehors du contexte des
pratiques d'autres sciences et disciplines connexes. Par exemple, Le
\emph{Flores Almagesti} -- réécriture de l'Almageste datant du \textsc{xv}\ieme par
l'astronome Giovanni Bianchini -- présente une partie algébrique à
l'ouverture mathématique, induisant la présence de nouveaux types de
diagrammes d'inspiration euclidienne. Pour retracer la source de ces
derniers, il est nécessaire de considérer les traités d'Euclide ou
autres travaux d'algèbre. Ceux-ci ne sont pas des traités
\emph{astronomiques}, bien qu'il ne soit pas certain que ces disinctions
contemporaines aient été aussi rigide à l'époque et aient eu un
quelconque sens pour les acteurs historiques. Les sources byzantines
confirment cette complexité~: les diagrammes y sont nommés
\emph{katagraphai}, indépendamment du domaine scientifique auquel ils
appartiennent. Également, de nombreux travaux astronomiques sont groupés
dans des témoins qui contiennent des œuvres issus de domaines divers.
C'est le cas avec les sources chinoises, comme le \emph{Chongzhen
lishu}, qui se présente généralement annexé d'une série de traités
mathématiques. Par conséquent, les diagrammes euclidiens ont été gardés
lors de la préparation des données, et l'algorithme de détection les
classe comme ``diagramme'', même s'ils ne constituent pas l'objet
principal des chercheur.ses.

En ce qui concerne les autres types de diagrammes non strictement
astronomiques (géométriques, harmoniques, logiques, illustrations de
constellations), une approche plus sélective a été adopté afin d'éviter
un modèle trop maximalistes. Ces éléments, bien que potentiellement
intéressants, n'ont pas été inclus dans la phase de détection
automatique.

Ainsi l'œil de la \cv contraint à des choix méthodologique
potentiellement inconfortables, mais en même temps il peut aider à
mesurer les impulsions des chercheur.ses, à mieux définir les objectifs de
recherche et à prioriser les éléments les plus pertinents. Ainsi, la
vision par ordinateur oblige les chercheur.ses à s'adapter à une logique
algorithmique qui, tout en limitant certaines interprétations
subjectives, offre l'opportunité de développer des modèles conceptuels et des méthodologies très rigoureuses.
                    

              


\vspace{2cm}


Les apports et les réflexions du projet \gaga reflètent la problématique globale de ce
mémoire~: comment concevoir des cadres pour l'environnement de la
recherche en Humanités Numériques, tout en répondant aux exigences
scientifiques de chaque projet. On voit se filer avec ce cas d'étude un écosystème complexe dans lequel s'agencent des systèmes et protocoles généraux et extensibles, adaptables aux besoins locaux, comme la \tei, ou SegmOnto, ou encore des \api pour la récupération des données. De nombreux défis restent cependant à relever, montrant que l'interopérabilité universelle est un idéal difficilement atteignable.

L'utilisation de l'\ia porte la question du partage des ressources à un niveau d'importance supérieur, puisque les modèles peuvent être spécialisés, et les corpus enrichis produits peuvent constituer des données d'entraînement. L'intervention du chercheur.se pour corriger les résultats des traitements automatiques est donc un aspect important à prendre en compte pour assurer la fiabilité scientifique de ses corpus annotés. 

\gaga visait à l'élaboration d'un
outil d'acquisition et d'enrichissement des données capable de se
détacher d'un corpus spécifique et sur ce point se voulait preuve de concept, mais le projet
n'a pas pleinement atteint ses objectifs, car la chaîne de traitement est restée très orientée vers les corpus de la \bnf. Il a ainsi mis en
évidence la complexité de concilier les exigences d'une infrastructure
générique et les besoins spécifiques des données, montrant que la modularité s'inscrit avant tout dans le temps. 

Comme le montre le cas de e-Scriptorium, une chaîne de traitement, si elle est assez généraliste pour tolérer une grande diversité de données, participe à la cohérence des pratiques et à la mutualisation des outils de la recherche. Les modalités d'accès (notebooks, plateforme, scripts etc.) impactent la prise en main et l'adoption des outils. 

Plus généralement, les recherches menées sur l'enrichissement et l'exploration de larges corpus grâce aux outils d'\ia ouvrent de nouvelles perspectives en terme de trouvabilité et de fouille de données visuelles ou textuelles. Cette valeur ajoutée est liée au passage du format \jpeg à des formats balisés et sémantiquement riche qui permettent des exploitations ou l'indexation. 
            
        \clearemptydoublepage
        
\chapter*{Conclusion partielle}
Comme le dit \citeauthor{jacquot_decrire_2017}\footcite{jacquot_decrire_2017}, face aux problématiques de traitement du volume, ``une réponse peut résider dans la généralisation des outils et
des méthodes ainsi que dans le partage des récits de projets de
recherche antérieurs, forme d'apprentissage et d'échange importante.''

Un trait saillant dans la progression vers l'automatisation est la
confrontation de deux besoins~: la collaboration et la spécialisation. D'un côté, la communauté scientifique voudrait mutualiser les outils et les méthodes, afin d'encourager le partage d'une expertise interdisciplinaire, des moyens financiers, et d'améliorer la maintenabilité, contribuant
ainsi à la robustesse des développements. Un tel outil, à l'image de la plateforme e-Scriptorium, qui porte et guide les processus de transcription de texte, et dont la portabilité facilite l'intégration dans divers écosystèmes de recherche, contribue aussi à la cohérence des méthodes scientifiques au sein d'une équipe de recherche, ou entre plusieurs projets. De l'autre côté, il faut prendre
en compte les spécificités locales des données, les besoins des
utilisateur.rices et les enjeux propres aux projets de recherche qui portent
la création des outils de traitement. Les solutions doivent être
adaptées aux particularités des corpus et aux questions de recherche
spécifiques, ce qui peut parfois entrer en conflit avec la nécessité de
généralisation. 

On aura voulu montrer dans cette première partie que l'ambition de créer un
outil générique -- visant à l'enrichissement et la sémantification de la
donnée -- se heurte à des dynamiques collectives et à la multiplicité des
acteurs, autant qu'elle en profite. La philosophie de l'ouverture, à la
fois des données et du code, est un atout (libre accès aux sources,
appui sur des scripts et des projets de recherche existants, rencontres
avec d'autres projets, création des synergies avec d'autres équipes ou
institutions et mutualisation des moyens financiers et humains), et une
double contrainte. La première est la nécessité d'assurer une
interopérabilité technique pour ouvrir en retour les corpus et garantir
la reproductibilité des résultats. La deuxième porte sur la prise en
compte de l'hétérogénéité des données, ce qui demande de trouver le
juste équilibre entre précision et flexibilité en terme de
description.

Nous sommes restés volontairement vagues sur la question des modèles de
vision artificielle et les réseaux de neurones~: ils sont pourtant la
clé de voûte des processus d'enrichissement et d'exploration des données. La construction et l'entraînement des réseaux de neurones s'inscrit aussi dans une balance
délicate entre généralisation et spécialisation. Les modèles doivent
être suffisamment généralistes pour reconnaître et interpréter une large
variété de motifs provenant de contextes divers, ce qui leur permet
d'être applicables à la diversité des scénarios rencontrés et de gérer
l'hétérogénéité des grands corpus~: par exemple la grande variété
orthographique, morphologique et de mise en page dans le cadre du projet
\gaga. Mais l'efficacité des modèles de vision exige aussi leur
spécialisation sur des jeux de données étroitement ciblés. Par
conséquent, les chercheur.ses et ingénieur.es doivent constamment équilibrer
ces deux exigences pour développer des modèles robustes, polyvalents,
mais aussi suffisamment précis pour répondre aux besoins spécifiques des
applications en recherche.

    \part{Mise en œuvre et exploitation de la \emph{Computer Vision}}

\chapter*{Introduction partielle}

Les diagrammes astronomiques offrent une perspective unique sur la
compréhension et la diffusion du savoir à travers les
siècles dans l'histoire afro-eurasienne. Des centaines de manuscrits
numérisés et des milliers de diagrammes sont disponibles. Nous avons
constaté dans la première partie de ce travail les dimensions et
l'hétérogénéité importantes caractérisant ce corpus de recherche, ainsi
que l'impossibilité de réaliser des annotations à grande échelle pour
chaque tâche spécifique. Pour assister les chercheur.ses dans le travail de
fouille et d'analyse des documents historiques, diverses méthodes
d'apprentissage profond (\dl) ont été développées et
implémentées sur la plateforme.

Il ne s'agit pas seulement d'implémenter des outils d'\ia efficaces,
mais aussi d'en exploiter les résultats. Pour \eida, la finalité est de
pouvoir produire des éditions scientifiquement satisfaisantes des
diagrammes.

Qu'est-ce que la vision artificielle~? Comment rendre les modèles
efficaces sur les sources historiques~? Notre propos s'articulera autour
de trois axes principaux~: le fonctionnement des modèles de vision, les
enjeux tenant aux jeux de données pour leur entraînement, et la
conceptualisation de l'outil d'édition qui repose sur les résultats des
algorithmes.

\clearemptydoublepage
    
        \hypertarget{chapitre-4-modele}{%
        \chapter{Les modèles de \emph{Computer Vision}}\label{chapitre-4-modele}}

            Dans des termes très simples, pour faire exécuter une tâche à la
machine, deux solutions existent. La première consiste à écrire un
programme, dont l'expert métier a explicité les règles. Le programme est
entièrement rédigé par un.e développeur.se. Il effectue une tâche précise,
chaque conjecture spécifique doit être prévue et son traitement
clairement formulé. Si le code produit des erreurs, il doit être modifié
par le.a développeur.se. La deuxième approche consiste à donner au programme
la capacité de se modifier lui-même, sans que cette modification soit
explicitement rédigée. L'expert métier doit coder l'architecture du modèle et
annoter des exemples ; puis un seul algorithme (d'apprentissage) suffit pour
traiter de multiples cas réels. En cas d'erreur de l'algorithme, il faut agir non plus
sur le programme mais sur les exemples d'apprentissage. L'objectif est
d'apprendre à généraliser pour prédire sur des exemples non vus pendant
l'apprentissage\footcite[p.7]{chollet_apprentissage_2020}.

\begin{kwote}         
``On peut ainsi opposer un programme \emph{classique}, qui utilise une
procédure et les données qu'il reçoit en entrée pour produire en sortie
des réponses, à un programme \emph{d'apprentissage automatique}, qui
utilise les données et les réponses afin de produire la procédure qui
permet d'obtenir les secondes à partir des premières.''\footcite[p.1-2]{azencott_introduction_2022}
\end{kwote} 

L'émergence de l'apprentissage machine a alors ouvert de nouvelles
perspectives en permettant de modéliser des interactions complexes entre
les données. Les modèles de \dl sont également plus
résilients face aux variations et au bruit présents dans les
données\footcite{juneja_deep_2023}. Ainsi,
l'application de l'apprentissage profond se révèle particulièrement
pertinente dans le cadre d'\eida, car elle permet de relever le défi
de la sémantification des images, y compris lorsqu'il s'agit de sources
historiques complexes.

        \hypertarget{les-reseaux-de-neurone}{%
        \section{Les réseaux de neurones, ou la généralisation prise au sens
        mathématique}\label{les-reseaux-de-neurone}}
         Le module de base est un package pour la gestion documentaire, duquel
l'application ne peut se détacher. Celui-ci inclut tout d'abord des
formulaires pour l'intégration des documents dans la base de données. Le
modèle de données permet de décrire différentes entités qui, bien que
liées dans leurs métadonnées, peuvent être intégrées indépendamment. Le
module de base permet également la création de \mans \iiif pour
chaque numérisation, permettant ensuite la visualisation des documents
grâce aux outils open-source dédiés. De ce fait, l'indexation de zones
d'image peut être réalisée manuellement via l'interface Mirador intégrée
à \sas. Ce noyau fonctionnel inclut en outre la sélection de lots de
documents (le ``panier''), sur lesquels pourront être effectués des
traitements groupés paramétrables.

Les briques fondamentales offrent donc les fonctionnalités essentielles
de gestion documentaire (intégration, modèle de données, \iiif). Les
traitements, quant à eux, sont gérés par des modules séparés, et c'est
sur cette structure que repose la modularité et l'évolutivité de
l'application.

Ci-après nous donnons une description détaillée de certaines de ces
fonctionnalités de base.

\hypertarget{description-des-donnees}{%
\subsection{Description des
données}\label{description-des-donnees}}

Le module de base contient un modèle de données suffisamment extensif
pour décrire efficacement une diversité de données, allant de documents
textuels historiques à des tableaux en histoire de l'art. La
tripartition entre témoin (\wit), série (qui contient un ensemble de
témoins), et contenu permet un alignement avec des corpus très
diversifiés et des données potentiellement hétéroclites, telles que des
manuscrits, des documents épistolaires, des inventaires de galeries
d'art, et même pourquoi pas des cartes\ldots{}

Pour ouvrir à cette large diversité de données, la liste des types de
pagination témoin doit être étendue \emph{a minima} d'un nouveau type
``other'', émancipant l'enregistrement des mentions de pagination. Les
développements futurs prévoient aussi la création d'un système pour
ajouter facilement un nouveau type\footnote{Le type de témoin est une
  métadonnée rentrée par l'utilisateur.rice lors de l'enregistrement du
  \wit dans la base de donnée. Il choisit le type dans une liste,
  originellement manuscrit, imprimé ou gravure sur bois.} de \wit
(tel que peinture, catalogue, etc.).

Au fil des développements, des débats ont émergé autour de l'ajout dans
le modèle de données d'un niveau de granularité supplémentaire pour
décrire des images ou zones d'images unitaires
(\graphicals), créant ainsi une entité détachée du fait
qu'elle provienne d'une extraction dans un document. Cette solution
aurait permis une description plus détaillée et plus fine des images,
importante pour des projets axés sur des images uniques, et aurait
favorisé un élargissement du spectre des type de sources pris en charge.
L'utilisateur.rice aurait pu soit importer une image unique (et de manière
optionnelle, la lier à un \wit) via un formulaire, soit sélectionner
une région d'image d'intérêt au sein des extractions (annotations \sas),
laquelle serait enregistrée comme \graphical, puis l'enrichir de
métadonnées. Dans les deux cas l'enregistrement d'un \graphical
aurait donné lieu à la création d'une \digit au format \jpeg.

Sans l'unité de description \graphical, les régions d'images
sont créées uniquement via les annotations \sas.

L'intégration de cette entité au sein du modèle aurait offert plusieurs
avantages en termes de cohérence et de flexibilité. En s'alignant sur
les structures existantes (\wits et \sers), elle aurait permis une
manipulation plus intuitive des images, facilitant ainsi les opérations
de recherche et la création de \emph{Sets} personnalisés. De plus, elle
aurait rationalisé la gestion des annotations \sas, permettant de
sélectionner les plus pertinentes dans la multitude existante.

Cependant, cette approche présente des limites, et on peut trouver des
alternatives. Tout d'abord, la coexistence de \graphicals avec les
annotations \sas, générées par des processus distincts, aurait pu créer
une certaine confusion quant à leur nature et à leur méthode de
création. De plus, la multiplication potentielle de milliers
d'enregistrements aurait pu impacter les performances de la base de
données et complexifier les requêtes. Enfin, le lien sémantique ambigu
et sujet à interprétation subjective entre \graphical et \wit
aurait compliqué les possibilités de corrélation.

Compte tenu de ces limites, il a semblé préférable de maintenir les
annotations \sas pour identifier les instances de base du modèle, sans
créer de nouvelle unité de description. La solution actuelle reste donc
basée sur la création manuelle ou automatique de zones dans les images
via \iiif et \sas, évitant les problèmes de redondance et de confusion.
Bien que l'entité \graphical n'ait pas été implémentée, les
fonctionnalités d'annotation et de sélection d'images sont assurées par
d'autres mécanismes. L'outil Mirador permet d'associer des tags aux
zones d'image, offrant ainsi une première couche d'enrichissement
sémantique. La sélection dans un \emph{set} personnalisé sera possible en
gardant en mémoire une référence contenant des coordonnées du
\emph{crop}. De plus, l'importation d'images individuelles est
réalisable en les considérant comme des \emph{Witness partiels}, ce qui
permet de les intégrer dans le \textit{workflow} existant. Toutefois
l'enrichissement sémantique à un niveau de granularité fin restera
limité~; et la dépendance à l'outil \sas constitue une potentielle dette
technique, susceptible de restreindre les évolutions futures du système.

Afin de mieux répondre aux exigences de modularité, l'évolution du
modèle de données s'oriente non pas vers une description individuelle
des documents, mais vers la gestion des traitements. Cette évolution
implique la création d'une entité \tr
liée à des ensembles de données (\ds et
\rs) potentiellement hétérogènes.

\hypertarget{principe-du-traitement}{%
\subsection{Principe du Traitement}\label{principe-du-traitement}}

Le but fondamental de la plateforme est de pouvoir effectuer plusieurs
actions sur les objets de la base. Afin d'assurer une meilleure
traçabilité et plus de flexibilité, la plateforme abandonne les
lancements automatiques des processus\footnote{C'était initialement le
  cas de l'extraction des entités, dont le lancement était lié à une
  méthode de classe liée à la \digit après soumission d'un
  formulaire d'ajout d'un \wit ou d'une \ser. L'action se
  lançait immédiatement après enregistrement des images d'une
  numérisation dans la plateforme.} au profit d'un système basé sur
l'entité \tr. Chaque traitement est associé à un ensemble
d'objets traités ensemble (\ds ou \rs), à un jeu de
paramètres et à un résultat. Ces informations sont stockées dans une
table dédiée. Cette approche facilite la gestion et le trackage des
processus (notamment, les utilisateur.rices sont notifiés par e-mail à la fin
du \textit{processing}), permet aux utilisateur.rices de consulter un historique de
leurs actions et offre la possibilité de créer des \textit{workflows}
personnalisés en passant par un formulaire de lancement unique mais
extensif.

En permettant de regrouper des documents de types différents (\wos,
\sers, \wits) dans des \dss, on offre à
l'utilisateur.rice la flexibilité de lancer des actions sur des ensembles
d'entités hétérogènes et granulaires. Le traitement est ensuite réparti
sur les entités de niveau inférieur (les témoins). Les \wits ainsi
sélectionnés peuvent être soumis à une large gamme de traitements~: des
fonctions déjà implémentées comme l'exportation (avec choix du
format), l'extraction, la vectorisation, la recherche de similarité~; ou
de nouveaux traitements personnalisés, tels que la visualisation sur une
frise chronologique ou une carte. La modularité de la plateforme est
assurée par un formulaire de lancement configurable, permettant de
l'adapter à différents scénarios d'utilisation, et à l'ajout de modules
personnalisés.

Le \rs fonctionne similairement au \ds, à un niveau
de granularité inférieur (à l'échelle de la zone d'image)\footnote{À
  l'été 2024, l'entité n'existe pas encore dans la base de données, mais
  le processus d'envoi du traitement et les modes de communication entre
  l'application et l'\api prévoient la possibilité de lancer l'inférence
  des modèles sur un ensemble de régions extraites.}.

\hypertarget{extraction-des-zones-dimage-manuelle}{%
\subsection{Extraction manuelle des zones d'image}\label{extraction-des-zones-dimage-manuelle}}

Le choix de la méthode d'extraction des régions d'intérêt dans les
documents constitue un élément clé de la modularité de la plateforme.
Les utilisateur.rices peuvent opter pour une extraction manuelle ou une
extraction automatique basée sur des algorithmes de vision par
ordinateur, adaptée aux traitements à plus grande échelle.

Après importation d'un enregistrement, le flux de travail procède à la
création de \mans \iiif pour chaque numérisation
(\digit) afin de permettre une visualisation grâce à la
plateforme Mirador. Le module de base autorise par la suite
l'extraction manuelle de zones d'intérêt au sein des images. Cette
fonctionnalité est particulièrement utile pour les projets ne souhaitant
pas recourir à des méthodes entièrement automatisées de vision par
ordinateur. L'outil \sas permet de créer des annotations, c'est-à-dire de
définir des régions d'intérêt spécifiques dans les numérisations, et de
les indexer directement dans les \mans \iiif correspondants,
enrichissant ainsi les ressources numériques. De plus, les
développements futurs prévoient la possibilité d'importer des fichiers
d'annotation préexistants en format .\textsc{txt} afin de pouvoir les indexer
manuellement. Par conséquent, le \textit{workflow} de base ne comporte aucun
traitement automatique basé sur la vision (et de fait éventuellement
trop gourmand en puissance de calcul).

L'extraction, qu'elle soit manuelle ou automatique, constitue le
fondement du reste des processus. Une interface est disponible pour
sélectionner un ensemble de documents et effectuer des actions
spécifiques sur les témoins annotés, via le formulaire de traitement qui
s'étend selon un choix de module configuré. Ainsi l'utilisateur.rice n'est
pas limité par un contexte initial, à l'origine deux étapes
indissociables et incontournables (importation et extraction), pour
pouvoir effectuer d'autres actions. Cette modularité permet de
s'affranchir d'un \textit{workflow} linéaire et prédéfini, offrant ainsi une plus
grande adaptabilité aux besoins spécifiques et aux ressources
matérielles des projets.

Pour conclure, l'existence de ce module de base répond à des besoins
élémentaires des projets de recherche en études visuelles. Il fournit un
outil qui permet d'agréger toutes les sources primaires qui concernent
le sujet, de décrire les sources et de les mettre en relation. Il offre
en outre la possibilité d'extraire et visualiser des contenus d'intérêt
(les ``crops'' d'images), ciblant ainsi les instances de base qui
intéressent les chercheur.ses.


        \hypertarget{des-traitements-et-des-architectures-diverses}{%
        \section{Des traitements et des architectures
        diverses}\label{des-traitements-et-des-architectures-diverses}}
         Spécialiser un modèle d'intelligence artificielle implique de lui
fournir des données pertinentes, diversifiées, et en quantité
suffisante. Cependant, pour certains domaines, dont l'histoire fait
partie, le volume de données disponible est insuffisant. Ce constat est
d'autant plus vrai dans le cas des diagrammes issus de traités
astronomiques~: les corpus de documents scientifiques historiques
contiennent généralement du texte en majeure partie, des tables et des
images, négligeant souvent les diagrammes.\footnote{Exception faite du
  corpus S-VED (\cite{buttner_cordeep_2022}), collection
  d'illustration très diverses contenant entre autre des diagrammes
  historiques. Mais les primitives ne sont pas annotées.}. De plus, ils
sont dénués d'annotations précises sur les éléments constitutifs des
pages~; c'est sans parler de l'inexistence d'un corpus de diagrammes
dont les primitives sont annotées. Or l'annotation est une tâche
chronophage et fastidieuse. Le recours aux données synthétique répond,
mais en partie seulement, à ces problématiques.

\hypertarget{datasets-synthetiques}{%
\subsection{\emph{datasets} synthétiques}\label{datasets-synthetiques}}

Les \textit{datasets} synthétiques sont générés par des algorithmes ou des
méthodes de simulation pour imiter des données réelles, sans être
directement extraites de sources existantes. De tels jeux de données
sont utilisés lorsque les données réelles sont limitées ou difficiles à
obtenir, mais qu'il est cependant nécessaire de contrôler spécifiquement
les caractéristiques des données d'entraînement\footcite{buttner_cordeep_2022}. La génération
d'images a pour but de fabriquer des ensembles de données plus vastes,
plus diversifiés, très variables et assez complexes, répondant aux
caractéristiques des objets d'intérêt du projet, et surtout étiquetés
automatiquement, sans recourir à l'annotation manuelle.

Ces données synthétiques sont assez ressemblantes et complexes pour être
exploitées. Par exemple, docExtractor est un modèle off-the-shell (au
même titre que \yolo) envisagé dans le cadre de la tâche d'extraction des
diagrammes, et qui se veut sépcifique aux données historiques, car il
est entraîné sur des données produites par un générateur de documents
historiques synthétiques~: SynDoc\footcite{monnier_docextractor_2020}. SynDoc
génère des images de manière aléatoire en combinant des éléments
graphiques (fonds, images, texte et bruit) provenant d'un jeu d'image
défini (constitué de 177 images de pages, 15 contextes, plus de 8000
œuvres d'art provenant de WikiArt, des lettrines générées à partir d'une
lettre aléatoire avec 91 fonts possibles, et des dessins, schémas et
textes tirés d'articles aléatoires sur Wikipedia, avec plus de 400
fonts). Les différents éléments s'agencent, intégrant sur le fond
images, texte et bruit, offrant des combinasons et des mises en pages
assez complexes. Chaque élément de contenu est pré-annoté, éliminant
ainsi le besoin d'annotations manuelles pour ces pages.

          \begin{figure}[H]
          \begin{center}
          \includegraphics[height=6.5cm]{figues/syndoc.jpg}
          \end{center}
          \caption{Données synthétiques générées par SynDoc.\footcite[p.46]{norindr_traitement_2023}}
          \label{fig:syndoc} \end{figure}

Pour entraîner le modèle de vectorisation, il a de même été nécessaire
d'utiliser des données synthétiques. Parce qu'annoter les primitives
géométriques dans des images de diagrammes complexes est très
chronophage, le modèle de vectorisation a été pré-formé sur des corpus
artificiels générés dynamiquement. Le script de génération des données
d'entraînement choisit aléatoirement un arrière-plan, y ajoute des mots,
des nombres et des glyphes puis crée artificiellement un diagramme en
insérant des segments, des cercles et des arcs. Le script est conçu pour
que ces diagrammes aient une forte probabilité de présenter des formes
très caractéristiques comme les cercles concentriques et tangents, les
lignes parallèles et les arcs connectés, afin de simuler les structures
typiques. Les primitives sont dessinées avec des
variations aléatoires d'opacité, de largeur et de couleur. Les cercles
peuvent être remplis ou vides. Enfin, du bruit est ajouté en appliquant
un flou gaussien, et en supprimant de petites régions du diagramme pour
imiter la dégradation des documents historiques. Les données
d'entraînement ainsi générées présentent des configurations assez
complexes.

          \begin{figure}[H]
          \begin{center}
          \includegraphics[height=7cm]{figues/vecto_synthetic_data.png}
          \end{center}
          \caption{Données synthétiques générées pour l'entraînement du modèle de vectorisation.\footcite[Figure issue de la présentation de Syrine Kalelli à l'occasion de la conférence \eida 2024~:][]{noauthor_eida_nodate-1}}
          \label{fig:vecto_synthetic} \end{figure}

Enfin, le modèle de similarité présente un troisième exemple, puisque
SegSwap est pré-entraîné sur de la donnée synthétique. Le script de
génération prend des parties aléatoires d'une images et les copie-colle
au-dessus d'une autre image. Les trois images (source, cible et
superposition) sont placées dans le même dataset d'entraînement, ainsi
le modèle apprend à retrouver ce qui, dans la superposition, vient de la
source, et ce qui vient de la cible.

          \begin{figure}[H]
          \begin{center}
          \includegraphics[height=3cm]{figues/segswap_blended_images.png}
          \end{center}
          \caption{Données d'entraînement du modèle Segswap.}
          \label{fig:segswap} \end{figure}

\hypertarget{les-donnees-reelles}{%
\subsection{Les données réelles}\label{les-donnees-reelles}}

S'appuyer sur les modèles \textit{off-the-shelf}, sur de larges \textit{datasets}
généralistes, ou sur des données synthétiques permet une implémentation
facilitée de la vision dans des projets et constitue une base solide.
Toutefois, les sources tenant aux deux projets (\vhs et \eida) sont trop
spécifiques pour se contenter de modèles généralistes ou formés sur des
données artificielles. Même si ces derniers peuvent offrir des performances
de base, ils risquent de manquer de précision et de sensibilité aux
particularités des documents historiques. Les corpus artificiels
présentent des configurations délibérément complexes pour s'approcher le
plus possible des difficultés que le modèle pourrrait rencontrer sur les
données réelle. Elles sont cependant irréalistes et insuffisantes pour
permettre aux modèles de généraliser sur des diagrammes réels.

En atteste la comparaison des performances de docExtractor et \yolov sur
les données d'\eida. docExtractor\footcite{monnier_docextractor_2020}, entraîné
sur des données synthétiques mimant les documents historiques serait en
théorie plus adapté au traitement d'images de pages de manuscrits, avec
du texte et des illustration côté à côte, d'autant qu'il intègre des
outils de traitement du texte (notamment pour la segmentation des
lignes)\footnote{\eida envisage l'implémentation d'un outil d'extraction
  et transcription des labels et des textes qui entourent les diagrammes}.
Pourtant, sans fine-tuning sur des données réelles, il présente des
performances équivalentes à celles de \yolov\footcite[p.45]{norindr_traitement_2023}. Cela
souligne que même les modèles off-the-shelf entraînés sur un corpus
assez spécifique et complexe, mais synthétique, ne dispense pas d'un
entraînement sur des données réelles, au même titre que les modèles très
généralistes comme \yolov.

Alors, le modèle de base \yolov tel que mis à disposition par
Ultralytics est entraîné sur de grands ensembles de données réelles, ce
qui constitue une base solide pour la classification des objets du
monde. L'utilisation de SynDoc permet ensuite de compléter
l'apprentissage initial en exposant le modèle à des exemples variés et
spécifiques aux documents historiques, augmentant ainsi sa capacité de
généralisation. Ces similis de manuscrits anciens offrent l'avantage de
pouvoir être produits en grandes quantités et de couvrir un large
éventail de scénarii et de configurations difficiles à obtenir dans des
ensembles de données réelles. Puis le modèle est entraîné sur les
données de \vhs, qui sont de réelles pages de documents historiques
contenant une large diversité d'illustrations. Ces données apporteront
une dimension supplémentaire de pertinence au modèle, en l'exposant à
des particularités des documents historiques réalistes. Enfin, \yolov
est entraîné sur les données d'\eida, qui sont orientées spécifiquement
vers les diagrammes, afin qu'il détecte uniquement ces derniers.

Quant au modèle de vectorisation développé par Syrine
Kalleli\footcite{kalleli_historical_2024}, il est formé
sur des données synthétiques générées à la volée par un script. Mais le
corpus de diagrammes d'\eida est particulièrement caractéristique et le
modèle n'aurait pu être optimal sans avoir appris sur des images de
diagrammes issus de manuscrits réels. Un corpus d'entraînement de 303
diagrammes extraits de manuscrits et de gravures a donc été constitué et
annoté par les historien.nes. Ces diagrammes sont issus de sources latines, arabes,
grecques, hébreuses ou chinoises, datant du \textsc{xii}\ieme au \textsc{xviii}\ieme siècle, et ils
présentent en guise d'étiquettes plus de 3000 lignes, cercles et arcs. Le
ré-entraînement a permis le transfert des connaissances acquises sur la
tâche de détection des primitives sur les données réalistes.

Il sera également possible d'obtenir des meilleurs résultats sur la
similarité grâce à une évaluation des scores (qui constitue un jeu de
données annotées) et le ré-entraînement du modèle, pour donner des
résultats plus adaptés à la spécificité des données historiques.

D'ailleurs, cette étape d'annotation (le choix des exemples et des
étiquettes) revêt des enjeux importants. L'apprentissage spécifique se
fait à partir de données sélectionnées par les chercheur.ses~: les exemples
sur lequel l'algorithme d'apprentissage va itérer définissent le modèle.
Il est nécessaire de constituer un échantillon de données aléatoire et
représentatif, et de l'annoter en fonction de ce que l'on souhaite
obtenir en prédiction.

L'annotation des jeux de données est non seulement une étape clé, mais
aussi un bel exemple de collaboration chercheur.ses-ingénieur.es. Elle
nécessite la définition de normes pertinentes et rigoureuses. Travail
minutieux et chronophage, l'étiquetage des données peut engendrer des
erreurs et du bruit dans les données, car elle implique la subjectivité
des chercheur.ses et le regard parfois trop précis sur les sources desquels
les annotateurs sont experts.

Voici un exemple rencontré lors de la préparation des données pour
entraîner un modèle de segmentation du contenu textuel. Les sources
arabes et chinoises sont particulièrement verbeuses et les diagrammes
sont très souvent entourés des blocs de commentaires se mélangeant alors
aux légendes et aux labels. Doit-on considérer ces commentaires comme
faisant partie des éléments que l'on souhaite identifier ou bien les ignorer
? Cette décision est importante car si on les ignore, le modèle risque
de passer à côté d'éléments textuels pertinents. En revanche, si on les
inclut, il ramènera des commentaires sans rapport direct avec le
diagramme observé. On voit ici comment la binarité des modèles, qui se
reflète dans les normes d'annotation, est problématique et constitue une
limite au \ml. Un compromis doit être trouvé entre
l'automatisation, qui requiert une normalisation, des définitions
claires et binaires, et la nuance dans l'interprétation des
sources\footnote{Dans le cadre du projet, il a toujors été plus
  intéressant d'opter pour une définition extensive des objets à
  détecter, car prévision d'une correction des traitement. Et il est
  plus facile de supprimer un élémént pas pertinent que d'aller en
  rechercher un, surtout compte tenu de la taille des corpus des
  chercheur.ses. Vaut aussi pour la préparation des données pour
  l'entraînement du modèle d'extraction.}.

La normalisation peut bénéficier à l'écosystème de recherche dans le
domaine de l'\htr et de l'\ocr. À ce titre, il est pertinent d'envisager
l'utilisation du vocabulaire contrôlé SegmOnto pour l'annotation du
contenu textuel entourant les diagrammes. Cela permettrait de créer des
jeux de données réutilisables, à partager avec des projets poursuivant
des objectifs similaires.\footnote{https://segmonto.github.io/}. Encore
une fois, un compromis doit être trouvé entre les besoins de description
des chercheur.ses et les possibilités offertes par les vocabulaires
contrôlés.

Un autre exemple concerne le dernier entraînement du modèle d'extraction
: les résultats montrent que des diagrammes sont encore détectés en
transparence. La question s'est alors posée de chercher à corriger ce
défaut en donnant au modèle, à l'occasion d'un nouvel entraînement,
d'avantage d'exemples négatifs (diagrammes visibles par transparence
mais non annotés). Or il est préférable de se contenter de la correction
ou suppression manuelle de ces prévisions erronées, garantissant que le
modèle parvienne à détecter les diagrammes presque effacés.

Pour assurer la rigueur et la cohérence des annotations, les décisions
prises entre les chercheur.ses et les ingénieur.es peuvent être l'objet d'une
documentation ou d'ateliers d'annotation.

\hypertarget{loeil-de-la-machine-avantages-et-limites}{%
\subsection{L'oeil de la machine~: avantages et
limites}\label{loeil-de-la-machine-avantages-et-limites}}

Bien qu'il soit possible d'optimiser les performances d'un modèle
d'apprentissage automatique en l'entraînant sur un ensemble de données
spécifique, son interprétation des données reste limitée car
fondamentalement binaire, ce qui le rend parfois déficient pour la
recherche en histoire. Ainsi, il gèrera difficilement les cas limites et
ambigüs. La décision d'inclure ou d'exclure ces cas particuliers de
l'ensemble d'entraînement implique un arbitrage délicat. D'un côté, une
inclusion trop restrictive peut compromettre les capacités de
généralisation du modèle, c'est-à-dire sa capacité à s'adapter à de
nouvelles données. À l'inverse, une inclusion trop permissive risque de
dégrader la précision du modèle sur les cas plus typiques. Les
chercheur.ses espérant obtenir un modèle maximaliste, quitte à accepter un
certain degré d'erreur et de devoir supprimer les faux
positifs, de nombreux cas limites ont été inclus. Le cas des diagrammes
visibles en transparence (expliqué précédemment) en est un exemple
éloquent.

Une autre difficulté réside dans la définition même du ``diagramme
astronomique''. Les limites de ce concept ne sont pas si claires et
définitives pour les chercheur.ses, et pourtant le modèle a besoin d'une
définition rigoureuse et cohérente. Il paraît en effet difficile de
considérer les diagrammes astronomiques en dehors du contexte des
pratiques d'autres sciences et disciplines connexes. Par exemple, Le
\emph{Flores Almagesti} -- réécriture de l'Almageste datant du \textsc{xv}\ieme par
l'astronome Giovanni Bianchini -- présente une partie algébrique à
l'ouverture mathématique, induisant la présence de nouveaux types de
diagrammes d'inspiration euclidienne. Pour retracer la source de ces
derniers, il est nécessaire de considérer les traités d'Euclide ou
autres travaux d'algèbre. Ceux-ci ne sont pas des traités
\emph{astronomiques}, bien qu'il ne soit pas certain que ces disinctions
contemporaines aient été aussi rigide à l'époque et aient eu un
quelconque sens pour les acteurs historiques. Les sources byzantines
confirment cette complexité~: les diagrammes y sont nommés
\emph{katagraphai}, indépendamment du domaine scientifique auquel ils
appartiennent. Également, de nombreux travaux astronomiques sont groupés
dans des témoins qui contiennent des œuvres issus de domaines divers.
C'est le cas avec les sources chinoises, comme le \emph{Chongzhen
lishu}, qui se présente généralement annexé d'une série de traités
mathématiques. Par conséquent, les diagrammes euclidiens ont été gardés
lors de la préparation des données, et l'algorithme de détection les
classe comme ``diagramme'', même s'ils ne constituent pas l'objet
principal des chercheur.ses.

En ce qui concerne les autres types de diagrammes non strictement
astronomiques (géométriques, harmoniques, logiques, illustrations de
constellations), une approche plus sélective a été adopté afin d'éviter
un modèle trop maximalistes. Ces éléments, bien que potentiellement
intéressants, n'ont pas été inclus dans la phase de détection
automatique.

Ainsi l'œil de la \cv contraint à des choix méthodologique
potentiellement inconfortables, mais en même temps il peut aider à
mesurer les impulsions des chercheur.ses, à mieux définir les objectifs de
recherche et à prioriser les éléments les plus pertinents. Ainsi, la
vision par ordinateur oblige les chercheur.ses à s'adapter à une logique
algorithmique qui, tout en limitant certaines interprétations
subjectives, offre l'opportunité de développer des modèles conceptuels et des méthodologies très rigoureuses.

\vspace{2cm}

En résumé, la diversité des architectures de réseaux de neurones permet
d'adapter les modèles aux spécificités des données et des tâches à
traiter, tout en optimisant les performances et l'efficacité
computationnelle. Cette flexibilité est essentielle pour répondre aux
multiples défis posés par le traitement automatique des documents
historiques. Deux approches se distinguent~: l'utilisation de modèles
\textit{off-the-shelf} ou d'architectures spécifiques. Traditionnellement, les
ingénieur.es concevaient des architectures de réseau de neurones sur
mesure, adaptées spécifiquement aux problèmes à résoudre. Cette
approche, bien que permettant une optimisation fine des performances,
est souvent coûteuse en temps et en ressources. Ces dernières années,
l'émergence de modèles pré-entraînés sur des jeux de données massifs a
ouvert de nouvelles perspectives. Ces modèles \textit{off-the-shelf} (ou
\textit{pre-trained models}), déjà qualitatifs, demandent cependant à être spécialisés sur des corpus. Le \textit{fine-tuning}
consiste alors à adapter ces modèles à une tâche spécifique en les
entraînant sur un jeu de données plus petit et plus pertinent.

L'approche de corpus de dimensions importantes, ne permettant pas une analyse
manuelle, trop chronophage, est permise par la vision artificielle.
L'utilisation des modèles de vision présente cependant des limites~:
notamment, la complexité de leur fonctionnement reste assez opaque aux
non spécialistes, ce qui peut être un inconvénient là où la transparence
et l'explicabilité sont importantes, typiquement pour des projets
transversaux caractéristiques des humanités numériques. D'autant que la
collaboration entre les chercheur.ses spécialistes des sources et les
ingénieur.es \textit{data scientists} est essentielle~: les deux pôles doivent
collaborer pour entraîner les modèles et les rendre performants sur les
données spécifiques. C'est pourquoi l'optimisation des modèles de vision
ne s'obtient pas sans un effort du côté des équipes disciplinaires, qui
vont réunir et annoter des jeux de données importants pour entraîner
efficacement les modèles.
               
            
        \clearemptydoublepage

\hypertarget{chapitre-5-la-donnee}{%
\chapter{La donnée~: générique ou caractéristique, réelle
ou
synthétique}\label{chapitre-5-la-donnee}}

            
Le \ml repose sur deux piliers~: l'algorithme d'une part,
qui est la procédure que l'on fait tourner sur les données pour produire
un modèle, et d'autre part les données, qui sont les exemples à partir
desquels l'algorithme ajuste les poids du modèle (il \emph{apprend}).
L'entraînement consiste à faire tourner un algorithme d'apprentissage
sur un jeu de données (très) important et accompagné de ses étiquettes
ou annotations (correspondant aux résultats espérés du modèle). La
disponibilité de corpus annotés assez divers et importants est un défi à
relever, particulièrement dans le domaine de l'histoire des sciences.
L'un des principaux challenges de la reconnaissance sémantique des
éléments visuels dans les documents historiques est leur grande
variabilité, ainsi que, et surtout, la rareté générale de jeux de
données historiques cohérents axés sur leur détection. De plus, dans la majorité des cas où des éléments graphiques
ont été rassemblés, ils n'ont pas été annotés selon des classes
sémantiques issus d'ontologies normalisées. En outre, les
\emph{datasets} disponibles sont souvent trop spécifiques, concernant
une période ou un support précis, tels que des manuscrits écrits à la
main, des livres imprimés ou des journaux. On pourra par exemple citer
le \emph{Newspaper Navigator Dataset}\footcite{lee_newspaper_2020} qui contient des
éléments visuels issus de 16 millions de journaux des Etats-Unis publiés
en 1789 et 1963 ou encore le \emph{HORAE dataset}\footcite{noauthor_horae_nodate} qui se
concentre sur les livres de prière de la fin du \ma. En histoire
des sciences astronomiques ou mathématiques, aucun projet n'a encore
entrepris de collecte d'envergure, hormis peut-être le projet
Sphaera\footcite{noauthor_sphere_nodate} (travail sur les premiers manuscrits) ou le projet \vhs, auquel \eida se greffe.

Cette rareté s'explique cependant~: l'annotation des jeux de données
représente une tâche chronophage qui nécessite des moyens humains
importants et l'implication d'experts capables de reconnaître et de
classer avec précision les éléments pertinents dans les images.

Par conséquent, il est nécessaire de trouver des solutions pour remédier à
la rareté des jeux de données d'images de documents historiques
annotés, telles que l'utilisation de techniques de génération de
données synthétiques ou l'utilisation de modèle généraux tirant partie
du \textit{crowdsourcing} pour leur entraînement. On verra que ces stratégies sont néanmoins
insuffisantes, et que la constitution de jeux de données réelles et
contextuelles est nécessaire pour ne pas sacrifier la pertinence des
modèles à l'efficience de leur fabrication.

        \hypertarget{partir-modele-generaliste}{%
        \section{Partir d'un modèle
        généraliste}\label{partir-modele-generaliste}}
        
        Le module de base est un package pour la gestion documentaire, duquel
l'application ne peut se détacher. Celui-ci inclut tout d'abord des
formulaires pour l'intégration des documents dans la base de données. Le
modèle de données permet de décrire différentes entités qui, bien que
liées dans leurs métadonnées, peuvent être intégrées indépendamment. Le
module de base permet également la création de \mans \iiif pour
chaque numérisation, permettant ensuite la visualisation des documents
grâce aux outils open-source dédiés. De ce fait, l'indexation de zones
d'image peut être réalisée manuellement via l'interface Mirador intégrée
à \sas. Ce noyau fonctionnel inclut en outre la sélection de lots de
documents (le ``panier''), sur lesquels pourront être effectués des
traitements groupés paramétrables.

Les briques fondamentales offrent donc les fonctionnalités essentielles
de gestion documentaire (intégration, modèle de données, \iiif). Les
traitements, quant à eux, sont gérés par des modules séparés, et c'est
sur cette structure que repose la modularité et l'évolutivité de
l'application.

Ci-après nous donnons une description détaillée de certaines de ces
fonctionnalités de base.

\hypertarget{description-des-donnees}{%
\subsection{Description des
données}\label{description-des-donnees}}

Le module de base contient un modèle de données suffisamment extensif
pour décrire efficacement une diversité de données, allant de documents
textuels historiques à des tableaux en histoire de l'art. La
tripartition entre témoin (\wit), série (qui contient un ensemble de
témoins), et contenu permet un alignement avec des corpus très
diversifiés et des données potentiellement hétéroclites, telles que des
manuscrits, des documents épistolaires, des inventaires de galeries
d'art, et même pourquoi pas des cartes\ldots{}

Pour ouvrir à cette large diversité de données, la liste des types de
pagination témoin doit être étendue \emph{a minima} d'un nouveau type
``other'', émancipant l'enregistrement des mentions de pagination. Les
développements futurs prévoient aussi la création d'un système pour
ajouter facilement un nouveau type\footnote{Le type de témoin est une
  métadonnée rentrée par l'utilisateur.rice lors de l'enregistrement du
  \wit dans la base de donnée. Il choisit le type dans une liste,
  originellement manuscrit, imprimé ou gravure sur bois.} de \wit
(tel que peinture, catalogue, etc.).

Au fil des développements, des débats ont émergé autour de l'ajout dans
le modèle de données d'un niveau de granularité supplémentaire pour
décrire des images ou zones d'images unitaires
(\graphicals), créant ainsi une entité détachée du fait
qu'elle provienne d'une extraction dans un document. Cette solution
aurait permis une description plus détaillée et plus fine des images,
importante pour des projets axés sur des images uniques, et aurait
favorisé un élargissement du spectre des type de sources pris en charge.
L'utilisateur.rice aurait pu soit importer une image unique (et de manière
optionnelle, la lier à un \wit) via un formulaire, soit sélectionner
une région d'image d'intérêt au sein des extractions (annotations \sas),
laquelle serait enregistrée comme \graphical, puis l'enrichir de
métadonnées. Dans les deux cas l'enregistrement d'un \graphical
aurait donné lieu à la création d'une \digit au format \jpeg.

Sans l'unité de description \graphical, les régions d'images
sont créées uniquement via les annotations \sas.

L'intégration de cette entité au sein du modèle aurait offert plusieurs
avantages en termes de cohérence et de flexibilité. En s'alignant sur
les structures existantes (\wits et \sers), elle aurait permis une
manipulation plus intuitive des images, facilitant ainsi les opérations
de recherche et la création de \emph{Sets} personnalisés. De plus, elle
aurait rationalisé la gestion des annotations \sas, permettant de
sélectionner les plus pertinentes dans la multitude existante.

Cependant, cette approche présente des limites, et on peut trouver des
alternatives. Tout d'abord, la coexistence de \graphicals avec les
annotations \sas, générées par des processus distincts, aurait pu créer
une certaine confusion quant à leur nature et à leur méthode de
création. De plus, la multiplication potentielle de milliers
d'enregistrements aurait pu impacter les performances de la base de
données et complexifier les requêtes. Enfin, le lien sémantique ambigu
et sujet à interprétation subjective entre \graphical et \wit
aurait compliqué les possibilités de corrélation.

Compte tenu de ces limites, il a semblé préférable de maintenir les
annotations \sas pour identifier les instances de base du modèle, sans
créer de nouvelle unité de description. La solution actuelle reste donc
basée sur la création manuelle ou automatique de zones dans les images
via \iiif et \sas, évitant les problèmes de redondance et de confusion.
Bien que l'entité \graphical n'ait pas été implémentée, les
fonctionnalités d'annotation et de sélection d'images sont assurées par
d'autres mécanismes. L'outil Mirador permet d'associer des tags aux
zones d'image, offrant ainsi une première couche d'enrichissement
sémantique. La sélection dans un \emph{set} personnalisé sera possible en
gardant en mémoire une référence contenant des coordonnées du
\emph{crop}. De plus, l'importation d'images individuelles est
réalisable en les considérant comme des \emph{Witness partiels}, ce qui
permet de les intégrer dans le \textit{workflow} existant. Toutefois
l'enrichissement sémantique à un niveau de granularité fin restera
limité~; et la dépendance à l'outil \sas constitue une potentielle dette
technique, susceptible de restreindre les évolutions futures du système.

Afin de mieux répondre aux exigences de modularité, l'évolution du
modèle de données s'oriente non pas vers une description individuelle
des documents, mais vers la gestion des traitements. Cette évolution
implique la création d'une entité \tr
liée à des ensembles de données (\ds et
\rs) potentiellement hétérogènes.

\hypertarget{principe-du-traitement}{%
\subsection{Principe du Traitement}\label{principe-du-traitement}}

Le but fondamental de la plateforme est de pouvoir effectuer plusieurs
actions sur les objets de la base. Afin d'assurer une meilleure
traçabilité et plus de flexibilité, la plateforme abandonne les
lancements automatiques des processus\footnote{C'était initialement le
  cas de l'extraction des entités, dont le lancement était lié à une
  méthode de classe liée à la \digit après soumission d'un
  formulaire d'ajout d'un \wit ou d'une \ser. L'action se
  lançait immédiatement après enregistrement des images d'une
  numérisation dans la plateforme.} au profit d'un système basé sur
l'entité \tr. Chaque traitement est associé à un ensemble
d'objets traités ensemble (\ds ou \rs), à un jeu de
paramètres et à un résultat. Ces informations sont stockées dans une
table dédiée. Cette approche facilite la gestion et le trackage des
processus (notamment, les utilisateur.rices sont notifiés par e-mail à la fin
du \textit{processing}), permet aux utilisateur.rices de consulter un historique de
leurs actions et offre la possibilité de créer des \textit{workflows}
personnalisés en passant par un formulaire de lancement unique mais
extensif.

En permettant de regrouper des documents de types différents (\wos,
\sers, \wits) dans des \dss, on offre à
l'utilisateur.rice la flexibilité de lancer des actions sur des ensembles
d'entités hétérogènes et granulaires. Le traitement est ensuite réparti
sur les entités de niveau inférieur (les témoins). Les \wits ainsi
sélectionnés peuvent être soumis à une large gamme de traitements~: des
fonctions déjà implémentées comme l'exportation (avec choix du
format), l'extraction, la vectorisation, la recherche de similarité~; ou
de nouveaux traitements personnalisés, tels que la visualisation sur une
frise chronologique ou une carte. La modularité de la plateforme est
assurée par un formulaire de lancement configurable, permettant de
l'adapter à différents scénarios d'utilisation, et à l'ajout de modules
personnalisés.

Le \rs fonctionne similairement au \ds, à un niveau
de granularité inférieur (à l'échelle de la zone d'image)\footnote{À
  l'été 2024, l'entité n'existe pas encore dans la base de données, mais
  le processus d'envoi du traitement et les modes de communication entre
  l'application et l'\api prévoient la possibilité de lancer l'inférence
  des modèles sur un ensemble de régions extraites.}.

\hypertarget{extraction-des-zones-dimage-manuelle}{%
\subsection{Extraction manuelle des zones d'image}\label{extraction-des-zones-dimage-manuelle}}

Le choix de la méthode d'extraction des régions d'intérêt dans les
documents constitue un élément clé de la modularité de la plateforme.
Les utilisateur.rices peuvent opter pour une extraction manuelle ou une
extraction automatique basée sur des algorithmes de vision par
ordinateur, adaptée aux traitements à plus grande échelle.

Après importation d'un enregistrement, le flux de travail procède à la
création de \mans \iiif pour chaque numérisation
(\digit) afin de permettre une visualisation grâce à la
plateforme Mirador. Le module de base autorise par la suite
l'extraction manuelle de zones d'intérêt au sein des images. Cette
fonctionnalité est particulièrement utile pour les projets ne souhaitant
pas recourir à des méthodes entièrement automatisées de vision par
ordinateur. L'outil \sas permet de créer des annotations, c'est-à-dire de
définir des régions d'intérêt spécifiques dans les numérisations, et de
les indexer directement dans les \mans \iiif correspondants,
enrichissant ainsi les ressources numériques. De plus, les
développements futurs prévoient la possibilité d'importer des fichiers
d'annotation préexistants en format .\textsc{txt} afin de pouvoir les indexer
manuellement. Par conséquent, le \textit{workflow} de base ne comporte aucun
traitement automatique basé sur la vision (et de fait éventuellement
trop gourmand en puissance de calcul).

L'extraction, qu'elle soit manuelle ou automatique, constitue le
fondement du reste des processus. Une interface est disponible pour
sélectionner un ensemble de documents et effectuer des actions
spécifiques sur les témoins annotés, via le formulaire de traitement qui
s'étend selon un choix de module configuré. Ainsi l'utilisateur.rice n'est
pas limité par un contexte initial, à l'origine deux étapes
indissociables et incontournables (importation et extraction), pour
pouvoir effectuer d'autres actions. Cette modularité permet de
s'affranchir d'un \textit{workflow} linéaire et prédéfini, offrant ainsi une plus
grande adaptabilité aux besoins spécifiques et aux ressources
matérielles des projets.

Pour conclure, l'existence de ce module de base répond à des besoins
élémentaires des projets de recherche en études visuelles. Il fournit un
outil qui permet d'agréger toutes les sources primaires qui concernent
le sujet, de décrire les sources et de les mettre en relation. Il offre
en outre la possibilité d'extraire et visualiser des contenus d'intérêt
(les ``crops'' d'images), ciblant ainsi les instances de base qui
intéressent les chercheur.ses.
        
        \hypertarget{quelles-donnees}{%
        \section{Quelles données pour le spécialiser~?}\label{quelles-donnuees}}
        
        Spécialiser un modèle d'intelligence artificielle implique de lui
fournir des données pertinentes, diversifiées, et en quantité
suffisante. Cependant, pour certains domaines, dont l'histoire fait
partie, le volume de données disponible est insuffisant. Ce constat est
d'autant plus vrai dans le cas des diagrammes issus de traités
astronomiques~: les corpus de documents scientifiques historiques
contiennent généralement du texte en majeure partie, des tables et des
images, négligeant souvent les diagrammes.\footnote{Exception faite du
  corpus S-VED (\cite{buttner_cordeep_2022}), collection
  d'illustration très diverses contenant entre autre des diagrammes
  historiques. Mais les primitives ne sont pas annotées.}. De plus, ils
sont dénués d'annotations précises sur les éléments constitutifs des
pages~; c'est sans parler de l'inexistence d'un corpus de diagrammes
dont les primitives sont annotées. Or l'annotation est une tâche
chronophage et fastidieuse. Le recours aux données synthétique répond,
mais en partie seulement, à ces problématiques.

\hypertarget{datasets-synthetiques}{%
\subsection{\emph{datasets} synthétiques}\label{datasets-synthetiques}}

Les \textit{datasets} synthétiques sont générés par des algorithmes ou des
méthodes de simulation pour imiter des données réelles, sans être
directement extraites de sources existantes. De tels jeux de données
sont utilisés lorsque les données réelles sont limitées ou difficiles à
obtenir, mais qu'il est cependant nécessaire de contrôler spécifiquement
les caractéristiques des données d'entraînement\footcite{buttner_cordeep_2022}. La génération
d'images a pour but de fabriquer des ensembles de données plus vastes,
plus diversifiés, très variables et assez complexes, répondant aux
caractéristiques des objets d'intérêt du projet, et surtout étiquetés
automatiquement, sans recourir à l'annotation manuelle.

Ces données synthétiques sont assez ressemblantes et complexes pour être
exploitées. Par exemple, docExtractor est un modèle off-the-shell (au
même titre que \yolo) envisagé dans le cadre de la tâche d'extraction des
diagrammes, et qui se veut sépcifique aux données historiques, car il
est entraîné sur des données produites par un générateur de documents
historiques synthétiques~: SynDoc\footcite{monnier_docextractor_2020}. SynDoc
génère des images de manière aléatoire en combinant des éléments
graphiques (fonds, images, texte et bruit) provenant d'un jeu d'image
défini (constitué de 177 images de pages, 15 contextes, plus de 8000
œuvres d'art provenant de WikiArt, des lettrines générées à partir d'une
lettre aléatoire avec 91 fonts possibles, et des dessins, schémas et
textes tirés d'articles aléatoires sur Wikipedia, avec plus de 400
fonts). Les différents éléments s'agencent, intégrant sur le fond
images, texte et bruit, offrant des combinasons et des mises en pages
assez complexes. Chaque élément de contenu est pré-annoté, éliminant
ainsi le besoin d'annotations manuelles pour ces pages.

          \begin{figure}[H]
          \begin{center}
          \includegraphics[height=6.5cm]{figues/syndoc.jpg}
          \end{center}
          \caption{Données synthétiques générées par SynDoc.\footcite[p.46]{norindr_traitement_2023}}
          \label{fig:syndoc} \end{figure}

Pour entraîner le modèle de vectorisation, il a de même été nécessaire
d'utiliser des données synthétiques. Parce qu'annoter les primitives
géométriques dans des images de diagrammes complexes est très
chronophage, le modèle de vectorisation a été pré-formé sur des corpus
artificiels générés dynamiquement. Le script de génération des données
d'entraînement choisit aléatoirement un arrière-plan, y ajoute des mots,
des nombres et des glyphes puis crée artificiellement un diagramme en
insérant des segments, des cercles et des arcs. Le script est conçu pour
que ces diagrammes aient une forte probabilité de présenter des formes
très caractéristiques comme les cercles concentriques et tangents, les
lignes parallèles et les arcs connectés, afin de simuler les structures
typiques. Les primitives sont dessinées avec des
variations aléatoires d'opacité, de largeur et de couleur. Les cercles
peuvent être remplis ou vides. Enfin, du bruit est ajouté en appliquant
un flou gaussien, et en supprimant de petites régions du diagramme pour
imiter la dégradation des documents historiques. Les données
d'entraînement ainsi générées présentent des configurations assez
complexes.

          \begin{figure}[H]
          \begin{center}
          \includegraphics[height=7cm]{figues/vecto_synthetic_data.png}
          \end{center}
          \caption{Données synthétiques générées pour l'entraînement du modèle de vectorisation.\footcite[Figure issue de la présentation de Syrine Kalelli à l'occasion de la conférence \eida 2024~:][]{noauthor_eida_nodate-1}}
          \label{fig:vecto_synthetic} \end{figure}

Enfin, le modèle de similarité présente un troisième exemple, puisque
SegSwap est pré-entraîné sur de la donnée synthétique. Le script de
génération prend des parties aléatoires d'une images et les copie-colle
au-dessus d'une autre image. Les trois images (source, cible et
superposition) sont placées dans le même dataset d'entraînement, ainsi
le modèle apprend à retrouver ce qui, dans la superposition, vient de la
source, et ce qui vient de la cible.

          \begin{figure}[H]
          \begin{center}
          \includegraphics[height=3cm]{figues/segswap_blended_images.png}
          \end{center}
          \caption{Données d'entraînement du modèle Segswap.}
          \label{fig:segswap} \end{figure}

\hypertarget{les-donnees-reelles}{%
\subsection{Les données réelles}\label{les-donnees-reelles}}

S'appuyer sur les modèles \textit{off-the-shelf}, sur de larges \textit{datasets}
généralistes, ou sur des données synthétiques permet une implémentation
facilitée de la vision dans des projets et constitue une base solide.
Toutefois, les sources tenant aux deux projets (\vhs et \eida) sont trop
spécifiques pour se contenter de modèles généralistes ou formés sur des
données artificielles. Même si ces derniers peuvent offrir des performances
de base, ils risquent de manquer de précision et de sensibilité aux
particularités des documents historiques. Les corpus artificiels
présentent des configurations délibérément complexes pour s'approcher le
plus possible des difficultés que le modèle pourrrait rencontrer sur les
données réelle. Elles sont cependant irréalistes et insuffisantes pour
permettre aux modèles de généraliser sur des diagrammes réels.

En atteste la comparaison des performances de docExtractor et \yolov sur
les données d'\eida. docExtractor\footcite{monnier_docextractor_2020}, entraîné
sur des données synthétiques mimant les documents historiques serait en
théorie plus adapté au traitement d'images de pages de manuscrits, avec
du texte et des illustration côté à côte, d'autant qu'il intègre des
outils de traitement du texte (notamment pour la segmentation des
lignes)\footnote{\eida envisage l'implémentation d'un outil d'extraction
  et transcription des labels et des textes qui entourent les diagrammes}.
Pourtant, sans fine-tuning sur des données réelles, il présente des
performances équivalentes à celles de \yolov\footcite[p.45]{norindr_traitement_2023}. Cela
souligne que même les modèles off-the-shelf entraînés sur un corpus
assez spécifique et complexe, mais synthétique, ne dispense pas d'un
entraînement sur des données réelles, au même titre que les modèles très
généralistes comme \yolov.

Alors, le modèle de base \yolov tel que mis à disposition par
Ultralytics est entraîné sur de grands ensembles de données réelles, ce
qui constitue une base solide pour la classification des objets du
monde. L'utilisation de SynDoc permet ensuite de compléter
l'apprentissage initial en exposant le modèle à des exemples variés et
spécifiques aux documents historiques, augmentant ainsi sa capacité de
généralisation. Ces similis de manuscrits anciens offrent l'avantage de
pouvoir être produits en grandes quantités et de couvrir un large
éventail de scénarii et de configurations difficiles à obtenir dans des
ensembles de données réelles. Puis le modèle est entraîné sur les
données de \vhs, qui sont de réelles pages de documents historiques
contenant une large diversité d'illustrations. Ces données apporteront
une dimension supplémentaire de pertinence au modèle, en l'exposant à
des particularités des documents historiques réalistes. Enfin, \yolov
est entraîné sur les données d'\eida, qui sont orientées spécifiquement
vers les diagrammes, afin qu'il détecte uniquement ces derniers.

Quant au modèle de vectorisation développé par Syrine
Kalleli\footcite{kalleli_historical_2024}, il est formé
sur des données synthétiques générées à la volée par un script. Mais le
corpus de diagrammes d'\eida est particulièrement caractéristique et le
modèle n'aurait pu être optimal sans avoir appris sur des images de
diagrammes issus de manuscrits réels. Un corpus d'entraînement de 303
diagrammes extraits de manuscrits et de gravures a donc été constitué et
annoté par les historien.nes. Ces diagrammes sont issus de sources latines, arabes,
grecques, hébreuses ou chinoises, datant du \textsc{xii}\ieme au \textsc{xviii}\ieme siècle, et ils
présentent en guise d'étiquettes plus de 3000 lignes, cercles et arcs. Le
ré-entraînement a permis le transfert des connaissances acquises sur la
tâche de détection des primitives sur les données réalistes.

Il sera également possible d'obtenir des meilleurs résultats sur la
similarité grâce à une évaluation des scores (qui constitue un jeu de
données annotées) et le ré-entraînement du modèle, pour donner des
résultats plus adaptés à la spécificité des données historiques.

D'ailleurs, cette étape d'annotation (le choix des exemples et des
étiquettes) revêt des enjeux importants. L'apprentissage spécifique se
fait à partir de données sélectionnées par les chercheur.ses~: les exemples
sur lequel l'algorithme d'apprentissage va itérer définissent le modèle.
Il est nécessaire de constituer un échantillon de données aléatoire et
représentatif, et de l'annoter en fonction de ce que l'on souhaite
obtenir en prédiction.

L'annotation des jeux de données est non seulement une étape clé, mais
aussi un bel exemple de collaboration chercheur.ses-ingénieur.es. Elle
nécessite la définition de normes pertinentes et rigoureuses. Travail
minutieux et chronophage, l'étiquetage des données peut engendrer des
erreurs et du bruit dans les données, car elle implique la subjectivité
des chercheur.ses et le regard parfois trop précis sur les sources desquels
les annotateurs sont experts.

Voici un exemple rencontré lors de la préparation des données pour
entraîner un modèle de segmentation du contenu textuel. Les sources
arabes et chinoises sont particulièrement verbeuses et les diagrammes
sont très souvent entourés des blocs de commentaires se mélangeant alors
aux légendes et aux labels. Doit-on considérer ces commentaires comme
faisant partie des éléments que l'on souhaite identifier ou bien les ignorer
? Cette décision est importante car si on les ignore, le modèle risque
de passer à côté d'éléments textuels pertinents. En revanche, si on les
inclut, il ramènera des commentaires sans rapport direct avec le
diagramme observé. On voit ici comment la binarité des modèles, qui se
reflète dans les normes d'annotation, est problématique et constitue une
limite au \ml. Un compromis doit être trouvé entre
l'automatisation, qui requiert une normalisation, des définitions
claires et binaires, et la nuance dans l'interprétation des
sources\footnote{Dans le cadre du projet, il a toujors été plus
  intéressant d'opter pour une définition extensive des objets à
  détecter, car prévision d'une correction des traitement. Et il est
  plus facile de supprimer un élémént pas pertinent que d'aller en
  rechercher un, surtout compte tenu de la taille des corpus des
  chercheur.ses. Vaut aussi pour la préparation des données pour
  l'entraînement du modèle d'extraction.}.

La normalisation peut bénéficier à l'écosystème de recherche dans le
domaine de l'\htr et de l'\ocr. À ce titre, il est pertinent d'envisager
l'utilisation du vocabulaire contrôlé SegmOnto pour l'annotation du
contenu textuel entourant les diagrammes. Cela permettrait de créer des
jeux de données réutilisables, à partager avec des projets poursuivant
des objectifs similaires.\footnote{https://segmonto.github.io/}. Encore
une fois, un compromis doit être trouvé entre les besoins de description
des chercheur.ses et les possibilités offertes par les vocabulaires
contrôlés.

Un autre exemple concerne le dernier entraînement du modèle d'extraction
: les résultats montrent que des diagrammes sont encore détectés en
transparence. La question s'est alors posée de chercher à corriger ce
défaut en donnant au modèle, à l'occasion d'un nouvel entraînement,
d'avantage d'exemples négatifs (diagrammes visibles par transparence
mais non annotés). Or il est préférable de se contenter de la correction
ou suppression manuelle de ces prévisions erronées, garantissant que le
modèle parvienne à détecter les diagrammes presque effacés.

Pour assurer la rigueur et la cohérence des annotations, les décisions
prises entre les chercheur.ses et les ingénieur.es peuvent être l'objet d'une
documentation ou d'ateliers d'annotation.

\hypertarget{loeil-de-la-machine-avantages-et-limites}{%
\subsection{L'oeil de la machine~: avantages et
limites}\label{loeil-de-la-machine-avantages-et-limites}}

Bien qu'il soit possible d'optimiser les performances d'un modèle
d'apprentissage automatique en l'entraînant sur un ensemble de données
spécifique, son interprétation des données reste limitée car
fondamentalement binaire, ce qui le rend parfois déficient pour la
recherche en histoire. Ainsi, il gèrera difficilement les cas limites et
ambigüs. La décision d'inclure ou d'exclure ces cas particuliers de
l'ensemble d'entraînement implique un arbitrage délicat. D'un côté, une
inclusion trop restrictive peut compromettre les capacités de
généralisation du modèle, c'est-à-dire sa capacité à s'adapter à de
nouvelles données. À l'inverse, une inclusion trop permissive risque de
dégrader la précision du modèle sur les cas plus typiques. Les
chercheur.ses espérant obtenir un modèle maximaliste, quitte à accepter un
certain degré d'erreur et de devoir supprimer les faux
positifs, de nombreux cas limites ont été inclus. Le cas des diagrammes
visibles en transparence (expliqué précédemment) en est un exemple
éloquent.

Une autre difficulté réside dans la définition même du ``diagramme
astronomique''. Les limites de ce concept ne sont pas si claires et
définitives pour les chercheur.ses, et pourtant le modèle a besoin d'une
définition rigoureuse et cohérente. Il paraît en effet difficile de
considérer les diagrammes astronomiques en dehors du contexte des
pratiques d'autres sciences et disciplines connexes. Par exemple, Le
\emph{Flores Almagesti} -- réécriture de l'Almageste datant du \textsc{xv}\ieme par
l'astronome Giovanni Bianchini -- présente une partie algébrique à
l'ouverture mathématique, induisant la présence de nouveaux types de
diagrammes d'inspiration euclidienne. Pour retracer la source de ces
derniers, il est nécessaire de considérer les traités d'Euclide ou
autres travaux d'algèbre. Ceux-ci ne sont pas des traités
\emph{astronomiques}, bien qu'il ne soit pas certain que ces disinctions
contemporaines aient été aussi rigide à l'époque et aient eu un
quelconque sens pour les acteurs historiques. Les sources byzantines
confirment cette complexité~: les diagrammes y sont nommés
\emph{katagraphai}, indépendamment du domaine scientifique auquel ils
appartiennent. Également, de nombreux travaux astronomiques sont groupés
dans des témoins qui contiennent des œuvres issus de domaines divers.
C'est le cas avec les sources chinoises, comme le \emph{Chongzhen
lishu}, qui se présente généralement annexé d'une série de traités
mathématiques. Par conséquent, les diagrammes euclidiens ont été gardés
lors de la préparation des données, et l'algorithme de détection les
classe comme ``diagramme'', même s'ils ne constituent pas l'objet
principal des chercheur.ses.

En ce qui concerne les autres types de diagrammes non strictement
astronomiques (géométriques, harmoniques, logiques, illustrations de
constellations), une approche plus sélective a été adopté afin d'éviter
un modèle trop maximalistes. Ces éléments, bien que potentiellement
intéressants, n'ont pas été inclus dans la phase de détection
automatique.

Ainsi l'œil de la \cv contraint à des choix méthodologique
potentiellement inconfortables, mais en même temps il peut aider à
mesurer les impulsions des chercheur.ses, à mieux définir les objectifs de
recherche et à prioriser les éléments les plus pertinents. Ainsi, la
vision par ordinateur oblige les chercheur.ses à s'adapter à une logique
algorithmique qui, tout en limitant certaines interprétations
subjectives, offre l'opportunité de développer des modèles conceptuels et des méthodologies très rigoureuses.

\vspace{2cm}

L'optimisation d'un modèle revient à trouver un équilibre entre la
disponibilité des données et leur réalisme. Trop peu de données, c'est
un modèle qui sur-apprend -- qui colle trop aux données -- ou qui
sous-apprend -- qui est incapable de modéliser la relation entre
l'entrée et la prédiction attendue. On ne peut donc pas se contenter des
données réelles dont le volume fait défaut dans le cas des documents
historiques. Néanmoins se limiter aux données trop générales ou
synthétiques ne suffit pas non plus, au risque de construire un modèle
trop généraliste. Le modèle doit se spécialiser sur un corpus et
l'utilisation de données réelles est essentielle pour en capturer la
complexité et la diversité.

L'annotation des jeux de données se trouve au cœur de cette tension~:
comment concilier la nécessité de standardiser les données pour assurer
la cohérence et la réutilisabilité, tout en préservant la richesse et la
complexité des données telles qu'elles sont perçues par les experts~? En
effet, si la standardisation est indispensable pour former des modèles
performant et combler le manque de données réelles, elle peut parfois
aplatir les nuances et les subtilités que seuls les chercheur.ses sont en
mesure de saisir. Encore une fois, un équiblibre doit être trouvé.
            
        \clearemptydoublepage
        
\hypertarget{chapitre-6-vers-edition}{%
\chapter{Un outils d'édition des diagrammes pour la cohésion des pratiques}\label{chapitre-6-vers-edition}}

            L'astronome Martin Rees décrit le but de la science de la manière
suivante~:

\begin{kwote}
``The aim of science is to unify disparate ideas, so we don't need to
remember them all. I mean we don't need to recognize the fall of every
apple, because Newton told us they all fall the same way.''\footcite{rees_cosmic_2013}
\end{kwote}

Cette remarque saisit une différence essentielle entre les sciences
dures et les sciences humaines. Les chercheur.ses en sciences humaines
s'intéressent aux pommes en raison de leur
diversité. Ils veulent avoir une idée de ce qu'est une pomme générique,
tout en observant les formes et nuances de chaque pomme. De même, les
chercheur.ses d'\eida veulent pouvoir produire une version unifiée des
diagrammes présents dans les sources primaires, tout en gardant un accès
aux diverses formes qu'ils ont pris à travers le temps et l'espace,
comme on réussit à le faire pour le texte. Cependant le constat de
l'absence de dispositif adéquat ou de recommandation pour l'édition des
figures s'impose. Actuellement, chaque chercheur.se, projet ou éditeur
utilise ses propres méthodes.

Ainsi, une des finalités du projet \eida est de réfléchir aux normes d'une édition native numérique d'un corpus de diagrammes,
qui autoriserait l'étude de leurs dimensions matérielles et
historiographiques, ainsi qu'un outil numérique qui définirait des
pratiques partagées dans la communauté scientifique. 

\hypertarget{perspectives-ouvertes-par-eida}{%
\section{Perspectives ouvertes par EIDA}\label{perspectives-ouvertes-par-eida}}

Le module de base est un package pour la gestion documentaire, duquel
l'application ne peut se détacher. Celui-ci inclut tout d'abord des
formulaires pour l'intégration des documents dans la base de données. Le
modèle de données permet de décrire différentes entités qui, bien que
liées dans leurs métadonnées, peuvent être intégrées indépendamment. Le
module de base permet également la création de \mans \iiif pour
chaque numérisation, permettant ensuite la visualisation des documents
grâce aux outils open-source dédiés. De ce fait, l'indexation de zones
d'image peut être réalisée manuellement via l'interface Mirador intégrée
à \sas. Ce noyau fonctionnel inclut en outre la sélection de lots de
documents (le ``panier''), sur lesquels pourront être effectués des
traitements groupés paramétrables.

Les briques fondamentales offrent donc les fonctionnalités essentielles
de gestion documentaire (intégration, modèle de données, \iiif). Les
traitements, quant à eux, sont gérés par des modules séparés, et c'est
sur cette structure que repose la modularité et l'évolutivité de
l'application.

Ci-après nous donnons une description détaillée de certaines de ces
fonctionnalités de base.

\hypertarget{description-des-donnees}{%
\subsection{Description des
données}\label{description-des-donnees}}

Le module de base contient un modèle de données suffisamment extensif
pour décrire efficacement une diversité de données, allant de documents
textuels historiques à des tableaux en histoire de l'art. La
tripartition entre témoin (\wit), série (qui contient un ensemble de
témoins), et contenu permet un alignement avec des corpus très
diversifiés et des données potentiellement hétéroclites, telles que des
manuscrits, des documents épistolaires, des inventaires de galeries
d'art, et même pourquoi pas des cartes\ldots{}

Pour ouvrir à cette large diversité de données, la liste des types de
pagination témoin doit être étendue \emph{a minima} d'un nouveau type
``other'', émancipant l'enregistrement des mentions de pagination. Les
développements futurs prévoient aussi la création d'un système pour
ajouter facilement un nouveau type\footnote{Le type de témoin est une
  métadonnée rentrée par l'utilisateur.rice lors de l'enregistrement du
  \wit dans la base de donnée. Il choisit le type dans une liste,
  originellement manuscrit, imprimé ou gravure sur bois.} de \wit
(tel que peinture, catalogue, etc.).

Au fil des développements, des débats ont émergé autour de l'ajout dans
le modèle de données d'un niveau de granularité supplémentaire pour
décrire des images ou zones d'images unitaires
(\graphicals), créant ainsi une entité détachée du fait
qu'elle provienne d'une extraction dans un document. Cette solution
aurait permis une description plus détaillée et plus fine des images,
importante pour des projets axés sur des images uniques, et aurait
favorisé un élargissement du spectre des type de sources pris en charge.
L'utilisateur.rice aurait pu soit importer une image unique (et de manière
optionnelle, la lier à un \wit) via un formulaire, soit sélectionner
une région d'image d'intérêt au sein des extractions (annotations \sas),
laquelle serait enregistrée comme \graphical, puis l'enrichir de
métadonnées. Dans les deux cas l'enregistrement d'un \graphical
aurait donné lieu à la création d'une \digit au format \jpeg.

Sans l'unité de description \graphical, les régions d'images
sont créées uniquement via les annotations \sas.

L'intégration de cette entité au sein du modèle aurait offert plusieurs
avantages en termes de cohérence et de flexibilité. En s'alignant sur
les structures existantes (\wits et \sers), elle aurait permis une
manipulation plus intuitive des images, facilitant ainsi les opérations
de recherche et la création de \emph{Sets} personnalisés. De plus, elle
aurait rationalisé la gestion des annotations \sas, permettant de
sélectionner les plus pertinentes dans la multitude existante.

Cependant, cette approche présente des limites, et on peut trouver des
alternatives. Tout d'abord, la coexistence de \graphicals avec les
annotations \sas, générées par des processus distincts, aurait pu créer
une certaine confusion quant à leur nature et à leur méthode de
création. De plus, la multiplication potentielle de milliers
d'enregistrements aurait pu impacter les performances de la base de
données et complexifier les requêtes. Enfin, le lien sémantique ambigu
et sujet à interprétation subjective entre \graphical et \wit
aurait compliqué les possibilités de corrélation.

Compte tenu de ces limites, il a semblé préférable de maintenir les
annotations \sas pour identifier les instances de base du modèle, sans
créer de nouvelle unité de description. La solution actuelle reste donc
basée sur la création manuelle ou automatique de zones dans les images
via \iiif et \sas, évitant les problèmes de redondance et de confusion.
Bien que l'entité \graphical n'ait pas été implémentée, les
fonctionnalités d'annotation et de sélection d'images sont assurées par
d'autres mécanismes. L'outil Mirador permet d'associer des tags aux
zones d'image, offrant ainsi une première couche d'enrichissement
sémantique. La sélection dans un \emph{set} personnalisé sera possible en
gardant en mémoire une référence contenant des coordonnées du
\emph{crop}. De plus, l'importation d'images individuelles est
réalisable en les considérant comme des \emph{Witness partiels}, ce qui
permet de les intégrer dans le \textit{workflow} existant. Toutefois
l'enrichissement sémantique à un niveau de granularité fin restera
limité~; et la dépendance à l'outil \sas constitue une potentielle dette
technique, susceptible de restreindre les évolutions futures du système.

Afin de mieux répondre aux exigences de modularité, l'évolution du
modèle de données s'oriente non pas vers une description individuelle
des documents, mais vers la gestion des traitements. Cette évolution
implique la création d'une entité \tr
liée à des ensembles de données (\ds et
\rs) potentiellement hétérogènes.

\hypertarget{principe-du-traitement}{%
\subsection{Principe du Traitement}\label{principe-du-traitement}}

Le but fondamental de la plateforme est de pouvoir effectuer plusieurs
actions sur les objets de la base. Afin d'assurer une meilleure
traçabilité et plus de flexibilité, la plateforme abandonne les
lancements automatiques des processus\footnote{C'était initialement le
  cas de l'extraction des entités, dont le lancement était lié à une
  méthode de classe liée à la \digit après soumission d'un
  formulaire d'ajout d'un \wit ou d'une \ser. L'action se
  lançait immédiatement après enregistrement des images d'une
  numérisation dans la plateforme.} au profit d'un système basé sur
l'entité \tr. Chaque traitement est associé à un ensemble
d'objets traités ensemble (\ds ou \rs), à un jeu de
paramètres et à un résultat. Ces informations sont stockées dans une
table dédiée. Cette approche facilite la gestion et le trackage des
processus (notamment, les utilisateur.rices sont notifiés par e-mail à la fin
du \textit{processing}), permet aux utilisateur.rices de consulter un historique de
leurs actions et offre la possibilité de créer des \textit{workflows}
personnalisés en passant par un formulaire de lancement unique mais
extensif.

En permettant de regrouper des documents de types différents (\wos,
\sers, \wits) dans des \dss, on offre à
l'utilisateur.rice la flexibilité de lancer des actions sur des ensembles
d'entités hétérogènes et granulaires. Le traitement est ensuite réparti
sur les entités de niveau inférieur (les témoins). Les \wits ainsi
sélectionnés peuvent être soumis à une large gamme de traitements~: des
fonctions déjà implémentées comme l'exportation (avec choix du
format), l'extraction, la vectorisation, la recherche de similarité~; ou
de nouveaux traitements personnalisés, tels que la visualisation sur une
frise chronologique ou une carte. La modularité de la plateforme est
assurée par un formulaire de lancement configurable, permettant de
l'adapter à différents scénarios d'utilisation, et à l'ajout de modules
personnalisés.

Le \rs fonctionne similairement au \ds, à un niveau
de granularité inférieur (à l'échelle de la zone d'image)\footnote{À
  l'été 2024, l'entité n'existe pas encore dans la base de données, mais
  le processus d'envoi du traitement et les modes de communication entre
  l'application et l'\api prévoient la possibilité de lancer l'inférence
  des modèles sur un ensemble de régions extraites.}.

\hypertarget{extraction-des-zones-dimage-manuelle}{%
\subsection{Extraction manuelle des zones d'image}\label{extraction-des-zones-dimage-manuelle}}

Le choix de la méthode d'extraction des régions d'intérêt dans les
documents constitue un élément clé de la modularité de la plateforme.
Les utilisateur.rices peuvent opter pour une extraction manuelle ou une
extraction automatique basée sur des algorithmes de vision par
ordinateur, adaptée aux traitements à plus grande échelle.

Après importation d'un enregistrement, le flux de travail procède à la
création de \mans \iiif pour chaque numérisation
(\digit) afin de permettre une visualisation grâce à la
plateforme Mirador. Le module de base autorise par la suite
l'extraction manuelle de zones d'intérêt au sein des images. Cette
fonctionnalité est particulièrement utile pour les projets ne souhaitant
pas recourir à des méthodes entièrement automatisées de vision par
ordinateur. L'outil \sas permet de créer des annotations, c'est-à-dire de
définir des régions d'intérêt spécifiques dans les numérisations, et de
les indexer directement dans les \mans \iiif correspondants,
enrichissant ainsi les ressources numériques. De plus, les
développements futurs prévoient la possibilité d'importer des fichiers
d'annotation préexistants en format .\textsc{txt} afin de pouvoir les indexer
manuellement. Par conséquent, le \textit{workflow} de base ne comporte aucun
traitement automatique basé sur la vision (et de fait éventuellement
trop gourmand en puissance de calcul).

L'extraction, qu'elle soit manuelle ou automatique, constitue le
fondement du reste des processus. Une interface est disponible pour
sélectionner un ensemble de documents et effectuer des actions
spécifiques sur les témoins annotés, via le formulaire de traitement qui
s'étend selon un choix de module configuré. Ainsi l'utilisateur.rice n'est
pas limité par un contexte initial, à l'origine deux étapes
indissociables et incontournables (importation et extraction), pour
pouvoir effectuer d'autres actions. Cette modularité permet de
s'affranchir d'un \textit{workflow} linéaire et prédéfini, offrant ainsi une plus
grande adaptabilité aux besoins spécifiques et aux ressources
matérielles des projets.

Pour conclure, l'existence de ce module de base répond à des besoins
élémentaires des projets de recherche en études visuelles. Il fournit un
outil qui permet d'agréger toutes les sources primaires qui concernent
le sujet, de décrire les sources et de les mettre en relation. Il offre
en outre la possibilité d'extraire et visualiser des contenus d'intérêt
(les ``crops'' d'images), ciblant ainsi les instances de base qui
intéressent les chercheur.ses.

\hypertarget{une-problematique-de-standardisation}{%
\section{Une problématique d'unification des pratiques}\label{une-problematique-de-standardisation}}

Spécialiser un modèle d'intelligence artificielle implique de lui
fournir des données pertinentes, diversifiées, et en quantité
suffisante. Cependant, pour certains domaines, dont l'histoire fait
partie, le volume de données disponible est insuffisant. Ce constat est
d'autant plus vrai dans le cas des diagrammes issus de traités
astronomiques~: les corpus de documents scientifiques historiques
contiennent généralement du texte en majeure partie, des tables et des
images, négligeant souvent les diagrammes.\footnote{Exception faite du
  corpus S-VED (\cite{buttner_cordeep_2022}), collection
  d'illustration très diverses contenant entre autre des diagrammes
  historiques. Mais les primitives ne sont pas annotées.}. De plus, ils
sont dénués d'annotations précises sur les éléments constitutifs des
pages~; c'est sans parler de l'inexistence d'un corpus de diagrammes
dont les primitives sont annotées. Or l'annotation est une tâche
chronophage et fastidieuse. Le recours aux données synthétique répond,
mais en partie seulement, à ces problématiques.

\hypertarget{datasets-synthetiques}{%
\subsection{\emph{datasets} synthétiques}\label{datasets-synthetiques}}

Les \textit{datasets} synthétiques sont générés par des algorithmes ou des
méthodes de simulation pour imiter des données réelles, sans être
directement extraites de sources existantes. De tels jeux de données
sont utilisés lorsque les données réelles sont limitées ou difficiles à
obtenir, mais qu'il est cependant nécessaire de contrôler spécifiquement
les caractéristiques des données d'entraînement\footcite{buttner_cordeep_2022}. La génération
d'images a pour but de fabriquer des ensembles de données plus vastes,
plus diversifiés, très variables et assez complexes, répondant aux
caractéristiques des objets d'intérêt du projet, et surtout étiquetés
automatiquement, sans recourir à l'annotation manuelle.

Ces données synthétiques sont assez ressemblantes et complexes pour être
exploitées. Par exemple, docExtractor est un modèle off-the-shell (au
même titre que \yolo) envisagé dans le cadre de la tâche d'extraction des
diagrammes, et qui se veut sépcifique aux données historiques, car il
est entraîné sur des données produites par un générateur de documents
historiques synthétiques~: SynDoc\footcite{monnier_docextractor_2020}. SynDoc
génère des images de manière aléatoire en combinant des éléments
graphiques (fonds, images, texte et bruit) provenant d'un jeu d'image
défini (constitué de 177 images de pages, 15 contextes, plus de 8000
œuvres d'art provenant de WikiArt, des lettrines générées à partir d'une
lettre aléatoire avec 91 fonts possibles, et des dessins, schémas et
textes tirés d'articles aléatoires sur Wikipedia, avec plus de 400
fonts). Les différents éléments s'agencent, intégrant sur le fond
images, texte et bruit, offrant des combinasons et des mises en pages
assez complexes. Chaque élément de contenu est pré-annoté, éliminant
ainsi le besoin d'annotations manuelles pour ces pages.

          \begin{figure}[H]
          \begin{center}
          \includegraphics[height=6.5cm]{figues/syndoc.jpg}
          \end{center}
          \caption{Données synthétiques générées par SynDoc.\footcite[p.46]{norindr_traitement_2023}}
          \label{fig:syndoc} \end{figure}

Pour entraîner le modèle de vectorisation, il a de même été nécessaire
d'utiliser des données synthétiques. Parce qu'annoter les primitives
géométriques dans des images de diagrammes complexes est très
chronophage, le modèle de vectorisation a été pré-formé sur des corpus
artificiels générés dynamiquement. Le script de génération des données
d'entraînement choisit aléatoirement un arrière-plan, y ajoute des mots,
des nombres et des glyphes puis crée artificiellement un diagramme en
insérant des segments, des cercles et des arcs. Le script est conçu pour
que ces diagrammes aient une forte probabilité de présenter des formes
très caractéristiques comme les cercles concentriques et tangents, les
lignes parallèles et les arcs connectés, afin de simuler les structures
typiques. Les primitives sont dessinées avec des
variations aléatoires d'opacité, de largeur et de couleur. Les cercles
peuvent être remplis ou vides. Enfin, du bruit est ajouté en appliquant
un flou gaussien, et en supprimant de petites régions du diagramme pour
imiter la dégradation des documents historiques. Les données
d'entraînement ainsi générées présentent des configurations assez
complexes.

          \begin{figure}[H]
          \begin{center}
          \includegraphics[height=7cm]{figues/vecto_synthetic_data.png}
          \end{center}
          \caption{Données synthétiques générées pour l'entraînement du modèle de vectorisation.\footcite[Figure issue de la présentation de Syrine Kalelli à l'occasion de la conférence \eida 2024~:][]{noauthor_eida_nodate-1}}
          \label{fig:vecto_synthetic} \end{figure}

Enfin, le modèle de similarité présente un troisième exemple, puisque
SegSwap est pré-entraîné sur de la donnée synthétique. Le script de
génération prend des parties aléatoires d'une images et les copie-colle
au-dessus d'une autre image. Les trois images (source, cible et
superposition) sont placées dans le même dataset d'entraînement, ainsi
le modèle apprend à retrouver ce qui, dans la superposition, vient de la
source, et ce qui vient de la cible.

          \begin{figure}[H]
          \begin{center}
          \includegraphics[height=3cm]{figues/segswap_blended_images.png}
          \end{center}
          \caption{Données d'entraînement du modèle Segswap.}
          \label{fig:segswap} \end{figure}

\hypertarget{les-donnees-reelles}{%
\subsection{Les données réelles}\label{les-donnees-reelles}}

S'appuyer sur les modèles \textit{off-the-shelf}, sur de larges \textit{datasets}
généralistes, ou sur des données synthétiques permet une implémentation
facilitée de la vision dans des projets et constitue une base solide.
Toutefois, les sources tenant aux deux projets (\vhs et \eida) sont trop
spécifiques pour se contenter de modèles généralistes ou formés sur des
données artificielles. Même si ces derniers peuvent offrir des performances
de base, ils risquent de manquer de précision et de sensibilité aux
particularités des documents historiques. Les corpus artificiels
présentent des configurations délibérément complexes pour s'approcher le
plus possible des difficultés que le modèle pourrrait rencontrer sur les
données réelle. Elles sont cependant irréalistes et insuffisantes pour
permettre aux modèles de généraliser sur des diagrammes réels.

En atteste la comparaison des performances de docExtractor et \yolov sur
les données d'\eida. docExtractor\footcite{monnier_docextractor_2020}, entraîné
sur des données synthétiques mimant les documents historiques serait en
théorie plus adapté au traitement d'images de pages de manuscrits, avec
du texte et des illustration côté à côte, d'autant qu'il intègre des
outils de traitement du texte (notamment pour la segmentation des
lignes)\footnote{\eida envisage l'implémentation d'un outil d'extraction
  et transcription des labels et des textes qui entourent les diagrammes}.
Pourtant, sans fine-tuning sur des données réelles, il présente des
performances équivalentes à celles de \yolov\footcite[p.45]{norindr_traitement_2023}. Cela
souligne que même les modèles off-the-shelf entraînés sur un corpus
assez spécifique et complexe, mais synthétique, ne dispense pas d'un
entraînement sur des données réelles, au même titre que les modèles très
généralistes comme \yolov.

Alors, le modèle de base \yolov tel que mis à disposition par
Ultralytics est entraîné sur de grands ensembles de données réelles, ce
qui constitue une base solide pour la classification des objets du
monde. L'utilisation de SynDoc permet ensuite de compléter
l'apprentissage initial en exposant le modèle à des exemples variés et
spécifiques aux documents historiques, augmentant ainsi sa capacité de
généralisation. Ces similis de manuscrits anciens offrent l'avantage de
pouvoir être produits en grandes quantités et de couvrir un large
éventail de scénarii et de configurations difficiles à obtenir dans des
ensembles de données réelles. Puis le modèle est entraîné sur les
données de \vhs, qui sont de réelles pages de documents historiques
contenant une large diversité d'illustrations. Ces données apporteront
une dimension supplémentaire de pertinence au modèle, en l'exposant à
des particularités des documents historiques réalistes. Enfin, \yolov
est entraîné sur les données d'\eida, qui sont orientées spécifiquement
vers les diagrammes, afin qu'il détecte uniquement ces derniers.

Quant au modèle de vectorisation développé par Syrine
Kalleli\footcite{kalleli_historical_2024}, il est formé
sur des données synthétiques générées à la volée par un script. Mais le
corpus de diagrammes d'\eida est particulièrement caractéristique et le
modèle n'aurait pu être optimal sans avoir appris sur des images de
diagrammes issus de manuscrits réels. Un corpus d'entraînement de 303
diagrammes extraits de manuscrits et de gravures a donc été constitué et
annoté par les historien.nes. Ces diagrammes sont issus de sources latines, arabes,
grecques, hébreuses ou chinoises, datant du \textsc{xii}\ieme au \textsc{xviii}\ieme siècle, et ils
présentent en guise d'étiquettes plus de 3000 lignes, cercles et arcs. Le
ré-entraînement a permis le transfert des connaissances acquises sur la
tâche de détection des primitives sur les données réalistes.

Il sera également possible d'obtenir des meilleurs résultats sur la
similarité grâce à une évaluation des scores (qui constitue un jeu de
données annotées) et le ré-entraînement du modèle, pour donner des
résultats plus adaptés à la spécificité des données historiques.

D'ailleurs, cette étape d'annotation (le choix des exemples et des
étiquettes) revêt des enjeux importants. L'apprentissage spécifique se
fait à partir de données sélectionnées par les chercheur.ses~: les exemples
sur lequel l'algorithme d'apprentissage va itérer définissent le modèle.
Il est nécessaire de constituer un échantillon de données aléatoire et
représentatif, et de l'annoter en fonction de ce que l'on souhaite
obtenir en prédiction.

L'annotation des jeux de données est non seulement une étape clé, mais
aussi un bel exemple de collaboration chercheur.ses-ingénieur.es. Elle
nécessite la définition de normes pertinentes et rigoureuses. Travail
minutieux et chronophage, l'étiquetage des données peut engendrer des
erreurs et du bruit dans les données, car elle implique la subjectivité
des chercheur.ses et le regard parfois trop précis sur les sources desquels
les annotateurs sont experts.

Voici un exemple rencontré lors de la préparation des données pour
entraîner un modèle de segmentation du contenu textuel. Les sources
arabes et chinoises sont particulièrement verbeuses et les diagrammes
sont très souvent entourés des blocs de commentaires se mélangeant alors
aux légendes et aux labels. Doit-on considérer ces commentaires comme
faisant partie des éléments que l'on souhaite identifier ou bien les ignorer
? Cette décision est importante car si on les ignore, le modèle risque
de passer à côté d'éléments textuels pertinents. En revanche, si on les
inclut, il ramènera des commentaires sans rapport direct avec le
diagramme observé. On voit ici comment la binarité des modèles, qui se
reflète dans les normes d'annotation, est problématique et constitue une
limite au \ml. Un compromis doit être trouvé entre
l'automatisation, qui requiert une normalisation, des définitions
claires et binaires, et la nuance dans l'interprétation des
sources\footnote{Dans le cadre du projet, il a toujors été plus
  intéressant d'opter pour une définition extensive des objets à
  détecter, car prévision d'une correction des traitement. Et il est
  plus facile de supprimer un élémént pas pertinent que d'aller en
  rechercher un, surtout compte tenu de la taille des corpus des
  chercheur.ses. Vaut aussi pour la préparation des données pour
  l'entraînement du modèle d'extraction.}.

La normalisation peut bénéficier à l'écosystème de recherche dans le
domaine de l'\htr et de l'\ocr. À ce titre, il est pertinent d'envisager
l'utilisation du vocabulaire contrôlé SegmOnto pour l'annotation du
contenu textuel entourant les diagrammes. Cela permettrait de créer des
jeux de données réutilisables, à partager avec des projets poursuivant
des objectifs similaires.\footnote{https://segmonto.github.io/}. Encore
une fois, un compromis doit être trouvé entre les besoins de description
des chercheur.ses et les possibilités offertes par les vocabulaires
contrôlés.

Un autre exemple concerne le dernier entraînement du modèle d'extraction
: les résultats montrent que des diagrammes sont encore détectés en
transparence. La question s'est alors posée de chercher à corriger ce
défaut en donnant au modèle, à l'occasion d'un nouvel entraînement,
d'avantage d'exemples négatifs (diagrammes visibles par transparence
mais non annotés). Or il est préférable de se contenter de la correction
ou suppression manuelle de ces prévisions erronées, garantissant que le
modèle parvienne à détecter les diagrammes presque effacés.

Pour assurer la rigueur et la cohérence des annotations, les décisions
prises entre les chercheur.ses et les ingénieur.es peuvent être l'objet d'une
documentation ou d'ateliers d'annotation.

\hypertarget{loeil-de-la-machine-avantages-et-limites}{%
\subsection{L'oeil de la machine~: avantages et
limites}\label{loeil-de-la-machine-avantages-et-limites}}

Bien qu'il soit possible d'optimiser les performances d'un modèle
d'apprentissage automatique en l'entraînant sur un ensemble de données
spécifique, son interprétation des données reste limitée car
fondamentalement binaire, ce qui le rend parfois déficient pour la
recherche en histoire. Ainsi, il gèrera difficilement les cas limites et
ambigüs. La décision d'inclure ou d'exclure ces cas particuliers de
l'ensemble d'entraînement implique un arbitrage délicat. D'un côté, une
inclusion trop restrictive peut compromettre les capacités de
généralisation du modèle, c'est-à-dire sa capacité à s'adapter à de
nouvelles données. À l'inverse, une inclusion trop permissive risque de
dégrader la précision du modèle sur les cas plus typiques. Les
chercheur.ses espérant obtenir un modèle maximaliste, quitte à accepter un
certain degré d'erreur et de devoir supprimer les faux
positifs, de nombreux cas limites ont été inclus. Le cas des diagrammes
visibles en transparence (expliqué précédemment) en est un exemple
éloquent.

Une autre difficulté réside dans la définition même du ``diagramme
astronomique''. Les limites de ce concept ne sont pas si claires et
définitives pour les chercheur.ses, et pourtant le modèle a besoin d'une
définition rigoureuse et cohérente. Il paraît en effet difficile de
considérer les diagrammes astronomiques en dehors du contexte des
pratiques d'autres sciences et disciplines connexes. Par exemple, Le
\emph{Flores Almagesti} -- réécriture de l'Almageste datant du \textsc{xv}\ieme par
l'astronome Giovanni Bianchini -- présente une partie algébrique à
l'ouverture mathématique, induisant la présence de nouveaux types de
diagrammes d'inspiration euclidienne. Pour retracer la source de ces
derniers, il est nécessaire de considérer les traités d'Euclide ou
autres travaux d'algèbre. Ceux-ci ne sont pas des traités
\emph{astronomiques}, bien qu'il ne soit pas certain que ces disinctions
contemporaines aient été aussi rigide à l'époque et aient eu un
quelconque sens pour les acteurs historiques. Les sources byzantines
confirment cette complexité~: les diagrammes y sont nommés
\emph{katagraphai}, indépendamment du domaine scientifique auquel ils
appartiennent. Également, de nombreux travaux astronomiques sont groupés
dans des témoins qui contiennent des œuvres issus de domaines divers.
C'est le cas avec les sources chinoises, comme le \emph{Chongzhen
lishu}, qui se présente généralement annexé d'une série de traités
mathématiques. Par conséquent, les diagrammes euclidiens ont été gardés
lors de la préparation des données, et l'algorithme de détection les
classe comme ``diagramme'', même s'ils ne constituent pas l'objet
principal des chercheur.ses.

En ce qui concerne les autres types de diagrammes non strictement
astronomiques (géométriques, harmoniques, logiques, illustrations de
constellations), une approche plus sélective a été adopté afin d'éviter
un modèle trop maximalistes. Ces éléments, bien que potentiellement
intéressants, n'ont pas été inclus dans la phase de détection
automatique.

Ainsi l'œil de la \cv contraint à des choix méthodologique
potentiellement inconfortables, mais en même temps il peut aider à
mesurer les impulsions des chercheur.ses, à mieux définir les objectifs de
recherche et à prioriser les éléments les plus pertinents. Ainsi, la
vision par ordinateur oblige les chercheur.ses à s'adapter à une logique
algorithmique qui, tout en limitant certaines interprétations
subjectives, offre l'opportunité de développer des modèles conceptuels et des méthodologies très rigoureuses.

\vspace{2cm}

En conclusion, afin de sortir des pratiques aléatoires et de garantir un
cadre cohérent et partagé des pratiques éditoriales, une
unification des méthodes d'édition de diagrammes astronomiques
s'impose. En s'affranchissant des contraintes de la page et de
l'imprimé, une édition numérique ouvre la voie à la définition de
pratiques normalisées -- servant la fonction de légitimation de l'édition scientifique -- et de nouvelles
formes d'interactivité -- appuyant la fonction de médiation de l'édition\footcite[Sur les trois fonctions de l'édition, voir][]{epron_ledition_2018}. L'illustration acquiert
ainsi un statut équivalent à celui du texte, en tant que vecteur
d'arguments scientifiques et témoin des pratiques des acteurs
historiques. Néanmoins, de nombreuses questions restent en suspens quand
à la mise en œuvre pratique de ces nouvelles méthodes.
            
        \clearemptydoublepage

        \chapter*{Conclusion partielle}

L'application de techniques de \cv aux documents historiques
se heurte à divers obstacles. Les modèles pré-entraînés sur des données
génériques peinent à capturer les spécificités visuelles et les
complexités propres aux données réelles. En outre, l'entraînement de modèles
spécifiques est contraint par la rareté de jeux de données historiques
annotées. Les annotations, aussi, doivent être assez précises pour
satisfaire les chercheur.ses et assez extensives pour permettre au modèle de
généraliser.

Toutefois l'émergence de nouvelles architectures de réseaux de neurones
et de nouvelles méthodes d'apprentissage profond permet d'améliorer les
performances des modèles sur des documents spécifiques. Par ailleurs, la
génération de données synthétiques constitue une approche prometteuse
pour pallier le manque de données réelles annotées. En simulant des
documents historiques, en imitant les dégradations et le trait manuel qui
les caractérise, il est possible d'enrichir les jeux de données
d'entraînement. Enfin, la collaboration entre les ingénieur.es de la
donnée et les spécialistes permet de définir des protocoles d'annotation
adaptés. En trouvant un compromis entre la finesse d'analyse requise par
les chercheur.ses et les contraintes de représentativité des modèles, il
est possible de créer des jeux de données réelles suffisamment
qualitatifs. Pour affiner les résultats, il est nécessaire d'intégrer un
processus de correction des sorties des modèles dans la boucle
d'apprentissage, afin d'améliorer progressivement leur précision en
récupérant des jeux de données annotés à partir des prédictions. Ces
processus de vérification sont aussi essentiels pour garantir la
pertinence des résultats et compenser la binarité de l'œil de la
machine.

Malgré cette étape de correction, l'automatisation des processus grâce
aux modèles de vision constitue un gain de temps pour les chercheur.ses et
permet de traiter des corpus dont la taille aurait rendu le traitement
manuel inenvisageable~: ils permettent d'extraire les illustrations des
documents, puis de chercher, pour chaque entité, leurs homologues les
plus similaires, et enfin de générer des représentations numériques
sémantiquement riches et manipulables à partir des images (\svgs). Ces
dernières pourrait ainsi servir de base à la création d'un outil qui
systématise les modalités d'édition des diagrammes, proposant une
méthodologie rigoureuse.

En effet, un outil définit un cadre et des pratiques, il porte une
vision des sources et une méthode. C'est ce sur quoi portera la partie
III. On voudra montrer quelles réflexions, défis et solutions tiennent
au développement d'une plateforme qui, tout en fournissant un cadre
méthodologique solide, offre une grande souplesse d'utilisation,
permettant aux chercheur.ses de personnaliser leurs \textit{workflows} en fonction de
leurs besoins spécifiques. En somme, comment concevoir un véritable
système d'information extensif capable de soutenir l'ensemble du
processus de traitement des illustrations présentes dans les documents
historiques, de l'acquisition des données à leur exploitation
scientifique.

    \part{Une plateforme pour une méthode reproductible et transposable}

\chapter*{Introduction partielle}

La modularité des projet \eida / \vhs repose en grande partie sur la
construction de la plateforme web \aikon~: outil complet pour la
recherche sur des documents historiques, permettant leur import,
leur stockage, leur analyse (par différentes méthodes) puis leur visualisation. Le but
est que tout projet tenant des \hn puisse utiliser la plateforme. Ainsi
l'application vise à être réutilisable et le code sera mis à disposition
en accès libre sur GitHub.

Cette plateforme, pensée pour être agnostique quant au domaine
d'application spécifiques, doit être capable d'accueillir des corpus de
données divers pour tout chercheur.se ou projet en étude visuelle
souhaitant adopter la méthodologie qu'elle porte. Elle doit également
prendre en compte la diversité des ressources humaines et matérielles
disponibles. Elle doit autoriser l'intégration modulable d'outils et algorithmes, notamment
ceux issus du domaine de la \cv, pour traiter des motifs
visuels et des objets présents dans les sources. En adoptant cette
approche, elle se veut environnement de recherche collaboratif où les
chercheur.ses peuvent partager leurs données, méthodes et résultats.

Le défi consiste donc à créer une application aux fonctionnalités
suffisamment spécifiques pour répondre aux besoins des deux projets qui
portent son développement (\vhs et \eida), tout en étant suffisamment
généralistes pour que la plateforme puisse être réemployée à l'avenir
par d'autres projets. L'objectif de l'application \aikon est alors de porter une
méthode de traitement et d'assurer sa possible transposition dans des
environnements techniques et disciplinaires différents. La stratégie de
développement consistera alors en l'assemblage de briques techniques,
indépendantes le plus possible, et réutilisables seules ou ensemble.

Le développement d'architectures flexibles et évolutives comporte des
défis liés à la conception de systèmes adaptables à des
besoins changeants et à des environnements hétérogènes. Il implique la
modularisation des modèles de données, l'interopérabilité des formats et
protocoles, l'architecture du code, les contraintes liées aux
ressources matérielles diverses (accès au \textit{hard-ware} et à la puissance de
calcul) et les enjeux liés à l'expérience utilisateur.rice (\ux/\ui).
Dans cette partie nous verrons comment la conception de l'application
\aikon entend répondre à ses enjeux.

Quelles sont les complexités liées à la conception d'un système
d'information capable de s'adapter à des domaines d'application variés
et d'évoluer dans le temps~? Nous explorerons deux piliers de cette
problématique~: l'optimisation des processus internes et la conception
d'interfaces utilisateur.rice intuitives, capable de guider l'utilisateur.rice
dans ces processus.

\clearemptydoublepage

\hypertarget{preambule}{%
\chapterNo{Préambule~: importance d'une plateforme modulaire}\label{preambule}}
        
      L'ouverture des codes, outils et données dans le contexte de la
recherche en \hn, et spécifiquement impliquant l'\ia, est
importante pour plusieurs raisons.

\emph{Traçabilité et Reproductibilité}

L'ouverture du code source de la plateforme garantit la traçabilité des
pipelines d'analyse, favorisant ainsi la transparence des méthodes de
recherche. La plateforme offre un cadre de travail reproductible et
permet de modéliser et standardiser les processus d'analyse, assurant
ainsi une cohérence méthodologique au sein des équipes de recherche et
entre différents projets. Cette approche contribue à renforcer la
confiance dans les résultats scientifiques et à faciliter leur
validation par la communauté. Elle favorise aussi le partage des méthodes au sein de la communauté des chercheur.ses. 

\emph{Modularité, Solidité et Pérennité}

L'objectif est de développer un outil qui pourra être réutilisé,
maintenu, et pourra évoluer dans le temps, contrairement à l'usage
qui voit souvent des outils de recherche devenir obsolètes et difficiles
à maintenir une fois les projets de recherche terminés.

Cette problématique est particulièrement prégnante dans les \hn, où les outils développés sont fréquemment abandonnés à la
fin des financements, faute de stratégie de pérennisation. Par exemple,
la difficulté de maintenir la plateforme \dishas est exprimée par
l'équipe DH, ne serait-ce qu'en raison du coût humain lié au temps passé
à son maintien. Qui plus est, l'accord des financements repose bien
souvent sur la justification par des livrables techniques innovants. Et
donc les outils développés précédemment ne sont pas maintenus et
abandonnés.

Un outil qui fonctionne aujourd'hui peut rapidement devenir obsolète ou
inadapté si sa conception n'intègre pas une capacité d'évolution. En
construisant un \si modulaire, il devient possible de prolonger la vie de
l'outil. Dans une optique de sobriété et dans la philosophie de l'\textit{open source}, cette démarche a aussi pour objectif d'éviter à de multiples
projets le développement d'outils aux fonctionnalités similaires.

\emph{Démocratisation des outils de \dl}

L'ouverture des données historiques ou patrimoniales, mais aussi des
technologies qui permettent de les manipuler (outils \textit{low tech},
publication du code, mais aussi et notamment outils d'\ia) produisent de
riche dynamiques de collaboration. Si l'utilisation de la vision
artificielle tend à se généraliser dans le traitement de ces données,
les projets de recherche ne disposent pas toujours des compétences en
interne pour intégrer le \ml, ni les ressources
temporelles, financières et humaines nécessaires. Ils peuvent alors profiter du
soutien d'une communauté et de l'existence d'outils génériques
spécialisables, réduisant ainsi les coûts de développement. En effet, les
algorithmes généralistes ne sont généralement pas adaptés aux données
historiques et sont donc difficiles à utiliser \textit{off-the-shelf} sans
ajustement. \aikon inclue donc des moyens d'export dans des formats
divers, pour créer des vérités de terrain et envisager l'entraînement des modèles, suivant l'exemple de e-Scriptorium.

Il y a donc un enjeu à penser des outils standards, en réfléchissant
techniquement à des solutions maintenables sur le long terme grâce à
des formats interopérables et des infrastructures flexibles, pour
permettre non seulement la reproductibilité des résultats mais aussi une
réutilisation des données produites dans des contextes divers.        
            
        \clearemptydoublepage
        
        \hypertarget{chapitre-7-processus-et-fonctionnalites}{%
\chapter{Processus et
Fonctionnalités}\label{chapitre-7-processus-et-fonctionnalites}}

  

La modularisation est une stratégie de développement applicatif visant à
décomposer un système complexe en unités de code autonomes, appelées
modules. Cette décomposition a pour objectif de minimiser les couplages
entre ces modules, facilitant ainsi leur développement, leur maintenance
et leur réutilisation.

\begin{kwote}
``La programmation modulaire est une solution favorisée pour la création
d'une application réutilisable, puisqu'elle permet le développement de
modules indépendants qui répondent à des besoins spécifiques, et qui
peuvent être réemployés par d'autres projets sans être dépendants du
reste de l'application. Ainsi, les différents traitements appliqués aux
numérisations déposées par les utilisateur.rices font appel à des outils
divers, indépendants, dédiés chacun à une tâche spécifique du \textit{workflow}~:
cette disposition permet l'amélioration et la modification de chacun des
modules sans impacter la structure globale de l'application, et permet
également la récupération d'éléments spécifiques par des programmes
futurs. Ainsi, le cœur de l'application \eida/\vhs (récemment baptisée \aikon) permet le dépôt, le
stockage et l'affichage des numérisations d'ouvrage, ainsi que la
correction des annotations. La détection d'objet est gérée par une \api
{[}\ldots{]}.''\footcite[p.52]{norindr_traitement_2023}
\end{kwote}

Comment penser des processus adaptables et un \textit{workflow} flexible~? Et
quel est l'impact sur les (infra)structures de la plateforme~? En somme,
d'un point de vue technique, comment se manifeste
l'extensivité de l'outil \aikon~?

\hypertarget{module-de-base}{%
\section{Module de base}\label{module-de-base}}

Le module de base est un package pour la gestion documentaire, duquel
l'application ne peut se détacher. Celui-ci inclut tout d'abord des
formulaires pour l'intégration des documents dans la base de données. Le
modèle de données permet de décrire différentes entités qui, bien que
liées dans leurs métadonnées, peuvent être intégrées indépendamment. Le
module de base permet également la création de \mans \iiif pour
chaque numérisation, permettant ensuite la visualisation des documents
grâce aux outils open-source dédiés. De ce fait, l'indexation de zones
d'image peut être réalisée manuellement via l'interface Mirador intégrée
à \sas. Ce noyau fonctionnel inclut en outre la sélection de lots de
documents (le ``panier''), sur lesquels pourront être effectués des
traitements groupés paramétrables.

Les briques fondamentales offrent donc les fonctionnalités essentielles
de gestion documentaire (intégration, modèle de données, \iiif). Les
traitements, quant à eux, sont gérés par des modules séparés, et c'est
sur cette structure que repose la modularité et l'évolutivité de
l'application.

Ci-après nous donnons une description détaillée de certaines de ces
fonctionnalités de base.

\hypertarget{description-des-donnees}{%
\subsection{Description des
données}\label{description-des-donnees}}

Le module de base contient un modèle de données suffisamment extensif
pour décrire efficacement une diversité de données, allant de documents
textuels historiques à des tableaux en histoire de l'art. La
tripartition entre témoin (\wit), série (qui contient un ensemble de
témoins), et contenu permet un alignement avec des corpus très
diversifiés et des données potentiellement hétéroclites, telles que des
manuscrits, des documents épistolaires, des inventaires de galeries
d'art, et même pourquoi pas des cartes\ldots{}

Pour ouvrir à cette large diversité de données, la liste des types de
pagination témoin doit être étendue \emph{a minima} d'un nouveau type
``other'', émancipant l'enregistrement des mentions de pagination. Les
développements futurs prévoient aussi la création d'un système pour
ajouter facilement un nouveau type\footnote{Le type de témoin est une
  métadonnée rentrée par l'utilisateur.rice lors de l'enregistrement du
  \wit dans la base de donnée. Il choisit le type dans une liste,
  originellement manuscrit, imprimé ou gravure sur bois.} de \wit
(tel que peinture, catalogue, etc.).

Au fil des développements, des débats ont émergé autour de l'ajout dans
le modèle de données d'un niveau de granularité supplémentaire pour
décrire des images ou zones d'images unitaires
(\graphicals), créant ainsi une entité détachée du fait
qu'elle provienne d'une extraction dans un document. Cette solution
aurait permis une description plus détaillée et plus fine des images,
importante pour des projets axés sur des images uniques, et aurait
favorisé un élargissement du spectre des type de sources pris en charge.
L'utilisateur.rice aurait pu soit importer une image unique (et de manière
optionnelle, la lier à un \wit) via un formulaire, soit sélectionner
une région d'image d'intérêt au sein des extractions (annotations \sas),
laquelle serait enregistrée comme \graphical, puis l'enrichir de
métadonnées. Dans les deux cas l'enregistrement d'un \graphical
aurait donné lieu à la création d'une \digit au format \jpeg.

Sans l'unité de description \graphical, les régions d'images
sont créées uniquement via les annotations \sas.

L'intégration de cette entité au sein du modèle aurait offert plusieurs
avantages en termes de cohérence et de flexibilité. En s'alignant sur
les structures existantes (\wits et \sers), elle aurait permis une
manipulation plus intuitive des images, facilitant ainsi les opérations
de recherche et la création de \emph{Sets} personnalisés. De plus, elle
aurait rationalisé la gestion des annotations \sas, permettant de
sélectionner les plus pertinentes dans la multitude existante.

Cependant, cette approche présente des limites, et on peut trouver des
alternatives. Tout d'abord, la coexistence de \graphicals avec les
annotations \sas, générées par des processus distincts, aurait pu créer
une certaine confusion quant à leur nature et à leur méthode de
création. De plus, la multiplication potentielle de milliers
d'enregistrements aurait pu impacter les performances de la base de
données et complexifier les requêtes. Enfin, le lien sémantique ambigu
et sujet à interprétation subjective entre \graphical et \wit
aurait compliqué les possibilités de corrélation.

Compte tenu de ces limites, il a semblé préférable de maintenir les
annotations \sas pour identifier les instances de base du modèle, sans
créer de nouvelle unité de description. La solution actuelle reste donc
basée sur la création manuelle ou automatique de zones dans les images
via \iiif et \sas, évitant les problèmes de redondance et de confusion.
Bien que l'entité \graphical n'ait pas été implémentée, les
fonctionnalités d'annotation et de sélection d'images sont assurées par
d'autres mécanismes. L'outil Mirador permet d'associer des tags aux
zones d'image, offrant ainsi une première couche d'enrichissement
sémantique. La sélection dans un \emph{set} personnalisé sera possible en
gardant en mémoire une référence contenant des coordonnées du
\emph{crop}. De plus, l'importation d'images individuelles est
réalisable en les considérant comme des \emph{Witness partiels}, ce qui
permet de les intégrer dans le \textit{workflow} existant. Toutefois
l'enrichissement sémantique à un niveau de granularité fin restera
limité~; et la dépendance à l'outil \sas constitue une potentielle dette
technique, susceptible de restreindre les évolutions futures du système.

Afin de mieux répondre aux exigences de modularité, l'évolution du
modèle de données s'oriente non pas vers une description individuelle
des documents, mais vers la gestion des traitements. Cette évolution
implique la création d'une entité \tr
liée à des ensembles de données (\ds et
\rs) potentiellement hétérogènes.

\hypertarget{principe-du-traitement}{%
\subsection{Principe du Traitement}\label{principe-du-traitement}}

Le but fondamental de la plateforme est de pouvoir effectuer plusieurs
actions sur les objets de la base. Afin d'assurer une meilleure
traçabilité et plus de flexibilité, la plateforme abandonne les
lancements automatiques des processus\footnote{C'était initialement le
  cas de l'extraction des entités, dont le lancement était lié à une
  méthode de classe liée à la \digit après soumission d'un
  formulaire d'ajout d'un \wit ou d'une \ser. L'action se
  lançait immédiatement après enregistrement des images d'une
  numérisation dans la plateforme.} au profit d'un système basé sur
l'entité \tr. Chaque traitement est associé à un ensemble
d'objets traités ensemble (\ds ou \rs), à un jeu de
paramètres et à un résultat. Ces informations sont stockées dans une
table dédiée. Cette approche facilite la gestion et le trackage des
processus (notamment, les utilisateur.rices sont notifiés par e-mail à la fin
du \textit{processing}), permet aux utilisateur.rices de consulter un historique de
leurs actions et offre la possibilité de créer des \textit{workflows}
personnalisés en passant par un formulaire de lancement unique mais
extensif.

En permettant de regrouper des documents de types différents (\wos,
\sers, \wits) dans des \dss, on offre à
l'utilisateur.rice la flexibilité de lancer des actions sur des ensembles
d'entités hétérogènes et granulaires. Le traitement est ensuite réparti
sur les entités de niveau inférieur (les témoins). Les \wits ainsi
sélectionnés peuvent être soumis à une large gamme de traitements~: des
fonctions déjà implémentées comme l'exportation (avec choix du
format), l'extraction, la vectorisation, la recherche de similarité~; ou
de nouveaux traitements personnalisés, tels que la visualisation sur une
frise chronologique ou une carte. La modularité de la plateforme est
assurée par un formulaire de lancement configurable, permettant de
l'adapter à différents scénarios d'utilisation, et à l'ajout de modules
personnalisés.

Le \rs fonctionne similairement au \ds, à un niveau
de granularité inférieur (à l'échelle de la zone d'image)\footnote{À
  l'été 2024, l'entité n'existe pas encore dans la base de données, mais
  le processus d'envoi du traitement et les modes de communication entre
  l'application et l'\api prévoient la possibilité de lancer l'inférence
  des modèles sur un ensemble de régions extraites.}.

\hypertarget{extraction-des-zones-dimage-manuelle}{%
\subsection{Extraction manuelle des zones d'image}\label{extraction-des-zones-dimage-manuelle}}

Le choix de la méthode d'extraction des régions d'intérêt dans les
documents constitue un élément clé de la modularité de la plateforme.
Les utilisateur.rices peuvent opter pour une extraction manuelle ou une
extraction automatique basée sur des algorithmes de vision par
ordinateur, adaptée aux traitements à plus grande échelle.

Après importation d'un enregistrement, le flux de travail procède à la
création de \mans \iiif pour chaque numérisation
(\digit) afin de permettre une visualisation grâce à la
plateforme Mirador. Le module de base autorise par la suite
l'extraction manuelle de zones d'intérêt au sein des images. Cette
fonctionnalité est particulièrement utile pour les projets ne souhaitant
pas recourir à des méthodes entièrement automatisées de vision par
ordinateur. L'outil \sas permet de créer des annotations, c'est-à-dire de
définir des régions d'intérêt spécifiques dans les numérisations, et de
les indexer directement dans les \mans \iiif correspondants,
enrichissant ainsi les ressources numériques. De plus, les
développements futurs prévoient la possibilité d'importer des fichiers
d'annotation préexistants en format .\textsc{txt} afin de pouvoir les indexer
manuellement. Par conséquent, le \textit{workflow} de base ne comporte aucun
traitement automatique basé sur la vision (et de fait éventuellement
trop gourmand en puissance de calcul).

L'extraction, qu'elle soit manuelle ou automatique, constitue le
fondement du reste des processus. Une interface est disponible pour
sélectionner un ensemble de documents et effectuer des actions
spécifiques sur les témoins annotés, via le formulaire de traitement qui
s'étend selon un choix de module configuré. Ainsi l'utilisateur.rice n'est
pas limité par un contexte initial, à l'origine deux étapes
indissociables et incontournables (importation et extraction), pour
pouvoir effectuer d'autres actions. Cette modularité permet de
s'affranchir d'un \textit{workflow} linéaire et prédéfini, offrant ainsi une plus
grande adaptabilité aux besoins spécifiques et aux ressources
matérielles des projets.

Pour conclure, l'existence de ce module de base répond à des besoins
élémentaires des projets de recherche en études visuelles. Il fournit un
outil qui permet d'agréger toutes les sources primaires qui concernent
le sujet, de décrire les sources et de les mettre en relation. Il offre
en outre la possibilité d'extraire et visualiser des contenus d'intérêt
(les ``crops'' d'images), ciblant ainsi les instances de base qui
intéressent les chercheur.ses.

\hypertarget{les-taches}{%
\section{Des tâches complémentaires~: parallélisation du
\emph{workflow}}\label{les-taches}}

Spécialiser un modèle d'intelligence artificielle implique de lui
fournir des données pertinentes, diversifiées, et en quantité
suffisante. Cependant, pour certains domaines, dont l'histoire fait
partie, le volume de données disponible est insuffisant. Ce constat est
d'autant plus vrai dans le cas des diagrammes issus de traités
astronomiques~: les corpus de documents scientifiques historiques
contiennent généralement du texte en majeure partie, des tables et des
images, négligeant souvent les diagrammes.\footnote{Exception faite du
  corpus S-VED (\cite{buttner_cordeep_2022}), collection
  d'illustration très diverses contenant entre autre des diagrammes
  historiques. Mais les primitives ne sont pas annotées.}. De plus, ils
sont dénués d'annotations précises sur les éléments constitutifs des
pages~; c'est sans parler de l'inexistence d'un corpus de diagrammes
dont les primitives sont annotées. Or l'annotation est une tâche
chronophage et fastidieuse. Le recours aux données synthétique répond,
mais en partie seulement, à ces problématiques.

\hypertarget{datasets-synthetiques}{%
\subsection{\emph{datasets} synthétiques}\label{datasets-synthetiques}}

Les \textit{datasets} synthétiques sont générés par des algorithmes ou des
méthodes de simulation pour imiter des données réelles, sans être
directement extraites de sources existantes. De tels jeux de données
sont utilisés lorsque les données réelles sont limitées ou difficiles à
obtenir, mais qu'il est cependant nécessaire de contrôler spécifiquement
les caractéristiques des données d'entraînement\footcite{buttner_cordeep_2022}. La génération
d'images a pour but de fabriquer des ensembles de données plus vastes,
plus diversifiés, très variables et assez complexes, répondant aux
caractéristiques des objets d'intérêt du projet, et surtout étiquetés
automatiquement, sans recourir à l'annotation manuelle.

Ces données synthétiques sont assez ressemblantes et complexes pour être
exploitées. Par exemple, docExtractor est un modèle off-the-shell (au
même titre que \yolo) envisagé dans le cadre de la tâche d'extraction des
diagrammes, et qui se veut sépcifique aux données historiques, car il
est entraîné sur des données produites par un générateur de documents
historiques synthétiques~: SynDoc\footcite{monnier_docextractor_2020}. SynDoc
génère des images de manière aléatoire en combinant des éléments
graphiques (fonds, images, texte et bruit) provenant d'un jeu d'image
défini (constitué de 177 images de pages, 15 contextes, plus de 8000
œuvres d'art provenant de WikiArt, des lettrines générées à partir d'une
lettre aléatoire avec 91 fonts possibles, et des dessins, schémas et
textes tirés d'articles aléatoires sur Wikipedia, avec plus de 400
fonts). Les différents éléments s'agencent, intégrant sur le fond
images, texte et bruit, offrant des combinasons et des mises en pages
assez complexes. Chaque élément de contenu est pré-annoté, éliminant
ainsi le besoin d'annotations manuelles pour ces pages.

          \begin{figure}[H]
          \begin{center}
          \includegraphics[height=6.5cm]{figues/syndoc.jpg}
          \end{center}
          \caption{Données synthétiques générées par SynDoc.\footcite[p.46]{norindr_traitement_2023}}
          \label{fig:syndoc} \end{figure}

Pour entraîner le modèle de vectorisation, il a de même été nécessaire
d'utiliser des données synthétiques. Parce qu'annoter les primitives
géométriques dans des images de diagrammes complexes est très
chronophage, le modèle de vectorisation a été pré-formé sur des corpus
artificiels générés dynamiquement. Le script de génération des données
d'entraînement choisit aléatoirement un arrière-plan, y ajoute des mots,
des nombres et des glyphes puis crée artificiellement un diagramme en
insérant des segments, des cercles et des arcs. Le script est conçu pour
que ces diagrammes aient une forte probabilité de présenter des formes
très caractéristiques comme les cercles concentriques et tangents, les
lignes parallèles et les arcs connectés, afin de simuler les structures
typiques. Les primitives sont dessinées avec des
variations aléatoires d'opacité, de largeur et de couleur. Les cercles
peuvent être remplis ou vides. Enfin, du bruit est ajouté en appliquant
un flou gaussien, et en supprimant de petites régions du diagramme pour
imiter la dégradation des documents historiques. Les données
d'entraînement ainsi générées présentent des configurations assez
complexes.

          \begin{figure}[H]
          \begin{center}
          \includegraphics[height=7cm]{figues/vecto_synthetic_data.png}
          \end{center}
          \caption{Données synthétiques générées pour l'entraînement du modèle de vectorisation.\footcite[Figure issue de la présentation de Syrine Kalelli à l'occasion de la conférence \eida 2024~:][]{noauthor_eida_nodate-1}}
          \label{fig:vecto_synthetic} \end{figure}

Enfin, le modèle de similarité présente un troisième exemple, puisque
SegSwap est pré-entraîné sur de la donnée synthétique. Le script de
génération prend des parties aléatoires d'une images et les copie-colle
au-dessus d'une autre image. Les trois images (source, cible et
superposition) sont placées dans le même dataset d'entraînement, ainsi
le modèle apprend à retrouver ce qui, dans la superposition, vient de la
source, et ce qui vient de la cible.

          \begin{figure}[H]
          \begin{center}
          \includegraphics[height=3cm]{figues/segswap_blended_images.png}
          \end{center}
          \caption{Données d'entraînement du modèle Segswap.}
          \label{fig:segswap} \end{figure}

\hypertarget{les-donnees-reelles}{%
\subsection{Les données réelles}\label{les-donnees-reelles}}

S'appuyer sur les modèles \textit{off-the-shelf}, sur de larges \textit{datasets}
généralistes, ou sur des données synthétiques permet une implémentation
facilitée de la vision dans des projets et constitue une base solide.
Toutefois, les sources tenant aux deux projets (\vhs et \eida) sont trop
spécifiques pour se contenter de modèles généralistes ou formés sur des
données artificielles. Même si ces derniers peuvent offrir des performances
de base, ils risquent de manquer de précision et de sensibilité aux
particularités des documents historiques. Les corpus artificiels
présentent des configurations délibérément complexes pour s'approcher le
plus possible des difficultés que le modèle pourrrait rencontrer sur les
données réelle. Elles sont cependant irréalistes et insuffisantes pour
permettre aux modèles de généraliser sur des diagrammes réels.

En atteste la comparaison des performances de docExtractor et \yolov sur
les données d'\eida. docExtractor\footcite{monnier_docextractor_2020}, entraîné
sur des données synthétiques mimant les documents historiques serait en
théorie plus adapté au traitement d'images de pages de manuscrits, avec
du texte et des illustration côté à côte, d'autant qu'il intègre des
outils de traitement du texte (notamment pour la segmentation des
lignes)\footnote{\eida envisage l'implémentation d'un outil d'extraction
  et transcription des labels et des textes qui entourent les diagrammes}.
Pourtant, sans fine-tuning sur des données réelles, il présente des
performances équivalentes à celles de \yolov\footcite[p.45]{norindr_traitement_2023}. Cela
souligne que même les modèles off-the-shelf entraînés sur un corpus
assez spécifique et complexe, mais synthétique, ne dispense pas d'un
entraînement sur des données réelles, au même titre que les modèles très
généralistes comme \yolov.

Alors, le modèle de base \yolov tel que mis à disposition par
Ultralytics est entraîné sur de grands ensembles de données réelles, ce
qui constitue une base solide pour la classification des objets du
monde. L'utilisation de SynDoc permet ensuite de compléter
l'apprentissage initial en exposant le modèle à des exemples variés et
spécifiques aux documents historiques, augmentant ainsi sa capacité de
généralisation. Ces similis de manuscrits anciens offrent l'avantage de
pouvoir être produits en grandes quantités et de couvrir un large
éventail de scénarii et de configurations difficiles à obtenir dans des
ensembles de données réelles. Puis le modèle est entraîné sur les
données de \vhs, qui sont de réelles pages de documents historiques
contenant une large diversité d'illustrations. Ces données apporteront
une dimension supplémentaire de pertinence au modèle, en l'exposant à
des particularités des documents historiques réalistes. Enfin, \yolov
est entraîné sur les données d'\eida, qui sont orientées spécifiquement
vers les diagrammes, afin qu'il détecte uniquement ces derniers.

Quant au modèle de vectorisation développé par Syrine
Kalleli\footcite{kalleli_historical_2024}, il est formé
sur des données synthétiques générées à la volée par un script. Mais le
corpus de diagrammes d'\eida est particulièrement caractéristique et le
modèle n'aurait pu être optimal sans avoir appris sur des images de
diagrammes issus de manuscrits réels. Un corpus d'entraînement de 303
diagrammes extraits de manuscrits et de gravures a donc été constitué et
annoté par les historien.nes. Ces diagrammes sont issus de sources latines, arabes,
grecques, hébreuses ou chinoises, datant du \textsc{xii}\ieme au \textsc{xviii}\ieme siècle, et ils
présentent en guise d'étiquettes plus de 3000 lignes, cercles et arcs. Le
ré-entraînement a permis le transfert des connaissances acquises sur la
tâche de détection des primitives sur les données réalistes.

Il sera également possible d'obtenir des meilleurs résultats sur la
similarité grâce à une évaluation des scores (qui constitue un jeu de
données annotées) et le ré-entraînement du modèle, pour donner des
résultats plus adaptés à la spécificité des données historiques.

D'ailleurs, cette étape d'annotation (le choix des exemples et des
étiquettes) revêt des enjeux importants. L'apprentissage spécifique se
fait à partir de données sélectionnées par les chercheur.ses~: les exemples
sur lequel l'algorithme d'apprentissage va itérer définissent le modèle.
Il est nécessaire de constituer un échantillon de données aléatoire et
représentatif, et de l'annoter en fonction de ce que l'on souhaite
obtenir en prédiction.

L'annotation des jeux de données est non seulement une étape clé, mais
aussi un bel exemple de collaboration chercheur.ses-ingénieur.es. Elle
nécessite la définition de normes pertinentes et rigoureuses. Travail
minutieux et chronophage, l'étiquetage des données peut engendrer des
erreurs et du bruit dans les données, car elle implique la subjectivité
des chercheur.ses et le regard parfois trop précis sur les sources desquels
les annotateurs sont experts.

Voici un exemple rencontré lors de la préparation des données pour
entraîner un modèle de segmentation du contenu textuel. Les sources
arabes et chinoises sont particulièrement verbeuses et les diagrammes
sont très souvent entourés des blocs de commentaires se mélangeant alors
aux légendes et aux labels. Doit-on considérer ces commentaires comme
faisant partie des éléments que l'on souhaite identifier ou bien les ignorer
? Cette décision est importante car si on les ignore, le modèle risque
de passer à côté d'éléments textuels pertinents. En revanche, si on les
inclut, il ramènera des commentaires sans rapport direct avec le
diagramme observé. On voit ici comment la binarité des modèles, qui se
reflète dans les normes d'annotation, est problématique et constitue une
limite au \ml. Un compromis doit être trouvé entre
l'automatisation, qui requiert une normalisation, des définitions
claires et binaires, et la nuance dans l'interprétation des
sources\footnote{Dans le cadre du projet, il a toujors été plus
  intéressant d'opter pour une définition extensive des objets à
  détecter, car prévision d'une correction des traitement. Et il est
  plus facile de supprimer un élémént pas pertinent que d'aller en
  rechercher un, surtout compte tenu de la taille des corpus des
  chercheur.ses. Vaut aussi pour la préparation des données pour
  l'entraînement du modèle d'extraction.}.

La normalisation peut bénéficier à l'écosystème de recherche dans le
domaine de l'\htr et de l'\ocr. À ce titre, il est pertinent d'envisager
l'utilisation du vocabulaire contrôlé SegmOnto pour l'annotation du
contenu textuel entourant les diagrammes. Cela permettrait de créer des
jeux de données réutilisables, à partager avec des projets poursuivant
des objectifs similaires.\footnote{https://segmonto.github.io/}. Encore
une fois, un compromis doit être trouvé entre les besoins de description
des chercheur.ses et les possibilités offertes par les vocabulaires
contrôlés.

Un autre exemple concerne le dernier entraînement du modèle d'extraction
: les résultats montrent que des diagrammes sont encore détectés en
transparence. La question s'est alors posée de chercher à corriger ce
défaut en donnant au modèle, à l'occasion d'un nouvel entraînement,
d'avantage d'exemples négatifs (diagrammes visibles par transparence
mais non annotés). Or il est préférable de se contenter de la correction
ou suppression manuelle de ces prévisions erronées, garantissant que le
modèle parvienne à détecter les diagrammes presque effacés.

Pour assurer la rigueur et la cohérence des annotations, les décisions
prises entre les chercheur.ses et les ingénieur.es peuvent être l'objet d'une
documentation ou d'ateliers d'annotation.

\hypertarget{loeil-de-la-machine-avantages-et-limites}{%
\subsection{L'oeil de la machine~: avantages et
limites}\label{loeil-de-la-machine-avantages-et-limites}}

Bien qu'il soit possible d'optimiser les performances d'un modèle
d'apprentissage automatique en l'entraînant sur un ensemble de données
spécifique, son interprétation des données reste limitée car
fondamentalement binaire, ce qui le rend parfois déficient pour la
recherche en histoire. Ainsi, il gèrera difficilement les cas limites et
ambigüs. La décision d'inclure ou d'exclure ces cas particuliers de
l'ensemble d'entraînement implique un arbitrage délicat. D'un côté, une
inclusion trop restrictive peut compromettre les capacités de
généralisation du modèle, c'est-à-dire sa capacité à s'adapter à de
nouvelles données. À l'inverse, une inclusion trop permissive risque de
dégrader la précision du modèle sur les cas plus typiques. Les
chercheur.ses espérant obtenir un modèle maximaliste, quitte à accepter un
certain degré d'erreur et de devoir supprimer les faux
positifs, de nombreux cas limites ont été inclus. Le cas des diagrammes
visibles en transparence (expliqué précédemment) en est un exemple
éloquent.

Une autre difficulté réside dans la définition même du ``diagramme
astronomique''. Les limites de ce concept ne sont pas si claires et
définitives pour les chercheur.ses, et pourtant le modèle a besoin d'une
définition rigoureuse et cohérente. Il paraît en effet difficile de
considérer les diagrammes astronomiques en dehors du contexte des
pratiques d'autres sciences et disciplines connexes. Par exemple, Le
\emph{Flores Almagesti} -- réécriture de l'Almageste datant du \textsc{xv}\ieme par
l'astronome Giovanni Bianchini -- présente une partie algébrique à
l'ouverture mathématique, induisant la présence de nouveaux types de
diagrammes d'inspiration euclidienne. Pour retracer la source de ces
derniers, il est nécessaire de considérer les traités d'Euclide ou
autres travaux d'algèbre. Ceux-ci ne sont pas des traités
\emph{astronomiques}, bien qu'il ne soit pas certain que ces disinctions
contemporaines aient été aussi rigide à l'époque et aient eu un
quelconque sens pour les acteurs historiques. Les sources byzantines
confirment cette complexité~: les diagrammes y sont nommés
\emph{katagraphai}, indépendamment du domaine scientifique auquel ils
appartiennent. Également, de nombreux travaux astronomiques sont groupés
dans des témoins qui contiennent des œuvres issus de domaines divers.
C'est le cas avec les sources chinoises, comme le \emph{Chongzhen
lishu}, qui se présente généralement annexé d'une série de traités
mathématiques. Par conséquent, les diagrammes euclidiens ont été gardés
lors de la préparation des données, et l'algorithme de détection les
classe comme ``diagramme'', même s'ils ne constituent pas l'objet
principal des chercheur.ses.

En ce qui concerne les autres types de diagrammes non strictement
astronomiques (géométriques, harmoniques, logiques, illustrations de
constellations), une approche plus sélective a été adopté afin d'éviter
un modèle trop maximalistes. Ces éléments, bien que potentiellement
intéressants, n'ont pas été inclus dans la phase de détection
automatique.

Ainsi l'œil de la \cv contraint à des choix méthodologique
potentiellement inconfortables, mais en même temps il peut aider à
mesurer les impulsions des chercheur.ses, à mieux définir les objectifs de
recherche et à prioriser les éléments les plus pertinents. Ainsi, la
vision par ordinateur oblige les chercheur.ses à s'adapter à une logique
algorithmique qui, tout en limitant certaines interprétations
subjectives, offre l'opportunité de développer des modèles conceptuels et des méthodologies très rigoureuses.

\hypertarget{infrastructures}{%
\section{(Infra)structures}\label{infrastructures}}

La construction d'une plateforme extensive et modulaire pour
démocratiser l'accès à un outil de gestion et de traitement de la donnée
visuelle implique une réflexion approfondie sur les architectures
matérielles et logicielles. Pour toucher des publics diversifiés de la communauté de la recherche, il est
essentiel de penser des infrastructures matérielles diverses, plus ou
moins puissantes et abordables, intégrant ou non des composants comme
les \gpu (qui permettent d'accélérer les calculs intensifs nécessaires à
l'\ia). En effet, la gestion efficace des ressources, la scalabilité,
et la performance sont des aspects à prendre en compte pour
que ces outils puissent être utilisés de manière fiable. Parallèlement, l'architecture logicielle doit être flexible et
évolutive. 

\hypertarget{hardware-une-api-sur-le-gpu}{%
\subsection{Hardware~: une API sur le GPU}\label{hardware-une-api-sur-le-gpu}}

La séparation physique de l'inférence des modèles tient un rôle
important dans l'ouverture de la plateforme.

Le type d'infrastructure de calcul, notamment le \cpu ou le \gpu,
implique des différences dans le traitement et l'analyse des données,
chacun offrant des capacités distinctes adaptées à des besoins
spécifiques. Un \cpu (Central Processing Unit) est le processeur
principal d'un ordinateur, conçu pour gérer une large gamme de tâches
générales et basiques, et utilisé pour les besoins quotidiens. Un \gpu
(Graphics Processing Unit) est spécialisé dans le
traitement des éléments graphiques. Il est conçu pour effectuer un grand
nombre de calculs simples en parallèle grâce à ses nombreux cœurs, ce
qui le rend extrêmement efficace pour des tâches nécessitant un
traitement massif et simultané de données. L'utilisation d'un \gpu est
souvent nécessaire pour les tâches d'\ia, notamment en vision
artificielle, car ces tâches impliquent souvent des opérations de calcul
intensives et parallélisables. Un \gpu, avec sa capacité à gérer des
milliers de \textit{threads} en parallèle, permet d'accélérer l'entraînement
et l'inférence des modèles de vision artificielle, rendant le traitement
plus rapide et plus efficace que sur \cpu.

Discover-Demo est une \api développée comme un module de l'application, répondant au besoin de séparer les algorithmes de vision du reste
de l'application. Cette séparation permet une plus grande flexibilité
dans l'utilisation des ressources de calcul. Elle tourne sur le \gpu
Dishas-ia, dédié quasi exclusivement aux besoins de l'équipe d'histoire
des sciences du \syrte.

          \begin{figure}[H]
	\begin{center}
		\includegraphics[height=7cm]{figues/com_hard_ware.png}
	\end{center}
	\caption{Organisation et communication des infrastructures.}
	\label{fig:com} \end{figure}

Malgré l'importance accordée à l'\ia,
l'interface web et l'\api associée sont conçues pour une analyse complète
des documents historiques, allant de leur importation et stockage à
leurs traitements (divers) et visualisations. Dans une optique
d'extensivité, elles ne doivent être rattachées à aucun processus d'analyse
prédéterminée. Ainsi, toutes les étapes peuvent être
effectuées manuellement ou à l'aide d'algorithmes automatisés.
L'application de base n'intègre pas de traitement de \cv, mais permet de gérer une base de données et des sources avec
leurs numérisations, utilisant le standard \iiif. Elle permet
l'indexation manuelle de zones d'images dans \sas via l'interface Mirador,
permettant la sélection de zones d'images d'intérêt. Pour cela,
l'extraction automatique de zones d'images est séparée en un nouveau
module, mais les fonctionnalités de base de l'application incluent
toujours les outils nécessaires pour effectuer des annotations manuelles
de régions. Cela comprend toutes les fonctions pour indexer un fichier
texte dans \sas, visualiser les régions annotées, et exporter les
résultats. Il devient alors envisageable d'importer des résultats de traitement
(fichiers d'annotation de régions ou de paires de régions similaires) et
de les indexer manuellement pour permettre leur visualisation et analyse
ultérieure.

Chaque traitement peut donc être
réalisé via l'inférence des modèles de vision sur \gpu (comme c'est le
cas pour \eida grâce à l'\api), par l'import d'un fichier de résultats,
manuellement, ou potentiellement par des méthodes locales sur \cpu
(\yolov, par exemple, est assez léger pour tourner en local). La
plateforme permet ainsi une adaptation à des environnements matériels
divers, laissant la possibilité de réaliser les traitements soit
automatiquement via l'\ia, soit manuellement.

Séparer les modèles de vision c'est aussi permettre une bascule vers des
modèles spécialisés. Les modèle développés dans le cadre du projet sont
disponibles mais peuvent facilement être réentraînés pour correspondre
spécifiquement aux données de l'utilisateur.rice, prenant en compte les
besoins de sa recherche.

Pour conclure, grâce à cette séparation des composants \textit{hard-ware}, la
plateforme répond efficacement à une diversité de besoins et permet son
intégration dans des projets aux ressources matérielles variées. Même
sans ressource matérielle capable de faire tourner les modèles de
vision, les utilisateur.rices peuvent toujours exploiter la plateforme web.
Cette conception offre un accès aux outils et méthode à des utilisateur.rices
divers, allant des projets sans ingénieur dédié pour le
développement, aux équipes de recherche disposant de leurs propres
ingénieurs, en passant par des doctorants indépendants ayant des
compétences en programmation mais sans accès à un serveur. L'outil est
pensé pour s'adapter à des environnements variés, des configurations
légères fonctionnant en local, jusqu'à des projets disposant de
ressources matérielles importantes comme un \gpu.

\hypertarget{software-des-modules-separes}{%
\subsection{Software~: des modules
séparés}\label{software-des-modules-separes}}

Le modèle MVC (Model-View-Controler) est une architecture logicielle qui
segmente une application en trois composantes interconnectées. Le Modèle
est chargé de la gestion des données et de la logique métier de
l'application, assurant la manipulation et l'administration des
informations. La Vue est responsable de la présentation visuelle des
données, les mettant en forme visuellement dans un \textit{template}. Le
'Contrôleur', sert d'intermédiaire entre le 'Modèle' et la 'Vue'~: il reçoit
les entrées de l'utilisateur.rice via la 'Vue', traite ces entrées, puis
interagit avec le 'Modèle' pour actualiser les données et, enfin, met à
jour ces modifications dans la Vue. Cette séparation des préoccupations
permet une organisation plus rigoureuse du code, facilitant ainsi la
maintenance, la réutilisabilité et le développement parallèle de chaque
composante.

Le cycle action → mise à jour → affichage induit par ce patron est bien
adapté aux applications web, il est à ce titre utilisé par nombre
d'entre elles, dont \eida fait partie, et par de nombreux \textit{frameworks},
Django y compris.

Bien que le modèle MVC offre déjà une structure prenant en compte la
séparation des préoccupation, \eida cherche à aller au-delà, proposant
une architecture encore plus flexible. La plateforme est conçue pour
permettre aux développeur.ses d'ajouter ou de supprimer des fonctionnalités
de manière indépendante. Cette approche permet de personnaliser
l'application en fonction des besoins spécifiques de chaque projet, sans
avoir à modifier le cœur du système. Les utilisateur.rices peuvent ainsi
partir de la base de la plateforme et la compléter avec des modules sur
mesure.

Voici une transcription de l'arborescence des fichiers de l'application
:

\begin{verbatim}
app/
├── config/
├── logs/
├── mediafiles/
├── regions/
├── similarity/
├── vectorization/
│   ├── templates/
│   ├── __init__.py
│   ├── const.py
│   ├── tasks.py
│   ├── urls.py
│   ├── utils.py
│   ├── views.py
├── webapp/
├── webpack/
├── __init__.py
├── manage.py
├── requirements-base.txt
├── requirements-dev.txt
├── requirements-prod.txt
cantaloupe/
celery/
docs/
gunicorn/
sas/
scripts/
├── .gitignore
├── .pre-commit-config.yaml
├── README.md
├── run.sh
\end{verbatim}

Il a été créées plusieurs unités fonctionnelles pouvant inclure leurs vues, \textit{templates}, utilitaires, etc. Cette approche
permet aux développeur.ses de découper l'application tout en factorisant le code dédié à
plusieurs tâches, ainsi les modules partagent des \textit{statics}, un fichier
de configuration global et des fonctions utilitaires.

Chaque sous-dossier dans \texttt{app/} représente un module fonctionnel. Le
répertoire \texttt{webapp/} contient le module de base, tandis que \texttt{webpack/} est
dédié aux interfaces\footnote{Voir le \hyperlink{chapitre-8-interfaces}{chapitre suivant}}. Les modules
additionnels, autonomes, peuvent s'interfacer les uns avec les autres,
et être développés puis testés de indépendamment. Pour un exemple
détaillé du contenu des fichiers, une description du module dédié à
la vectorisation développée pendant se trouve en annexe \ref{module_vecto}. 

La variable \texttt{INSTALLED\_APPS} du fichier de configuration global permet de
personnaliser l'application en activant ou désactivant les modules
souhaités.

\emph{Modularité}

Cette architecture s'inscrit dans une stratégie de développement
applicatif ouverte et évolutive. La division en unités fonctionnelles et
indépendantes permet d'ajouter ou de supprimer les fonctionnalité
complémentaires au module de base. Ce cadre de développement modulaire
garantit que l'application reste adaptable aux exigences évolutives des
chercheur.ses et des institutions partenaires, la rendant plus robuste et
tolérante aux usages extérieurs et autorisant alors le réemploi du code
par des projets ayant besoin d'effectuer des traitements divers sur
du matériel documentaire pictural numérisé. \aikon est utilisable par des
projets divers, ce qui réduit le temps de développement et ouvre la voie
à des partenariats, aidant alors à pérenniser les outils.

\emph{Maintenance}

L'organisation modulaire du code facilite également la maintenance et
les mises à jour. Avec une telle structure, il devient plus facile
d'isoler les composants pour le développement et le débogage. Les
développeur.ses peuvent travailler sur un module spécifique sans interférer
avec les autres parties du projet. De plus,
l'implémentation d'un module indépendant provoquera moins de conflit
lors du déploiement.

\vspace{2cm}

En conclusion, \aikon est
composée de plusieurs modules applicatifs, elle est extensible et facile à maintenir (l'organisation des processus se reflétant
dans l'arborescence de fichiers), tout
en facilitant la gestion des dépendances et l'optimisation des
performances. 

 \begin{figure}[H]
	\begin{center}
		\includegraphics[height=6cm]{figues/separate.png}
	\end{center}
	\caption{Représentation schématique des fonctionnalités et processus de la plateforme.*}
	\label{fig:fonct_separes} \end{figure}

Cette approche modulaire est particulièrement bénéfique
pour les projets complexes nécessitant une collaboration entre plusieurs
équipes de développement (c'est d'ailleurs le cas dans le cadre de la collaboration \eida/\vhs), assurant la qualité et la robustesse du code,
et permet la modularité de la plateforme.
            
        \clearemptydoublepage
        
\hypertarget{chapitre-8-interfaces}{%
\chapter{Interfaces}\label{chapitre-8-interfaces}}


L'interface crée une zone d'échange et de contact entre l'application et l'utilisateur.rice, permettant d'échanger des informations grâce à l'adoption de règles communes. En cela, elle est bien plus qu'une simple couche superficielle. Elle
constitue le point de rencontre entre deux mondes~: celui de l'humain et
celui de la machine. Concevoir une interface, c'est penser à la manière
dont les utilisateur.rices vont interagir avec un système. C'est définir les
éléments visuels, les commandes, les actions possibles et la logique qui
régit ces interactions. L'interface est ainsi l'aboutissement d'un
processus de design qui vise à rendre une expérience utilisateur.rice aussi
intuitive et efficace que possible. Et, à ce titre, elle jouera un rôle
décisif dans la mise à disposition et l'adoption des méthodes d'analyse
basées sur l'\ia.

L'intégration du \dl aux pratiques des chercheur.ses en sciences
historiques peut être favorisée par le développement de la plateforme \aikon, qui leur
permet d'exploiter simplement ces outils pour traiter leurs sources.
L'interface graphique sert donc d'intermédiaire entre les chercheur.ses et les
algorithmes de vision artificielle. Elle facilite en outre le dépôt et
la gestion structurée des données sources et des métadonnées associées
et simplifie l'accès aux outils de traitement de l'images. Cette
plateforme à interface graphique doit être adaptée à une grande
diversité de documents à traiter, et pensée pour accueillir des
utilisateur.rices divers, aux compétences numériques et aux questions de
recherche variées.

\begin{kwote}
``{[}L{]}'activité de conception des plateformes digitales, chères aux
Humanités Numériques en tant que ces dernières se veulent productrices
d'outils et d'instruments, doit être considérée par nature comme
relevant d'un travail de design et, par conséquent, en intégrer la
culture créative et la philosophie dès le commencement, dans l'esprit
d'un « design des programmes » (Masure, 2014) qui doit permettre de
développer et d'améliorer le design des plateformes. Car, on le sait,
la cause principale de la réussite (et donc de l'adoption par une
communauté) d'un service numérique réside dans la haute qualité
d'expérience utilisateur.rice ( User eXperience ) qu'il est capable de
délivrer aux usagers.''\footcite{clavert_2dh_2015}
\end{kwote}

L'\ux \textit{design}, approche centrée utilisateur.rice, vise à optimiser les
interactions entre un utilisateur.rice et un système. Il s'appuie sur des
méthodes issues des sciences cognitives pour concevoir des interfaces
intuitives et efficaces. L'\ui \textit{design}, partie intégrante de l'\ux,
se concentre sur l'aspect visuel de l'interface, en cherchant à créer
une expérience esthétique et cohérente. Le projet \eida illustre
parfaitement l'importance de l'\ux/\ui \textit{design} dans le domaine de la
recherche. En plaçant l'utilisateur.rice final (le chercheur.se) au centre de
ses préoccupations, \eida vise à développer une plateforme qui le guidera
dans l'adoption des méthodes et la prise en main de la chaîne de
traitement. Ainsi un outil de recherche, pour être efficace, doit
disposer d'une interface utilisateur.rice pensée pour optimiser les \textit{workflows}
des chercheur.ses. Une bonne ergonomie, une navigation fluide et une
organisation claire des informations permettent d'assurer la cohérence
des méthodes et processus. Ainsi, l'outil pourra devenir un véritable
levier de productivité, modélisant des
processus unifiés au sein des équipes. Une interface intuitive et
agréable facilite l'apprentissage et l'utilisation de l'outil, même par
des utilisateur.rices ayant des niveaux d'expertise variés~; elle favorise
ainsi l'engagement et l'adoption, par le biais de l'outil numérique, d'une
méthodologie.

\hypertarget{faire-le-lien-avec-les-chercheur.ses}{%
\section{Faire le lien avec les
chercheur.ses}\label{faire-le-lien-avec-les-chercheur.ses}}

Le module de base est un package pour la gestion documentaire, duquel
l'application ne peut se détacher. Celui-ci inclut tout d'abord des
formulaires pour l'intégration des documents dans la base de données. Le
modèle de données permet de décrire différentes entités qui, bien que
liées dans leurs métadonnées, peuvent être intégrées indépendamment. Le
module de base permet également la création de \mans \iiif pour
chaque numérisation, permettant ensuite la visualisation des documents
grâce aux outils open-source dédiés. De ce fait, l'indexation de zones
d'image peut être réalisée manuellement via l'interface Mirador intégrée
à \sas. Ce noyau fonctionnel inclut en outre la sélection de lots de
documents (le ``panier''), sur lesquels pourront être effectués des
traitements groupés paramétrables.

Les briques fondamentales offrent donc les fonctionnalités essentielles
de gestion documentaire (intégration, modèle de données, \iiif). Les
traitements, quant à eux, sont gérés par des modules séparés, et c'est
sur cette structure que repose la modularité et l'évolutivité de
l'application.

Ci-après nous donnons une description détaillée de certaines de ces
fonctionnalités de base.

\hypertarget{description-des-donnees}{%
\subsection{Description des
données}\label{description-des-donnees}}

Le module de base contient un modèle de données suffisamment extensif
pour décrire efficacement une diversité de données, allant de documents
textuels historiques à des tableaux en histoire de l'art. La
tripartition entre témoin (\wit), série (qui contient un ensemble de
témoins), et contenu permet un alignement avec des corpus très
diversifiés et des données potentiellement hétéroclites, telles que des
manuscrits, des documents épistolaires, des inventaires de galeries
d'art, et même pourquoi pas des cartes\ldots{}

Pour ouvrir à cette large diversité de données, la liste des types de
pagination témoin doit être étendue \emph{a minima} d'un nouveau type
``other'', émancipant l'enregistrement des mentions de pagination. Les
développements futurs prévoient aussi la création d'un système pour
ajouter facilement un nouveau type\footnote{Le type de témoin est une
  métadonnée rentrée par l'utilisateur.rice lors de l'enregistrement du
  \wit dans la base de donnée. Il choisit le type dans une liste,
  originellement manuscrit, imprimé ou gravure sur bois.} de \wit
(tel que peinture, catalogue, etc.).

Au fil des développements, des débats ont émergé autour de l'ajout dans
le modèle de données d'un niveau de granularité supplémentaire pour
décrire des images ou zones d'images unitaires
(\graphicals), créant ainsi une entité détachée du fait
qu'elle provienne d'une extraction dans un document. Cette solution
aurait permis une description plus détaillée et plus fine des images,
importante pour des projets axés sur des images uniques, et aurait
favorisé un élargissement du spectre des type de sources pris en charge.
L'utilisateur.rice aurait pu soit importer une image unique (et de manière
optionnelle, la lier à un \wit) via un formulaire, soit sélectionner
une région d'image d'intérêt au sein des extractions (annotations \sas),
laquelle serait enregistrée comme \graphical, puis l'enrichir de
métadonnées. Dans les deux cas l'enregistrement d'un \graphical
aurait donné lieu à la création d'une \digit au format \jpeg.

Sans l'unité de description \graphical, les régions d'images
sont créées uniquement via les annotations \sas.

L'intégration de cette entité au sein du modèle aurait offert plusieurs
avantages en termes de cohérence et de flexibilité. En s'alignant sur
les structures existantes (\wits et \sers), elle aurait permis une
manipulation plus intuitive des images, facilitant ainsi les opérations
de recherche et la création de \emph{Sets} personnalisés. De plus, elle
aurait rationalisé la gestion des annotations \sas, permettant de
sélectionner les plus pertinentes dans la multitude existante.

Cependant, cette approche présente des limites, et on peut trouver des
alternatives. Tout d'abord, la coexistence de \graphicals avec les
annotations \sas, générées par des processus distincts, aurait pu créer
une certaine confusion quant à leur nature et à leur méthode de
création. De plus, la multiplication potentielle de milliers
d'enregistrements aurait pu impacter les performances de la base de
données et complexifier les requêtes. Enfin, le lien sémantique ambigu
et sujet à interprétation subjective entre \graphical et \wit
aurait compliqué les possibilités de corrélation.

Compte tenu de ces limites, il a semblé préférable de maintenir les
annotations \sas pour identifier les instances de base du modèle, sans
créer de nouvelle unité de description. La solution actuelle reste donc
basée sur la création manuelle ou automatique de zones dans les images
via \iiif et \sas, évitant les problèmes de redondance et de confusion.
Bien que l'entité \graphical n'ait pas été implémentée, les
fonctionnalités d'annotation et de sélection d'images sont assurées par
d'autres mécanismes. L'outil Mirador permet d'associer des tags aux
zones d'image, offrant ainsi une première couche d'enrichissement
sémantique. La sélection dans un \emph{set} personnalisé sera possible en
gardant en mémoire une référence contenant des coordonnées du
\emph{crop}. De plus, l'importation d'images individuelles est
réalisable en les considérant comme des \emph{Witness partiels}, ce qui
permet de les intégrer dans le \textit{workflow} existant. Toutefois
l'enrichissement sémantique à un niveau de granularité fin restera
limité~; et la dépendance à l'outil \sas constitue une potentielle dette
technique, susceptible de restreindre les évolutions futures du système.

Afin de mieux répondre aux exigences de modularité, l'évolution du
modèle de données s'oriente non pas vers une description individuelle
des documents, mais vers la gestion des traitements. Cette évolution
implique la création d'une entité \tr
liée à des ensembles de données (\ds et
\rs) potentiellement hétérogènes.

\hypertarget{principe-du-traitement}{%
\subsection{Principe du Traitement}\label{principe-du-traitement}}

Le but fondamental de la plateforme est de pouvoir effectuer plusieurs
actions sur les objets de la base. Afin d'assurer une meilleure
traçabilité et plus de flexibilité, la plateforme abandonne les
lancements automatiques des processus\footnote{C'était initialement le
  cas de l'extraction des entités, dont le lancement était lié à une
  méthode de classe liée à la \digit après soumission d'un
  formulaire d'ajout d'un \wit ou d'une \ser. L'action se
  lançait immédiatement après enregistrement des images d'une
  numérisation dans la plateforme.} au profit d'un système basé sur
l'entité \tr. Chaque traitement est associé à un ensemble
d'objets traités ensemble (\ds ou \rs), à un jeu de
paramètres et à un résultat. Ces informations sont stockées dans une
table dédiée. Cette approche facilite la gestion et le trackage des
processus (notamment, les utilisateur.rices sont notifiés par e-mail à la fin
du \textit{processing}), permet aux utilisateur.rices de consulter un historique de
leurs actions et offre la possibilité de créer des \textit{workflows}
personnalisés en passant par un formulaire de lancement unique mais
extensif.

En permettant de regrouper des documents de types différents (\wos,
\sers, \wits) dans des \dss, on offre à
l'utilisateur.rice la flexibilité de lancer des actions sur des ensembles
d'entités hétérogènes et granulaires. Le traitement est ensuite réparti
sur les entités de niveau inférieur (les témoins). Les \wits ainsi
sélectionnés peuvent être soumis à une large gamme de traitements~: des
fonctions déjà implémentées comme l'exportation (avec choix du
format), l'extraction, la vectorisation, la recherche de similarité~; ou
de nouveaux traitements personnalisés, tels que la visualisation sur une
frise chronologique ou une carte. La modularité de la plateforme est
assurée par un formulaire de lancement configurable, permettant de
l'adapter à différents scénarios d'utilisation, et à l'ajout de modules
personnalisés.

Le \rs fonctionne similairement au \ds, à un niveau
de granularité inférieur (à l'échelle de la zone d'image)\footnote{À
  l'été 2024, l'entité n'existe pas encore dans la base de données, mais
  le processus d'envoi du traitement et les modes de communication entre
  l'application et l'\api prévoient la possibilité de lancer l'inférence
  des modèles sur un ensemble de régions extraites.}.

\hypertarget{extraction-des-zones-dimage-manuelle}{%
\subsection{Extraction manuelle des zones d'image}\label{extraction-des-zones-dimage-manuelle}}

Le choix de la méthode d'extraction des régions d'intérêt dans les
documents constitue un élément clé de la modularité de la plateforme.
Les utilisateur.rices peuvent opter pour une extraction manuelle ou une
extraction automatique basée sur des algorithmes de vision par
ordinateur, adaptée aux traitements à plus grande échelle.

Après importation d'un enregistrement, le flux de travail procède à la
création de \mans \iiif pour chaque numérisation
(\digit) afin de permettre une visualisation grâce à la
plateforme Mirador. Le module de base autorise par la suite
l'extraction manuelle de zones d'intérêt au sein des images. Cette
fonctionnalité est particulièrement utile pour les projets ne souhaitant
pas recourir à des méthodes entièrement automatisées de vision par
ordinateur. L'outil \sas permet de créer des annotations, c'est-à-dire de
définir des régions d'intérêt spécifiques dans les numérisations, et de
les indexer directement dans les \mans \iiif correspondants,
enrichissant ainsi les ressources numériques. De plus, les
développements futurs prévoient la possibilité d'importer des fichiers
d'annotation préexistants en format .\textsc{txt} afin de pouvoir les indexer
manuellement. Par conséquent, le \textit{workflow} de base ne comporte aucun
traitement automatique basé sur la vision (et de fait éventuellement
trop gourmand en puissance de calcul).

L'extraction, qu'elle soit manuelle ou automatique, constitue le
fondement du reste des processus. Une interface est disponible pour
sélectionner un ensemble de documents et effectuer des actions
spécifiques sur les témoins annotés, via le formulaire de traitement qui
s'étend selon un choix de module configuré. Ainsi l'utilisateur.rice n'est
pas limité par un contexte initial, à l'origine deux étapes
indissociables et incontournables (importation et extraction), pour
pouvoir effectuer d'autres actions. Cette modularité permet de
s'affranchir d'un \textit{workflow} linéaire et prédéfini, offrant ainsi une plus
grande adaptabilité aux besoins spécifiques et aux ressources
matérielles des projets.

Pour conclure, l'existence de ce module de base répond à des besoins
élémentaires des projets de recherche en études visuelles. Il fournit un
outil qui permet d'agréger toutes les sources primaires qui concernent
le sujet, de décrire les sources et de les mettre en relation. Il offre
en outre la possibilité d'extraire et visualiser des contenus d'intérêt
(les ``crops'' d'images), ciblant ainsi les instances de base qui
intéressent les chercheur.ses.

\hypertarget{choix-techniques}{%
\section{Choix techniques}\label{choix-techniques}}

Spécialiser un modèle d'intelligence artificielle implique de lui
fournir des données pertinentes, diversifiées, et en quantité
suffisante. Cependant, pour certains domaines, dont l'histoire fait
partie, le volume de données disponible est insuffisant. Ce constat est
d'autant plus vrai dans le cas des diagrammes issus de traités
astronomiques~: les corpus de documents scientifiques historiques
contiennent généralement du texte en majeure partie, des tables et des
images, négligeant souvent les diagrammes.\footnote{Exception faite du
  corpus S-VED (\cite{buttner_cordeep_2022}), collection
  d'illustration très diverses contenant entre autre des diagrammes
  historiques. Mais les primitives ne sont pas annotées.}. De plus, ils
sont dénués d'annotations précises sur les éléments constitutifs des
pages~; c'est sans parler de l'inexistence d'un corpus de diagrammes
dont les primitives sont annotées. Or l'annotation est une tâche
chronophage et fastidieuse. Le recours aux données synthétique répond,
mais en partie seulement, à ces problématiques.

\hypertarget{datasets-synthetiques}{%
\subsection{\emph{datasets} synthétiques}\label{datasets-synthetiques}}

Les \textit{datasets} synthétiques sont générés par des algorithmes ou des
méthodes de simulation pour imiter des données réelles, sans être
directement extraites de sources existantes. De tels jeux de données
sont utilisés lorsque les données réelles sont limitées ou difficiles à
obtenir, mais qu'il est cependant nécessaire de contrôler spécifiquement
les caractéristiques des données d'entraînement\footcite{buttner_cordeep_2022}. La génération
d'images a pour but de fabriquer des ensembles de données plus vastes,
plus diversifiés, très variables et assez complexes, répondant aux
caractéristiques des objets d'intérêt du projet, et surtout étiquetés
automatiquement, sans recourir à l'annotation manuelle.

Ces données synthétiques sont assez ressemblantes et complexes pour être
exploitées. Par exemple, docExtractor est un modèle off-the-shell (au
même titre que \yolo) envisagé dans le cadre de la tâche d'extraction des
diagrammes, et qui se veut sépcifique aux données historiques, car il
est entraîné sur des données produites par un générateur de documents
historiques synthétiques~: SynDoc\footcite{monnier_docextractor_2020}. SynDoc
génère des images de manière aléatoire en combinant des éléments
graphiques (fonds, images, texte et bruit) provenant d'un jeu d'image
défini (constitué de 177 images de pages, 15 contextes, plus de 8000
œuvres d'art provenant de WikiArt, des lettrines générées à partir d'une
lettre aléatoire avec 91 fonts possibles, et des dessins, schémas et
textes tirés d'articles aléatoires sur Wikipedia, avec plus de 400
fonts). Les différents éléments s'agencent, intégrant sur le fond
images, texte et bruit, offrant des combinasons et des mises en pages
assez complexes. Chaque élément de contenu est pré-annoté, éliminant
ainsi le besoin d'annotations manuelles pour ces pages.

          \begin{figure}[H]
          \begin{center}
          \includegraphics[height=6.5cm]{figues/syndoc.jpg}
          \end{center}
          \caption{Données synthétiques générées par SynDoc.\footcite[p.46]{norindr_traitement_2023}}
          \label{fig:syndoc} \end{figure}

Pour entraîner le modèle de vectorisation, il a de même été nécessaire
d'utiliser des données synthétiques. Parce qu'annoter les primitives
géométriques dans des images de diagrammes complexes est très
chronophage, le modèle de vectorisation a été pré-formé sur des corpus
artificiels générés dynamiquement. Le script de génération des données
d'entraînement choisit aléatoirement un arrière-plan, y ajoute des mots,
des nombres et des glyphes puis crée artificiellement un diagramme en
insérant des segments, des cercles et des arcs. Le script est conçu pour
que ces diagrammes aient une forte probabilité de présenter des formes
très caractéristiques comme les cercles concentriques et tangents, les
lignes parallèles et les arcs connectés, afin de simuler les structures
typiques. Les primitives sont dessinées avec des
variations aléatoires d'opacité, de largeur et de couleur. Les cercles
peuvent être remplis ou vides. Enfin, du bruit est ajouté en appliquant
un flou gaussien, et en supprimant de petites régions du diagramme pour
imiter la dégradation des documents historiques. Les données
d'entraînement ainsi générées présentent des configurations assez
complexes.

          \begin{figure}[H]
          \begin{center}
          \includegraphics[height=7cm]{figues/vecto_synthetic_data.png}
          \end{center}
          \caption{Données synthétiques générées pour l'entraînement du modèle de vectorisation.\footcite[Figure issue de la présentation de Syrine Kalelli à l'occasion de la conférence \eida 2024~:][]{noauthor_eida_nodate-1}}
          \label{fig:vecto_synthetic} \end{figure}

Enfin, le modèle de similarité présente un troisième exemple, puisque
SegSwap est pré-entraîné sur de la donnée synthétique. Le script de
génération prend des parties aléatoires d'une images et les copie-colle
au-dessus d'une autre image. Les trois images (source, cible et
superposition) sont placées dans le même dataset d'entraînement, ainsi
le modèle apprend à retrouver ce qui, dans la superposition, vient de la
source, et ce qui vient de la cible.

          \begin{figure}[H]
          \begin{center}
          \includegraphics[height=3cm]{figues/segswap_blended_images.png}
          \end{center}
          \caption{Données d'entraînement du modèle Segswap.}
          \label{fig:segswap} \end{figure}

\hypertarget{les-donnees-reelles}{%
\subsection{Les données réelles}\label{les-donnees-reelles}}

S'appuyer sur les modèles \textit{off-the-shelf}, sur de larges \textit{datasets}
généralistes, ou sur des données synthétiques permet une implémentation
facilitée de la vision dans des projets et constitue une base solide.
Toutefois, les sources tenant aux deux projets (\vhs et \eida) sont trop
spécifiques pour se contenter de modèles généralistes ou formés sur des
données artificielles. Même si ces derniers peuvent offrir des performances
de base, ils risquent de manquer de précision et de sensibilité aux
particularités des documents historiques. Les corpus artificiels
présentent des configurations délibérément complexes pour s'approcher le
plus possible des difficultés que le modèle pourrrait rencontrer sur les
données réelle. Elles sont cependant irréalistes et insuffisantes pour
permettre aux modèles de généraliser sur des diagrammes réels.

En atteste la comparaison des performances de docExtractor et \yolov sur
les données d'\eida. docExtractor\footcite{monnier_docextractor_2020}, entraîné
sur des données synthétiques mimant les documents historiques serait en
théorie plus adapté au traitement d'images de pages de manuscrits, avec
du texte et des illustration côté à côte, d'autant qu'il intègre des
outils de traitement du texte (notamment pour la segmentation des
lignes)\footnote{\eida envisage l'implémentation d'un outil d'extraction
  et transcription des labels et des textes qui entourent les diagrammes}.
Pourtant, sans fine-tuning sur des données réelles, il présente des
performances équivalentes à celles de \yolov\footcite[p.45]{norindr_traitement_2023}. Cela
souligne que même les modèles off-the-shelf entraînés sur un corpus
assez spécifique et complexe, mais synthétique, ne dispense pas d'un
entraînement sur des données réelles, au même titre que les modèles très
généralistes comme \yolov.

Alors, le modèle de base \yolov tel que mis à disposition par
Ultralytics est entraîné sur de grands ensembles de données réelles, ce
qui constitue une base solide pour la classification des objets du
monde. L'utilisation de SynDoc permet ensuite de compléter
l'apprentissage initial en exposant le modèle à des exemples variés et
spécifiques aux documents historiques, augmentant ainsi sa capacité de
généralisation. Ces similis de manuscrits anciens offrent l'avantage de
pouvoir être produits en grandes quantités et de couvrir un large
éventail de scénarii et de configurations difficiles à obtenir dans des
ensembles de données réelles. Puis le modèle est entraîné sur les
données de \vhs, qui sont de réelles pages de documents historiques
contenant une large diversité d'illustrations. Ces données apporteront
une dimension supplémentaire de pertinence au modèle, en l'exposant à
des particularités des documents historiques réalistes. Enfin, \yolov
est entraîné sur les données d'\eida, qui sont orientées spécifiquement
vers les diagrammes, afin qu'il détecte uniquement ces derniers.

Quant au modèle de vectorisation développé par Syrine
Kalleli\footcite{kalleli_historical_2024}, il est formé
sur des données synthétiques générées à la volée par un script. Mais le
corpus de diagrammes d'\eida est particulièrement caractéristique et le
modèle n'aurait pu être optimal sans avoir appris sur des images de
diagrammes issus de manuscrits réels. Un corpus d'entraînement de 303
diagrammes extraits de manuscrits et de gravures a donc été constitué et
annoté par les historien.nes. Ces diagrammes sont issus de sources latines, arabes,
grecques, hébreuses ou chinoises, datant du \textsc{xii}\ieme au \textsc{xviii}\ieme siècle, et ils
présentent en guise d'étiquettes plus de 3000 lignes, cercles et arcs. Le
ré-entraînement a permis le transfert des connaissances acquises sur la
tâche de détection des primitives sur les données réalistes.

Il sera également possible d'obtenir des meilleurs résultats sur la
similarité grâce à une évaluation des scores (qui constitue un jeu de
données annotées) et le ré-entraînement du modèle, pour donner des
résultats plus adaptés à la spécificité des données historiques.

D'ailleurs, cette étape d'annotation (le choix des exemples et des
étiquettes) revêt des enjeux importants. L'apprentissage spécifique se
fait à partir de données sélectionnées par les chercheur.ses~: les exemples
sur lequel l'algorithme d'apprentissage va itérer définissent le modèle.
Il est nécessaire de constituer un échantillon de données aléatoire et
représentatif, et de l'annoter en fonction de ce que l'on souhaite
obtenir en prédiction.

L'annotation des jeux de données est non seulement une étape clé, mais
aussi un bel exemple de collaboration chercheur.ses-ingénieur.es. Elle
nécessite la définition de normes pertinentes et rigoureuses. Travail
minutieux et chronophage, l'étiquetage des données peut engendrer des
erreurs et du bruit dans les données, car elle implique la subjectivité
des chercheur.ses et le regard parfois trop précis sur les sources desquels
les annotateurs sont experts.

Voici un exemple rencontré lors de la préparation des données pour
entraîner un modèle de segmentation du contenu textuel. Les sources
arabes et chinoises sont particulièrement verbeuses et les diagrammes
sont très souvent entourés des blocs de commentaires se mélangeant alors
aux légendes et aux labels. Doit-on considérer ces commentaires comme
faisant partie des éléments que l'on souhaite identifier ou bien les ignorer
? Cette décision est importante car si on les ignore, le modèle risque
de passer à côté d'éléments textuels pertinents. En revanche, si on les
inclut, il ramènera des commentaires sans rapport direct avec le
diagramme observé. On voit ici comment la binarité des modèles, qui se
reflète dans les normes d'annotation, est problématique et constitue une
limite au \ml. Un compromis doit être trouvé entre
l'automatisation, qui requiert une normalisation, des définitions
claires et binaires, et la nuance dans l'interprétation des
sources\footnote{Dans le cadre du projet, il a toujors été plus
  intéressant d'opter pour une définition extensive des objets à
  détecter, car prévision d'une correction des traitement. Et il est
  plus facile de supprimer un élémént pas pertinent que d'aller en
  rechercher un, surtout compte tenu de la taille des corpus des
  chercheur.ses. Vaut aussi pour la préparation des données pour
  l'entraînement du modèle d'extraction.}.

La normalisation peut bénéficier à l'écosystème de recherche dans le
domaine de l'\htr et de l'\ocr. À ce titre, il est pertinent d'envisager
l'utilisation du vocabulaire contrôlé SegmOnto pour l'annotation du
contenu textuel entourant les diagrammes. Cela permettrait de créer des
jeux de données réutilisables, à partager avec des projets poursuivant
des objectifs similaires.\footnote{https://segmonto.github.io/}. Encore
une fois, un compromis doit être trouvé entre les besoins de description
des chercheur.ses et les possibilités offertes par les vocabulaires
contrôlés.

Un autre exemple concerne le dernier entraînement du modèle d'extraction
: les résultats montrent que des diagrammes sont encore détectés en
transparence. La question s'est alors posée de chercher à corriger ce
défaut en donnant au modèle, à l'occasion d'un nouvel entraînement,
d'avantage d'exemples négatifs (diagrammes visibles par transparence
mais non annotés). Or il est préférable de se contenter de la correction
ou suppression manuelle de ces prévisions erronées, garantissant que le
modèle parvienne à détecter les diagrammes presque effacés.

Pour assurer la rigueur et la cohérence des annotations, les décisions
prises entre les chercheur.ses et les ingénieur.es peuvent être l'objet d'une
documentation ou d'ateliers d'annotation.

\hypertarget{loeil-de-la-machine-avantages-et-limites}{%
\subsection{L'oeil de la machine~: avantages et
limites}\label{loeil-de-la-machine-avantages-et-limites}}

Bien qu'il soit possible d'optimiser les performances d'un modèle
d'apprentissage automatique en l'entraînant sur un ensemble de données
spécifique, son interprétation des données reste limitée car
fondamentalement binaire, ce qui le rend parfois déficient pour la
recherche en histoire. Ainsi, il gèrera difficilement les cas limites et
ambigüs. La décision d'inclure ou d'exclure ces cas particuliers de
l'ensemble d'entraînement implique un arbitrage délicat. D'un côté, une
inclusion trop restrictive peut compromettre les capacités de
généralisation du modèle, c'est-à-dire sa capacité à s'adapter à de
nouvelles données. À l'inverse, une inclusion trop permissive risque de
dégrader la précision du modèle sur les cas plus typiques. Les
chercheur.ses espérant obtenir un modèle maximaliste, quitte à accepter un
certain degré d'erreur et de devoir supprimer les faux
positifs, de nombreux cas limites ont été inclus. Le cas des diagrammes
visibles en transparence (expliqué précédemment) en est un exemple
éloquent.

Une autre difficulté réside dans la définition même du ``diagramme
astronomique''. Les limites de ce concept ne sont pas si claires et
définitives pour les chercheur.ses, et pourtant le modèle a besoin d'une
définition rigoureuse et cohérente. Il paraît en effet difficile de
considérer les diagrammes astronomiques en dehors du contexte des
pratiques d'autres sciences et disciplines connexes. Par exemple, Le
\emph{Flores Almagesti} -- réécriture de l'Almageste datant du \textsc{xv}\ieme par
l'astronome Giovanni Bianchini -- présente une partie algébrique à
l'ouverture mathématique, induisant la présence de nouveaux types de
diagrammes d'inspiration euclidienne. Pour retracer la source de ces
derniers, il est nécessaire de considérer les traités d'Euclide ou
autres travaux d'algèbre. Ceux-ci ne sont pas des traités
\emph{astronomiques}, bien qu'il ne soit pas certain que ces disinctions
contemporaines aient été aussi rigide à l'époque et aient eu un
quelconque sens pour les acteurs historiques. Les sources byzantines
confirment cette complexité~: les diagrammes y sont nommés
\emph{katagraphai}, indépendamment du domaine scientifique auquel ils
appartiennent. Également, de nombreux travaux astronomiques sont groupés
dans des témoins qui contiennent des œuvres issus de domaines divers.
C'est le cas avec les sources chinoises, comme le \emph{Chongzhen
lishu}, qui se présente généralement annexé d'une série de traités
mathématiques. Par conséquent, les diagrammes euclidiens ont été gardés
lors de la préparation des données, et l'algorithme de détection les
classe comme ``diagramme'', même s'ils ne constituent pas l'objet
principal des chercheur.ses.

En ce qui concerne les autres types de diagrammes non strictement
astronomiques (géométriques, harmoniques, logiques, illustrations de
constellations), une approche plus sélective a été adopté afin d'éviter
un modèle trop maximalistes. Ces éléments, bien que potentiellement
intéressants, n'ont pas été inclus dans la phase de détection
automatique.

Ainsi l'œil de la \cv contraint à des choix méthodologique
potentiellement inconfortables, mais en même temps il peut aider à
mesurer les impulsions des chercheur.ses, à mieux définir les objectifs de
recherche et à prioriser les éléments les plus pertinents. Ainsi, la
vision par ordinateur oblige les chercheur.ses à s'adapter à une logique
algorithmique qui, tout en limitant certaines interprétations
subjectives, offre l'opportunité de développer des modèles conceptuels et des méthodologies très rigoureuses.

\vspace{2cm}

On aura voulu montrer dans cette section l'importance du travail des
interfaces, qui font le lien entre le chercheur.se et la donnée d'une part,
et entre le chercheur.se et des pratiques d'autre part. Plus que de simples
ornements, des interfaces performants impactent l'analyse des données,
ainsi que l'adhésion et l'efficacité des utilisateur.rices. Elles constituent
le pont entre les chercheur.ses, souvent peu familiers des bases de
données, et les données complexes qu'ils produisent. Une interface
permet de faciliter la navigation entre les sources, même si elles sont
hétérogènes, et permet aux chercheur.ses de retrouver facilement les données
dont ils ont besoin, grâce à des fonctionnalités de recherche
pertinentes. L'utilisation de formulaires de saisie réduit les erreurs
et permet une gestion normalisée des métadonnées, favorisant la
trouvabilité et la réutilisation des données par d'autres chercheur.ses.
Des outils de visualisation leur permettent une exploration de leurs données centrée sur l'élément graphique, unité de base du modèle.

L'interface permet aussi de guider le chercheur.se dans un protocole.
Unifier l'accès aux données et aux modèles de vision par ordinateur
implique d'élaborer des \textit{workflows}, facilitant la mise en œuvre de méthodes standardisées. L'interface facilite leur prise en main,
favorisant ainsi la cohérence des méthodes au sein des
équipes, ou entre plusieurs équipes de recherche (à l'instar de
l'annotation des prédictions, destinées à être réutilisés pour
l'entraînement des modèles). Une interface bien conçue peut donc
faciliter notamment le travail en équipe, améliorant le partage des pratiques.

Afin de répondre à ces exigences, le travail de front est clé. La
plateforme \aikon a été pensée pour proposer des interfaces performantes pour les fonctionnalités de recherche et de visualisation diverses. Ce travail pour rendre
la plateforme réactive et accueillante anticipe en outre la valorisation et la médiation des données de la recherche vers un public plus large (via la construction d'une plateforme
publique). Les choix techniques sont faits dans ce sens, l'utilisation
d'un \textit{framework front-end} permettant d'améliorer les performances et de
garantir une fluidité d'interaction, tandis que l'adoption d'un
\textit{framework CSS} assure une uniformité et une cohérence visuelle, dotant la
plateforme d'une identité graphique forte.

        \clearemptydoublepage

        \chapter*{Conclusion partielle}

Le travail des chercheur.ses sur de grandes quantités de données
nécessitent le développement d'outils adaptés. La performance des
modèles d'\ia ne suffit pas~: il faut les rendre opérationnels au sein
des environnements de recherche. Là se situe l'enjeu de la plateforme
\aikon, dont l'objectif est de rendre compatibles les outils techniques
et les pratiques des ss. Elle a pour ambition de façonner un
véritable \si incluant les outils de \cv pour
systématiser les traitements, des fonctionnalités de gestion
documentaire, le tout en préservant le rôle central de l'interprétation
humaine.

L'objectif est aussi de concevoir une plateforme adaptable à une
multitude de projets de recherche en études visuelles. En adoptant une
architecture modulaire, divers projets peuvent l'exploiter comme une
`coquille', en personnalisant les protocoles selon leurs besoins
spécifiques. Cette flexibilité contribue à la pérennité et l'évolution
de l'outil.

À l'heure actuelle, l'application est à déployer sur des serveurs
personnels. Cependant, l'objectif à long terme est de développer une
plateforme en ligne, à l'exemple d'e-Scriptorium, dédiée à l'analyse
automatique de documents multimodaux. En offrant des fonctionnalités de
correction des traitement et d'exploration de résultats, elle permettra
la création de corpus enrichis. Et en rationalisant les méthodes de
travail des historiens, elle ouvrira de nouvelles perspectives en
matière de collaboration et de partage des outils comme des données. Se
conformant aux principes \fair et aux normes internationales d'interopérabilité, elle pourra faciliter la migration des données vers
d'autre systèmes (via des \api), et permettre ainsi les
comparaisons avec d'autres corpus. Elle favoriserait ainsi son intégration
dans des écosystèmes de recherche plus larges.

\clearemptydoublepage
    
    \chapterNo{Conclusion}
    \addcontentsline{toc}{chapter}{Conclusion}
    \begin{kwote}
``Les outils façonnent la pensée. Ce que nous pouvons penser et ce que nous pouvons dire résulte d’une dynamique dans laquelle les outils et les techniques jouent un rôle fondamental.''\footcite[p.31]{epron_ledition_2018}
\end{kwote}

\citeauthor{epron_ledition_2018} avancent l'idée selon laquelle les outils ne sont pas de simples instruments passifs, mais des acteur dans la construction de la pensée. Ils ne se limitent pas à faciliter des tâches, mais façonnent une conceptualisation du monde et des idées. Les outils de la recherche, qu'ils soient numériques ou analogiques, structurent des gestes intellectuels. Les outils numériques ne sont pas de simples remédiations des outils traditionnels. Ils offrent de nouvelles possibilités, de nouvelles façons d'interagir avec l'information et de la manipuler. Ils peuvent catalyser les dynamiques collaboratives autour de ces nouvelles méthodes et ces nouvelles perceptives. Un cadre technique contient un cadre conceptuel, et il y a donc un enjeu à forger des outils de traitement de la donnée qui vont dans le sens du partage des pratiques.

Ce travail s'est attaché à exposer l'importance et les moyens de partager le développement et l'utilisation des outils numériques de la recherche, en prenant pour exemple l'application du \dl à la sémantification et l'enrichissement des sources en histoire de l'astronomie. Les questionnements, impliquant la modélisation, l'accès à la donnée, et l'élaboration des modèles de \cv, s'orientent vers la conception d'un \si dédié à des documents numérisés variés et hétérogènes, et au traitement de leurs éléments visuels. Les outils de \dl sont mis à disposition via une plateforme web modulaire et réemployable dans différents contextes (\aikon).

La première partie de ce mémoire, qui portait sur le contexte disciplinaire des projets \eida/\vhs, a mis en évidence la complexité des données traitées, ainsi que la diversité des besoins des chercheur.ses. En conséquence, une modélisation de données flexible et interopérable, ne renonçant pas aux exigences de description des sources, a constitué un axe de réflexion. Par ailleurs, la variété des modes d'accès aux données est une piste d'analyse importante. Les données visuelles et les besoins spécifiques des historien.nes, notamment l'accès à leurs propres images stockées localement, nécessitent des solutions techniques adaptées. Bien que \iiif offre un cadre prometteur pour l'échange de données volumineuses, il ne constitue pas une solution universelle.

Dans un second temps, nous avons présenté sur les enjeux et défis liés à l'utilisation du \dl et de la \cv dans le traitement des données historiques, notamment les moyens de gérer la faible disponibilité de corpus annotés. Nous nous sommes ensuite penché sur une application des traitements d'\ia, en dressant les exigences fonctionnelles d'un outil d'édition des \svgs (sorties de l'algorithme de vectorisation) intégré à la plateforme. Cet outil a vocation à fédérer les pratiques des chercheur.ses pour l'édition des diagrammes astronomiques. 

La troisième partie est consacrée aux spécificités techniques de la conception d'une plateforme modulaire destinée à héberger les outils de gestion de données et les instruments de \cv. Cette plateforme présente une architecture modulaire, reposant sur une séparation claire des fonctionnalités dans des modules et des composants applicatifs distincts. L'inférence des modèles, qui exige une puissance de calcul importante, est optimisée par l'utilisation d'un \gpu sur lequel tourne une \api prévue à cet effet. Par ailleurs, le choix d'une bibliothèque \textit{front-end} permet de créer des interfaces utilisateur.rice performantes, facilitant ainsi l'interaction des chercheur.ses avec un processus de travail itératif, alternant phases de traitement par les modèles de vision et corrections afin de garantir la qualité des résultats. Ces derniers constituent des corpus annotés qualitatifs qui pourront être réutilisés dans le cadre de l'entraînement des modèles. 

\textit{La modularité des outils~: pourquoi~?} 

Cette étude de cas a montré l'importance de l'accessibilité et du partage au sein de la communauté de la recherche, non seulement des données, mais aussi des outils qui vont permettre de les traiter. Cette démarche favorise à la fois l'évolutivité et la pérennité des outils numériques, permet d'établir des pratiques communes au sein de la communauté scientifique et, enfin, démocratise l'accès à des outils innovants qui ouvrent de nouvelles perspectives d'analyse des sources.

Les outils taillés trop spécifiquement sur un projet ou sur une question de recherche sont fragiles. Leur cycle de vie est intimement lié aux financements, ce qui les expose à l'obsolescence dès lors que les ressources s'épuisent. De tels développements, souvent coûteux en temps et en moyens (financiers comme humains), ne garantissent pas la pérennité des outils. Pour assurer leur survie et favoriser leur évolution, il est alors intéressant de s'inscrire dans des projets collaboratifs, et en conséquence, d'adopter une approche modulaire. En fédérant les efforts et en mutualisant les ressources, il est possible non seulement de stimuler l'innovation mais aussi de garantir la maintenance à long terme de ces outils, les rendant ainsi plus robustes et pérennes.

L'ouverture des outils et l'extensivité des chaînes de traitement des données a aussi des implications scientifiques d'importance. Ces aspects inscrivent une démarche scientifique dans l'optique de la collaboration et du partage des pratiques, favorisant leur reproductibilité, et renforçant la transparence des méthodes. On l'aura évoqué dans le \hyperlink{chapitre-6-vers-edition}{chapitre 6}, la diversité des pratiques d'édition scientifique des diagrammes présents dans les traités d'histoire des sciences témoigne de la richesse des points de vue sur les sources, mais peut aussi constituer un obstacle à la coopération au sein du champ de recherche. La multiplication des outils, des méthodologies et des normes entraîne une fragmentation des pratiques et remet en question l'utilité scientifique des contenus produits. Chaque chercheur.se ou chaque équipe développe ses propres outils, avec ses propres formats, et ses normes adaptées à leur angle d'approche épistémologique et heuristique. Devant ce constat, l'élaboration d'un outil d'édition numérique peut alors porter un cadre de travail partageable. 

S'adapter à l'hétérogénéité des données et des contextes matériels ne favorise pas seulement la collaboration et la cohérence des pratiques au sein d'un champ de recherche, mais facilite également l'accès aux outils d'\ia pour l'analyse à grande échelle des corpus, ouvrant ainsi de nouvelles perspectives sur les sources.

La plateforme \aikon et son interface charpentent alors un véritable environnent de recherche numérique, dédié à l'analyse des sources, orienté vers l'interprétation par les chercheur.ses. En centralisant les ressources et en offrant des fonctionnalités d'annotation et de visualisation, cet outil fait office de nouveau laboratoire, et change la manière dont les chercheur.ses interagissent avec les données historiques, permettant au domaine des études visuelles d'exploiter le paradigme du \textit{big data}. Elle est conçue pour s'adapter au traitement de documents numérisés divers, pour exploiter les éléments graphiques qu'ils contiennent.

\textit{Un outil généraliste et adaptable à plusieurs projets de recherche~: comment~?}

Le cas \aikon a en outre permis d'identifier les principaux enjeux liés à la mise en œuvre de la modularité. Nous nous sommes interrogés sur les fondements d'un outil suffisamment polyvalent et offrant une base solide pour permettre sa spécialisation.

Premièrement, la modélisation de la donnée est un aspect à considérer. Se référer à des modèles conceptuels standards est alors essentiel pour disposer de concepts larges et adaptables aux données du patrimoine. Si le modèle de données adopté n'est pas taillé spécifiquement pour décrire les source de \eida, il est néanmoins suffisamment précis pour répondre aux besoins actuels des chercheur.ses, et bénéficie de sa généralité pour s'adapter aux sources de \vhs et à une grande diversité de types de documents, permettant l'étude des éléments graphiques qu'ils contiennent. 

Le domaine de recherche de la \cv, dans son essence même, illustre en outre les dynamiques collaboratives dans la construction des instruments de traitement de la donnée. En partant de modèles génériques pré-entraînés, il est possible de construire des architectures complexes spécialisées sur des tâches et des données particulières. Pour assurer la cohérence et la réutilisabilité des données, il est important d'encourager l'utilisation de vocabulaires contrôlés et de normes pour l'annotation des données.

Une attention particulière a été portée aux interfaces de la plateforme \aikon, pour faciliter le travail des chercheur.ses et la diffusion de données. Un outil doté d'une interface graphique accueille un vaste éventail d'utilisateur.rice et anticipe en outre la médiation des résultats des recherches vers un public plus large. 

D'autre part, le recours à des protocoles standards et interopérables, tels que le \iiif, assure la compatibilité des systèmes. Les formats de sortie des traitements sont aussi des formats libres et manipulables~: \textsc{txt}, Numpy ou \svg. 

La personnalisation s'exprime par ailleurs dans le fait qu'\aikon peut être déployée dans différents environnements matériels, la possibilité d'installation en locale démontrant sa légèreté et sa portabilité. Mais les problématiques liées à l'ouverture et la modularité du code concernent aussi la conception d'une architecture qui puisse s'adapter à une augmentation du volume de données et du nombre d'utilisateur.rices. Ainsi les besoins côté ingénierie vont parfois rentrer en conflit avec cette volonté de légèreté si importante pour inclure un maximum de contextes de recherche et d'utilisateur.rices différents. La gestion de volumes de données croissants ou des outils d'\ia plus lourds requiert du matériel et des architectures applicatives plus robustes que celles développées avec les moyens des laboratoire de recherche en \shs. Pour l'instant, \aikon est taillé pour -- et utilisé sur -- des corpus qui, bien que larges, restent limités comparés aux collections des bibliothèques par exemple. Un passage à l'échelle reposerait sur l'implication de consortiums ou d'institutions, qui disposent des technologies et des infrastructures capables de gérer de grandes quantités de données de manière efficace. Nous conclurons et ouvrirons sur cette idée~: la recherche constitue le terreau de l'innovation~; néanmoins, son passage à l'échelle institutionnelle exige des compétences en ingénierie qui transcendent ceux des projets de recherche, même collaboratifs.
    

\appendix
    \part*{Annexes}	
    \addcontentsline{toc}{part}{Annexes}
    
    \chapter[Évolution du modèle de données]{\label{data_models}Évolution du modèle de données}
	    \section{Modèle de données initial de l'application VHS}
	\begin{figure}[H]
		\centering
		\includegraphics[width=16cm]{figues/vhs_data_model.png}
		\caption{Modèle de données de l'application \vhs avant refonte.*}
		\label{fig:vhs_data_model}
	\end{figure}

\section{Modèle de données de l'application VHS/EIDA}
	\begin{figure}[H]
		\centering
		\includegraphics[width=16cm]{figues/new_model.png}
		\caption{Nouveau modèle de données de \eida appliqué à \vhs.*}
		\label{fig:eida_data_model}
	\end{figure}

 \section{Modèle de données de l'application AIKON}
	\begin{figure}[H]
		\centering
		\includegraphics[width=16cm]{figues/model_aikon.png}
		\caption{Nouveau modèle de données en vue de la refonte de l'application \aikon.*}
		\label{fig:aikon_model}
	\end{figure}

 
	    \clearemptydoublepage
	
    \chapter[Module vectorisation~: description des développements]{\label{module_vecto}Module vectorisation~: description des développements}
	    Les méthodes développées pendant mon stage ont posé les bases d’un \textit{workflow}, qui a très rapidement évolué. Ces méthodes ont ensuite été itérativement raffinées pour orienter la plateforme vers plus de modularité, notamment grâce à l'implémentation de l'instance \tr, sur laquelle repose désormais le lancement des actions. Cette annexe présente le développement du \textit{workflow} de lancement de la vectorisation les \wits, ainsi que son évolution.  

\section{Workflow}
	\begin{itemize}
    \item Requête utilisateur.rice dans l’application \aikon~;
    \item Lancement d'un \tr~:
    \begin{itemize}
        \item Identification des \wits~;
        \item Parser les \mans et ramener les \URLs des régions d'images dans une liste~;
        \item Création d'un fichier \json 
        \item Envoie du \json via requête \http POST à l’\api \textit{endpoint}~;
    \end{itemize}
    \item Inférence du modèle dans Discover-Demo~: 
    \begin{itemize}
        \item Vérification du \json~; 
        \item Lancement d'une tâche~; 
        \item Enregistrement des images~;  
        \item Inférence avec le modèle~: écriture des fichiers \svgs~; 
        \item Envoi d’un \textsc{zip} contenant les fichiers à l'application via requête \http POST~;
    \end{itemize}
    \item Récupération du \textsc{zip} au \textit{endpoint} de l’application~;
    \begin{itemize}
        \item Dézip, lecture et écriture des \svgs en \textit{backend} dans l’application \aikon~;
        \item Vérification dans le répertoire de résultats (dossier dans \texttt{mediafiles}) que deux fichiers du même nom n’existent pas~;
        \item S’il existe déjà un fichier du même nom, il est écrasé par le nouveau~;
    \end{itemize}
    \item Apparition des visualisations dans l'interface~;
\end{itemize}

\section{Settings}

	\subsection{AIKON}
	
Le fichier de configuration (\texttt{.env}) de l'application doit contenir l'\URL de l'\api à laquelle elle se connecte. Afin d'activer les fonctionnalités de vectorisation, il est aussi nécessaire de spécifier le module correspondant dans les paramètres de configuration.  

	\begin{lstlisting}[language=python, frame=single, breaklines=true, caption={Extrait du fichier de configuration de l'application.}]
# Computer vision apps to install
ADDITIONAL_MODULES=regions,similarity,vectorization
 \end{lstlisting}


	\subsection{Discover-Demo}
	
Dans le fichier \texttt{.env}, la variable \texttt{INSTALLED\_APPS} contient la liste des modules à charger. Pour activer les fonctionnalités, notamment la vectorisation, il est nécessaire d'ajouter les modules correspondants à cette liste.
	
	\begin{lstlisting}[language=python, frame=single, breaklines=true, caption={Extrait du fichier de configuration de l'\api.}]
# apps (folder names) to be imported to the API
INSTALLED_APPS=dticlustering,watermarks,similarity,region,vectorization\end{lstlisting}

\section{Envoi d'un traitement de vectorisation}

\subsection{Initialisation de la requête}

Le processus de vectorisation était initialement déclenché via l'interface d'administration Django, en utilisant le mécanisme des \texttt{@admin.action} (des fonctions appelées avec une liste d’objets sélectionnés depuis la page de liste pour modification, elles agissent donc au niveau du témoin). Cette approche, bien que fonctionnelle, présentait des limites~: notamment concernant le suivi des traitements et leur application sur différents objets de la base.

Une refonte des processus a été entreprise par Jade Norindr, Ségolène Albouy et Fouad Aouinti. Cette évolution a conduit à la création d'un formulaire dédié au lancement de tous les traitements, standardisant ainsi les modes de communication avec l'\api. Ce formulaire, accessible depuis l'interface utilisateur.rice, permet d'appliquer le traitement sur un ensemble d'objets précédemment sélectionnés. 

	\begin{figure}[h]
	\centering
	\includegraphics[width=15cm]{figues/form_traitement.png}
	\caption{Capture d'écran du formulaire d'envoi d'un traitement dans \aikon.}
	\label{fig:eida_send_manifest}
	\end{figure}

Le lancement de la vectorisation se fait donc désormais au niveau de l'entité \tr reliée à un ensemble de témoins. Le formulaire de lancement permet de créer une instance de cette entité, et par conséquent, tous les témoins associés seront soumis au processus de vectorisation.

\subsection{Création du \textsc{json}}

Initialement la fonction utilitaire \texttt{vectorization\_request\_for\_one} était utilisée pour formater un fichier \json à envoyer au \textit{endpoint} de l'\api. Afin de traiter par lots un ensemble de témoins, la fonction \texttt{vectorization\_request} permettait d'itérer sur une liste de témoins, en appelant récursivement la fonction \texttt{vectorization\_request\_for\_one} pour chacun d'entre eux.

\begin{lstlisting}[language=python, frame=single, breaklines=true, caption={Extrait de la fonction gérant l'envoi de \json à l'\api dans \texttt{app/vectorization/utils.py}}]
try:
	response = requests.post(
	url=f"{CV_API_URL}/vectorization/start",
	json={
		"doc_id": regions.get_ref(),
		"model": f"{VECTO_MODEL_EPOCH}",
		"images": get_regions_urls(regions),
		"callback": f"{APP_URL}/{APP_NAME}/get-vectorization",
		},
	)
\end{lstlisting}

Avec les récents développements, le \json utilisé pour communiquer avec l'\api est généré par la fonction \texttt{prepare\_request}, présente dans le fichier \texttt{utils.py} de chaque module. 

La boucle externe itère sur les \wits et vérifie qu'il n'y a pas déjà de vectorisation effectuée. La boucle interne itère sur les annotations des \wits. Le dictionnaire \texttt{regions\_dic} est créé pour mapper les références des témoins annotés aux \URLs des annotations. La fonction renvoie un dictionnaire contenant les données nécessaires pour la requête de vectorisation.

\begin{lstlisting}[language=python, frame=single, breaklines=true, caption={Structurer les \URLs de régions d'images.}]
def prepare_request(witnesses, treatment_id):
    regions_list = []
    regions_dic = {}

    try:
        for witness in witnesses:
            if witness.has_vectorization():
                log(
                    f"[vectorization_request] Witness {witness.get_ref()} already has vectorizations"
                )
                pass
            else:
                regions_list.extend(witness.get_regions())

        if regions_list:
            for regions in regions_list:
                regions_dic.update({regions.get_ref(): get_regions_urls(regions)})

            return {
                "experiment_id": f"{treatment_id}",
                "documents": regions_dic,
                "model": f"{VECTO_MODEL_EPOCH}",
                "callback": f"{APP_URL}/{APP_NAME}/get-vectorization",  # URL to which the SVG zip file must be sent back
                "tracking_url": f"{APP_URL}/{APP_NAME}/api-progress",
            }

        else:
            return {
                "message": f"No regions to vectorize for all the selected {WIT}es"
                if APP_LANG == "en"
                else f"Pas de regions à vectoriser pour tous les {WIT}s sélectionnés"
            }

    except Exception as e:
        log(
            f"[prepare_request] Failed to prepare data for vectorization request",
            e,
        )
        raise Exception(
            f"[prepare_request] Failed to prepare data for vectorization request"
        )
	\end{lstlisting}
 
La fonction reçoit une liste de \wits, générée par une tâche appelée par une méthode \texttt{post\_save} de l'instance de \tr~:

\begin{lstlisting}[language=python, frame=single, breaklines=true, caption={Méthode \texttt{post\_save du \tr.}}]
# vhs/app/webapp/models/treatment.py
@receiver(post_save, sender=Treatment)
def treatment_post_save(sender, instance, created, **kwargs):
    if created:
        get_all_witnesses.delay(instance)
\end{lstlisting}


\begin{lstlisting}[language=python, frame=single, breaklines=true, caption={Ramener tous les \wits à partir des entités reliées aux \tr et lancer de la tâche.}]
# vhs/app/webapp/tasks.py
@celery_app.task
def get_all_witnesses(treatment):
    try:
        witnesses = treatment.get_witnesses()
        treatment.start_task(witnesses)
    except Exception as e:
        treatment.on_task_error(
            {
                "error": f"Error when retrieving documents from set: {e}",
                "notify": treatment.notify_email,
            },
        )
\end{lstlisting}

\section{Discover-Demo~: module vectorization}

\subsection{Réception à l'\emph{endpoint} de l'API~:}

L'\emph{endpoint} (dans \texttt{api/app/vectorization/routes.py}) sert de passerelle pour lancer la tâche de vectorisation. Il valide la requête entrante, extrait les paramètres nécessaires et transfère la tâche au processus en arrière-plan pour son exécution. 

 \begin{lstlisting}[language=python, frame=single, breaklines=true, caption={\emph{Endpoint} \texttt{start\_vectorization}.}]

@blueprint.route("start", methods=["POST"])
@shared_routes.get_client_id
@shared_routes.error_wrapper
def start_vectorization(client_id):
    """
    A list of images to download + relevant data
    """
    if not request.is_json:
        return "No JSON in request: Vectorization task aborted!"

    json_param = request.get_json()
    console(json_param, color="cyan")

    experiment_id = json_param.get("experiment_id")
    documents = json_param.get("documents", {})
    model = json_param.get("model", None)

    notify_url = json_param.get("callback", None)
    tracking_url = json_param.get("tracking_url")

    return shared_routes.start_task(
        compute_vectorization,
        experiment_id,
        {
            "documents": documents,
            "model": model,
            "notify_url": notify_url,
            "tracking_url": tracking_url,
        },
    )
 
	\end{lstlisting}

 \subsection{Configuration de la tâche de fond}

Dans le fichier \texttt{api/app/vectorization/tasks.py}, la fonction \texttt{compute\_vectorization} définit un acteur Dramatiq qui exécute la tâche de vectorisation en \textit{background}. Elle initialise une nouvelle instance de la classe \texttt{LoggedComputeVectorization}, en lui passant le logger et les paramètres fournis. Elle appelle la méthode \texttt{run\_task} sur l'objet créé, qui va lancer le processus. 

 \begin{lstlisting}[language=python, frame=single, breaklines=true, caption={Tâche \texttt{dramatiq} \texttt{compute\_vectorization}.}]
@dramatiq.actor(time_limit=1000 * 60 * 60, max_retries=0, queue_name=VEC_QUEUE)
def compute_vectorization(
    experiment_id: str,
    documents: dict,
    model: str,
    notify_url: Optional[str] = None,
    tracking_url: Optional[str] = None,
    logger: TLogger = LoggerHelper,
):
    """
    Run vecto task on lists of URL
    """
    vectorization_task = LoggedComputeVectorization(
        logger,
        experiment_id=experiment_id,
        documents=documents,
        model=model,
        notify_url=notify_url,
        tracking_url=tracking_url
    )
    vectorization_task.run_task()

	\end{lstlisting}

 \subsection{inférence avec le modèle}

La classe \texttt{LoggedComputeVectorization} gère les processus de vectorisation en utilisant les paramètres fournis. La méthode \texttt{run\_task} lance sur l'instance le téléchargement des données, l'inférence du modèle, le traitement des résultats (renvoi des \svgs sous forme de \textsc{zip}) et la notification, avec journalisation tout au long du processus.

Dans le fichier \texttt{api/app/vectorization/lib/vectorization.py}~: 

 \begin{lstlisting}[language=python, frame=single, breaklines=true, caption={Classe \texttt{ComputeVectorization}.}]
 class ComputeVectorization:
    def __init__(
        self,
        experiment_id: str,
        documents: dict,
        model: Optional[str] = None,
        notify_url: Optional[str] = None,
        tracking_url: Optional[str] = None,
    ):
        self.experiment_id = experiment_id
        self.documents = documents
        self.model = model
        self.notify_url = notify_url
        self.tracking_url = tracking_url
        self.client_id = "default"
        self.imgs = []

    def run_task(self):
        pass

    def check_dataset(self):
        if len(list(self.documents.keys())) == 0:
            return False
        return True

    def task_update(self, event, message=None):
        if self.tracking_url:
            send_update(self.experiment_id, self.tracking_url, event, message)
            return True
        else:
            return False


class LoggedComputeVectorization(LoggingTaskMixin, ComputeVectorization):
    def run_task(self):
        if not self.check_dataset():
            self.print_and_log_warning(f"[task.vectorization] No documents to download")
            self.task_update("ERROR", f"[API ERROR] Failed to download documents for vectorization")
            return

        error_list = []

        try:
            for doc_id, document in self.documents.items():
                self.print_and_log(
                    f"[task.vectorization] Vectorization task triggered for {doc_id} !"
                )
                self.task_update("STARTED")

                self.download_dataset(doc_id, document)
                self.process_inference(doc_id)
                self.send_zip(doc_id)

            self.task_update("SUCCESS", error_list if error_list else None)

        except Exception as e:
            self.print_and_log(f"Error when computing vectorizations", e=e)
            self.task_update("ERROR", "[API ERROR] Vectorization task failed")

    def download_dataset(self, doc_id, document):
        self.print_and_log(
            f"[task.vectorization] Dowloading images...", color="blue"
        )
        for image_id, url in document.items():
            try:
                if not is_downloaded(doc_id, image_id):
                    self.print_and_log(
                        f"[task.vectorization] Downloading image {image_id}"
                    )
                    download_img(url, doc_id, image_id)

            except Exception as e:
                self.print_and_log(
                    f"[task.vectorization] Unable to download image {image_id}", e
                )

    def process_inference(self, doc_id):
        model_folder = Path(MODEL_PATH) 
        model_config_path = f"{model_folder}/config_cfg.py" 
        epoch = DEFAULT_EPOCHS if self.model is None else self.model
        model_checkpoint_path = f"{model_folder}/checkpoint{epoch}.pth"
        args = SLConfig.fromfile(model_config_path)
        args.device = 'cuda'
        args.num_select = 200

        corpus_folder = Path(IMG_PATH)
        image_paths = glob.glob(str(corpus_folder / doc_id) + "/*.jpg")
        output_dir = VEC_RESULTS_PATH / doc_id
        os.makedirs(output_dir, exist_ok=True)

        model, criterion, postprocessors = build_model_main(args)
        checkpoint = torch.load(model_checkpoint_path, map_location='cpu')
        model.load_state_dict(checkpoint['model'])
        model.eval()

        args.dataset_file = 'synthetic'
        args.mode = "primitives"
        args.relative = False
        args.common_queries = True
        args.eval = True
        args.coco_path = "data/synthetic_processed"
        args.fix_size = False
        args.batch_size = 1
        args.boxes_only = False
        vslzr = COCOVisualizer()
        id2name = {0: 'line', 1: 'circle', 2: 'arc'}
        primitives_to_show = ['line', 'circle', 'arc']

        torch.cuda.empty_cache()
        transform = T.Compose([
            T.RandomResize([800], max_size=1333),
            T.ToTensor(),
            T.Normalize([0.485, 0.456, 0.406], [0.229, 0.224, 0.225])
        ])

        with torch.no_grad():
            for image_path in image_paths:
                try:
                    self.print_and_log(
                        f"[task.vectorization] Processing {image_path}", color="blue"
                    )
                    # Load and process image
                    im_name = os.path.basename(image_path)[:-4]
                    image = Image.open(image_path).convert("RGB")
                    im_shape = image.size
                    input_image, _ = transform(image, None)
                    size = torch.Tensor([input_image.shape[1], input_image.shape[2]])

                    # Model inference
                    output = model.cuda()(input_image[None].cuda())
                    output = postprocessors['param'](output, torch.Tensor([[im_shape[1], im_shape[0]]]).cuda(), to_xyxy=False)[0]

                    threshold, arc_threshold = 0.3, 0.3
                    scores = output['scores']
                    labels = output['labels']
                    boxes = output['parameters']
                    select_mask = ((scores > threshold) & (labels != 2)) | ((scores > arc_threshold) & (labels == 2))
                    labels = labels[select_mask]
                    boxes = boxes[select_mask]
                    scores = scores[select_mask]
                    pred_dict = {'parameters': boxes, 'labels': labels, 'scores': scores}
                    lines, line_scores, circles, circle_scores, arcs, arc_scores = get_outputs_per_class(pred_dict)

                    # Postprocess the outputs
                    lines, line_scores = remove_duplicate_lines(lines, im_shape, line_scores)
                    lines, line_scores = remove_small_lines(lines, im_shape, line_scores)
                    circles, circle_scores = remove_duplicate_circles(circles, im_shape, circle_scores)
                    arcs, arc_scores = remove_duplicate_arcs(arcs, im_shape, arc_scores)
                    arcs, arc_scores = remove_arcs_on_top_of_circles(arcs, circles, im_shape, arc_scores)
                    arcs, arc_scores = remove_arcs_on_top_of_lines(arcs, lines, im_shape, arc_scores)

                    # Generate and save SVG
                    self.print_and_log(f"[task.vectorization] Drawing {image_path}", color="blue")
                    #shutil.copy2(image_path, output_dir)
                    #décommenter cette ligne si on veut obtenir les images dans le répertoire de sortie
                    diagram_name = Path(image_path).stem
                    image_name = os.path.basename(image_path)
                    lines = lines.reshape(-1, 2, 2)
                    arcs = arcs.reshape(-1, 3, 2)

                    dwg = svgwrite.Drawing(str(output_dir / f"{diagram_name}.svg"), profile="tiny", size=im_shape)
                    dwg.add(dwg.image(href=image_name, insert=(0, 0), size=im_shape))
                    dwg = write_svg_dwg(dwg, lines, circles, arcs, show_image=False, image=None)
                    dwg.save(pretty=True)

                    ET.register_namespace('', "http://www.w3.org/2000/svg")
                    ET.register_namespace('xlink', "http://www.w3.org/1999/xlink")
                    ET.register_namespace('sodipodi', "http://sodipodi.sourceforge.net/DTD/sodipodi-0.dtd")
                    ET.register_namespace('inkscape', "http://www.inkscape.org/namespaces/inkscape")

                    file_name = output_dir / f"{diagram_name}.svg"
                    tree = ET.parse(file_name)
                    root = tree.getroot()

                    root.set('xmlns:inkscape', 'http://www.inkscape.org/namespaces/inkscape')
                    root.set('xmlns:sodipodi', 'http://sodipodi.sourceforge.net/DTD/sodipodi-0.dtd')
                    root.set('inkscape:version', '1.3 (0e150ed, 2023-07-21)')

                    arc_regex = re.compile(r'[aA]')
                    for path in root.findall('{http://www.w3.org/2000/svg}path'):
                        d = path.get('d', '')
                        if arc_regex.search(d):
                            path.set('sodipodi:type', 'arc')
                            path.set('sodipodi:arc-type', 'arc')
                            path_parsed = parse_path(d)
                            for e in path_parsed:
                                if isinstance(e, Line):
                                    continue
                                elif isinstance(e, Arc):
                                    center, radius, start_angle, end_angle, p0, p1 = get_arc_param([e])
                                    path.set('sodipodi:cx', f'{center[0]}')
                                    path.set('sodipodi:cy', f'{center[1]}')
                                    path.set('sodipodi:rx', f'{radius}')
                                    path.set('sodipodi:ry', f'{radius}')
                                    path.set('sodipodi:start', f'{start_angle}')
                                    path.set('sodipodi:end', f'{end_angle}')

                    tree.write(file_name, xml_declaration=True)

                    self.print_and_log(f"[task.vectorization] SVG for {image_path} drawn", color="yellow")

                except Exception as e:
                    self.print_and_log(f"[task.vectorization] Failed to process {image_path}", e)

            self.print_and_log(f"[task.vectorization] Task over", color="yellow")

    def send_zip(self, doc_id):
        """
        Zip le répertoire et envoie ce répertoire via POST à l'URL spécifiée.
        """
        try:
            output_dir = VEC_RESULTS_PATH / doc_id
            zip_path = output_dir / f"{doc_id}.zip"
            self.print_and_log(f"[task.vectorization] Zipping directory {output_dir}", color="blue")

            zip_directory(output_dir, zip_path)
            self.print_and_log(f"[task.vectorization] Sending zip {zip_path} to {self.notify_url}", color="blue")

            with open(zip_path, 'rb') as zip_file:
                response = requests.post(
                    url=self.notify_url,
                    files={
                        "file": zip_file,
                    },
                    data={
                        "experiment_id": self.experiment_id,
                        "model": self.model,
                    },
                )
            
            if response.status_code == 200:
                self.print_and_log(f"[task.vectorization] Zip sent successfully to {self.notify_url}", color="yellow")
            else:
                self.print_and_log(f"[task.vectorization] Failed to send zip to {self.notify_url}. Status code: {response.status_code}", color="red")
        
        except Exception as e:
            self.print_and_log(f"[task.vectorization] Failed to zip and send directory {output_dir}", e)
	\end{lstlisting}

\section{Réception des résultats dans AIKON}

Pour que les vectorisations soient retournées de l'\api à \aikon après l'inférence du modèle, un \textit{endpoint} est créé, dont le routage avec la bonne \URL est définie dans le fichier \texttt{urls.py} du module.

\begin{lstlisting}[language=python, frame=single, breaklines=true, caption={Routage de l'\textit{endpoint} pour la réception de vectorisations.}]
 path(
        f"{APP_NAME}/get-vectorization",
        receive_vectorization,
        name="get-vectorization",
    ),\end{lstlisting}

Cet \textit{endpoint} appelle la fonction \texttt{receive\_vectorization} du fichier \texttt{views.py} du module \texttt{vectorization}, qui reçoit un fichier .\textsc{zip} via une POST request de l'\api.  Grâce à une fonction utilitaire du fichier \texttt{utils.py} (\texttt{save\_svg\_files}), l'archive est extraite, les fichiers sont lus et écris dans le dossier \texttt{mediafiles}. Si les fichiers existent déjà, ils sont écrasés et réécrits. 

\begin{lstlisting}[language=python, frame=single, breaklines=true, caption={\emph{Endpoint} \texttt{receive\_vectorization} pour le retour des vectorisations dans l'application.}]
@csrf_exempt
def receive_vectorization(request):
    """
    Endpoint to receive a ZIP file containing SVG files and save them to the media directory.
    """
    if "file" not in request.FILES:
        return JsonResponse({"error": "No file received"}, status=400)

    file = request.FILES["file"]
    # treatment_id = request.DATA["experiment_id"]

    if file.name == "":
        return JsonResponse({"error": "File name is empty"}, status=400)

    if file and file.name.endswith(".zip"):
        try:
            temp_zip_path = default_storage.save("temp.zip", file)
            temp_zip_file = default_storage.path(temp_zip_path)

            save_svg_files(temp_zip_file)
            default_storage.delete(temp_zip_path)

            return JsonResponse(
                {"message": "Files successfully uploaded and extracted"}, status=200
            )
        except Exception as e:
            return JsonResponse({"error": str(e)}, status=500)
    else:
        return JsonResponse({"error": "Unsupported file type"}, status=400)
	\end{lstlisting}

Fonction utilitaire pour traiter le contenu du \textsc{zip}~: 

 \begin{lstlisting}[language=python, frame=single, breaklines=true, caption={Fonction utilitaire pour le traitement du contenu de l'archive.}]
 def save_svg_files(zip_file):
    """
    Extracts SVG files from a ZIP file and saves them to the SVG_PATH directory.
    """
    # Vérifie si le répertoire SVG_PATH existe, sinon le crée
    if not os.path.exists(SVG_PATH):
        os.makedirs(SVG_PATH)

    try:
        with zipfile.ZipFile(zip_file, "r") as zip_ref:
            for file_info in zip_ref.infolist():
                # TODO do not save jpg file
                # Vérifie si le fichier est un fichier SVG
                if file_info.filename.endswith(".svg"):
                    file_path = os.path.join(
                        SVG_PATH, os.path.basename(file_info.filename)
                    )

                    # Supprime le fichier existant s'il y en a un
                    if os.path.exists(file_path):
                        os.remove(file_path)

                    # Extrait le fichier SVG et l'écrit dans le répertoire spécifié
                    with zip_ref.open(file_info) as svg_file:
                        with open(file_path, "wb") as output_file:
                            output_file.write(svg_file.read())
    except Exception as e:
        log(f"[save_svg_files] Error when extracting SVG files from ZIP file", e)
        return False
    return True
	\end{lstlisting}



	    \clearemptydoublepage

	\chapter[Interfaces~: développement des vues personnalisées]{\label{dvt_interfaces}Interfaces~: développement des vues personnalisées}
	Le début de mon stage s'est focalisé sur le développement d'interfaces. Plus spécifiquement, j'ai travaillé sur l'affichage des 'régions d'image', anciennement appelées 'annotations' (correspondant aux extractions), ainsi que des vectorisations associées. J'ai pu ainsi approfondir ma compréhension du \textit{framework} Django en me familiarisant avec le processus de création de vues personnalisées. Néanmoins, ces vues ont été amenées à évoluer au fil des développements effectués par l'équipe de développeur.ses du projet. En effet, le principe de la 'vue' est d'interfacer le rendu graphique et le modèle de données, et ce dernier a été amené à évoluer au cours de mon stage. Cette annexe présente les développements effectués et certaines de leurs évolutions. 

\section{\texttt{views.py} pour traiter la logique des requêtes}

Ce fichier contient les fonctions ou classes appelées 'vues' qui traitent les requêtes \http envoyées par l'utilisateur.rice. Chaque vue correspond généralement à une fonctionnalité spécifique de l'application. Elle récupère des données, souvent formatées grâce à des méthodes de classes, elle peut les traiter, et retourne une réponse \http (comme une page \html, un fichier \json, etc.).

Lorsque l'utilisateur.rice visite une \URL spécifique, Django appelle la vue correspondante pour générer la réponse.

Par exemple, l'\URL qui affiche les résultats de l'extraction peut appeler la fonction suivante~:

\begin{lstlisting}[language=python, frame=single, breaklines=true, caption={Vue pour l'affichage des extraction en \enquote{dump}.}]
# /app/webapp/views.py

@login_required(login_url=f"/{APP_NAME}-admin/login/")
def show_all_annotations(request, anno_ref):
    passed, anno = check_ref(anno_ref, "Annotation")
    if not passed:
        return JsonResponse(anno)

    if not ENV("DEBUG"):
        credentials(f"{SAS_APP_URL}/", ENV("SAS_USERNAME"), ENV("SAS_PASSWORD"))

    _, all_annos = formatted_annotations(anno)
    all_crops = [
        (canvas_nb, coord, img_file)
        for canvas_nb, coord, img_file in all_annos
        if coord
    ]

    paginator = Paginator(all_crops, 50)
    try:
        page_annos = paginator.page(request.GET.get("page"))
    except PageNotAnInteger:
        page_annos = paginator.page(1)
    except EmptyPage:
        page_annos = paginator.page(paginator.num_pages)

    return render(
        request,
        "show_crops.html",
        context={
            "anno": anno,
            "page_annos": page_annos,
            "all_crops": all_crops,
            "url_manifest": anno.gen_manifest_url(version=MANIFEST_V2),
            "anno_ref": anno_ref,
        },
    )
\end{lstlisting}

La fonction envoie au \textit{template} des métadonnées, ainsi que les coordonnées des annotations, pour chaque page annotée. Les régions d'images seront appelées dynamiquement dans le \textit{template} grâce à la reconstruction à la volée des \URLs \iiif. 

L'action utilisateur.rice pour exporter l'ensemble des \textit{crops} de diagramme d'un témoin en \jpeg et sous forme de \textsc{zip} appelle la vue suivante~: 

\begin{lstlisting}[language=python, frame=single, breaklines=true, caption={Vue pour l'export de l'ensemble des diagrammes du témoin affiché.}]
@login_required(login_url=f"/{APP_NAME}-admin/login/")
def export_all_crops(request, anno_ref):
    passed, anno = check_ref(anno_ref, "Annotation")
    if not passed:
        return JsonResponse(anno)

    if not ENV("DEBUG"):
        credentials(f"{SAS_APP_URL}/", ENV("SAS_USERNAME"), ENV("SAS_PASSWORD"))

    urls_list = []

    _, all_annos = formatted_annotations(anno)
    all_crops = [
        (canvas_nb, coord, img_file)
        for canvas_nb, coord, img_file in all_annos
        if coord
    ]

    for canvas_nb, coord, img_file in all_crops:
        urls_list.extend(gen_iiif_url(img_file, 2, f"{c[0]}/full/0") for c in coord)

    return zip_img(urls_list)
\end{lstlisting}

Enfin, une vue a été créée pour exporter uniquement les diagrammes sélectionnés. La liste est créée côté navigateur grâce à du JavaScript.  

\begin{lstlisting}[language=python, frame=single, breaklines=true, caption={Vue pour l'export d'une liste de diagrammes.}]
@login_required(login_url=f"/{APP_NAME}-admin/login/")
def export_selected_crops(request):

    urls_list = json.loads(request.POST.get("listeURL"))

    return zip_img(urls_list)
\end{lstlisting}

Les vues font appel à de nombreuses autres fonctions utilitaires ou méthodes de classe, par exemple \texttt{formatted\_annotations}, situé dans un fichier du dossier utilitaire dédié aux annotations \iiif (\texttt{app/webapp/utils/iiif/annotation.py}). Elle génère une liste structurée d'annotations. Pour chaque image, dans les données fournies par l'entité \textit{Regions}, la fonction récupère les annotations associées. Si des annotations existent pour une image donnée, elle extrait pour chacune ses coordonnées et son identifiant. 

\section{\texttt{url.py} pour le routage des \URLs}

Ce fichier définit les correspondances entre les \URLs entrées par l'utilisateur.rice et les vues dans \texttt{views.py}. Il lie les requêtes \URL spécifiques aux fonctions appropriées. Lorsqu'une requête est faite par l'utilisateur.rice, Django utilise \texttt{urls.py} pour déterminer quelle vue doit être exécutée en fonction de l'\URL.

Indexation des vues dans le fichier \texttt{urls.py}~:

\begin{lstlisting}[language=python, frame=single, breaklines=true, caption={Contenu du ficher \texttt{urls.py}~: import des \texttt{views}.}]
#/app/config/urls.py
from app.webapp.views import (
...
show_all_annotations,
export_all_crops,
export_selected_crops,
...
)
\end{lstlisting}

\begin{lstlisting}[language=python, frame=single, breaklines=true, caption={Contenu du ficher \texttt{urls.py}~: routage des fonctions et des \URLs.}]
urlpatterns = [
    ...
    path(
        f"{APP_NAME}/<str:anno_ref>/show-all-annotations",
        show_all_annotations,
        name="show-all-annotations",
    ),

    path(
        f"{APP_NAME}/export-crops/<str:anno_ref>",
        export_all_crops,
        name="export-crops",
    ),

    path(
        f"{APP_NAME}/export-selected-crops",
        export_selected_crops,
        name="export-selected-crops",
    ),

    path(
        f"{APP_NAME}/<str:anno_ref>/show-vectorization",
        show_vectorization,
        name="show-vectorization",
    ),
    ...
]
\end{lstlisting}

\section{Les \emph{templates}}

Les \textit{templates} sont des fichiers \html (avec éventuellement des balises Django) qui définissent la présentation de la réponse envoyée à l'utilisateur.rice. Ils permettent de séparer la logique de l'application (dans les vues) de la présentation (dans les \textit{templates}).

\begin{lstlisting}[language=HTML5, frame=single, breaklines=true, caption={Template \html pour l'affichage des crops de diagrammes.}]
{#/app/webapp/templates/show_crops.html#}


    <div class="toolbar">
        <div class="title">
            <b>{{ anno|capfirst }}</b>
        </div>

        <div style="display: flex; flex-direction: row;">
            <a href="">
                <button class="export-button" type="submit">
                    <i class="fa-regular fa-file-zipper"></i>&nbsp;
                    
                        Download all
                    
                        Télécharger toutes les
                    
                    annotations
                </button>
            </a>

            <form id="export" action="" method="post">
                
                <input type="hidden" id="listeURL" name="listeURL" value="">
                <button class="export-button" type="submit">
                    <i class="fa-regular fa-file-zipper"></i>&nbsp;
                    
                        Download selected annotations
                    
                        Télécharger les annotations sélectionnées
                    
                </button>
            </form>

            
                <a href="" target="_blank">
                    <button class="edit-button" type="submit">
                        <i class="fa fa-pencil"></i>&nbsp;
                        
                            Edit
                        
                            Éditer les
                        
                        annotations
                    </button>
                </a>
            
        </div>
    </div>




<div class="tabs-crops">
    <div class="row">
        <div class="tab-buttons">
            <button class="btn-change active-tab">
                Page view
            
                Vue page
            
            </button>
            <button class="btn-change">
                Dump view
            
                Vue générale
            
            </button>
        </div>
    </div>

    <div class="tab-bodies">
        <div class="row" style="display:block;">
            <div class="column">
                <table class="anno-table" style="margin-top: 0em;">
                    
                        <tr class="anno-row">
                            <td class="anno-td page-col">
                                <a href="{{ img_file|img_to_iiif}}" target="_blank">
                                    <img class="page-preview" src="{{ img_file|img_to_iiif:"full/350,/0" }}" alt="Click to see real size image">
                                </a>
                                <h3><a href="{{ img_file|img_to_iiif }}" target="_blank">Page {{ canvas_nb }}</a></h3>
                            </td>
                            <td class="anno-td anno-col">
                                
                                    <div id="ill_{{ id }}" class="anno-div">
                                        
                                            <a href="{{ img_file|img_to_iiif:region_full }}" target="_blank">
                                                <img src="{{ img_file|img_to_iiif:region }}" alt="scan preview">
                                            </a>
                                            <br>
                                            <input id="bbox_{{ id }}" type="checkbox" name="crop_checkbox" value="{{ img_file|img_to_iiif:region_full }}" onchange="updateSelectedImageURLs()">
                                            <label for="bbox_{{ id }}">
                                                SelectSélectionner
                                            </label>
                                        
                                    </div>
                                
                            </td>
                        </tr>
                    
                </table>
                
            </div>
        </div>

        <div style="display:none;">
            <div class="grid-container" style="margin-top: .5em;">
                
                    
                        
                            <div class="grid-item">
                                <a href="{{ img_file|img_to_iiif:region_full }}" target="_blank">
                                    <img src="{{ img_file|img_to_iiif:region }}" alt="scan preview">
                                </a>
                                <h3><a href="{{ img_file|img_to_iiif }}" target="_blank">Page {{ canvas_nb }}</a></h3>
                                <input id="bbox_{{ id }}" type="checkbox" name="crop_checkbox" value="{{ img_file|img_to_iiif:region_full }}" onchange="updateSelectedImageURLs()">
                                <label for="bbox_{{ id }}">
                                    SelectSélectionner
                                </label>
                            </div>
                        
                    
                
            </div>
        </div>
    </div>
</div>

<script>
    var checkboxes = document.querySelectorAll('input[name="crop_checkbox"]');
    var selectedImageURLs = [];
    // Fonction pour mettre à jour la liste des URL sélectionnées
    function updateSelectedImageURLs() {
        // Réinitialiser la liste des URL sélectionnées
        selectedImageURLs = [];

        // Parcourir toutes les cases à cocher
        checkboxes.forEach(function(checkbox) {
            // Vérifier si la case est cochée
            if (checkbox.checked) {
                // Récupérez l'URL de l'image à partir de la value de la case cochée
                var imageURL = checkbox.value;
                // Ajouter l'URL de l'image à la liste des URL sélectionnées
                selectedImageURLs.push(imageURL);
                }
            }
        );
        // Afficher la liste des URLs sélectionnées
        console.log(selectedImageURLs);
        var jsonString = JSON.stringify(selectedImageURLs);

        // liste => hidden input
        document.getElementById("listeURL").value = jsonString;
        // Vérifier si la valeur du champ est correcte
        console.log(document.getElementById("listeURL").value);
    };

    Array.from(document.querySelectorAll('.tabs-crops')).forEach((tab_container, TabID) => {
        const registers = tab_container.querySelector('.tab-buttons');
        const bodies = tab_container.querySelector('.tab-bodies');

        Array.from(registers.children).forEach((el, i) => {
          el.setAttribute('aria-controls', `${TabID}_${i}`)
          bodies.children[i]?.setAttribute('id', `${TabID}_${i}`)

          el.addEventListener('click', (ev) => {
            let activeRegister = registers.querySelector('.active-tab');
            activeRegister.classList.remove('active-tab')
            activeRegister = el;
            activeRegister.classList.add('active-tab')
            changeBody(registers, bodies, activeRegister)
          })
      })
    })

    function changeBody(registers, bodies, activeRegister) {
        selectedImageURLs = [];

        checkboxes.forEach(function(checkbox) {
        checkbox.checked = false;
        });

        Array.from(registers.children).forEach((el, i) => {

            if (bodies.children[i]) {
                bodies.children[i].style.display = el == activeRegister~? 'block'~: 'none'
            }

            el.setAttribute('aria-expanded', el == activeRegister~? 'true'~: 'false')
        })
    }
</script>


\end{lstlisting}

Ce qui donne deux affichages possibles pour les extractions~: 

          \begin{figure}[H]
          \begin{center}
          \includegraphics[height=7cm]{figues/vue_dump.png}
          \includegraphics[height=7cm]{figues/vue_contexte.png}
          \end{center}
          \caption{Affichage des extractions, vue \emph{dump} et avec la page de manuscrit en regard.}
          \label{fig:old_interface_1} \end{figure}

Des vues similaires ont été créées pour afficher les résultats de la vectorisation. 

\section{Autres développements }

Vues pour l'export des \svgs et des images \jpeg~: 

\begin{lstlisting}[language=python, frame=single, breaklines=true, caption={Vue pour l'export de tous les \svgs et des images associées.}]

@login_required(login_url=f"/{APP_NAME}-admin/login/")
def export_all_images_and_svgs(request, anno_ref):
    passed, anno = check_ref(anno_ref, "Annotation")
    if not passed:
        return JsonResponse(anno)

    if not ENV("DEBUG"):
        credentials(f"{SAS_APP_URL}/", ENV("SAS_USERNAME"), ENV("SAS_PASSWORD"))

    urls_list = []
    path_list = []

    _, all_annos = formatted_annotations(anno)
    all_crops = [
        (canvas_nb, coord, img_file)
        for canvas_nb, coord, img_file in all_annos
        if coord
    ]

    for canvas_nb, coord, img_file in all_crops:
        urls_list.extend(gen_iiif_url(img_file, 2, f"{c[0]}/full/0") for c in coord)
        vecto_path = f"{img_file[:-4]}_{''.join(c[0] for c in coord)}.svg"
        # Vérifie si le chemin existe
        if os.path.exists(os.path.join(SVG_PATH, vecto_path)):
            path_list.append(vecto_path)

    return zip_images_and_files(urls_list, path_list)

\end{lstlisting}

\begin{lstlisting}[language=python, frame=single, breaklines=true, caption={Vue pour l'export des \svgs et des images \jpeg sélectionnés.}]
@login_required(login_url=f"/{APP_NAME}-admin/login/")
def export_selected_imgs_and_svgs(request):
    images_list = json.loads(request.POST.get("liste_images"))
    urls_list = []
    paths_list = []
    for element in images_list:
        if is_url(element):
            urls_list.append(element)
        else:
            paths_list.append(element)
    return zip_images_and_files(urls_list, paths_list)
\end{lstlisting}

Des fonctions utilitaires sont utiles pour assurer l'export~: 

\begin{lstlisting}[language=python, frame=single, breaklines=true, caption={Fonctions utilitaires pour compresser et télécharger les images.}]

def zip_img(img_list, zip_name=f"{APP_NAME}_export"):
    buffer = io.BytesIO()
    with zipfile.ZipFile(buffer, "w") as z:
        for img_path in img_list:
            img_name = f"{url_to_name(img_path)}.jpg"
            if urlparse(img_path).scheme == "":
                z.write(f"{IMG_PATH}/{img_name}", img_name)
            else:
                response = requests.get(img_path)
                if response.status_code == 200:
                    z.writestr(img_name, response.content)
                else:
                    log(f"[zip_img] Fail to download img: {img_path}")
                    pass

    response = HttpResponse(
        buffer.getvalue(), content_type="application/x-zip-compressed"
    )
    response["Content-Disposition"] = f"attachment; filename={zip_name}.zip"
    return response


def zip_files(filenames_contents, zip_name=f"{APP_NAME}_export"):
    # filenames_contents = [(filename1, content1), (filename2, content2), ...]
    buffer = io.BytesIO()
    with zipfile.ZipFile(buffer, "w") as z:
        for filename_content in filenames_contents:
            filename, content = filename_content
            z.writestr(filename, content)

    response = HttpResponse(
        buffer.getvalue(), content_type="application/x-zip-compressed"
    )
    response["Content-Disposition"] = f"attachment; filename={zip_name}.zip"
    return response


def zip_images_and_files(img_list, file_list, zip_name=f"{APP_NAME}_export"):
    buffer = io.BytesIO()
    with zipfile.ZipFile(buffer, "w") as z:
        # Ajouter des images à partir des URLs ou du répertoire local
        for img_path in img_list:
            img_name = f"{url_to_name(img_path)}.jpg"
            if urlparse(img_path).scheme == "":
                try:
                    z.write(f"{SVG_PATH}/{img_name}", img_name)
                except FileNotFoundError:
                    log(f"[zip_images_and_files] Local image not found: {img_path}")
            else:
                response = requests.get(img_path)
                if response.status_code == 200:
                    z.writestr(img_name, response.content)
                else:
                    log(f"[zip_images_and_files] Fail to download image: {img_path}")

        # Ajouter des fichiers à partir du répertoire mediafiles
        for file_path in file_list:
            try:
                with open(f"{SVG_PATH}/{file_path}", "rb") as f:
                    z.writestr(file_path, f.read())
            except FileNotFoundError:
                log(f"[zip_images_and_files] Local file not found: {file_path}")

    response = HttpResponse(
        buffer.getvalue(), content_type="application/x-zip-compressed"
    )
    response["Content-Disposition"] = f"attachment; filename={zip_name}.zip"
    return response

\end{lstlisting}

\textit{Template} pour afficher les \textit{crops} de diagrammes avec les vectorisations associées. Le \textit{template} et le rendu sont très similaire à ceux concernant l'affichage des diagrammes uniquement~: 

\begin{lstlisting}[language=HTML5, frame=single, breaklines=true, caption={Template \html pour afficher les extractions et leurs vectorisations.}]





{{ anno }}


    <div class="toolbar">
        <div class="title">
            <b>{{ anno|capfirst }}</b>
        </div>
        <div style="display: flex; flex-direction: row;">
            <a href="" target="_blank">
                <button class="export-button" type="submit">
                    <i class="fa-solid fa-eye"></i>
                    
                        Visualize all
                    
                        Voir toutes les
                    
                    annotations
                </button>
            </a>
            <a href="">
                <button class="export-button" type="submit">
                    <i class="fa-regular fa-file-zipper"></i>&nbsp;
                    
                        Download all
                    
                        Télécharger toutes les
                    
                    vectorisations
                </button>
            </a>

            <form id="export" action="" method="post">
                
                <input type="hidden" id="liste_images" name="liste_images" value="">
                <button class="export-button" type="submit">
                    <i class="fa-regular fa-file-zipper"></i>&nbsp;
                    
                        Download selected annotations
                    
                        Télécharger les vectorisations sélectionnées
                    
                </button>
            </form>
        </div>
    </div>




<div class="tabs-crops">
    <div class="row">
        <div class="tab-buttons">
            <button class="btn-change active-tab">
                Page view
            
                Vue page
            
            </button>
            <button class="btn-change">
                Dump view
            
                Vue générale
            
            </button>
        </div>
    </div>


<div class="tab-bodies">

    <div class="row" style="display:block;">
        <div class="column">
            <table class="anno-table" style="margin-top: 0em;">
                
                    <tr class="anno-row">
                        <td class="anno-td page-col">
                            <a href="{{ img_file|img_to_iiif}}" target="_blank">
                                <img class="page-preview" src="{{ img_file|img_to_iiif:"full/350,/0" }}" alt="Click to see real size image">
                            </a>
                            <h3><a href="{{ img_file|img_to_iiif }}" target="_blank">Page {{ canvas_nb }}</a></h3>
                        </td>
                        <td class="anno-td anno-col">
                            
                                <div id="ill_{{ id }}" class="anno-div grid-item" style=" display: flex; flex-wrap: wrap; justify-content: center;">
                                    
                                        <a href="{{ img_file|img_to_iiif:region_full }}" target="_blank">
                                            <img src="{{ img_file|img_to_iiif:region }}" alt="scan preview">
                                        </a>
                                        <img src="svg/{{ img_file|jpg_to_none }}_{{ coords }}.svg" class="img-fluid" alt="{{ img_file|jpg_to_none }}_{{ coords }}.svg">
                                        <a href="" target="_blank">VisualizeVisualiser</a>
                                        <br>
                                        <input type="checkbox" name="vecto_checkbox" value='["{{ img_file|jpg_to_none }}_{{ coords }}.svg", "{{ img_file|img_to_iiif:region }}"]' onchange="updateSelectedImageURLs()">
                                        <label for="checkbox">
                                            SelectSélectionner
                                        </label>
                                    
                                </div>
                            
                        </td>
                    </tr>
                
            </table>
            
        </div>
    </div>

    <div style="display: none;">
        <div style="margin-top: 5%;">
            <div class="grid-container">
                
                    
                        <div class="grid-item">
                            <div style="display: flex; flex-direction: row;">
                                
                                    <a href="{{ img_file|img_to_iiif:small }}" target="_blank">
                                        <img src="{{ img_file|img_to_iiif:small }}" class="img-fluid" alt="{{ img_file }}" style="margin-right: 10px;">
                                    </a>
                                            <img src="svg/{{ img_file|jpg_to_none }}_{{ coords }}.svg" class="img-fluid" alt="{{ img_file|jpg_to_none }}_{{ coords }}.svg">
                                            <a href="{ % url 'img-and-svg' img_file=img_file coords=coords %}" target="_blank">VisualizeVisualiser</a>
                                </a>
                                <input type="checkbox" name="vecto_checkbox" value='["{{ img_file|jpg_to_none }}_{{ coords }}.svg", "{{ img_file|img_to_iiif:small }}"]' onchange="updateSelectedImageURLs()">
                                
                            </div>
                            <h3><a href="{{ img_file|img_to_iiif }}" target="_blank">Page {{ canvas_nb }}</a></h3>
                        </div>
                    
                
            </div>
        </div>
    =</div>

</div>
</div>


<script>
    document.addEventListener('DOMContentLoaded', () => {
        const tabContainers = document.querySelectorAll('.tabs-crops');

        tabContainers.forEach((tabContainer, tabID) => {
            const registerButtons = tabContainer.querySelectorAll('.tab-buttons .btn-change');
            const tabBodies = tabContainer.querySelectorAll('.tab-bodies > div');

            registerButtons.forEach((button, index) => {
                button.setAttribute('aria-controls', `${tabID}_${index}`);
                if (tabBodies[index]) {
                    tabBodies[index].setAttribute('id', `${tabID}_${index}`);
                }

                button.addEventListener('click', () => {
                    setActiveTab(registerButtons, tabBodies, button);
                });
            });
        });

        function setActiveTab(registerButtons, tabBodies, activeButton) {
            registerButtons.forEach((button, index) => {
                if (button === activeButton) {
                    button.classList.add('active-tab');
                    if (tabBodies[index]) {
                        tabBodies[index].style.display = 'block';
                    }
                } else {
                    button.classList.remove('active-tab');
                    if (tabBodies[index]) {
                        tabBodies[index].style.display = 'none';
                    }
                }
                button.setAttribute('aria-expanded', button === activeButton~? 'true'~: 'false');
            });
        }
    });

    var checkboxes = document.querySelectorAll('input[name="vecto_checkbox"]');
    var selectedImages = [];

    function updateSelectedImageURLs() {
    // Clear the selectedImages array before processing checkboxes
    selectedImages.length = 0; // More efficient way to clear the array

    // Loop through all checkboxes
    checkboxes.forEach(function(checkbox) {
        if (checkbox.checked) {
        try {
            // Parse the JSON string from checkbox value (handle potential errors)
            var parsedImages = JSON.parse(checkbox.value);
            // Add parsed images to selectedImages array (assuming correct format)
            selectedImages.push(...parsedImages);
        } catch (error) {
            console.error("Error parsing JSON from checkbox:", checkbox.value, error);
        }
        }
    });

    // Convert the selectedImages array to JSON string (handle empty array)
    var jsonString = selectedImages.length > 0~? JSON.stringify(selectedImages)~: "";

    // Update the hidden input "liste_images" with the JSON string
    document.getElementById("liste_images").value = jsonString;

    // Log the JSON string for verification (optional)
    console.log(selectedImages);
    console.log(jsonString);
    }

    </script>


\end{lstlisting}

\section{Manipuler le format SVG}

Le stage m'a également permis de me familiariser avec le format \svg. J'ai développé une interface basique permettant de manipuler des fichiers \svgs, en implémentant des fonctionnalités de base (Fig. \ref{fig:manip_vecto})~: 
\begin{itemize}
    \item supprimer une primitive (en cliquant dessus)~; 
    \item passer les primitives en noir et blanc~;
    \item \textit{fade} l'image en arrière plan~;
    \item télécharger au format \jpeg l'image et le format vectoriel superposés, et en gardant les modifications effectuées.
\end{itemize}

  \begin{figure}[H]
	\begin{center}
		\includegraphics[height=7cm]{figues/manip_vecto.png}
	\end{center}
	\caption{Interface de manipulation des fichiers \svgs, avec exemple d'image modifiée téléchargée.}
	\label{fig:manip_vecto} \end{figure}

J'ai pu ainsi approfondir mes connaissances en JavaScript et découvrir les spécificités du format \svg. La stratégie a consisté à écrire le contenu des fichiers dans le \dom et ajouter de l'interaction grâce à du JavaScript. 

\begin{lstlisting}[language=HTML5, frame=single, breaklines=true, caption={\textit{Template} \html pour la manipulation des fichiers \svgs.}]





{{ regions }}


    {{ block.super }}
    <link rel="stylesheet" href="">
    <link rel="stylesheet" href="">
    <link href="https://cdn.jsdelivr.net/npm/bootstrap-icons@1.8.0/font/bootstrap-icons.css" rel="stylesheet">




    <div class="toolbar">
        <div class="title">
            <b>{{ regions|capfirst }} | page {{ canvas_nb }}</b>
        </div>
    </div>



<div class="container">
    <div class="sidebar">
        <div class="switch-container">
            <label class="switch">
                <input type="checkbox" id="change-stroke-color">
                <span class="slider round"></span>
            </label>
            <label for="change-stroke-color">All lines in blackNoir et blanc</label>
        </div>

        <div>
            <input type="range" min="0" max="1" step="0.01" value="1" id="opacityRange">
            <label for="opacityRange">Toggle Image OpacityModuler l'opacité de l'image</label>
        </div>

        <div>
            <a id="downloadLink" href="#" download="image.jpg">Télécharger l'image</a>
        </div>
    </div>

    <div class="image-container">
        {{ svg_content|safe }}
    </div>
</div>


<script>
    document.addEventListener('DOMContentLoaded', function() {

        const svgElement = document.querySelector('svg'); // Sélectionner l'élément SVG
        const imageElement = document.querySelector('svg image');

        // Charger l'image de fond
        function fond() {
            
            const backgroundImage = "{{ img_file|img_to_iiif:small }}";
            
            imageElement.setAttribute('xlink:href', backgroundImage);
        }
        fond();

        // Changer la couleur des traits
        var toggleSwitch = document.getElementById('change-stroke-color');
        var elementsWithStroke = document.querySelectorAll('[stroke]');
        var originalStrokeColors = [];

        elementsWithStroke.forEach(function(element) {
            originalStrokeColors.push(element.getAttribute('stroke'));
        });

        function changeStrokeColorToBlack() {
            elementsWithStroke.forEach(function(element) {
                element.setAttribute('stroke', 'black');
            });
        }

        function restoreOriginalStrokeColors() {
            elementsWithStroke.forEach(function(element, index) {
                element.setAttribute('stroke', originalStrokeColors[index]);
            });
        }

        toggleSwitch.addEventListener('change', function() {
            if (this.checked) {
                changeStrokeColorToBlack();
            } else {
                restoreOriginalStrokeColors();
            }
        });

        // Modifier l'opacité de l'image
        const opacityRange = document.getElementById("opacityRange");
        opacityRange.addEventListener("input", function() {
            const opacityValue = this.value;
            imageElement.style.opacity = opacityValue;
        });

        // Faire disparaître les éléments onclick
        function removeElementOnClick(event) {
            event.target.remove();
        }

        const circles = document.querySelectorAll('svg circle');
        const paths = document.querySelectorAll('svg path');

        circles.forEach(circle => {
            circle.addEventListener('click', removeElementOnClick);
        });

        paths.forEach(path => {
            path.addEventListener('click', removeElementOnClick);
        });

    // Convertir le SVG modifié en image (PNG ou JPEG)
    function convertModifiedSVGToImage(format, quality, callback) {
        // Sérialiser le SVG modifié
        const svgString = new XMLSerializer().serializeToString(svgElement);

        // Créer un canvas pour dessiner l'image
        const canvas = document.createElement('canvas');
        const context = canvas.getContext('2d');

        // Charger l'image de fond dans le canvas
        const image = new Image();
        image.setAttribute('crossOrigin', 'anonymous');
        image.onload = function() {
            // Le canvas a la même taille que l'image
            canvas.width = image.width;
            canvas.height = image.height;

            // Remplir le canvas avec un fond blanc
            context.fillStyle = 'white';
            context.fillRect(0, 0, canvas.width, canvas.height);

            // Appliquer l'opacité de l'image de fond
            const imageOpacity = parseFloat(window.getComputedStyle(imageElement).opacity);
            context.globalAlpha = isNaN(imageOpacity)~? 1.0~: imageOpacity;
            context.drawImage(image, 0, 0, canvas.width, canvas.height);

            // Réinitialiser l'opacité pour tracer le SVG
            context.globalAlpha = 1.0;

            // Dessiner le SVG modifié sur le canvas
            const svgBlob = new Blob([svgString], { type: 'image/svg+xml;charset=utf-8' });
            const DOMURL = window.URL || window.webkitURL || window;
            const svgUrl = DOMURL.createObjectURL(svgBlob);

            const svgImage = new Image();
            svgImage.onload = function() {
                // Dessiner le SVG à la bonne taille
                context.drawImage(svgImage, 0, 0, canvas.width, canvas.height);

                // Télécharger l'image convertie au format spécifié
                const dataUrl = canvas.toDataURL(`image/${format}`, quality);
                callback(dataUrl); // Utiliser le callback pour manipuler le dataUrl
            };
            svgImage.src = svgUrl;
        };
        image.src = imageElement.getAttribute('xlink:href');
    }

    // Fonction de téléchargement
    function downloadImage(format, quality) {
        convertModifiedSVGToImage(format, quality, function(dataUrl) {
            const downloadLink = document.createElement('a');
            downloadLink.href = dataUrl;
            downloadLink.setAttribute('download', `image.${format}`);
            document.body.appendChild(downloadLink);
            downloadLink.click();
            document.body.removeChild(downloadLink);
        });
    }

    // Lien pour télécharger en JPEG
    const downloadLink = document.getElementById('downloadLink');
    downloadLink.addEventListener('click', function(event) {
        event.preventDefault();
        downloadImage('jpeg', 0.8);
    });

    });
    </script>




\end{lstlisting}

Cette interface est associée à une vue qui renvoie au \textit{template} les informations nécessaires, notamment le contenu du fichier \svg. 

\begin{lstlisting}[language=python, frame=single, breaklines=true, caption={Vue pour relier les données à leur affichage dans l'interface de manipulation.}]
@login_required
def show_crop_vectorization(request, img_file, coords, regions, canvas_nb):
    svg_filename = f"{jpg_to_none(img_file)}_{coords}.svg"
    svg_path = os.path.join(SVG_PATH, svg_filename)

    if not os.path.exists(svg_path):
        print(f"File {svg_path} not found")

    with open(svg_path, "r", encoding="utf-8") as file:
        svg_content = file.read()

    return render(
        request,
        "crop_vecto.html",
        context={
            "img_file": img_file,
            "coords": coords,
            "svg_content": svg_content,
            "regions": regions,
            "canvas_nb": canvas_nb,
        },
    )
\end{lstlisting}

\section{Nouvelles interfaces}

Avec la refonte des interfaces effectuée par les développeuses du projet, ces codes ne sont actuellement plus utilisés dans l'application \aikon, l'utilisation d'un \textit{framework front-end} ayant totalement changé les logiques d'affichage. J'ai cependant eu l'opportunité de poser fondations de la page d'accueil de la plateforme (Fig. \ref{fig:acceuil})~: 

	\begin{figure}[H]
	\begin{center}
		\includegraphics[height=8cm]{figues/accueil.png}
	\end{center}
	\caption{Page d'accueil de la plateforme \aikon}
	\label{fig:accueil} \end{figure}
	\clearemptydoublepage
	
	\chapter[Interfaces~: documentation]{\label{doc_interfaces}Nouvelles interfaces~: documentation}
	\includepdf[scale=1,pages=1-]{templates/annexes/doc_interfaces.pdf}
	\clearemptydoublepage

\clearemptydoublepage

\backmatter
    \printacronyms[title=Liste des acronymes,toctitle=Acronymes, type=\acronymtype]
    \printglossary 
  \listoffigures
  \addtocontents{lof}{\protect\footnotetext{Les schémas suivis d'un astérisque sont prélevés ou inspirés de la documentation technique interne créée par les développeur.ses du projet \eida et de l'\iscd.}}
    \tableofcontents
	
\end{document}