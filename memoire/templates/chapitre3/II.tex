\gaga mène une réflexion
méthodologique sur la portabilité des modèles, leur diffusion et le
partage de grands ensembles de données annotées selon des normes
communes. Les modèles d'\ia, notamment ceux dédiés à la
reconnaissance du texte manuscrit (\htr) et au traitement automatique des
langues (\tal), requièrent des données d'entraînement spécifiques. 

Mais si chaque projet de recherche annotait ses corpus selon ses propres
exigences, ils engendreraient fatalement des silos de données non
réutilisables. Pour garantir la réutilisation des données, il est
impératif d'établir des normes et des standards.

Le projet porte alors le développement d'une syntaxe
d'annotation générique pour harmoniser la segmentation des pages des
vérités de terrain, afin de constituer des corpus d'entraînement réutilisables. \gaga propose une
approche très inclusive en identifiant des éléments textuels communs à
une large variété de documents, manuscrits comme imprimés. Cette
démarche donne lieu à la définition d'un vocabulaire contrôlé permettant
ainsi de construire des corpus annotés compatibles avec différents contextes\footcite[``Using a common vocabulary to annotate zones called SegmOnto (that is still evolving), we have developed a generic workflow to analyse the layout, OCRise the text, and convert the ALTO output into minimally encoded TEI files (\dots).''][p.2]{janes_towards_2021}~:
SegmOnto\footnote{https://github.com/SegmOnto}

Les étapes de lemmatisation et d'étiquetage morphologique (POS-tagging) effectuées par les modèles de \tal sur le texte extrait des pages numérisées visent à normaliser le
langage en réduisant les mots à leur forme canonique (lemme) et en
identifiant leur catégorie grammaticale. Cette normalisation est
essentielle pour faciliter des analyses ultérieures telles que la
collation et la stylométrie. Elle permet une analyse comparative des textes
malgré la grande variabilité inhérentes aux
langues historiques, pour lesquelles l'absence de normes orthographiques entraîne une
grande diversité de graphies. La préparation des données, là aussi, a donné lieu à une
réflexion méthodologique sur les défis liés à la standardisation des
annotations linguistiques dans les corpus diachroniques.

Selon \citeauthor{gabay_standardizing_2020}~:
\begin{kwote}                     
	``With the development of big corpora of various periods, it becomes
	crucial to standardise linguistic annotation (e.g.~lemmas, POS tags,
	morphological annotation) to increase the interoperability of the data
	produced, despite diachronic variations.''\footcite[p.2]{gabay_standardizing_2020}
\end{kwote}  

\citeauthor{gabay_standardizing_2020}\footcite{gabay_standardizing_2020} relèvent la
difficulté de mettre en œuvre un cadre technique qui prenne en compte
les pratiques d'annotation déjà établies et propres à des états de la langue. Pourtant garantir une
interopérabilité minimale avec les corpus existants est essentielle pour
maximiser la valeur ajoutée des nouvelles données.

\begin{kwote}                     
	``Such a task cannot be done without taking into account longstanding
	annotation practices, in order to allow (minimal) interoperability with
	already existing datasets. Such a statement is sadly easier said than
	done, because EMF is an intermediary stage between medieval (12th-15th
	c.) and late modern and contemporary (from c.~1750) French, two states
	of language that tend to have different needs regarding annotation: EMF
	is then caught in between two (potentially incompatible) practices, one
	for each extreme of the continuum.''\footcite[p.2]{gabay_standardizing_2020}
\end{kwote}     

Il est particulièrement complexe de trouver un équilibre entre une
description linguistique trop fine, qui pourrait limiter
la réutilisabilité des corpus, et une description trop générale, qui pourrait
manquer de précision. De plus, les besoins en annotation varient considérablement
entre le français médiéval et le français moderne, et les systèmes
d'annotation de ses deux états de la langue sont difficilement
réconciliables~: or une large partie des sources de \gaga se
situent dans l'entre-deux. Trouver un compromis qui satisfait les
exigences spécifiques de chaque période est complexe.

L'harmonisation des annotations vise à favoriser la diffusion des corpus
de données sur HTR-United, une plateforme collaborative dédiée au
catalogage de vérités de terrain pour l'\htr et l'\ocr, principalement en
français\footcite{chague_htr-united_2021}. Cette base de
données, hébergée sur GitHub, centralise des images et leurs
transcriptions produites par différents projets de recherche, offrant
ainsi une diversité de jeux de données diachroniques et géographiques
pour l'entraînement de modèles \htr.

\gaga vise aussi à la diffusion des modèles en eux-même, qui peuvent alors être réutilisés et spécialisés. Par ailleurs, le projet
s'appuie sur des outils existants, par exemple Deucalion, une boîte à
outils de traitement automatique des langues (\tal) conçue pour être
interopérable avec d'autres systèmes\footnote{https://github.com/chartes/deucalion-model-af}.