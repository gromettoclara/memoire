L'industrie éditoriale destinée au milieu académique détermine depuis le
\textsc{xix}\ieme siècle les standards desquelles héritent l'édition
contemporaine\footcite{epron_ledition_2018}. Or, dans les traités historiques
d'astronomie, il n'existe pas de norme pour la reproduction des
diagrammes. Face à ce constat, il est nécessaire de mener une réflexion
sur les enjeux épistémologiques et historiographiques du partage des pratiques d'édition des diagrammes, et sur les besoins de la
mise en œuvre d'une édition numérique des diagrammes.

\hypertarget{sortir-des-pratiques-aleatoires}{%
\subsection{Sortir des pratiques
aléatoires}\label{sortir-des-pratiques-aleatoires}}

Adolphe Rome, dans l'introduction du premier volume de son édition des
commentaires de Pappus et Theon sur l'Almageste de Ptolémée, a remarqué
à propos des figures qu'``à leur manière, elles constituent également un
texte à éditer''\footcite[Commentaires de Pappus et de Théon d'Alexandre sur
l'Almageste, ed.~by A. Rome (Rome, 1931), i, p.~xxiv.][p.394]{jardine_critical_2010}. En
tant qu'objet d'étude à part entière, porteur d'indices important, ils
doivent être considérés au même titre que le texte comme des témoins des
pratiques des sciences astronomiques et de la transmission des
connaissances

Or il n'existe ni conventions, ni normes pour une édition des traités
astronomiques qui retracent efficacement les métamorphoses des images
comme du texte de manière équivalente ; ce qui est surprenant car après
tout, les éditeurs sont confrontés, avec les diagrammes, aux même
problématiques que celles concernant les textes, à savoir des erreurs et
variations. \citeauthor{jardine_critical_2010} parlent même de ``remarkable failure to take seriously
the editing of diagrams''\footcite[p.393]{jardine_critical_2010}, et ce même dans les éditions
numériques.

\begin{kwote}
``Text can be handled with \tei-compliant \xml ; a similar standard does not
yet exist for diagrams.''\footcite[p.77]{roughan_digital_2014}
\end{kwote}

Les éditeurs font alors des choix dans la manière de représenter les
diagrammes. Notamment, dans les manuscrits, il arrive régulièrement que
les figures rentrent en contradiction avec le texte qui les accompagne.
Jardine et Jardine remarquent alors que les éditions canoniques de ces
grands traités d'astronomie corrigent les diagrammes afin de les rendre
conformes au texte. Mais les modifications sont apportées sans
commentaire, une pratique inenvisageable en ce qui concerne le
texte.\footnote{\citeauthor{jardine_critical_2010} donnent
  en exemple des cas où la casse des labels est modifiée pour s'aligner
  avec le texte : \emph{Kepleri opera omnia}, ed.~by Ch. Frisch
  (Frankfurt and Erlangen, 1858; facs. Hildesheim, 1971), i, 282, Fig. 9
  N. Jardine's, in \emph{The birth of history and philosophy of science:
  Kepler's A defense of Tycho against Ursus with essays on its
  provenance and significance} (Cambridge, 1984; rev. edn, 1988). V.
  Bialas's, \emph{Johannes Kepler: Gesammelte Werke}, xx/1 (Munich,
  1988). (\cite[p.411]{jardine_critical_2010})}

En outre, les éditeurs ont tendance à moderniser les conventions
visuelles pour adapter les diagrammes à leurs lecteurs contemporains. En
voici un exemple typique :

\begin{kwote}
``In Heiberg's edition {[}of Archimedis \emph{Archimedis Opera
Omnia}{]}, medieval methods of representing diagrams are quietly
abandoned in favor of more modern methods. Heiberg is not alone in
silently modernizing the diagrams in his editions : many editors of
mathematical texts diverge from the manuscripts in their presentation of
diagrams. In some cases they''correct'' shapes ; in other they add more
generality to diagram figures which in the manuscripts were
overspecified. (For instance a manuscript diagram might have a right
triangle but the text state that any triangle would do -- often modern
editors choose to draw an irregular triangle instead so that their
readers are not misled into thinking that the proof requires a right
triangle.)``\footcite[p.78]{roughan_digital_2014}
\end{kwote}

La pratique de la correction ou adaptation dénote une perception très
contemporaine du diagramme comme simple accompagnement des textes
scientifiques auxquels ils sont soumis. Les illustrations sont
considérées comme outil de compréhension du contenu textuel destiné au
lecteur. Or les diagrammes médiévaux et du début de l'époque moderne ne
se contentent pas d'illustrer le texte. Il est probable (c'est la position des chercheurs du projet \eida) qu'une grande variété de pratiques et de manières d'utiliser les diagrammes aient coexisté dans l'astronomie ancienne, tissant des relations complexes aux textes et aux données tabulaires. Par conséquent, imposer un principe d'édition unique, qui arrête un emploi des diagrammes, revient à contraindre artificiellement les sources, en occultant la diversité des représentations. Il est nécessaire de trouver, grâce au numérique, un environnement éditoriale plus plastique qui permette de représenter différentes modalités d'utilisations des diagrammes et de leur rapport aux autres éléments des sources (textes et tables).

Dans l'édition papier, les tentatives d'édition pour les diagrammes
reposent sur leur mise en texte. Les variations sont codifiées et
expliquées\footcite[Voir les édition de]{de_young_editing_2014}.
Les corrections, reconstructions, avec modernisation des conventions
font également l'objet d'un accompagnement textuel\footcite[p.395]{jardine_critical_2010}.
Selon \citeauthor{jardine_critical_2010}, bien que plus satisfaisantes que la
modification sans signalement, ces pratiques, trop dépendantes des
présupposés de l'éditeur et des objectifs éditoriaux (l'établissement du
\textit{stemma}, l'attribution de la production ou encore une étude de la
réception et la transmission), méritent d'être questionnées.

Une première problématique concerne la mise en page et le placement du
diagramme. L'adoption de la convention consistant à centrer l'image
entre deux blocs de texte masque la manière dont l'emplacement peut
s'avérer signifiant, les diagrammes centrés servant souvent de bases aux
démonstrations textuelles, tandis que les diagrammes marginaux
constituent généralement des gloses ou des commentaires sur le texte. La
question de l'emplacement ne devrait pas être complètement éludée, mais
comment faire cela avec l'imprimé sans sacrifier la lisibilité\footcite[p.400]{jardine_critical_2010} ? 

Un second problème concerne l'observation des conventions graphiques.
Leur compréhension n'est évidente que dans un contexte et une culture
donnée (si on identifie immédiatement un code barre ou un QR Code, un
alien qui arrive sur Terre ou un scribe du \ma pourrait croire à
une simple tâche). Sans une étude des normes utilisées par les acteurs
historiques, le jugement des variations et des erreurs demeure
fondamentalement ambiguë : il est difficile de distinguer ce qui est le
produit d'une convention et donc un standard, et ce qui relève de
l'erreur, ou de l'innovation ; il est impossible de trancher avec
certitude. Or ces variations, selon l'objectif de l'édition, tendent à
être éludées, les éditeurs les considérant comme des erreurs minimes qui
ne nécessitent pas un signalement.\footnote{``The basic task of editing
  diagrams centers around things that go wrong with the reproduction of
  diagram lines and points as diagrams are copied from one manuscript to
  another. Just as when editing a verbal text, the principles of editing
  include the recording as clearly as possible of all significant
  variations within the tradition and indicating transparently any
  editorial interference in the text.''\cite[p.229]{de_young_editing_2014}. La
  notion de ``significant variation'' est problématique, car les
  chercheurs n'accordent pas tous la même importance à tous les types de
  variation pouvant affecter le diagramme.}

Une étude exhaustive de l'ensemble du matériau visuel est indispensable
au risque de rentrer dans un cercle vicieux herméneutique empêchant
toute étude de cette riche grammaire visuelle, mais elle est complexe à
mettre en œuvre, notamment dans un format \textit{print}\footcite[p.398]{jardine_critical_2010}.
C'est pourquoi \citeauthor{jardine_critical_2010} plaident pour une approche qui met en
évidence les conventions historiques (la disposition des diagrammes par
rapport au texte, les utilisations de la perspective, de la couleur
etc.) des diagrammes plutôt que de les uniformiser.

Pour résumer, les méthodes d'édition des diagrammes contenus dans les
traités éludent des informations importantes. Les éditeurs sont
confrontés à un dilemme : concilier la production d'une version unifiée
et lisible tout en préservant la richesse et la diversité des
informations contenues dans les sources originales. Les supports
imprimés présentant des limites face à cet objectif, le recours à
l'édition numérique est une voie à explorer. Les chercheurs d'\eida s'y
penchent pour élaborer une méthode d'édition capable de rendre compte de
la complexité des processus de création, de transmission et de réception
des diagrammes.

\hypertarget{edition-numeriue-solution}{%
\subsection{L'édition numérique : une solution
?}\label{edition-numeriue-solution}}

\begin{kwote}
``We predict that this brave new world of digital editing will
eventually produce a much better understanding and appreciation of the
many lively roles of diagrams in early-modern astronomy.''\footcite[p.411]{jardine_critical_2010}
\end{kwote}

Quelles sont les avantages et les apports d'une édition numérique ?

\hypertarget{un-format-enrichi}{%
\subsubsection{Un format enrichi}\label{un-format-enrichi}}

Les langages de balisage comme le \xml (utilisé dans le format \svg) permettent d'expliciter le sens et la valeur de certaines parties des diagrammes, ce qui s'avère très utile d'un point de vue scientifique, car les données deviennent exploitables à un niveau de détail très fin. 

Le format \svg permet de verrouiller ensemble des complexes géométriques
et/ou des labels, d'effectuer des recherches de similarité dynamique, ou
encore d'insérer des identifiants. Ces derniers rendent alors possible
des formes de citation permettant de documenter et justifier des choix
éditoriaux.

Le projet \emph{Archimedes} (College of the Holy Cross), par exemple,
présente une méthode d'édition numérique et diplomatique des diagrammes
mathématiques de tradition grecque présents dans les sources médiévales,
en utilisant le \emph{framework} \CITE comme outil de citation. Il y est
utilisé pour lier une édition diplomatique d'un diagramme à la preuve
de sa transcription, à la proposition mathématique associée, et pour
faire tenir ensemble toute l'édition numérique du traité\footcite{roughan_digital_2014}.

L'architecture \CITE\footcite{blackwell_cite_2019} a été
initialement développée pour le projet \textit{Homer Multitext}\footcite{noauthor_homer_nodate}. Elle fournit
une méthode pour l'identification, la récupération, la manipulation et
l'intégration de données d'une bibliothèque numérique via des \urns,
notamment via des mécanismes pour relier les différents objets. \CITE
offre un cadre, une méthode pour la création et la gestion d'éditions
numériques complexes, en particulier pour faire tenir ensemble des
éléments hétérogènes (comme les images et leurs transcriptions),
associant le tout dans un index qui organise la donnée en triplets.

Initialement pensé pour le texte, l'utilisation du standard de citation
\CITE dans le cadre du projet \emph{Archimedes} montre la manière dont il
peut être appliqué à l'édition numérique des diagrammes, combinant texte
et images et permettant l'identification et le lien entre les éléments à
un niveau de granularité très fin.

Le standard de citation \CITE, traditionnellement utilisé dans l'analyse
textuelle, a fait l'objet d'une adaptation pour répondre aux besoins de
l'édition diplomatique numérique des diagrammes dans le cadre du projet
\textit{Archimedes}. Il permet d'atteindre un niveau de granularité très fin dans
l'identification et l'interconnexion des éléments constitutifs des
documents, qu'ils soient textuels ou visuels. À titre d'exemple, chaque
objet géométrique dont il est question dans les propositions
mathématiques du texte sont associés à deux ensembles de labels
distincts : ceux utilisés dans le texte pour le désigner et ceux qui le
marquent directement sur la figure, les deux pouvant différer. L'objet
est au premier ensemble via son identifiant (attribut \texttt{id}) assigné à
partir des labels qui lui réfèrent dans le texte (le cercle ABC du
texte a un identifiant \texttt{circleABC}). Ils sont reliés au second
ensemble de labels via des triplets rassemblés dans l'index et selon la
syntaxe : \texttt{circleABC} \texttt{hasLabel} A, par exemple. Cependant, si
le projet \emph{Archimedes} a besoin de mettre du liant entre les
entités numériques, c'est au titre d'une édition complète des traités,
comprenant textes et illustrations, ce qui n'est pas le projet de la
plateforme d'édition d'\eida, relativisant la pertinence de l'utilisation du cadre \CITE dans le cadre
d'une édition uniquement des illustrations\footcite[p.81]{roughan_digital_2014}.

\hypertarget{interactivite}{%
\subsubsection{Interactivité}\label{interactivite}}

Une édition numérique autoriserait des formes d'interactivité
impossibles à obtenir dans le \textit{print} :

Déjà, des scripts de reconnaissance des formes et des
labels, basés sur la lecture du format \svg, permettraient d'identifier
automatiquement la présence ou absence de certains éléments graphiques
dans les sources. La modulation des critères de recherche, en incluant
ou excluant certains calques (par exemple, en considérant ou non la
couleur), permet d'affiner les résultats, prenant en compte toute la
complexité des figures. De telles recherches effectuées dans le corpus de
sources pour retrouver des configurations similaires offriraient la
possibilité d'opérer un tri dynamique dans le corpus de sources, avec
pour ambition d'automatiser la génération d'un apparat critique.

La possibilité d'avoir un accès direct à l'image source -- sans
altération de la lisibilité de l'image éditée -- renforce la rigueur
scientifique de l'édition. Cette approche permet d'éviter les biais
interprétatifs liés à une description textuelle et offre aux chercheur.ses
une vision synoptique des variantes au sein d'un corpus. De plus, la
capacité à contextualiser chaque diagramme en le reliant à sa page de
manuscrit d'origine facilite l'analyse des facteurs matériels liés à sa
réception et sa transmission.

Enfin, les visualisations d'une édition pourraient inclure un
environnement de type \emph{sandbox} pour des comparaisons et superpositions
interactives. Les utilisateur.rices pourraient comparer les images originales
(les témoins) et éditées, et accéder à la page source pour une
vérification détaillée des décisions éditoriales et corrections. Cette
multiplicité de perspectives faciliterait l'analyse approfondie des
diagrammes, l'édition numérique pourrait ainsi résolument appuyer les
analyses historiographiques et les processus d'interprétation des
chercheur.ses, ce que souligne \citeauthor{roughan_digital_2014} :

\begin{kwote}
``When mathematical documents are digitized, diagrams should receive the
same attention as text does : both are vehicles of the mathematical
argument.''\footcite[p.78]{roughan_digital_2014}
\end{kwote}

\hypertarget{exigences-et-dissensus}{%
\subsection{Exigences et dissensus}\label{exigences-et-dissensus}}

À l'heure de la rédaction de ce mémoire, les modalités des divers types
d'éditions suscitent encore de nombreuses interrogations parmi les
chercheur.ses d'\eida. En effet, si le texte hérite d'une tradition
éditoriale de près de deux siècles et des apports de l'ecdotique et de
la philologie, les normes restent à inventer pour le matériau graphique,
ce qui soulève inévitablement des questions épistémologiques et
heuristiques. Le système inventé pour le texte est-il transférable ?
C'est toute la question : certains chercheurs veulent s'inscrire dans
les traditions philologiques textuelles, d'autres entendent s'en écarter
pour mieux s'adapter à l'objet image.

\dishas, dont les objets d'étude sont les tables astronomiques, arrête
trois types d'édition~: l'édition ``diplomatique'' d'une table originale
(type A), l'édition ``critique'' basée sur plusieurs sources (type B),
et l'édition ``recalculée'' qui s'appuie sur un modèle mathématique
reconstruit par l'éditeur et pas nécessairement lié à une quelconque
source. Déjà, \dishas optait pour des dénominations délibérément neutres
(types A, B, C) pour les types d'édition des tables, afin de pouvoir
s'éloigner des catégories canoniques et hérités de la tradition
textuelle. Un alignement avec les trois types d'édition présents dans
\dishas constitue un angle d'approche mais ce système n'est pas
complètement transférable non plus.

\hypertarget{type-a-ou-edition-diplomatique}{%
\subsubsection{Type A ou édition diplomatique
?}\label{type-a-ou-edition-diplomatique}}

Dans \dishas, les éditions de type A s'appuient sur un seul témoin et se
veulent au plus proche du texte. Il n'est cependant pas possible de
parler d'édition diplomatique dans la mesure où aucune donnée de mise en
page n'est reproduite. Elles constituent ensuite la base des éditions
``critiques'' qui s'appuient sur plusieurs témoins.

\begin{kwote}
``Plusieurs éditions peuvent être réalisées d'un même original item, et
une édition peut elle-même s'appuyer sur plusieurs éditions préalables.
Le travail de transcription d'une table étant sujet à interprétation, il
est ainsi considéré dans DISHAS qu'une édition critique de plusieurs
témoins constitue en réalité l'édition de plusieurs éditions. Une simple
transcription constitue déjà en ce sens, une édition classique (éditions
de type A {[}\ldots{]}).''\footcite[p.24]{albouy_mediation_2019}
\end{kwote}

Dans le cadre du projet \eida, la catégorie correspondante s'appuierait
sur une vectorisation automatique corrigée. Cela soulève la question de
l'intervention des chercheur.ses : quand passe-t-elle de la correction à
l'édition ? Les chercheur.ses doivent définir ce qui constitue pour eux une
``édition type A'' et ce qu'elle apporte de plus par rapport à la simple
correction d'une transcription par des modèles d'\ia.

Le Projet \textit{Archimedes}, dans sa vision de l'édition diplomatique des
diagrammes, choisit de maintenir un parallèle avec le texte, comme
l'explique \citeauthor{roughan_digital_2014} :

\begin{kwote}
``For a comparison : digitally, diplomatically transcribing the text of
a manuscript accomplishes certain goals. Such a transcription makes the
text in the manuscript more accessible to those unfamiliar with the
manuscript's paleography and more accessible when it is faded or
obscured. It also makes the text in the manuscipt machine-actionnable
and allows for annotations to mark certain features (personal names,
unclear text, expanded abbreviations, etc.).''\footcite[p.78]{roughan_digital_2014}
\end{kwote}

\begin{kwote}
``A digital, diplomatic transcription therefore should make the diagram
in the manuscript more accessible, make the information contained in the
diagram machine-actionable, and allow for editorial
annotations.''\footcite[p.78]{roughan_digital_2014}
\end{kwote}

Les éditions diplomatiques du projet \textit{Archimedes} apportent une valeur
ajoutée par rapport à une simple image raster, en améliorant la
lisibilité et en réduisant le bruit de la numérisation. Enrichies par
rapport à une simple transcription vectorielle, elles intègrent des
identificateurs uniques normalisés pour chaque élément géométrique. Ces
identifiants combinent le terme contemporain désignant la forme avec les
labels issus de la proposition mathématique associée.

Il y a donc un enjeu, dans le cadre d'\eida, à penser des protocoles
statuant sur des états (au sens presque chimique du terme) d'une
transcription, et à articuler ces états avec les protocoles de
versionning. Les chercheur.ses devront alors décider sur quelle version
baser leurs éditions critiques.

\hypertarget{sur-quels-objets-baser-une-edition-corrigee}{%
\subsubsection{Sur quels objets baser une édition corrigée
?}\label{sur-quels-objets-baser-une-edition-corrigee}}

Dans \dishas, pour réaliser une édition de type C, le.la chercheur.se récupère
les paramètres d'une table témoin et recalcule l'ensemble de ses valeurs
à l'aide de modèles mathématiques contemporains. Ce type d'édition
permet, entre autres, de comparer les techniques arithmétiques anciennes
et modernes, en observant par exemple les variations de certaines
constantes mathématiques. Dans le modèle de données, les tables
corrigées renvoient aux scénarios de calcul utilisés.

Une édition corrigée d'un diagramme repose sur la proposition
mathématique qu'il illustre et correspond à ce qui est textuellement
décrit. Une telle édition permet d'observer les déviations par rapport à
la démonstration (témoignant des modalités de réception) ou bien les
phénomènes d'\emph{overspecification}, c'est-à-dire la tendance à
représenter des formes très spécifiques comme des triangles rectangles
ou équilatéraux, alors que la preuve textuelle ne les mentionne pas
(suggérant plutôt des représentations quelconques). Cependant, dans
\eida, ces éditions ne s'appuient sur aucun objet informatique, car les
textes des manuscrits ne sont pas intégrés à la plateforme, ce qui est
scientifiquement insatisfaisant et fait douter de la possibilité
d'intégrer ce type d'édition dans la plateforme.

\hypertarget{peut-on-vraiment-parler-dedition-critique}{%
\subsubsection{Peut-on vraiment parler d'édition critique
?}\label{peut-on-vraiment-parler-dedition-critique}}

Une édition critique s'appuie sur un ensemble de témoins dont les
vectorisations ont été corrigées et figure donc un apparat critique. Le
défi consiste à imaginer un outil d'édition scientifique des diagrammes
pensé pour donner un accès aux variantes pour chaque complexe de
primitives et/ou label verrouillé sous l'œil expert du.de la chercheur.se. Les
débats portent sur les modalités d'accès aux variations.

Certains chercheur.ses du projet défendent une codification des types de
variantes directement sur le diagramme\footcite[``Electronic tools such
  as DRaFT allow us to indicate editorial actions by altering the weight
  of solid lines or by using various forms of dashed lines.''][p.229]{de_young_editing_2014}. Ce
système, conçu initialement pour le texte et l'imprimé, repose sur une
séquentialité et se transfère mal à l'immédiateté des images. En effet,
l'image présente une densité d'information qui se manifeste en une seule
fois, contrairement au texte qui se déploie progressivement. Par
ailleurs, de nombreux symboles graphiques et expressions visuelles sont
déjà utilisées dans les sources par les acteurs historiques. Ces deux
problématiques (la densité d'information et la difficile dissociation des
annotations éditoriales des éléments du diagramme) rendent le résultat
rapidement chargé et illisible. De plus, l'absence d'un langage
symbolique partagé au sein de la communauté scientifique limite les
possibilités de collaboration autour de cette méthode, pouvant susciter
des problèmes d'interprétation\footcite[``Since there is as yet no
  strong consensus on how to indicate various forms of editor
  interference within diagrams, there is the potential for confusing or
  contradictory indicators to develop.''][p.229]{de_young_editing_2014}. Par
ailleurs, est-il même possible de répertorier et symboliser de manière
exhaustive tous les types de variation possibles ? Cette méthode a
néanmoins l'avantage d'être transposable facilement du numérique à
l'imprimé.

La proposition défendue par d'autres chercheur.ses est celle d'une
représentation résolument numérique (quitte à ce qu'elle le soit
uniquement). La visualisation de l'édition devrait alors fournir un
apparat critique interactif, uniquement visuelle, généré automatiquement
sur la base de l'analyse systématique des fichiers \svg. Ainsi l'édition
se détacherait de toute description textuelle comme du langage visuel
des diagrammes. L'idée initiale consisterait à générer un apparat critique
qui ne montrerait pas les variantes mais les invariantes, via une
recherche dynamique de complexes similaires dans le corpus de sources,
avec système de calques (couleur, épaisseur, etc.) à prendre en compte
ou ignorer.

Cependant, dans la tradition philologique textuelle, un certain nombre
de notions ont été définies et fondent la légitimité d'une édition
critique, spécifiquement les concepts d'apparat critique et de
collation.

\begin{kwote}
``The critical apparatus of readings may be positive, that is mention
all the witnesses in each note, or negative, that is mention only the
witnesses disagreeing with the chosen reading.''\footcite[p.348]{moureau_apparatus_2015}
\end{kwote}

Soit la phrase ``Tom sort pour acheter une pizza''. Les manuscrits A, B,
C, D ont ``pizza'', tandis que les manuscrits E, F, G ont ``piazza''.
``Pizza'' est la bonne lecture. Un appareil positif notera : ``pizza A B
C D : piazza E F G'' tandis qu'un appareil négatif notera : ``pizza{]}
piazza E F G''. La différence tient essentiellement dans la manière de
séparer le ``bon mot'' de ``la variante''. Finalement, tant l'appareil
positif que négatif nécessitent de tabler sur ce qui est juste et ce qui
est faux.

La collation désigne l'acte d'enregistrer les différences entre les
textes préservés dans les manuscrits. Ces différences peuvent concerner
: des fautes d'orthographe, des inversions de mots ou de syntagmes, des
suppressions, des mots illisibles, des corrections faites par le scribe
du texte principal, des corrections faites par d'autres, etc. Le mot
collation vient du latin \textit{collatio} qui signifie `rassemblement,
assemblage'. En ce sens, le terme peut donc être employé pour désigner
le rassemblement de tous les diagrammes témoins qui présentent une
configuration similaire, grâce à une recherche de similarité dans le
corpus de fichiers \svgs sur lequel l'édition est basée.

Mais pour qu'une information figure dans un apparat critique au sens
philologique du terme, c'est-à-dire pour qu'elle soit pertinente pour
l'intérêt d'une édition, elle doit présenter une variation par rapport à
un cadre invariant. Dans des termes simples, si tous les manuscrits
avaient ``pizza'', l'apparat critique serait vide car il n'y aurait rien de
remarquable à ce sujet. Si les manuscrits présentaient deux autre
graphies, et une suppression, les trois variantes seraient répertoriées,
distinguées et associées ensemble aux témoins qui les présentent. C'est
pourquoi ``apparat critique'' ne serait pas un bon terme dans le
langage des éditeurs pour désigner les résultats d'une simple recherche
de similarité dans les fichiers \svgs\footcite[PAGE]{trovato_everything_2014}.

La mise en œuvre d'une collation automatique visant à produire un
apparat critique positif d'un diagramme nécessite une double
segmentation : une segmentation en classes de similarité (invariantes
vs.~variantes) et une classification fine des variantes selon le type de
transformation visuelle. Cette classification pose le défi d'identifier,
parmi la multitude de variations possibles, celles qui sont à la fois
détectables par les algorithmes et pertinentes pour l'analyse.

Cela démontre que l'élaboration d'une typologie des éditions de
diagramme ne peut se faire par simple analogie avec les textes. Les
classifications existantes, notamment celle proposée par \dishas,
fournissent un cadre de référence utile. Cependant, la spécificité des
opérations éditoriales sur les images nécessite une adaptation et une
extension de ces classifications. Ces enjeux terminologiques soulignent
le besoin d'une réflexion collective pour parvenir à une standardisation
des termes et des concepts.

