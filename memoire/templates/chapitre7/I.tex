Le module de base est un package pour la gestion documentaire, duquel
l'application ne peut se détacher. Celui-ci inclut tout d'abord des
formulaires pour l'intégration des documents dans la base de données. Le
modèle de données permet de décrire différentes entités qui, bien que
liées dans leurs métadonnées, peuvent être intégrées indépendamment. Le
module de base permet également la création de \mans \iiif pour
chaque numérisation, permettant ensuite la visualisation des documents
grâce aux outils open-source dédiés. De ce fait, l'indexation de zones
d'image peut être réalisée manuellement via l'interface Mirador intégrée
à \sas. Ce noyau fonctionnel inclut en outre la sélection de lots de
documents (le ``panier''), sur lesquels pourront être effectués des
traitements groupés paramétrables.

Les briques fondamentales offrent donc les fonctionnalités essentielles
de gestion documentaire (intégration, modèle de données, \iiif). Les
traitements, quant à eux, sont gérés par des modules séparés, et c'est
sur cette structure que repose la modularité et l'évolutivité de
l'application.

Ci-après nous donnons une description détaillée de certaines de ces
fonctionnalités de base.

\hypertarget{description-des-donnees}{%
\subsection{Description des
données}\label{description-des-donnees}}

Le module de base contient un modèle de données suffisamment extensif
pour décrire efficacement une diversité de données, allant de documents
textuels historiques à des tableaux en histoire de l'art. La
tripartition entre témoin (\wit), série (qui contient un ensemble de
témoins), et contenu permet un alignement avec des corpus très
diversifiés et des données potentiellement hétéroclites, telles que des
manuscrits, des documents épistolaires, des inventaires de galeries
d'art, et même pourquoi pas des cartes\ldots{}

Pour ouvrir à cette large diversité de données, la liste des types de
pagination témoin doit être étendue \emph{a minima} d'un nouveau type
``other'', émancipant l'enregistrement des mentions de pagination. Les
développements futurs prévoient aussi la création d'un système pour
ajouter facilement un nouveau type\footnote{Le type de témoin est une
  métadonnée rentrée par l'utilisateur.rice lors de l'enregistrement du
  \wit dans la base de donnée. Il choisit le type dans une liste,
  originellement manuscrit, imprimé ou gravure sur bois.} de \wit
(tel que peinture, catalogue, etc.).

Au fil des développements, des débats ont émergé autour de l'ajout dans
le modèle de données d'un niveau de granularité supplémentaire pour
décrire des images ou zones d'images unitaires
(\graphicals), créant ainsi une entité détachée du fait
qu'elle provienne d'une extraction dans un document. Cette solution
aurait permis une description plus détaillée et plus fine des images,
importante pour des projets axés sur des images uniques, et aurait
favorisé un élargissement du spectre des type de sources pris en charge.
L'utilisateur.rice aurait pu soit importer une image unique (et de manière
optionnelle, la lier à un \wit) via un formulaire, soit sélectionner
une région d'image d'intérêt au sein des extractions (annotations \sas),
laquelle serait enregistrée comme \graphical, puis l'enrichir de
métadonnées. Dans les deux cas l'enregistrement d'un \graphical
aurait donné lieu à la création d'une \digit au format \jpeg.

Sans l'unité de description \graphical, les régions d'images
sont créées uniquement via les annotations \sas.

L'intégration de cette entité au sein du modèle aurait offert plusieurs
avantages en termes de cohérence et de flexibilité. En s'alignant sur
les structures existantes (\wits et \sers), elle aurait permis une
manipulation plus intuitive des images, facilitant ainsi les opérations
de recherche et la création de \emph{Sets} personnalisés. De plus, elle
aurait rationalisé la gestion des annotations \sas, permettant de
sélectionner les plus pertinentes dans la multitude existante.

Cependant, cette approche présente des limites, et on peut trouver des
alternatives. Tout d'abord, la coexistence de \graphicals avec les
annotations \sas, générées par des processus distincts, aurait pu créer
une certaine confusion quant à leur nature et à leur méthode de
création. De plus, la multiplication potentielle de milliers
d'enregistrements aurait pu impacter les performances de la base de
données et complexifier les requêtes. Enfin, le lien sémantique ambigu
et sujet à interprétation subjective entre \graphical et \wit
aurait compliqué les possibilités de corrélation.

Compte tenu de ces limites, il a semblé préférable de maintenir les
annotations \sas pour identifier les instances de base du modèle, sans
créer de nouvelle unité de description. La solution actuelle reste donc
basée sur la création manuelle ou automatique de zones dans les images
via \iiif et \sas, évitant les problèmes de redondance et de confusion.
Bien que l'entité \graphical n'ait pas été implémentée, les
fonctionnalités d'annotation et de sélection d'images sont assurées par
d'autres mécanismes. L'outil Mirador permet d'associer des tags aux
zones d'image, offrant ainsi une première couche d'enrichissement
sémantique. La sélection dans un \emph{set} personnalisé sera possible en
gardant en mémoire une référence contenant des coordonnées du
\emph{crop}. De plus, l'importation d'images individuelles est
réalisable en les considérant comme des \emph{Witness partiels}, ce qui
permet de les intégrer dans le \textit{workflow} existant. Toutefois
l'enrichissement sémantique à un niveau de granularité fin restera
limité~; et la dépendance à l'outil \sas constitue une potentielle dette
technique, susceptible de restreindre les évolutions futures du système.

Afin de mieux répondre aux exigences de modularité, l'évolution du
modèle de données s'oriente non pas vers une description individuelle
des documents, mais vers la gestion des traitements. Cette évolution
implique la création d'une entité \tr
liée à des ensembles de données (\ds et
\rs) potentiellement hétérogènes.

\hypertarget{principe-du-traitement}{%
\subsection{Principe du Traitement}\label{principe-du-traitement}}

Le but fondamental de la plateforme est de pouvoir effectuer plusieurs
actions sur les objets de la base. Afin d'assurer une meilleure
traçabilité et plus de flexibilité, la plateforme abandonne les
lancements automatiques des processus\footnote{C'était initialement le
  cas de l'extraction des entités, dont le lancement était lié à une
  méthode de classe liée à la \digit après soumission d'un
  formulaire d'ajout d'un \wit ou d'une \ser. L'action se
  lançait immédiatement après enregistrement des images d'une
  numérisation dans la plateforme.} au profit d'un système basé sur
l'entité \tr. Chaque traitement est associé à un ensemble
d'objets traités ensemble (\ds ou \rs), à un jeu de
paramètres et à un résultat. Ces informations sont stockées dans une
table dédiée. Cette approche facilite la gestion et le trackage des
processus (notamment, les utilisateur.rices sont notifiés par e-mail à la fin
du \textit{processing}), permet aux utilisateur.rices de consulter un historique de
leurs actions et offre la possibilité de créer des \textit{workflows}
personnalisés en passant par un formulaire de lancement unique mais
extensif.

En permettant de regrouper des documents de types différents (\wos,
\sers, \wits) dans des \dss, on offre à
l'utilisateur.rice la flexibilité de lancer des actions sur des ensembles
d'entités hétérogènes et granulaires. Le traitement est ensuite réparti
sur les entités de niveau inférieur (les témoins). Les \wits ainsi
sélectionnés peuvent être soumis à une large gamme de traitements~: des
fonctions déjà implémentées comme l'exportation (avec choix du
format), l'extraction, la vectorisation, la recherche de similarité~; ou
de nouveaux traitements personnalisés, tels que la visualisation sur une
frise chronologique ou une carte. La modularité de la plateforme est
assurée par un formulaire de lancement configurable, permettant de
l'adapter à différents scénarios d'utilisation, et à l'ajout de modules
personnalisés.

Le \rs fonctionne similairement au \ds, à un niveau
de granularité inférieur (à l'échelle de la zone d'image)\footnote{À
  l'été 2024, l'entité n'existe pas encore dans la base de données, mais
  le processus d'envoi du traitement et les modes de communication entre
  l'application et l'\api prévoient la possibilité de lancer l'inférence
  des modèles sur un ensemble de régions extraites.}.

\hypertarget{extraction-des-zones-dimage-manuelle}{%
\subsection{Extraction manuelle des zones d'image}\label{extraction-des-zones-dimage-manuelle}}

Le choix de la méthode d'extraction des régions d'intérêt dans les
documents constitue un élément clé de la modularité de la plateforme.
Les utilisateur.rices peuvent opter pour une extraction manuelle ou une
extraction automatique basée sur des algorithmes de vision par
ordinateur, adaptée aux traitements à plus grande échelle.

Après importation d'un enregistrement, le flux de travail procède à la
création de \mans \iiif pour chaque numérisation
(\digit) afin de permettre une visualisation grâce à la
plateforme Mirador. Le module de base autorise par la suite
l'extraction manuelle de zones d'intérêt au sein des images. Cette
fonctionnalité est particulièrement utile pour les projets ne souhaitant
pas recourir à des méthodes entièrement automatisées de vision par
ordinateur. L'outil \sas permet de créer des annotations, c'est-à-dire de
définir des régions d'intérêt spécifiques dans les numérisations, et de
les indexer directement dans les \mans \iiif correspondants,
enrichissant ainsi les ressources numériques. De plus, les
développements futurs prévoient la possibilité d'importer des fichiers
d'annotation préexistants en format .\textsc{txt} afin de pouvoir les indexer
manuellement. Par conséquent, le \textit{workflow} de base ne comporte aucun
traitement automatique basé sur la vision (et de fait éventuellement
trop gourmand en puissance de calcul).

L'extraction, qu'elle soit manuelle ou automatique, constitue le
fondement du reste des processus. Une interface est disponible pour
sélectionner un ensemble de documents et effectuer des actions
spécifiques sur les témoins annotés, via le formulaire de traitement qui
s'étend selon un choix de module configuré. Ainsi l'utilisateur.rice n'est
pas limité par un contexte initial, à l'origine deux étapes
indissociables et incontournables (importation et extraction), pour
pouvoir effectuer d'autres actions. Cette modularité permet de
s'affranchir d'un \textit{workflow} linéaire et prédéfini, offrant ainsi une plus
grande adaptabilité aux besoins spécifiques et aux ressources
matérielles des projets.

Pour conclure, l'existence de ce module de base répond à des besoins
élémentaires des projets de recherche en études visuelles. Il fournit un
outil qui permet d'agréger toutes les sources primaires qui concernent
le sujet, de décrire les sources et de les mettre en relation. Il offre
en outre la possibilité d'extraire et visualiser des contenus d'intérêt
(les ``crops'' d'images), ciblant ainsi les instances de base qui
intéressent les chercheur.ses.