\chapterNo{Résumé}
\addcontentsline{toc}{chapter}{Résumé}
\medskip	

\textbf{Résumé~:} Ce mémoire porte sur l'intégration des outils numériques dans la recherche en \shs, notamment l'utilisation de la \cv pour l'enrichissement et la sémantification des sources historiques. Le projet central étudié est \eida, porteur du développement de la plateforme \aikon, cette dernière mettant à disposition des outils de \dl pour l'extraction et l'analyse des diagrammes astronomiques de tradition ptoléméenne.

Le mémoire pose la question du niveau de spécificité ou de généralité à prévoir dans le développement des outils numériques~: comment construire une chaîne de traitement à la fois flexible, adaptée à des données variées, et capable de répondre à des besoins spécifiques~? Via cette question, ce travail explore les opportunités liés à l'intégration des technologies numériques dans la recherche en \shs, et relève les défis tenant au partage des outils, impactant le partage des pratiques et des méthodes. 

\medskip

\textbf{Abstract~:} This thesis focuses on the integration of digital tools in research within the humanities and social sciences, particularly the use of \cv for the enrichment and semantic annotation of historical sources. The central project under study is \eida, which is responsible for the development of the \textsc{aikon} platform. This platform provides \dl tools for the extraction and analysis of astronomical diagrams from the Ptolemaic tradition.

The thesis addresses the question of the level of specificity or generality to be anticipated in the development of digital tools~: how can one construct a processing pipeline that is both flexible, adaptable to various types of data, and capable of meeting specific needs~? Through this inquiry, the work explores the opportunities related to the integration of digital technologies in humanities and social sciences research and highlights the challenges associated with tool sharing, which in turn affects the sharing of practices and methods.

\medskip

\textbf{Mots-clés~:} histoire des science~; humanités numérique~; diagrammes astronomiques~; \ia~; \dl~; Python~; modularité~; standardisation technique~; interopérabilité.\\

\medskip

\textbf{Informations bibliographiques~:} Clara GROMETTO, \textit{Le partage des outils numériques dans la recherche, Élaboration d'une plateforme extensive pour le traitement des données visuelles}, mémoire de master \og Technologies numériques appliquées à l'histoire~\fg, dir. Ségolène Albouy, École nationale des chartes, 2024.
	
\clearemptydoublepage