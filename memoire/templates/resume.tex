\chapterNo{Résumé}
\addcontentsline{toc}{chapter}{Résumé}
\medskip	

Ce mémoire porte sur l'intégration des outils numériques dans la recherche en \shs, notamment l'utilisation de la \cv pour l'enrichissement et la sémantification des sources historiques. Le projet central étudié est \eida, porteur du développement de la plateforme \aikon, cette dernière mettant à disposition des outils de \dl pour l'extraction et l'analyse des diagrammes astronomiques de tradition ptoléméenne.

Le mémoire pose la question du niveau de spécificité ou de généralité à prévoir dans le développement des outils numériques~: comment construire une chaîne de traitement à la fois flexible, adaptée à des données variées, et capable de répondre à des besoins spécifiques~? Via cette question, ce travail explore les opportunités liés à l'intégration des technologies numériques dans la recherche en \shs, et relève les défis tenant au partage des outils, impactant le partage des pratiques et des méthodes. 

\textbf{Mots-clés~:} histoire des science~; humanités numérique~; diagrammes astronomiques~; \ia~; \dl~; Python~; modularité~; standardisation technique~; interopérabilité.\\

\textbf{Informations bibliographiques~:} Clara GROMETTO, \textit{La standardisation des outils numériques dans la recherche, Élaboration d'une plateforme extensive pour le traitement des données visuelles basé sur la vision artificielle}, mémoire de master \og Technologies numériques appliquées à l'histoire~\fg, dir. Ségolène Albouy, École nationale des chartes, 2024.
	
\clearemptydoublepage