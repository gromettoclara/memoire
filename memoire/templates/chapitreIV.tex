Dans des termes très simples, pour faire exécuter une tâche à la
machine, deux solutions existent. La première consiste à écrire un
programme, dont l'expert métier a explicité les règles. Le programme est
entièrement rédigé par un.e développeur.se. Il effectue une tâche précise,
chaque conjecture spécifique doit être prévue et son traitement
clairement formulé. Si le code produit des erreurs, il doit être modifié
par le.a développeur.se. La deuxième approche consiste à donner au programme
la capacité de se modifier lui-même, sans que cette modification soit
explicitement rédigée. L'expert métier doit coder l'architecture du modèle et
annoter des exemples ; puis un seul algorithme (d'apprentissage) suffit pour
traiter de multiples cas réels. En cas d'erreur de l'algorithme, il faut agir non plus
sur le programme mais sur les exemples d'apprentissage. L'objectif est
d'apprendre à généraliser pour prédire sur des exemples non vus pendant
l'apprentissage\footcite[p.7]{chollet_apprentissage_2020}.

\begin{kwote}         
``On peut ainsi opposer un programme \emph{classique}, qui utilise une
procédure et les données qu'il reçoit en entrée pour produire en sortie
des réponses, à un programme \emph{d'apprentissage automatique}, qui
utilise les données et les réponses afin de produire la procédure qui
permet d'obtenir les secondes à partir des premières.''\footcite[p.1-2]{azencott_introduction_2022}
\end{kwote} 

L'émergence de l'apprentissage machine a alors ouvert de nouvelles
perspectives en permettant de modéliser des interactions complexes entre
les données. Les modèles de \dl sont également plus
résilients face aux variations et au bruit présents dans les
données\footcite{juneja_deep_2023}. Ainsi,
l'application de l'apprentissage profond se révèle particulièrement
pertinente dans le cadre d'\eida, car elle permet de relever le défi
de la sémantification des images, y compris lorsqu'il s'agit de sources
historiques complexes.

        \hypertarget{les-reseaux-de-neurone}{%
        \section{Les réseaux de neurones, ou la généralisation prise au sens
        mathématique}\label{les-reseaux-de-neurone}}
         Le module de base est un package pour la gestion documentaire, duquel
l'application ne peut se détacher. Celui-ci inclut tout d'abord des
formulaires pour l'intégration des documents dans la base de données. Le
modèle de données permet de décrire différentes entités qui, bien que
liées dans leurs métadonnées, peuvent être intégrées indépendamment. Le
module de base permet également la création de \mans \iiif pour
chaque numérisation, permettant ensuite la visualisation des documents
grâce aux outils open-source dédiés. De ce fait, l'indexation de zones
d'image peut être réalisée manuellement via l'interface Mirador intégrée
à \sas. Ce noyau fonctionnel inclut en outre la sélection de lots de
documents (le ``panier''), sur lesquels pourront être effectués des
traitements groupés paramétrables.

Les briques fondamentales offrent donc les fonctionnalités essentielles
de gestion documentaire (intégration, modèle de données, \iiif). Les
traitements, quant à eux, sont gérés par des modules séparés, et c'est
sur cette structure que repose la modularité et l'évolutivité de
l'application.

Ci-après nous donnons une description détaillée de certaines de ces
fonctionnalités de base.

\hypertarget{description-des-donnees}{%
\subsection{Description des
données}\label{description-des-donnees}}

Le module de base contient un modèle de données suffisamment extensif
pour décrire efficacement une diversité de données, allant de documents
textuels historiques à des tableaux en histoire de l'art. La
tripartition entre témoin (\wit), série (qui contient un ensemble de
témoins), et contenu permet un alignement avec des corpus très
diversifiés et des données potentiellement hétéroclites, telles que des
manuscrits, des documents épistolaires, des inventaires de galeries
d'art, et même pourquoi pas des cartes\ldots{}

Pour ouvrir à cette large diversité de données, la liste des types de
pagination témoin doit être étendue \emph{a minima} d'un nouveau type
``other'', émancipant l'enregistrement des mentions de pagination. Les
développements futurs prévoient aussi la création d'un système pour
ajouter facilement un nouveau type\footnote{Le type de témoin est une
  métadonnée rentrée par l'utilisateur.rice lors de l'enregistrement du
  \wit dans la base de donnée. Il choisit le type dans une liste,
  originellement manuscrit, imprimé ou gravure sur bois.} de \wit
(tel que peinture, catalogue, etc.).

Au fil des développements, des débats ont émergé autour de l'ajout dans
le modèle de données d'un niveau de granularité supplémentaire pour
décrire des images ou zones d'images unitaires
(\graphicals), créant ainsi une entité détachée du fait
qu'elle provienne d'une extraction dans un document. Cette solution
aurait permis une description plus détaillée et plus fine des images,
importante pour des projets axés sur des images uniques, et aurait
favorisé un élargissement du spectre des type de sources pris en charge.
L'utilisateur.rice aurait pu soit importer une image unique (et de manière
optionnelle, la lier à un \wit) via un formulaire, soit sélectionner
une région d'image d'intérêt au sein des extractions (annotations \sas),
laquelle serait enregistrée comme \graphical, puis l'enrichir de
métadonnées. Dans les deux cas l'enregistrement d'un \graphical
aurait donné lieu à la création d'une \digit au format \jpeg.

Sans l'unité de description \graphical, les régions d'images
sont créées uniquement via les annotations \sas.

L'intégration de cette entité au sein du modèle aurait offert plusieurs
avantages en termes de cohérence et de flexibilité. En s'alignant sur
les structures existantes (\wits et \sers), elle aurait permis une
manipulation plus intuitive des images, facilitant ainsi les opérations
de recherche et la création de \emph{Sets} personnalisés. De plus, elle
aurait rationalisé la gestion des annotations \sas, permettant de
sélectionner les plus pertinentes dans la multitude existante.

Cependant, cette approche présente des limites, et on peut trouver des
alternatives. Tout d'abord, la coexistence de \graphicals avec les
annotations \sas, générées par des processus distincts, aurait pu créer
une certaine confusion quant à leur nature et à leur méthode de
création. De plus, la multiplication potentielle de milliers
d'enregistrements aurait pu impacter les performances de la base de
données et complexifier les requêtes. Enfin, le lien sémantique ambigu
et sujet à interprétation subjective entre \graphical et \wit
aurait compliqué les possibilités de corrélation.

Compte tenu de ces limites, il a semblé préférable de maintenir les
annotations \sas pour identifier les instances de base du modèle, sans
créer de nouvelle unité de description. La solution actuelle reste donc
basée sur la création manuelle ou automatique de zones dans les images
via \iiif et \sas, évitant les problèmes de redondance et de confusion.
Bien que l'entité \graphical n'ait pas été implémentée, les
fonctionnalités d'annotation et de sélection d'images sont assurées par
d'autres mécanismes. L'outil Mirador permet d'associer des tags aux
zones d'image, offrant ainsi une première couche d'enrichissement
sémantique. La sélection dans un \emph{set} personnalisé sera possible en
gardant en mémoire une référence contenant des coordonnées du
\emph{crop}. De plus, l'importation d'images individuelles est
réalisable en les considérant comme des \emph{Witness partiels}, ce qui
permet de les intégrer dans le \textit{workflow} existant. Toutefois
l'enrichissement sémantique à un niveau de granularité fin restera
limité~; et la dépendance à l'outil \sas constitue une potentielle dette
technique, susceptible de restreindre les évolutions futures du système.

Afin de mieux répondre aux exigences de modularité, l'évolution du
modèle de données s'oriente non pas vers une description individuelle
des documents, mais vers la gestion des traitements. Cette évolution
implique la création d'une entité \tr
liée à des ensembles de données (\ds et
\rs) potentiellement hétérogènes.

\hypertarget{principe-du-traitement}{%
\subsection{Principe du Traitement}\label{principe-du-traitement}}

Le but fondamental de la plateforme est de pouvoir effectuer plusieurs
actions sur les objets de la base. Afin d'assurer une meilleure
traçabilité et plus de flexibilité, la plateforme abandonne les
lancements automatiques des processus\footnote{C'était initialement le
  cas de l'extraction des entités, dont le lancement était lié à une
  méthode de classe liée à la \digit après soumission d'un
  formulaire d'ajout d'un \wit ou d'une \ser. L'action se
  lançait immédiatement après enregistrement des images d'une
  numérisation dans la plateforme.} au profit d'un système basé sur
l'entité \tr. Chaque traitement est associé à un ensemble
d'objets traités ensemble (\ds ou \rs), à un jeu de
paramètres et à un résultat. Ces informations sont stockées dans une
table dédiée. Cette approche facilite la gestion et le trackage des
processus (notamment, les utilisateur.rices sont notifiés par e-mail à la fin
du \textit{processing}), permet aux utilisateur.rices de consulter un historique de
leurs actions et offre la possibilité de créer des \textit{workflows}
personnalisés en passant par un formulaire de lancement unique mais
extensif.

En permettant de regrouper des documents de types différents (\wos,
\sers, \wits) dans des \dss, on offre à
l'utilisateur.rice la flexibilité de lancer des actions sur des ensembles
d'entités hétérogènes et granulaires. Le traitement est ensuite réparti
sur les entités de niveau inférieur (les témoins). Les \wits ainsi
sélectionnés peuvent être soumis à une large gamme de traitements~: des
fonctions déjà implémentées comme l'exportation (avec choix du
format), l'extraction, la vectorisation, la recherche de similarité~; ou
de nouveaux traitements personnalisés, tels que la visualisation sur une
frise chronologique ou une carte. La modularité de la plateforme est
assurée par un formulaire de lancement configurable, permettant de
l'adapter à différents scénarios d'utilisation, et à l'ajout de modules
personnalisés.

Le \rs fonctionne similairement au \ds, à un niveau
de granularité inférieur (à l'échelle de la zone d'image)\footnote{À
  l'été 2024, l'entité n'existe pas encore dans la base de données, mais
  le processus d'envoi du traitement et les modes de communication entre
  l'application et l'\api prévoient la possibilité de lancer l'inférence
  des modèles sur un ensemble de régions extraites.}.

\hypertarget{extraction-des-zones-dimage-manuelle}{%
\subsection{Extraction manuelle des zones d'image}\label{extraction-des-zones-dimage-manuelle}}

Le choix de la méthode d'extraction des régions d'intérêt dans les
documents constitue un élément clé de la modularité de la plateforme.
Les utilisateur.rices peuvent opter pour une extraction manuelle ou une
extraction automatique basée sur des algorithmes de vision par
ordinateur, adaptée aux traitements à plus grande échelle.

Après importation d'un enregistrement, le flux de travail procède à la
création de \mans \iiif pour chaque numérisation
(\digit) afin de permettre une visualisation grâce à la
plateforme Mirador. Le module de base autorise par la suite
l'extraction manuelle de zones d'intérêt au sein des images. Cette
fonctionnalité est particulièrement utile pour les projets ne souhaitant
pas recourir à des méthodes entièrement automatisées de vision par
ordinateur. L'outil \sas permet de créer des annotations, c'est-à-dire de
définir des régions d'intérêt spécifiques dans les numérisations, et de
les indexer directement dans les \mans \iiif correspondants,
enrichissant ainsi les ressources numériques. De plus, les
développements futurs prévoient la possibilité d'importer des fichiers
d'annotation préexistants en format .\textsc{txt} afin de pouvoir les indexer
manuellement. Par conséquent, le \textit{workflow} de base ne comporte aucun
traitement automatique basé sur la vision (et de fait éventuellement
trop gourmand en puissance de calcul).

L'extraction, qu'elle soit manuelle ou automatique, constitue le
fondement du reste des processus. Une interface est disponible pour
sélectionner un ensemble de documents et effectuer des actions
spécifiques sur les témoins annotés, via le formulaire de traitement qui
s'étend selon un choix de module configuré. Ainsi l'utilisateur.rice n'est
pas limité par un contexte initial, à l'origine deux étapes
indissociables et incontournables (importation et extraction), pour
pouvoir effectuer d'autres actions. Cette modularité permet de
s'affranchir d'un \textit{workflow} linéaire et prédéfini, offrant ainsi une plus
grande adaptabilité aux besoins spécifiques et aux ressources
matérielles des projets.

Pour conclure, l'existence de ce module de base répond à des besoins
élémentaires des projets de recherche en études visuelles. Il fournit un
outil qui permet d'agréger toutes les sources primaires qui concernent
le sujet, de décrire les sources et de les mettre en relation. Il offre
en outre la possibilité d'extraire et visualiser des contenus d'intérêt
(les ``crops'' d'images), ciblant ainsi les instances de base qui
intéressent les chercheur.ses.


        \hypertarget{des-traitements-et-des-architectures-diverses}{%
        \section{Des traitements et des architectures
        diverses}\label{des-traitements-et-des-architectures-diverses}}
         Spécialiser un modèle d'intelligence artificielle implique de lui
fournir des données pertinentes, diversifiées, et en quantité
suffisante. Cependant, pour certains domaines, dont l'histoire fait
partie, le volume de données disponible est insuffisant. Ce constat est
d'autant plus vrai dans le cas des diagrammes issus de traités
astronomiques~: les corpus de documents scientifiques historiques
contiennent généralement du texte en majeure partie, des tables et des
images, négligeant souvent les diagrammes.\footnote{Exception faite du
  corpus S-VED (\cite{buttner_cordeep_2022}), collection
  d'illustration très diverses contenant entre autre des diagrammes
  historiques. Mais les primitives ne sont pas annotées.}. De plus, ils
sont dénués d'annotations précises sur les éléments constitutifs des
pages~; c'est sans parler de l'inexistence d'un corpus de diagrammes
dont les primitives sont annotées. Or l'annotation est une tâche
chronophage et fastidieuse. Le recours aux données synthétique répond,
mais en partie seulement, à ces problématiques.

\hypertarget{datasets-synthetiques}{%
\subsection{\emph{datasets} synthétiques}\label{datasets-synthetiques}}

Les \textit{datasets} synthétiques sont générés par des algorithmes ou des
méthodes de simulation pour imiter des données réelles, sans être
directement extraites de sources existantes. De tels jeux de données
sont utilisés lorsque les données réelles sont limitées ou difficiles à
obtenir, mais qu'il est cependant nécessaire de contrôler spécifiquement
les caractéristiques des données d'entraînement\footcite{buttner_cordeep_2022}. La génération
d'images a pour but de fabriquer des ensembles de données plus vastes,
plus diversifiés, très variables et assez complexes, répondant aux
caractéristiques des objets d'intérêt du projet, et surtout étiquetés
automatiquement, sans recourir à l'annotation manuelle.

Ces données synthétiques sont assez ressemblantes et complexes pour être
exploitées. Par exemple, docExtractor est un modèle off-the-shell (au
même titre que \yolo) envisagé dans le cadre de la tâche d'extraction des
diagrammes, et qui se veut sépcifique aux données historiques, car il
est entraîné sur des données produites par un générateur de documents
historiques synthétiques~: SynDoc\footcite{monnier_docextractor_2020}. SynDoc
génère des images de manière aléatoire en combinant des éléments
graphiques (fonds, images, texte et bruit) provenant d'un jeu d'image
défini (constitué de 177 images de pages, 15 contextes, plus de 8000
œuvres d'art provenant de WikiArt, des lettrines générées à partir d'une
lettre aléatoire avec 91 fonts possibles, et des dessins, schémas et
textes tirés d'articles aléatoires sur Wikipedia, avec plus de 400
fonts). Les différents éléments s'agencent, intégrant sur le fond
images, texte et bruit, offrant des combinasons et des mises en pages
assez complexes. Chaque élément de contenu est pré-annoté, éliminant
ainsi le besoin d'annotations manuelles pour ces pages.

          \begin{figure}[H]
          \begin{center}
          \includegraphics[height=6.5cm]{figues/syndoc.jpg}
          \end{center}
          \caption{Données synthétiques générées par SynDoc.\footcite[p.46]{norindr_traitement_2023}}
          \label{fig:syndoc} \end{figure}

Pour entraîner le modèle de vectorisation, il a de même été nécessaire
d'utiliser des données synthétiques. Parce qu'annoter les primitives
géométriques dans des images de diagrammes complexes est très
chronophage, le modèle de vectorisation a été pré-formé sur des corpus
artificiels générés dynamiquement. Le script de génération des données
d'entraînement choisit aléatoirement un arrière-plan, y ajoute des mots,
des nombres et des glyphes puis crée artificiellement un diagramme en
insérant des segments, des cercles et des arcs. Le script est conçu pour
que ces diagrammes aient une forte probabilité de présenter des formes
très caractéristiques comme les cercles concentriques et tangents, les
lignes parallèles et les arcs connectés, afin de simuler les structures
typiques. Les primitives sont dessinées avec des
variations aléatoires d'opacité, de largeur et de couleur. Les cercles
peuvent être remplis ou vides. Enfin, du bruit est ajouté en appliquant
un flou gaussien, et en supprimant de petites régions du diagramme pour
imiter la dégradation des documents historiques. Les données
d'entraînement ainsi générées présentent des configurations assez
complexes.

          \begin{figure}[H]
          \begin{center}
          \includegraphics[height=7cm]{figues/vecto_synthetic_data.png}
          \end{center}
          \caption{Données synthétiques générées pour l'entraînement du modèle de vectorisation.\footcite[Figure issue de la présentation de Syrine Kalelli à l'occasion de la conférence \eida 2024~:][]{noauthor_eida_nodate-1}}
          \label{fig:vecto_synthetic} \end{figure}

Enfin, le modèle de similarité présente un troisième exemple, puisque
SegSwap est pré-entraîné sur de la donnée synthétique. Le script de
génération prend des parties aléatoires d'une images et les copie-colle
au-dessus d'une autre image. Les trois images (source, cible et
superposition) sont placées dans le même dataset d'entraînement, ainsi
le modèle apprend à retrouver ce qui, dans la superposition, vient de la
source, et ce qui vient de la cible.

          \begin{figure}[H]
          \begin{center}
          \includegraphics[height=3cm]{figues/segswap_blended_images.png}
          \end{center}
          \caption{Données d'entraînement du modèle Segswap.}
          \label{fig:segswap} \end{figure}

\hypertarget{les-donnees-reelles}{%
\subsection{Les données réelles}\label{les-donnees-reelles}}

S'appuyer sur les modèles \textit{off-the-shelf}, sur de larges \textit{datasets}
généralistes, ou sur des données synthétiques permet une implémentation
facilitée de la vision dans des projets et constitue une base solide.
Toutefois, les sources tenant aux deux projets (\vhs et \eida) sont trop
spécifiques pour se contenter de modèles généralistes ou formés sur des
données artificielles. Même si ces derniers peuvent offrir des performances
de base, ils risquent de manquer de précision et de sensibilité aux
particularités des documents historiques. Les corpus artificiels
présentent des configurations délibérément complexes pour s'approcher le
plus possible des difficultés que le modèle pourrrait rencontrer sur les
données réelle. Elles sont cependant irréalistes et insuffisantes pour
permettre aux modèles de généraliser sur des diagrammes réels.

En atteste la comparaison des performances de docExtractor et \yolov sur
les données d'\eida. docExtractor\footcite{monnier_docextractor_2020}, entraîné
sur des données synthétiques mimant les documents historiques serait en
théorie plus adapté au traitement d'images de pages de manuscrits, avec
du texte et des illustration côté à côte, d'autant qu'il intègre des
outils de traitement du texte (notamment pour la segmentation des
lignes)\footnote{\eida envisage l'implémentation d'un outil d'extraction
  et transcription des labels et des textes qui entourent les diagrammes}.
Pourtant, sans fine-tuning sur des données réelles, il présente des
performances équivalentes à celles de \yolov\footcite[p.45]{norindr_traitement_2023}. Cela
souligne que même les modèles off-the-shelf entraînés sur un corpus
assez spécifique et complexe, mais synthétique, ne dispense pas d'un
entraînement sur des données réelles, au même titre que les modèles très
généralistes comme \yolov.

Alors, le modèle de base \yolov tel que mis à disposition par
Ultralytics est entraîné sur de grands ensembles de données réelles, ce
qui constitue une base solide pour la classification des objets du
monde. L'utilisation de SynDoc permet ensuite de compléter
l'apprentissage initial en exposant le modèle à des exemples variés et
spécifiques aux documents historiques, augmentant ainsi sa capacité de
généralisation. Ces similis de manuscrits anciens offrent l'avantage de
pouvoir être produits en grandes quantités et de couvrir un large
éventail de scénarii et de configurations difficiles à obtenir dans des
ensembles de données réelles. Puis le modèle est entraîné sur les
données de \vhs, qui sont de réelles pages de documents historiques
contenant une large diversité d'illustrations. Ces données apporteront
une dimension supplémentaire de pertinence au modèle, en l'exposant à
des particularités des documents historiques réalistes. Enfin, \yolov
est entraîné sur les données d'\eida, qui sont orientées spécifiquement
vers les diagrammes, afin qu'il détecte uniquement ces derniers.

Quant au modèle de vectorisation développé par Syrine
Kalleli\footcite{kalleli_historical_2024}, il est formé
sur des données synthétiques générées à la volée par un script. Mais le
corpus de diagrammes d'\eida est particulièrement caractéristique et le
modèle n'aurait pu être optimal sans avoir appris sur des images de
diagrammes issus de manuscrits réels. Un corpus d'entraînement de 303
diagrammes extraits de manuscrits et de gravures a donc été constitué et
annoté par les historien.nes. Ces diagrammes sont issus de sources latines, arabes,
grecques, hébreuses ou chinoises, datant du \textsc{xii}\ieme au \textsc{xviii}\ieme siècle, et ils
présentent en guise d'étiquettes plus de 3000 lignes, cercles et arcs. Le
ré-entraînement a permis le transfert des connaissances acquises sur la
tâche de détection des primitives sur les données réalistes.

Il sera également possible d'obtenir des meilleurs résultats sur la
similarité grâce à une évaluation des scores (qui constitue un jeu de
données annotées) et le ré-entraînement du modèle, pour donner des
résultats plus adaptés à la spécificité des données historiques.

D'ailleurs, cette étape d'annotation (le choix des exemples et des
étiquettes) revêt des enjeux importants. L'apprentissage spécifique se
fait à partir de données sélectionnées par les chercheur.ses~: les exemples
sur lequel l'algorithme d'apprentissage va itérer définissent le modèle.
Il est nécessaire de constituer un échantillon de données aléatoire et
représentatif, et de l'annoter en fonction de ce que l'on souhaite
obtenir en prédiction.

L'annotation des jeux de données est non seulement une étape clé, mais
aussi un bel exemple de collaboration chercheur.ses-ingénieur.es. Elle
nécessite la définition de normes pertinentes et rigoureuses. Travail
minutieux et chronophage, l'étiquetage des données peut engendrer des
erreurs et du bruit dans les données, car elle implique la subjectivité
des chercheur.ses et le regard parfois trop précis sur les sources desquels
les annotateurs sont experts.

Voici un exemple rencontré lors de la préparation des données pour
entraîner un modèle de segmentation du contenu textuel. Les sources
arabes et chinoises sont particulièrement verbeuses et les diagrammes
sont très souvent entourés des blocs de commentaires se mélangeant alors
aux légendes et aux labels. Doit-on considérer ces commentaires comme
faisant partie des éléments que l'on souhaite identifier ou bien les ignorer
? Cette décision est importante car si on les ignore, le modèle risque
de passer à côté d'éléments textuels pertinents. En revanche, si on les
inclut, il ramènera des commentaires sans rapport direct avec le
diagramme observé. On voit ici comment la binarité des modèles, qui se
reflète dans les normes d'annotation, est problématique et constitue une
limite au \ml. Un compromis doit être trouvé entre
l'automatisation, qui requiert une normalisation, des définitions
claires et binaires, et la nuance dans l'interprétation des
sources\footnote{Dans le cadre du projet, il a toujors été plus
  intéressant d'opter pour une définition extensive des objets à
  détecter, car prévision d'une correction des traitement. Et il est
  plus facile de supprimer un élémént pas pertinent que d'aller en
  rechercher un, surtout compte tenu de la taille des corpus des
  chercheur.ses. Vaut aussi pour la préparation des données pour
  l'entraînement du modèle d'extraction.}.

La normalisation peut bénéficier à l'écosystème de recherche dans le
domaine de l'\htr et de l'\ocr. À ce titre, il est pertinent d'envisager
l'utilisation du vocabulaire contrôlé SegmOnto pour l'annotation du
contenu textuel entourant les diagrammes. Cela permettrait de créer des
jeux de données réutilisables, à partager avec des projets poursuivant
des objectifs similaires.\footnote{https://segmonto.github.io/}. Encore
une fois, un compromis doit être trouvé entre les besoins de description
des chercheur.ses et les possibilités offertes par les vocabulaires
contrôlés.

Un autre exemple concerne le dernier entraînement du modèle d'extraction
: les résultats montrent que des diagrammes sont encore détectés en
transparence. La question s'est alors posée de chercher à corriger ce
défaut en donnant au modèle, à l'occasion d'un nouvel entraînement,
d'avantage d'exemples négatifs (diagrammes visibles par transparence
mais non annotés). Or il est préférable de se contenter de la correction
ou suppression manuelle de ces prévisions erronées, garantissant que le
modèle parvienne à détecter les diagrammes presque effacés.

Pour assurer la rigueur et la cohérence des annotations, les décisions
prises entre les chercheur.ses et les ingénieur.es peuvent être l'objet d'une
documentation ou d'ateliers d'annotation.

\hypertarget{loeil-de-la-machine-avantages-et-limites}{%
\subsection{L'oeil de la machine~: avantages et
limites}\label{loeil-de-la-machine-avantages-et-limites}}

Bien qu'il soit possible d'optimiser les performances d'un modèle
d'apprentissage automatique en l'entraînant sur un ensemble de données
spécifique, son interprétation des données reste limitée car
fondamentalement binaire, ce qui le rend parfois déficient pour la
recherche en histoire. Ainsi, il gèrera difficilement les cas limites et
ambigüs. La décision d'inclure ou d'exclure ces cas particuliers de
l'ensemble d'entraînement implique un arbitrage délicat. D'un côté, une
inclusion trop restrictive peut compromettre les capacités de
généralisation du modèle, c'est-à-dire sa capacité à s'adapter à de
nouvelles données. À l'inverse, une inclusion trop permissive risque de
dégrader la précision du modèle sur les cas plus typiques. Les
chercheur.ses espérant obtenir un modèle maximaliste, quitte à accepter un
certain degré d'erreur et de devoir supprimer les faux
positifs, de nombreux cas limites ont été inclus. Le cas des diagrammes
visibles en transparence (expliqué précédemment) en est un exemple
éloquent.

Une autre difficulté réside dans la définition même du ``diagramme
astronomique''. Les limites de ce concept ne sont pas si claires et
définitives pour les chercheur.ses, et pourtant le modèle a besoin d'une
définition rigoureuse et cohérente. Il paraît en effet difficile de
considérer les diagrammes astronomiques en dehors du contexte des
pratiques d'autres sciences et disciplines connexes. Par exemple, Le
\emph{Flores Almagesti} -- réécriture de l'Almageste datant du \textsc{xv}\ieme par
l'astronome Giovanni Bianchini -- présente une partie algébrique à
l'ouverture mathématique, induisant la présence de nouveaux types de
diagrammes d'inspiration euclidienne. Pour retracer la source de ces
derniers, il est nécessaire de considérer les traités d'Euclide ou
autres travaux d'algèbre. Ceux-ci ne sont pas des traités
\emph{astronomiques}, bien qu'il ne soit pas certain que ces disinctions
contemporaines aient été aussi rigide à l'époque et aient eu un
quelconque sens pour les acteurs historiques. Les sources byzantines
confirment cette complexité~: les diagrammes y sont nommés
\emph{katagraphai}, indépendamment du domaine scientifique auquel ils
appartiennent. Également, de nombreux travaux astronomiques sont groupés
dans des témoins qui contiennent des œuvres issus de domaines divers.
C'est le cas avec les sources chinoises, comme le \emph{Chongzhen
lishu}, qui se présente généralement annexé d'une série de traités
mathématiques. Par conséquent, les diagrammes euclidiens ont été gardés
lors de la préparation des données, et l'algorithme de détection les
classe comme ``diagramme'', même s'ils ne constituent pas l'objet
principal des chercheur.ses.

En ce qui concerne les autres types de diagrammes non strictement
astronomiques (géométriques, harmoniques, logiques, illustrations de
constellations), une approche plus sélective a été adopté afin d'éviter
un modèle trop maximalistes. Ces éléments, bien que potentiellement
intéressants, n'ont pas été inclus dans la phase de détection
automatique.

Ainsi l'œil de la \cv contraint à des choix méthodologique
potentiellement inconfortables, mais en même temps il peut aider à
mesurer les impulsions des chercheur.ses, à mieux définir les objectifs de
recherche et à prioriser les éléments les plus pertinents. Ainsi, la
vision par ordinateur oblige les chercheur.ses à s'adapter à une logique
algorithmique qui, tout en limitant certaines interprétations
subjectives, offre l'opportunité de développer des modèles conceptuels et des méthodologies très rigoureuses.

\vspace{2cm}

En résumé, la diversité des architectures de réseaux de neurones permet
d'adapter les modèles aux spécificités des données et des tâches à
traiter, tout en optimisant les performances et l'efficacité
computationnelle. Cette flexibilité est essentielle pour répondre aux
multiples défis posés par le traitement automatique des documents
historiques. Deux approches se distinguent~: l'utilisation de modèles
\textit{off-the-shelf} ou d'architectures spécifiques. Traditionnellement, les
ingénieur.es concevaient des architectures de réseau de neurones sur
mesure, adaptées spécifiquement aux problèmes à résoudre. Cette
approche, bien que permettant une optimisation fine des performances,
est souvent coûteuse en temps et en ressources. Ces dernières années,
l'émergence de modèles pré-entraînés sur des jeux de données massifs a
ouvert de nouvelles perspectives. Ces modèles \textit{off-the-shelf} (ou
\textit{pre-trained models}), déjà qualitatifs, demandent cependant à être spécialisés sur des corpus. Le \textit{fine-tuning}
consiste alors à adapter ces modèles à une tâche spécifique en les
entraînant sur un jeu de données plus petit et plus pertinent.

L'approche de corpus de dimensions importantes, ne permettant pas une analyse
manuelle, trop chronophage, est permise par la vision artificielle.
L'utilisation des modèles de vision présente cependant des limites~:
notamment, la complexité de leur fonctionnement reste assez opaque aux
non spécialistes, ce qui peut être un inconvénient là où la transparence
et l'explicabilité sont importantes, typiquement pour des projets
transversaux caractéristiques des humanités numériques. D'autant que la
collaboration entre les chercheur.ses spécialistes des sources et les
ingénieur.es \textit{data scientists} est essentielle~: les deux pôles doivent
collaborer pour entraîner les modèles et les rendre performants sur les
données spécifiques. C'est pourquoi l'optimisation des modèles de vision
ne s'obtient pas sans un effort du côté des équipes disciplinaires, qui
vont réunir et annoter des jeux de données importants pour entraîner
efficacement les modèles.