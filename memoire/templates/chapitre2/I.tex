La dernière décennie observe un changement de paradigme dans la
recherche par la multiplication des actions concertées entre les
institutions culturelles et patrimoniales, les établissements de
l'enseignement supérieur, les chercheur.ses, les personnels et les
étudiants du monde. La production scientifique sans frontières qui en
résulte repose sur une nouvelle stratégie de coopération internationale
dans le cadre de partenariats qui facilitent la circulation des
connaissances. Or les outils numériques sont vecteurs d'affirmation de
ces modes d'irrigation du savoir.

\hypertarget{enjeux-de-la-recuperation-des-sources-numerisees}{%
\subsection{Enjeux de la récupération des sources
numérisées}\label{enjeux-de-la-recuperation-des-sources-numerisees}}

Dans le cadre d'un projet tel qu'\eida, qui repose sur des traitements
automatiques des sources par des algorithmes de vision par ordinateur,
la disponibilité des sources numérisées constitue un principe
fondamental de la démarche de recherche.

La numérisation, qui transforme un objet physique en fichier image
(matrice de pixels), opérant un passage de la matière en données, est un
premier pas pour accès et rendre visible les documents
patrimoniaux, notamment les sources de la recherche destinées à être
enrichies et étudiées. La numérisation n'a en effet pas vocation à
interpréter mais à citer l'objet, elle offre une image la plus proche
possible de l'objet en prenant en compte l'intégralité du support.

L'examen exhaustif des nombreuses sources inexploitées disponibles dans
les archives et les bibliothèques à l'aide des méthodes classiques n'est
pas viable. Cette limitation a conduit, ces dernières années, à
l'émergence de nombreuses initiatives visant à numériser ces collections
de documents historiques. Parmi ces initiatives, on peut citer Google
Books Search (\textsc{gbs}) et l'Open Content Alliance (\textsc{oca}). Objet d'une
attention particulière par les bibliothèques et les musées depuis les
années 90, la numérisation s'inscrit généralement pour les institutions
dans un vaste programme de mise à disposition des documents qui peuvent
alors circuler via les canaux d'internet, en facilitant la consultation,
spécialement à destination du monde académique.

Si la numérisation vise avant tout, outre les questions de conservation, à donner accès, l'enjeux de la
valorisation des documents, fonds et collections des bibliothèques et
des musées, est également présent. Quand on numérise pour faciliter la
consultation, on vise un public déjà familier du contenu.
L'affranchissement de l'hétérogénéité des supports permis par le passage
au format numérique ouvre en outre de nouvelles voies d'exploitations
techniques.

Certains grands opérateurs assurent alors des programmes ambitieux de
numérisation et de valorisation des contenus culturels. La \bnf en est
l'exemple le plus probant, numérisant plus d'un million de pages par
mois à partir de ses collections patrimoniales\footcite{noauthor_numerisation_nodate}. Depuis
2013, cette production est complétée par des documents numériques
produits dans le cadre des accords conclus par \bnf-Partenariats.

De vastes projets, à l'image de Gallica ou Européena, larges campagnes
de numérisation accompagnées de plateformes de mise à disposition et de
valorisation, voient alors le jour. Les collections se diffusent à
l'international~; Internet devient un espace permettant à des
utilisateur.rices aux profils variés d'exploiter ce patrimoine culturel
numérique au-delà des limites physiques de consultation des documents.

Malgré le soutien des acteurs publiques, les disparités nationales et
internationales subsistent en matière de numérisation et mise en ligne
des collections, disparités liées à la capacité des institutions à
porter ces projets souvent coûteux. De nombreux facteurs rentrent en jeu
: financements, qualité des données, volume de documents à traiter. Ces
inégalités constituent un biais important à prendre en compte, notamment
pour les projets de recherche s'appuyant -- à l'instar d'\eida -- sur des
sources internationales. On gardera donc à l'esprit que les corpus
numérisés rassemblés auprès des institutions chinoises, indiennes et
européennes ne sont qu'un fragment de la réalité des sources existentes,
tout en étant suffisamment représentatif pour permettre de tirer des
conclusions pertinentes.

Un autre défi tenant à la valorisation des collections et des projets de
recherche concerne la question juridique~: le respect du droit d'auteur.
Toute diffusion d'une numérisation doit se faire dans le respect de
l'ensemble des dispositions du code de la propriété intellectuelle, en
France il se décline en deux grandes catégories. Le droit moral est
perpétuel, inaliénable et imprescriptible, il comprend plusieurs
prérogatives~: le droit de paternité, de divulgation, la protection de
l'intégrité de l'œuvre. Le droit moral ne peut être cédé ou vendu, et il
subsiste même après la mort de l'auteur, étant alors transmissible à ses
héritiers. Le droit patrimonial concerne les aspects économiques de
l'œuvre et permet à l'auteur de tirer des revenus de son exploitation.
Il inclut principalement les droits de reproduction et de
représentation. Limité dans le temps, il dure toute la vie de l'auteur
puis 70 ans après sa mort. Après cette période, l'œuvre tombe dans le
domaine public et peut être librement utilisée par tous, tout en
respectant le droit moral de l'auteur. Les principes fondamentaux des
droits d'auteur sont largement reconnus à l'international, notamment
grâce à la Convention de Berne et aux accords \textsc{adpic}. Cependant, les
spécificités, notamment en matière de droit moral et de durée de
protection, peuvent varier selon les législations nationales.

Dans le cas de corpus sur lesquels ne s'appliquent pas de droits de
propriété intellectuelle, c'est-à-dire les corpus qui font partie du
domaine public, il renvient à la Bibliothèque d'établir quelle licence
va conditionner leur distribution et leur réutilisation. Elles
n'imposent généralement pas de nouvelles restrictions, privilégiant la
mise à disposition. Il existe cependant des exceptions~: même si les
manuscrits et les textes originaux peuvent ne plus être sous droits
d'auteur en raison de leur ancienneté, les copies numériques réalisées
par la Bibliothèque du Vatican sont protégées par leurs politiques
spécifiques de droit d'auteur. Toute reproduction ou diffusion sans
autorisation est interdite, une considération importante dans l'optique de la diffusion des corpus de recherche à un large public.

\hypertarget{projet-dune-plateforme-web}{%
\subsection{Projet d'une plateforme
web}\label{projet-dune-plateforme-web}}

La science ouverte renvoie au principes qui veulent rendre la recherche
-- ses méthodes et ses résultats -- accessible à tous. Les chercheur.ses
qui publient conformément à ses principes visent à plus de transparence,
de collaboration et à une recherche plus étendue et plus efficace. Les
principes de la recherche (ou science) ouverte veulent rendre les
méthodes de recherche et les données résultantes librement disponibles,
souvent via Internet, afin de soutenir la reproductibilité et,
potentiellement, la collaboration entre des projets largement diffusés.
À cet égard, elle est liée à la fois aux logiciels open-source et à la
science participative~: une approche de la recherche scientifique dans
laquelle des membres du grand public contribuent au processus, ceci à
divers degrés, de la collecte des données à l'encouragement à
l'éducation scientifique informelle.

Conformément aux principes de la science ouverte, \eida vise au long
terme le développement d'une interface web incluant à la fois le
front-end\footnote{Le développement web frontal, ou \textit{front-end}, désigne
  les productions d'une application avec lesquelles l'utilisateur.rice peut
  interagir directement.} et le \textit{back-office} pour collecter, étudier,
éditer et visualiser des diagrammes, modelée sur la plateforme \dishas
existante pour les tables astronomiques. La nouvelle plateforme
permettra à la communauté de recherche d'accéder aux outils utilisés
pour produire les résultats publiés par \eida, soit les outils de
traitement ou la future plateforme d'édition, aidant ainsi la communauté
scientifique à mener et publier des travaux portant sur les
illustrations \emph{a minima} à caractère historique, voire moins
spécifiques\footnote{Le niveau de spécialisation de ce système dépend
  justement des efforts de modularisation et d'interopérabilité dont la
  mise en œuvre sera discutée en partie III.}. L'application dédiée aux chercheur.ses de \eida et \vhs est développé en gardant à l'esprit son utilisation future par
un public plus vaste que les seuls chercheur.ses affiliés à ces projets.
Cette approche vise à rendre les outils accessibles et utilisables par
une audience diversifiée, tout en restant centrée sur le monde
académique.

À l'été 2024, il existe uniquement l'interface administrateur -- ou
\textit{back-office} -- pour la gestion de la base de données et la mise à
disposition des modèles de vision. L'accès à cette plateforme est
conditionnée par les droits admin accordés au chercheur.ses du projet, et à
tous ceux qui le demandent. À terme, cette plateforme est amenée à être
dotée d'une interface publique déployée en ligne, donnant un accès à la
base de données et aux résultats des traitements\footnote{Une version
  \emph{démo} de la plateforme est actuellement en développement pour
  tester les modèles de vision sur des images personnelles et sans
  stockage. Elle sera déployée sur les serveurs de l'École des Pont, en
  utilisant une instance de l'\api développée dans le cadre du projet
  \eida}. Les enjeux tenant à la construction de cette plateforme seront
détaillés en \hyperlink{partie-III}{partie III}.

En suivant les principes \fair, la plateforme garantira l'ouverture des
outils et des résultats de recherche, facilitant ainsi une collaboration
et une diffusion plus larges des connaissances.