
L'astronomie est issue d'une tradition continue vieille de près de 4000
ans qui transcende les cultures et les langues. Les théories, les
savoirs, les méthodes, au gré de leur diffusion, se mélangent aux
pratiques autochtones pour servir les usages locaux, qui bien souvent
renforcent des dynamiques de pouvoir en place, que ce dernier soit
politique ou religieux. Les sciences astrales revêtent ainsi une
importance culturelle majeure.

Une étude approfondie des sources révèle les processus par lesquels les
connaissances se sont enrichies et transformées au contact de
différentes cultures. Leur examen permet en outre de souligner les
spécificités et les innovations de chaque tradition, tout en montrant
les interconnexions et les influences réciproques qui ont façonné
l'évolution de l'astronomie. Les schémas de circulation sont dans de
nombreux cas de très grande portée géographique, chronologique et
culturelle. Ils relient des contextes de production de connaissances à
l'échelle afro-eurasienne et sur des périodes de siècles voire de
millénaires. La trace de ces transmissions supporte une vision connectée
et globale de l'histoire des cultures et du savoir.

La grande diversité des sources et des approches possibles rend
cependant difficile une approche globale. À ce titre, il est important
de cibler un objet d'étude : \eida se focalise ainsi essentiellement dans
la transmission de la tradition ptoléméenne, et le corpus se compose
donc de ressources manuscrites et imprimées relevant de cette tradition.

\hypertarget{ptolemee-modele-et-transmission}{%
\subsection{Ptolémée : modèle et
transmission}\label{ptolemee-modele-et-transmission}}

Ptolémée tient une place proéminente dans l'histoire de l'astronomie et
des mathématiques. Son nom reste associé à la conception d'un système
astronomique qui plaçait la Terre immobile au centre du monde, et dont
la mise en question, de Copernic à Newton, a commandé une révolution
scientifique.

Dans sa \emph{Syntaxe mathématique}, plus connue sous le titre
d'\emph{Almageste}, et dont la dernière observation consignée date de
141, il expose l'ensemble des connaissances astronomiques de son époque.
Notamment il perfectionne le modèle élaboré par Hipparque, à qui il
emprunte la découverte de l'excentricité des trajectoires apparentes du
Soleil et de la Lune par rapport à la Terre, et l'idée de composer ces
trajectoires à l'aide de deux mouvements distincts. Il élabore un
système géocentrique au moyen d'un ensemble complexe de trajectoires
circulaires des objets pris dans un mouvement uniforme : les déférents,
autour de la Terre, et les épicycles, dont les centres parcourent les
déférents\footcite{lequeux_systeme_nodate}.

En effet, les sociétés anciennes attendent des
corps astraux (soleil, lune, planètes et étoiles) un mouvement uniforme
et le plus ``parfait'' possible, c'est-à-dire un cercle. Pourtant la
trajectoire de ces corps, observée empiriquement, n'est pas circulaire.
Le modèle de Ptolémée explique ces imperfections en postulant que les
mouvement apparemment irréguliers sont dû à cette fameuse combinaison de
plusieurs trajectoires circulaires régulières vues depuis la Terre,
point statique. Les planètes se déplacent à vitesse uniforme sur un
cercle (l'épicycle) dont le centre se déplace à vitesse uniforme sur un
cercle coplanaire (le déférent), dont la Terre est le centre.

En plus de la description du mouvement des astres, Ptolémée dresse dans
l'\emph{Almageste} des tables établissant les positions de la lune et
prévoyant les périodes et les caractéristiques des éclipses avec une
précision inédite\footcite{raymond_jones_ptolemy_2024}, un catalogue des étoiles, un traité
complet de trigonométrie plane et sphérique et une description des
instruments nécessaires à un grand observatoire.

L'œuvre de Ptolémée fera référence, et en tant que synthèse des
connaissances astronomiques antérieures, sa transmission correspond
à celle de la vision des pratiques de l'astronomie grecque à son apogée,
et sa diffusion façonnera la production astronomique pendant
près de treize siècles. D'ailleurs le nom d'Almageste date de la
transmission par les civilisations arabes à l'occident.

En effet, lors de la chute de l'Empire Romain d'occident, la majeure
partie des ouvrages antiques sont perdus et la science occidentale
stagnera jusqu'au \textsc{xii}\ieme siècle. Elle continuera cependant à progresser
ailleurs : notamment dans le monde arabe et musulman. Dès le \textsc{viii}\ieme et
\textsc{ix}\ieme siècle, les Arabes vont traduire dans leur langue la plupart des
grands textes scientifiques de l'Antiquité, en particulier les œuvres
d'Aristote et l'\emph{Almageste} de Ptolémée. Sans remettre en cause le
géocentrisme et le système de Ptolémée, ils le perfectionnent et
l'amènent à un très grand degré de précision. Nécessaire à la stricte
observation des règles de l'islam, l'astronomie arabe se développe et se
diffuse, grâce aux travaux d'al-Biruni, al-Hazen ou al-Sufi\footcite{noauthor_monde_nodate}.

À partir du \textsc{xi}\ieme et surtout du \textsc{xii}\ieme siècle, au fil des conquêtes des
occidentaux en Espagne et en Sicile, les textes grecques sont traduits
en latin via la traduction arabe. La transmission des savoirs
gréco-arabes -- notamment les traductions arabo-latines de
l'\emph{Almageste} et du Livre des étoiles fixes d'al-Sufi -- ouvre la
voie à un renouveau scientifique dans l'Occident chrétien, permettant
ainsi l'essor des grandes universités européennes de l'époque (Paris,
Oxford, Bologne, etc). On redécouvre les modèles d'Aristote et de
Ptolémée en les adaptant aux conceptions chrétiennes\footcite[``Le
  système géocentrique devient le modèle astronomique et théologique de
  l'Église, qui ne remet pas en cause la sphéricité de la Terre''][]{noauthor_monde_nodate}.

Avant l'avènement de l'astronomie grecque, les Babyloniens, dès le premier
millénaire \jc, utilisaient des calculs arithmétiques pour prévoir
la position des planètes. Ces théories ont voyagé jusqu'en Perse et en
Inde, où elles ont été adaptées et combinées à des méthodes autochtones.
Les théories grecques de l'époque de Ptolémée et de son prédécesseur
Hipparque sont également parvenues jusqu'en Inde, créant un matériel
complexe dont les influences sont difficiles à démêler. Parmi les
pratiques empruntées aux théories grecques, on relève l'emploi de termes
-- par exemple le titre du canon \emph{Romaka Siddhanta} datant du début
du \textsc{v}\ieme et qui marque les origines de la science astronomique sanskrite --
ainsi que des modèles épicycliques et des méthodes de calculs requérant
des paramètres numériques hérités d'Hipparque\footnote{\cite[p.6-7]{mercier_studies_2004} in \cite[p.15]{albouy_mediation_2019}}.

La tradition chinoise se développe de manière relativement indépendante
et les échanges ne débutent qu'autour de 200 \jc. Elle se distingue
de celle des Grecs par un intérêt plus marqué pour la prédiction
d'événements singuliers plutôt que pour les théories cosmologiques
cherchant l'établissement d'un modèle d'organisation du ciel. En effet,
en Chine impériale, l'astronomie a une fonction politique. L'empereur
est considéré comme le Fils du Ciel et ainsi la régulation du
calendrier, ou bien le succès (ou l'échec) de ses astronomes à prédire
une éclipse, se reflétaient positivement ou négativement sur lui.
L'inclusion croissante des diagrammes dans les traités après les
missions jésuites à partir du \textsc{xvi}\ieme siècle révèle l'influence des
pratiques d'Europe de l'ouest.

Pour conclure, l'\emph{Almageste} se présente donc comme une sorte
d'encyclopédie des connaissances d'une époque qui s'est enrichie avec le
temps au point de rendre difficile l'appréciation de son état originel.
Œuvre sans cesse recopiée au cours des siècles, passant du grec à
l'arabe puis au latin, transmise à travers tout le bassin méditerranéen
et dominant le \ma occidental après avoir conquis l'Islām, chaque
traduction et chaque copie de l'Almageste n'ont pas seulement transmis
son contenu, mais l'ont aussi adapté et enrichi en fonction des
contextes culturels et scientifiques de chaque époque. L'œuvre
ptoléméenne a servi de base à de nombreux commentaires et traités,
intégrant progressivement des éléments de connaissance issus de diverses
traditions scientifiques, et illustrant ainsi l'interconnexion des
savoirs à travers les civilisations.

\begin{kwote}
``Certains indices dans les manuscrits révèlent les emprunts
intellectuels qui s'opèrent au fur et à mesure des copies ; les méthodes
de calcul, le tracé des diagrammes, la mise en page des tables, la
structuration des textes techniques, la mention d'auteurs antérieurs, la
réutilisation de paramètres astronomiques ou même la récurrence de
certaines erreurs sont autant de signes qui témoignent des échanges
culturels qui ont façonné la pratique de l'astronomie.''\footcite[p.14]{albouy_mediation_2019}
\end{kwote}

Comme l'entend \citeauthor{albouy_mediation_2019}, les diagrammes font partie des révélateurs des
échanges intellectuels.

\hypertarget{le-diagramme-vecteur-de-connaissances}{%
\subsection{Le diagramme vecteur de
connaissances}\label{le-diagramme-vecteur-de-connaissances}}

L'historiographie et l'histoire des sciences n'échappent pas au récent \textit{visual digital turn}\footcite[``Digital humanities research has
  focused primarily on the analysis of texts. This emphasis stems from
  the availability of technology to study digitized text. Optical
  character recognition allows researchers to use keywords to search and
  analyze digitized texts. However, archives of digitized sources also
  contain large numbers of images.''][]{wevers_visual_2020} général des
humanités, montrant à quel point la production et la diffusion du savoir
croisent les représentations visuelles rendant compte de ces
connaissance. De fait, on ne s'intéresse plus seulement au texte. Or les
astronomes, au fil de l'histoire, ont eu recours à un grande diversité
de matériaux. Les sources primaires sont constituées par des instruments
et des écrits, ces derniers eux-même hétérogènes. Dans les traités
anciens, on trouve des descriptions détaillées, des propositions
mathématiques, des tables de calcul, et des diagrammes illustrant
souvent le texte qu'ils accompagnent. Les sources primaires peuvent en
outre être enrichies de commentaires et de gloses, prose ou
illustrations, témoignant de la manière dont les connaissances ont été
transmises et interprétées. Elles révèlent également les méthodes
pédagogiques employées pour enseigner ces savoirs.

Au cœur de cette diversité, le diagramme, objet hybride pour deux
raisons : il combine un contenu géométrique (des lignes, arcs et
cercles) et des labels, et il entretient un lien (plus ou moins étroit)
avec le texte qui l'accompagne.

En tant que structure de pensée, la figuration géométrique -- une forme
de création de modèles associée à l'élaboration et à la résolution de
problèmes dans divers domaines de la pensée humaine liés au calcul
abstrait et à la modélisation des idées -- est aussi ancienne que
presque toute autre forme d'enregistrement des pensées et des idées. Des
diagrammes utilisés pour calculer la superficie de parcelles et de
terres apparaissent dans le Papyrus mathématique Rhind, la source la
plus importante qui subsiste pour l'histoire des mathématiques dans
l'Égypte ancienne\footcite[p.6]{safran_diagram_2022}. Instruments
de pensée et de démonstration, ils servent non seulement à transmettre
le savoir mais aussi à le produire. Et dans un étrange mouvement
métaréflexif, ils permettent aux historien.nes des sciences de produire la
connaissance sur ces anciennes traditions heuristiques et leur
transmission.

Les diagrammes sont, pour les astronomes, le support d'une pratique
scientifique, et sont ainsi révélateurs de leurs méthodes, de leur
contexte d'exercice et de leur conception de leur discipline. Ils
revêtent des rôles et des aspects différents, permettant d'identifier
des modes diagrammatiques\footnote{La ``diagrammatisation'' désigne
  assez largement l'investissement des acteurs dans la complexification
  des représentations visuelles des propositions scientifiques présentes
  dans les traités.} spécifiques d'un lieu ou d'une époque, et traçant
des lignes de diffusion des pratiques et des savoirs.

Au \ma, trois grandes cultures coexistent en Eurasie : les
cultures byzantine, islamique, et d'Europe occidentale. Elles
connaissent des évolutions différentes en termes linguistiques et
religieux ; cette diversité est vraie également pour les usages auxquels
les diagrammes astronomiques étaient destinés, pour les domaines dans
lesquels ils étaient reconnus comme des instruments de pédagogie et des
vecteurs de pensée, ainsi que pour la place accordée à la culture
visuelle plus généralement et aux modes de représentation
diagrammatiques. Pourtant cette coexistence donne lieu à des échanges
intellectuels, artistiques, diplomatiques et commerciaux. Les
traductions d'œuvres savantes, les transferts de manuscrits illustrent
la porosité des frontières du savoir et de l'interdépendance des
cultures. Les diagrammes astronomiques sont à ce titre témoins des
chemins de diffusion des connaissance et des pratiques des sciences.

\emph{Comment ces diagrammes parlent-ils aux historien.nes ?}

Les diagrammes peuvent être étudiés intrinsèquement (quelles conventions
gouvernaient le langage visuel, quelle fonction assumaient-ils ?) ou
extrinsèquement (pour comprendre la transmission de ces traditions et
ces pratiques entre l'Europe et l'Asie, en passant pas la péninsule
arabique).

On pourrait penser que les formes et éléments visuels
utilisés pour une démonstration géométrique soient universels, qu'ils
restent les mêmes quelle que soit la date et la langue de l'explication
textuelle, le grec, l'arabe ou le latin\ldots{} Et pourtant le contexte
géographique, temporel, et les aspects matériels liés aux technologies
d'inscription changent profondément le fonctionnement et les objectifs des diagrammes,
leurs objectifs.

L'évolution des conventions graphiques en sont un exemple frappant. Par
exemple, l'axe vertical de la Terre, bien que représenté à plat sur la
page, fut conventionnellement compris comme un axe perpendiculaire à la
coupe du globe. Au fil du temps, de nouvelles conventions graphiques ont
été adoptées, et pour les lecteurs d'aujourd'hui on représenterait
sûrement le globe terrestre avec sa profondeur pour expliciter la
représentation. Citons en outre la représentation des phases lunaires,
qui a connu une évolution concernant l'association des couleurs claire
et sombre à la pleine lune et à la nouvelle lune. Si aujourd'hui on
associerait plutôt la pleine lune à un aplat de couleur claire et la
nouvelle lune à une couleur sombre, les manuscrits médiévaux byzantins
adoptent le référentiel contraire.

De même, le rôle du diagrammes est fluctuant, et va au delà du simple
support démonstratif au service du texte. Par exemple ceux des
\emph{Traités logiques} d'Aristote ont probablement circulé
indépendamment du texte, même si tout indique que les écrits les
appelaient dès le départ. Cela souligne la distinction entre le
diagramme en tant qu'objet de démonstration et de discussion complétant
une proposition d'un côté, et le diagramme en tant qu'accompagnement des
textes scolaires, qui aide à la compréhension de l'autre\footcite[p.5]{safran_diagram_2022}. Le
diagramme peut ainsi constituer la preuve et le support d'une
réinterprétation de la proposition textuelle.

Par-dessus tout, les transformations subies au fil des copies et des
réceptions sont éloquentes pour les chercheurs. Bien que les diagrammes
soient initialement conçus pour clarifier et expliquer une proposition textuelle, ils peuvent
parfois être des vecteurs de confusion (ou d'innovation). Le même diagramme d'une même
œuvre soumis à un processus constant de transformation par les scribes,
les artistes ou les lecteurs/commentateurs. 

La recherche des erreurs
transmises a ainsi un intérêt philologique important. Les cas de
méprises et les malentendus sont peut-être plus nombreux que les cas de
compréhension fidèle lors de la traversée des frontières géographiques,
culturelles, religieuses et/ou linguistiques, et l'étude des erreurs et modifications
révèle leurs aspects heuristiques, autant qu'il peut amener à
l'établissement d'un stemma\footcite{raynaud_building_2014}.

L'importance des diagrammes dans les transmissions est illustrée par l'exemple des diagrammes attribués à al-Ḥajjāj, en lien avec la transmission arabe des \emph{Éléments} d'Euclide. Bien que la traduction originale d'al-Ḥajjāj soit perdue, les diagrammes retrouvés dans divers manuscrits montrent qu'il utilisait parfois des schémas différents de ceux adoptés dans la tradition arabe ultérieure. Ces diagrammes ont probablement joué un rôle clé dans l'élaboration d'une version alternative de la géométrie euclidienne, influençant ainsi la transmission vers l'Europe via les traductions latines et hébraïques\footcite{de_young_editing_2014}.

Ainsi, au travers des variations, similarités et évolutions des
diagrammes, les historien.nes peuvent reconstruire les pratiques
scientifiques des astronomes et comprendre les contextes culturels et
sociaux dans lesquels elles s'inscrivaient. De telles études permettent
aussi de tracer la circulation des sources dans le monde entre les
différentes cultures et de comprendre comment celle-ci s'approprient le
contenu. En somme, l'évaluation de phénomènes diagrammatiques
indéniablement disparates à travers des géographies éloignées permet
d'identifier des modalités d'échanges culturels et leur impact sur la
construction du savoir scientifique.

Les travaux antérieurs sur l'illustration scientifique se concentrent
essentiellement sur des types spécifiques de diagrammes, situés dans des
contextes déterminés chronologiquement et culturellement. Un exemple de
ce paradigme est le projet précédent ALFA, qui porte sur les sources
de tradition alfonsine un regard eurocentré. Cependant, à
l'aune des remarques précédentes, il devient pressant de dépasser cette
perspective centripète en étendant la portée géographique et temporelle
des projets ; ambition rendue possible par la disponibilité des sources
primaires en ligne, permettant la construction de bases de données
d'images de grande envergure. Ainsi peut être mise en œuvre une analyse
plus inclusive et diversifiée des sources iconographiques -- notamment
les diagrammes\footcite{husson_eida_2022}.

Comme le dit Jeffrey F. Hamburger dans un plaidoyer pour une étude
comparative des diagrammes astronomique : ``Diagrams can thus be seen
not as embodiements of eternal truths, but, rather, as culturally
embedded objects''\footcite[p.7]{safran_diagram_2022}.

\begin{kwote}
``(\ldots) to be effective, a cross-cultural comparison of diagrammatic
traditions must look beyond the prima facies appearence of the diagrams
under consideration to their underlying operations and the patterns of
thoughts that they both codified and were intended to
inculcate.''\footcite[p.3]{safran_diagram_2022}
\end{kwote}         

Les interrogations soulevées par le projet \eida se déclinent donc comme
suit : analyser l'articulation entre les fonctions documentaires et
épistémiques des diagrammes au sein de l'histoire des pratiques
astronomiques, interroger l'importance des diagrammes dans la
construction et la transmission des connaissances scientifiques, et
enfin identifier des schémas récurrents dans les modalités de
circulation de ces diagrammes. S'appuyant sur ces analyses, les
chercheurs pourront à leur tour tracer des lignes : au sens figuré
construire ``a web of connections linking the points represented by the
individual contributions together into a larger pattern''\footcite[p.10]{safran_diagram_2022} ; et
au sens propre, cette vision globale sur la vie des images et des œuvres
pouvant permettre l'établissement d'éditions critiques normalisées.