Comme mentionné précédemment, les objectifs disciplinaires du projet
consistent à étudier la diversité des fonctions et des modes de
circulation des diagrammes astronomiques à l'échelle afro-eurasienne et
sur une période de temps étendue. Cette approche trouve des assises
conceptuelles et techniques chez des projets similaire (\vhs) ou chez ses
proches parents (\dishas), sans verrouiller pour autant l'émergence
progressive des spécificités du projet.

\hypertarget{contributeurs-et-partenaires}{%
\subsection{Contributeurs et
partenaires}\label{contributeurs-et-partenaires}}

Coordonné par Mathieu Husson, \eida repose sur l'association fructueuse
entre l'équipe d'historien.nes basée au \syrte de l'Observatoire de Paris
prenant en charge la composante disciplinaire, et des chercheurs en
computer vision basés au laboratoire \imagine de l'école des Ponts. Ce
dernier possède l'expertise en vision artificielle et apprentissage
machine nécessaire à la réalisation des aspirations du projet, tandis
que les chercheur.ses de l'équipe d'histoire des sciences développent une
approche conceptuelle des sources.

Le projet \vhs (Vision artificielle et analyse Historique de la
circulation de l'illustration Scientifique), coordonné par l'Institut
des sciences du calcul et des données, est également partenaire d'\eida
pour le développement des algorithmes de détection et de la plateforme
de mise à disposition des modèles et des résultats. Mathieu Aubry,
chercheur titulaire membre de l'équipe \imagine et PI du projet \vhs, fait
le lien entre les deux équipes. Les deux projets profitent de son
expérience au sein de projets précurseurs comme le projet \textsc{anr}
EnHerit\footcite{noauthor_enhancing_nodate} sur
l'enrichissement des bases de données d'images du patrimoine.

\vhs s'intéresse tout comme \eida aux illustrations et leur évolution dans
les corpus scientifiques du \ma et de l'époque moderne. Les deux projets
non seulement partagent certains développements (en fonction cependant
de leurs objectifs spécifiques) mais se coordonnent aussi dans la
production de publications académiques, de contenus pédagogiques, et
l'organisation de séminaires pour la diffusion et la visibilité des
résultats de la recherche.

\eida/\vhs créent alors un espace d'échange entre l'histoire des sciences
et les pratiques contemporaines, concentrant l'expertise théorique des
sources et celle, technique, du \dl appliqué à la donnée
historique.

\hypertarget{projets-precedents}{%
\subsection{Projets précédents}\label{projets-precedents}}

\eida apparaît comme la suite logique de \dishas, large projet de
plateforme visant à rassembler les sources de la recherche en histoire
des sciences astrales portant sur des corpus de traditions multiples et
diverses, notamment des corpus chinois, sanskrits, arabes, latins et
hébreux. Toutefois, les sources primaires astronomiques relevant d'une
telle variété -- textes théoriques et poétiques, diagrammes, instruments
de mesures, entre autres -- il n'était pas possible d'inclure tous ces
objets dans une seule plateforme, les infrastructures numériques à
concevoir pour leur étude étant trop différentes. Il a été décidé
d'approcher ces types de documents les uns après les autres : si les
tables astronomiques ont été choisies pour constituer l'unité
fondamentale du projet \dishas, \eida porte sur les diagrammes. Fédérant
plusieurs projets de recherche en astronomie ancienne, le projet pilote
\dishas considère un corpus pluriel, afin d'étudier les pratiques des
astronomes et des milieux intellectuels, ainsi que la transmission et la
circulation du savoir dans ces milieux. La conception d'une plateforme
entend répondre à des problématiques de trouvabilité et de médiation des
données de la recherche en histoire de l'astronomie.

\begin{kwote}
``sans un effort de médiation la masse considérable des données
produites par la communauté scientifique en SHS peut rester inaccessible
et perdue dans des systèmes d'information obsolètes.''\footcite[p.xix]{albouy_mediation_2019}
\end{kwote}       

\dishas et sa jeune sœur \eida poursuivent donc un triple objectif commun
: l'accès à la donnée, l'accès à des outils d'analyse et d'édition, et
la valorisation des travaux des spécialistes. En élaborant des
plateformes sur mesure pour exposer les données de la recherche et
donner accès aux outils de traitement, des vecteurs d'analyse
alternatifs deviennent disponibles pour la communauté académique. Ainsi
\dishas propose des outils d'édition, d'analyse et visualisation de
tables astronomiques de traditions diverses allant de la Chine à
l'Europe. \eida poursuit des ambitions parallèles avec un autre objet :
les diagrammes. L'ensemble de ces projets vise au long terme à
l'élaboration d'un véritable système d'information se voulant outil
extensif pour le traitement, l'analyse et la mise à disposition des
sources en histoire de l'astronomie.

\textsc{alfa}, examinait des sources issues du haut
\ma en Europe, étudiant l'interaction entre diagrammes
astronomiques, tables numériques et textes dans les manuscrits de
tradition alfonsine. Les initiatives collectives telles que \dishas et \eida répondent aux besoins, au
sein de la discipline, d'étendre la typologie des sources étudiées et
de dépasser la perspective eurocentrée en élargissant la portée
géographique des projets. Elles ont alors permis de créer des pont entre les projets, à l'image de \dishas qui rassemble HAMSI et PAL, portant sur des corpus spécialisés, respectivement sur l'astronomie sanskrite et arabe.